%%  MATH5505 Ramsey Theory Lecture 6: Applications
%%
%%  by Thomas Britz 2018S1
%%

%% OK Overview of Ramsey Theory Chapter
%% OK The classical geometrical convexity(?) theorem using Ramsey's Theorem
%% OK Erdos' & co' application of Ramsey's Theorem
%% OK Others? The links and lists
%%%%  More Pigeonhole Principle stuff:
%% OK   Euclid's Theorem/Bezout's Identity
%% OK   Chinese Remainder Theorem
%% OK   Proizvolov's Identity
%%      Mankel's Theorem and mention Turan's Theorem
%%      Brouwer's Fixed Point Theorem.
%%      Also, maybe: Dirichlet's Approximation Theorem to show that x^2 - Dy^2 =1 or something always has a solution
%%%%    Or include in first lecture along with this theorem? Might be better...
%%%%    And move some of the 1st lecture stuff to this lecture?


\documentclass[12pt,a4paper,landscape]{article}
\special{landscape}

\usepackage{latexsym,amsfonts,amsmath,amssymb} %,calc,fancybox,epsfig,amscd,tabularx}
\usepackage{pstricks,pst-plot,pst-node,pst-tree}
%\usepackage{array,graphicx,epsfig}
%\usepackage{multicol,multirow,hyperref,rotating}
%\usepackage[T1]{fontenc}
%\usepackage{yfonts}


\mag=\magstep4

\newrgbcolor{green}{0.2 0.6 0.2}
\newrgbcolor{darkgreen}{0 0.4 0}
\newrgbcolor{graydarkgreen}{0.5 0.67 0.5}
\newrgbcolor{lightgreen}{0 0.75 0}
\newrgbcolor{skyblue}{0.5 0.80 1}
\newrgbcolor{darkblue}{0 0 0.6}
\newrgbcolor{darkred}{0.6 0 0}
\newrgbcolor{halfred}{.8 0 0}
\newrgbcolor{white}{1 1 1}
\newrgbcolor{nearlywhite}{0.95 0.95 0.95}
\newrgbcolor{offwhite}{0.9 0.9 0.9}
\newrgbcolor{lightgray}{0.85 0.85 0.85}
\newrgbcolor{halfgray}{0.8 0.8 0.8}
\newrgbcolor{altgray}{0.67 0.67 0.67}
\newrgbcolor{darkyellow}{1 0.94 0.15}
\newrgbcolor{halflightyellow}{1 1 0.4}
\newrgbcolor{lightyellow}{1 1 0.7}
\newrgbcolor{gold}{0.96 0.96 0.1}
\newrgbcolor{lightblue}{0.5 0.5 1}
\newrgbcolor{amber}{1 0.75 0}
\newrgbcolor{hotpink}{1 0.41 0.71}

\setlength{\textheight}{18.0truecm}
\setlength{\textwidth}{26.0truecm}
\setlength{\hoffset}{-12.0truecm}
\setlength{\voffset}{-5.2truecm}

\parindent 0in

%\setlength{\bigskipamount}{5ex plus1.5ex minus 2ex}
%\setlength{\parindent}{0cm}
%\setlength{\parskip}{0.2cm}

\psset{unit=6mm,linewidth=.06,dotscale=1.5,fillcolor=white,fillstyle=none,
 linecolor=gray,framearc=.3,shadowcolor=offwhite,shadow=true,shadowsize=.125,
 shadowangle=-45,dash=7pt 5pt}
%\psset{unit=.4,linecolor=gray,fillcolor=offwhite,shadowsize=.2,framearc=.3}
%\psset{unit=10mm,linewidth=.03}

\def\dedge{\ncline[linestyle=dashed]}

\def\np{\newpage}
\def\bl{\blue}
\def\bk{\black}
\def\wh{\white}
\def\rd{\red}
\def\lg{\lightgray}
\def\gr{\green}
\def\dr{\darkred}
\def\dg{\darkgreen}
\def\gdg{\graydarkgreen}
\def\dgy{\darkgray}
\def\ec{\dg}

\newcommand{\ora}[1]{\overrightarrow{#1}}

\newcommand{\llb}{\\[1mm]}
\newcommand{\lb}{\\[3mm]}
\newcommand{\blb}{\\[5mm]}
\newcommand{\hlb}{\\[48mm]}

\newcommand{\mynewpage}{\newpage\vspace*{-10mm}}

\newcommand{\vc}[1]{\begin{pmatrix}#1\end{pmatrix}}

\newcommand{\mytext}[1]{\text{\black#1\blue}}

\newcommand{\qbinom}[2]{\genfrac{[}{]}{0pt}{}{#1}{#2}}

\newcommand{\dl}{\psset{linestyle=dashed,linecolor=altgray,shadow=false}}
%\newcommand{\dl}{\psset{linestyle=dashed,linecolor=blue}}

\newcommand{\myframe}[1]{\pspicture(0,0)(0,0)\psset{unit=1cm,
 shadowcolor=offwhite,shadow=true,shadowangle=-45,linewidth=.03,linecolor=gray,
 fillcolor=lightgray,shadowsize=.15,framearc=.3}\psframe#1\endpspicture}

\newcommand{\mypicture}[1]{\pspicture(0,0)(0,0)\psset{unit=1cm,
 shadowcolor=offwhite,shadow=true,shadowangle=-45,linewidth=.03,linecolor=gray,
 fillcolor=lightgray,shadowsize=.15,framearc=.3}#1\endpspicture}

\newcommand{\mymatrix}[1]{{\psset{unit=4mm,linewidth=.03,linecolor=gray,fillstyle=solid,fillcolor=offwhite,shadow=false,framearc=0}\pspicture(0,0)(6,4)
  \psframe(0,0)(7,5)#1\psframe[linecolor=gray,fillcolor=offwhite,linewidth=.03,fillstyle=none,framearc=0](0,0)(7,5)\endpspicture}}

\newcommand{\mypair}[2]{{\psset{unit=4mm,linewidth=.03,linecolor=gray,fillstyle=solid,fillcolor=offwhite,shadow=false,framearc=0}\pspicture(0,.2)(2,.8)
  \psframe(0,0)(2,1)\darkgray\rput(.5,.5){#1}\rput(1.5,.5){#2}\psframe[linecolor=gray,fillcolor=offwhite,linewidth=.03,fillstyle=none,framearc=0](0,0)(2,1)\endpspicture}}

\newcommand{\mysixtuple}[6]{{\psset{unit=4mm,linewidth=.03,linecolor=gray,fillstyle=solid,fillcolor=offwhite,shadow=false,framearc=0}\pspicture(0,.2)(6,.8)
  \psframe(0,0)(6,1)\darkgray\rput(.5,.5){#1}\rput(1.5,.5){#2}\rput(2.5,.5){#3}\rput(3.5,.5){#4}\rput(4.5,.5){#5}\rput(5.5,.5){#6}
  \psframe[linecolor=gray,fillcolor=offwhite,linewidth=.03,fillstyle=none,framearc=0](0,0)(6,1)\endpspicture}}

\newcommand{\acset}{\psset{fillstyle=none,linecolor=red}}
\newcommand{\drr}{\psset{linecolor=darkred,fillcolor=red,linewidth=.034}}
\newcommand{\dbb}{\psset{linecolor=darkblue,fillcolor=blue,linewidth=.034}}
\newcommand{\bbl}{\psset{linecolor=blue,fillcolor=lightblue}}
\newcommand{\dbhg}{\psset{linecolor=darkblue,fillcolor=halfgray}}
\newcommand{\bhg}{\psset{linecolor=blue,fillcolor=halfgray}}
\newcommand{\rhg}{\psset{linecolor=red,fillcolor=halfgray}}
\newcommand{\ghg}{\psset{linecolor=green,fillcolor=halfgray}}
\newcommand{\dgg}{\psset{linecolor=darkgreen,fillcolor=green,linewidth=.034}}

\newcommand{\mychain}{\pspicture(0,0)(0,0)\dbb
  \psset{shadowsize=.2,shadow=true,shadowcolor=offwhite,shadowangle=-45,shadowsize=.15}
  \pscircle(0,0){.09}\pscircle(0,1){.09}\pscircle(0,2){.09}\pscircle(0,3){.09}
  \psline(0,.125)(0,.875)\psline(0,1.125)(0,1.875)\psline(0,2.125)(0,2.875)\endpspicture}

\newcommand{\myantichain}{\pspicture(0,0)(0,0)\drr
  \psset{shadowsize=.2,shadow=true,shadowcolor=offwhite,shadowangle=-45,shadowsize=.15}
  \pscircle(0,0){.09}\pscircle(1,0){.09}\pscircle(2,0){.09}\pscircle(3,0){.09}\endpspicture}

\newcommand{\mybullet}{{\psset{unit=6mm,dotscale=1.5,linewidth=.05,fillcolor=black,linecolor=black,framearc=.3,shadow=true}
  \pspicture(-.3,0)(.8,.4)\qdisk(.25,.2){.1}\pscircle*[linecolor=gray](.25,.2){.07}\endpspicture}}

\newcommand{\mysmallbullet}{{\psset{unit=4mm,dotscale=1.5,linewidth=.05,fillcolor=black,linecolor=black,framearc=.3,shadow=true}
  \pspicture(-.3,0)(.8,.4)\qdisk(.25,.2){.1}\pscircle*[linecolor=gray](.25,.2){.07}\endpspicture}}

\newcommand{\mybulletv}{{\psset{unit=6mm,dotscale=1.5,linewidth=.04,linecolor=black,fillstyle=solid,
  fillcolor=gray,shadow=false}\pspicture(0,0)(.5,.4)\pscircle(.25,.2){.15}\endpspicture}}

\newcommand{\mystar}{\pspicture(0,0)(0,0)\psset{unit=1cm}\gold\large$\star$\endpspicture
  \pspicture(0,0)(0,0)\psset{unit=1cm}\rput(-.23,.193){\small\white$\star$}\endpspicture}

\newcommand{\mystep}{{\psset{unit=6mm,dotscale=1.5,linewidth=.06,linecolor=darkgreen,fillstyle=solid,
  fillcolor=lightgreen,shadow=false}\pspicture(-.3,0)(.8,.4)\pscircle(.25,.2){.15}\endpspicture}}
\newcommand{\mystepv}{{\psset{unit=6mm,dotscale=1.5,linewidth=.06,linecolor=darkgreen,fillstyle=solid,
  fillcolor=lightgreen,shadow=false}\pspicture(0,0)(.5,.4)\pscircle(.25,.2){.15}\endpspicture}}
\newcommand{\mystepvv}{{\psset{unit=6mm,dotscale=1.5,linewidth=.06,linecolor=darkred,fillstyle=solid,
  fillcolor=red,shadow=false}\pspicture(0,0)(.5,.4)\pscircle(.25,.2){.15}\endpspicture}}
\newcommand{\mystepvvv}{{\psset{unit=6mm,dotscale=1.5,linewidth=.06,linecolor=darkred,fillstyle=solid,
  fillcolor=red,shadow=false}\pspicture(0,0)(.5,.4)\endpspicture}}

\newcommand{\myi}{{\psset{unit=6mm,dotscale=1.5,linewidth=.05,linecolor=darkgreen,fillstyle=none,shadow=false}%
  \pspicture(-.3,0)(.8,.4)\pscircle(.25,.2){.25}\rput(.25,.2){\tiny\dg1}\endpspicture}}
\newcommand{\myii}{{\psset{unit=6mm,dotscale=1.5,linewidth=.05,linecolor=darkgreen,fillstyle=none,shadow=false}%
  \pspicture(-.3,0)(.8,.4)\pscircle(.25,.2){.25}\rput(.25,.2){\tiny\dg2}\endpspicture}}
\newcommand{\myiii}{{\psset{unit=6mm,dotscale=1.5,linewidth=.05,linecolor=darkgreen,fillstyle=none,shadow=false}%
  \pspicture(-.3,0)(.8,.4)\pscircle(.25,.2){.25}\rput(.25,.2){\tiny\dg3}\endpspicture}}
\newcommand{\myiv}{{\psset{unit=6mm,dotscale=1.5,linewidth=.05,linecolor=darkgreen,fillstyle=none,shadow=false}%
  \pspicture(-.3,0)(.8,.4)\pscircle(.25,.2){.25}\rput(.25,.2){\tiny\dg4}\endpspicture}}
\newcommand{\myv}{{\psset{unit=6mm,dotscale=1.5,linewidth=.05,linecolor=darkgreen,fillstyle=none,shadow=false}%
  \pspicture(-.3,0)(.8,.4)\pscircle(.25,.2){.25}\rput(.25,.2){\tiny\dg5}\endpspicture}}
\newcommand{\myvi}{{\psset{unit=6mm,dotscale=1.5,linewidth=.05,linecolor=darkgreen,fillstyle=none,shadow=false}%
  \pspicture(-.3,0)(.8,.4)\pscircle(.25,.2){.25}\rput(.25,.2){\tiny\dg6}\endpspicture}}
\newcommand{\myvii}{{\psset{unit=6mm,dotscale=1.5,linewidth=.05,linecolor=darkgreen,fillstyle=none,shadow=false}%
  \pspicture(-.3,0)(.8,.4)\pscircle(.25,.2){.25}\rput(.25,.2){\tiny\dg7}\endpspicture}}
\newcommand{\myviii}{{\psset{unit=6mm,dotscale=1.5,linewidth=.05,linecolor=darkgreen,fillstyle=none,shadow=false}%
  \pspicture(-.3,0)(.8,.4)\pscircle(.25,.2){.25}\rput(.25,.2){\tiny\dg8}\endpspicture}}
\newcommand{\myix}{{\psset{unit=6mm,dotscale=1.5,linewidth=.05,linecolor=darkgreen,fillstyle=none,shadow=false}%
  \pspicture(-.3,0)(.8,.4)\pscircle(.25,.2){.25}\rput(.25,.2){\tiny\dg9}\endpspicture}}

\newcommand{\mywo}{{\psset{unit=6mm,dotscale=1.5,linewidth=.05,linecolor=darkgreen,fillcolor=white,fillstyle=solid,shadow=false}%
  \pspicture(-.3,0)(.8,.4)\pscircle(.25,.2){.25}\rput(.25,.2){\tiny\dg0}\endpspicture}}
\newcommand{\mywi}{{\psset{unit=6mm,dotscale=1.5,linewidth=.05,linecolor=darkgreen,fillcolor=white,fillstyle=solid,shadow=false}%
  \pspicture(-.3,0)(.8,.4)\pscircle(.25,.2){.25}\rput(.25,.2){\tiny\dg1}\endpspicture}}
\newcommand{\mywii}{{\psset{unit=6mm,dotscale=1.5,linewidth=.05,linecolor=darkgreen,fillcolor=white,fillstyle=solid,shadow=false}%
  \pspicture(-.3,0)(.8,.4)\pscircle(.25,.2){.25}\rput(.25,.2){\tiny\dg2}\endpspicture}}

\newcommand{\myO}{{\psset{unit=6mm,dotscale=1.5,linewidth=.06,linecolor=darkgreen,fillstyle=none,
  shadow=false}\pspicture(0,0)(0,0)\pscircle(-.2,.225){.4}\endpspicture}}

\newcommand{\myspace}{\psset{unit=6mm,dotscale=1.5,linewidth=.06}\pspicture(-.3,0)(.8,.4)\endpspicture}

\newcommand{\mydot}{\pscircle(0,0){.1}}
\newcommand{\mydotv}{\pscircle*[linecolor=gray](0,0){.1}\pscircle*[linecolor=blue](0,0){.06}}

\newcommand{\myhexagon}[2]{\psset{fillcolor=halfgray,linecolor=gray,shadow=false,fillstyle=none,linewidth=.06}
  \pspicture(-3.5,-2.6)(3,2.6)
    \pnode(-3  , 0  ){a}
    \pnode(-1.5, 2.6){b}
    \pnode( 1.5, 2.6){c}
    \pnode( 3  , 0  ){d}
    \pnode( 1.5,-2.6){e}
    \pnode(-1.5,-2.6){f}
    \pspolygon(a)(b)(c)(d)(e)(f)#1\psset{fillstyle=solid}
    \pscircle(a){.2}
    \pscircle(b){.2}
    \pscircle(c){.2}
    \pscircle(d){.2}
    \pscircle(e){.2}
    \pscircle(f){.2}#2
  \endpspicture}

\newcommand{\myksix}[2]{\myhexagon{\pspolygon(a)(c)(e)\pspolygon(b)(d)(f)\psline(a)(d)\psline(b)(e)\psline(c)(f)#1}{#2}}

\newcommand{\lss}{$\spadesuit$}
\newcommand{\lhs}{$\darkred\heartsuit$}
\newcommand{\lds}{$\darkred\diamondsuit$}
\newcommand{\lcs}{$\clubsuit$}

\newcommand{\hatmanv}{{%
 \psset{dotscale=1.5,dotsize=0.09,linewidth=.03,fillcolor=black,linecolor=black,shadow=false}
 \psline(0,0)(0.25,0)\psline(1.75,0)(2,0)\psline(0.25,0)(1,1.2)
 \psline(1.75,0)(1,1.2)\psline(1,1.2)(1,2.5)\psline(0.25,1.75)(1,2)
 \psline(1.75,1.75)(1,2)\pscircle(1,3){0.5}\psarc(1,3){.25}{-140}{-40}
 \psdots(.85,3.15)(1.15,3.15)\psset{fillstyle=solid}\psframe(.5,3.35)(1.5,3.45)
 \psframe(.7,3.35)(1.3,4.1)}}

\newcommand{\manv}{{%
 \psset{dotscale=1.5,dotsize=0.09,linewidth=.03,fillcolor=black,linecolor=black,shadow=false}%
 \pspicture(0,0)(2,3.5)%\psline(0,0)(0.25,0)\psline(1.75,0)(2,0)
 \psline(0.25,0)(1,1.2)
 \psline(1.75,0)(1,1.2)\psline(1,1.2)(1,2.5)\psline(0.25,1.75)(1,2)
 \psline(1.75,1.75)(1,2)\pscircle(1,3){0.5}\psarc(1,3){.25}{-140}{-40}
 \psdots(.85,3.15)(1.15,3.15)%\psset{fillstyle=solid}\psframe(.5,3.35)(1.5,3.45)\psframe(.7,3.35)(1.3,4.1)
 \endpspicture}}

\newcommand{\sadmanv}{{
 \psset{dotscale=1.5,dotsize=0.09,linewidth=.03,fillcolor=black,linecolor=black,shadow=false}
 \psline(0,0)(0.25,0)\psline(1.75,0)(2,0)\psline(0.25,0)(1,1.2)
 \psline(1.75,0)(1,1.2)\psline(1,1.2)(1,2.5)\psline(0.25,1.75)(1,2)
 \psline(1.75,1.75)(1,2)\pscircle(1,3){0.5}\psarc(1,2.65){.25}{40}{140}
 \psdots(.87,3.15)(1.13,3.15)\psset{fillstyle=solid}\psframe(.5,3.35)(1.5,3.45)
 \psframe(.7,3.35)(1.3,4.1)}}

\newcommand{\womanv}{{
 \psset{dotscale=1.5,dotsize=0.09,linewidth=.03,fillcolor=black,linecolor=black,shadow=false}
 \pspicture(0,0)(2,3.5)%
 \psline(0.5,0)(0.75,0)\psline(1.5,0)(1.25,0)\psline(0.75,0)(0.75,0.5)
 \psline(1.25,0)(1.25,0.5)\psline(1,2)(1,2.35)\psline(0.25,1.75)(1,2)
 \psline(1.75,1.75)(1,2)\pscircle(1,2.85){0.5}
 \psarc(1,2.85){.25}{-140}{-40}\psdots(.85,2.95)(1.15,2.95)
 \psset{linecolor=brown,linewidth=.1}
 \psarc(1,2.85){.5}{0}{180}\psarc(1.75,2.85){.25}{180}{270}
 \psarc(0.25,2.85){.25}{-90}{0}
 \pstriangle[linecolor=red,fillstyle=solid,fillcolor=red](1,.5)(1.2,1.75)\endpspicture}}

\newcommand{\spyv}{{%
 \psset{dotscale=1.5,dotsize=0.09,linewidth=.03,fillcolor=black,linecolor=black,shadow=false}
 \pspicture(0,0)(2,3.5)%\psline(0,0)(0.25,0)\psline(1.75,0)(2,0)
 \psline(0.25,0)(1,1.2)
 \psline(1.75,0)(1,1.2)\psline(1,1.2)(1,2.5)\psline(0.25,1.75)(1,2)
 \psline(1.75,1.75)(1,2)\pscircle(1,3){0.5}\psarc(1,3){.25}{-140}{-40}
 %\psdots(.85,3.15)(1.15,3.15)
 \psset{fillstyle=solid}
 \psframe( .725,3.075)( .975,3.225)
 \psframe(1.025,3.075)(1.275,3.225)
 \psellipse*(1,3.5)(.5,.14)
 \pspolygon(.6,3.5)(1.4,3.5)( .8,3.8)
 \pspolygon(.6,3.5)(1.4,3.5)(1.2,3.8)
 \endpspicture}}

\newcommand{\puzzledspyv}{{%
 \psset{dotscale=1.5,dotsize=0.09,linewidth=.03,fillcolor=black,linecolor=black,shadow=false}
 \pspicture(0,0)(2,3.5)%\psline(0,0)(0.25,0)\psline(1.75,0)(2,0)
 \psline(0.25,0)(1,1.2)
 \psline(1.75,0)(1,1.2)\psline(1,1.2)(1,2.5)\psline(0.25,1.75)(1,2)
 \psline(1.75,1.75)(1,2)\pscircle(1,3){0.5}
 % Mouth
 \psline(.83,2.8)(1.17,2.8)
 %\psdots(.85,3.15)(1.15,3.15)
 \psset{fillstyle=solid}
 \psframe( .725,3.075)( .975,3.225)
 \psframe(1.025,3.075)(1.275,3.225)
 \psellipse*(1,3.5)(.5,.14)
 \pspolygon(.6,3.5)(1.4,3.5)( .8,3.8)
 \pspolygon(.6,3.5)(1.4,3.5)(1.2,3.8)
 \endpspicture}}

\newcommand{\man}{\psset{unit=4mm}\manv}
\newcommand{\hatman}{\psset{unit=4mm}\hatmanv}
\newcommand{\sadman}{\psset{unit=4mm}\sadmanv}
\newcommand{\woman}{\psset{unit=4mm}\womanv}
\newcommand{\spy}{\psset{unit=4mm}\spyv}
\newcommand{\puzzledspy}{\psset{unit=4mm}\puzzledspyv}


\newcommand{\coursetitle}{\vphantom{ }\vspace{7mm}\begin{center}
    {\large\darkgreen MATH5505 \:\: Combinatorics}\\[2.5mm]
    UNSW 2018S1
    \end{center}}

\newcounter{gra}

\DeclareMathAlphabet{\mathscr}{OT1}{pzc}%
                                 {m}{it}

\newcommand{\ph}{\phantom}
\newcommand{\ds}{\displaystyle}
\newcommand{\mra}{\black\rightsquigarrow\blue}
\newcommand{\tvs}{\textvisiblespace}
\newcommand{\msp}{\,}
\newcommand{\mba}{\,\mypicture{\psset{shadow=false}\psline(-.04,-.15)(-.04,.4)}}
\newcommand{\mbq}{\,\mypicture{\psset{shadow=false}\psline(-.04,-.15)(-.04,.4)\rput(-0.04,0.7){{\darkgray?}}}}
\newcommand{\meq}{\black=\blue}
\newcommand{\msim}{\black\sim\blue}
\newcommand{\mequiv}{\black\equiv\blue}
\newcommand{\mapprox}{\black\approx\blue}
\newcommand{\mneq}{\red\neq\blue}
\newcommand{\mleq}{\black\leq\blue}
\newcommand{\mgeq}{\black\geq\blue}
\newcommand{\mgt}{\black>\blue}
\newcommand{\mlt}{\black<\blue}
\newcommand{\mto}{\black\to\blue}
\newcommand{\msubseteq}{\black\subseteq\blue}
\newcommand{\mnsubseteq}{\red\not\subseteq\blue}
\newcommand{\mequ}{\black\equiv\blue}
\newcommand{\mnequ}{\red\not\equiv\blue}
\newcommand{\myin}{\black\in\blue}
\newcommand{\mypl}{\black+\blue}
\newcommand{\mdef}{\black:=\blue}
\newcommand{\mnotin}{\red\notin\blue}
\newcommand{\mynotin}{\red\notin\blue}
\newcommand{\mywhere}{\quad\text{\black where\blue}\quad}
\newcommand{\myand}{\quad\text{\black and\blue}\quad}
\newcommand{\dnd}{\black\mathchoice{\mathrel{{\kern0.1em|\kern-0.4em/}}}
  {\mathrel{{\kern0.1em|\kern-0.4em/}}}{\mathrel{{\kern0.1em|\kern-0.33em/}}}
  {\mathrel{{\kern0.1em|\kern-0.2em/}}}\blue}
\newcommand{\mdiv}{\black\,|\,\blue}
\newcommand{\mymod}[1]{\:(\textrm{mod}\,#1)}
\newcommand{\mask}[1]{}
\newcommand{\ord}{\textrm{ord}\,}
\newcommand{\ns}{\negthickspace\negthickspace}
\newcommand{\nns}{\negthickspace\negthickspace\negthickspace\negthickspace}
\newcommand{\lns}{\hspace*{-.3mm}}
\newcommand{\ri}{\,i\,}% \,\mathrm{i}}
\newcommand{\di}{\mbox{$\dg\bullet$}}
\newcommand{\dah}{\mbox{$\dg\mathbf{-}$}}
\newcommand{\tp}{\mbox{{\dg\tt p}}}
\newcommand{\Var}{\mathrm{Var}}
\newcommand{\Arg}{\mathrm{Arg}}
\renewcommand{\Re}{\mathrm{Re}}
\renewcommand{\Im}{\mathrm{Im}}
\newcommand{\bbN}{\blue\mathbb{N}}
\newcommand{\bbZ}{\blue\mathbb{Z}}
\newcommand{\bbQ}{\blue\mathbb{Q}}
\newcommand{\bbR}{\blue\mathbb{R}}
\newcommand{\bbC}{\blue\mathbb{C}}
\newcommand{\bbF}{\blue\mathbb{F}}
\newcommand{\bbP}{\blue\mathbb{P}}
%\newcommand{\calR}{\blue\mathcal{R}}
%\newcommand{\calC}{\blue\mathcal{C}}
\renewcommand{\vec}[1]{\mathbf{\blue #1}}
%\newcommand{\pv}[1]{{\blue\begin{pmatrix}#1\end{pmatrix}}}
%\newcommand{\pvn}[1]{{\begin{pmatrix}#1\end{pmatrix}}}
%\newcommand{\spv}[1]{{\blue\left(\begin{smallmatrix}#1\end{smallmatrix}\right)}}
%\newcommand{\mpv}[1]{{\blue\Biggl(\!\!\begin{array}{r}#1\end{array}\!\!\Biggr)}}
%\newcommand{\augmv}[1]{{\left(\begin{array}{rr|r}#1\end{array}\right)}}
%\newcommand{\taugmv}[1]{{\left(\begin{array}{rrr|r}#1\end{array}\right)}}
%\newcommand{\qaugmv}[1]{{\left(\begin{array}{rrrr|r}#1\end{array}\right)}}
%\newcommand{\augm}[1]{{\blue\left(\begin{array}{rr|r}#1\end{array}\right)}}
%\newcommand{\daugm}[1]{{\blue\left(\begin{array}{rr|rr}#1\end{array}\right)}}
%\newcommand{\taugm}[1]{{\blue\left(\begin{array}{rrr|r}#1\end{array}\right)}}
%\newcommand{\traugm}[1]{{\blue\left(\begin{array}{rrr|rrr}#1\end{array}\right)}}
%\newcommand{\qaugm}[1]{{\blue\left(\begin{array}{rrrr|r}#1\end{array}\right)}}
%\newcommand{\qtaugm}[1]{{\blue\left(\begin{array}{rrrr|rrr}#1\end{array}\right)}}
%\newcommand{\mydet}[1]{{\blue\left|\begin{matrix}#1\end{matrix}\right|}}
%\newcommand{\mydets}[1]{{\blue\Bigl|\begin{matrix}#1\end{matrix}\Bigr|}}
%\newcommand{\arr}[1]{\overrightarrow{#1}}
\newcommand{\lcm}{\mathrm{lcm}}
\renewcommand{\mod}{\;\mathrm{mod}\;}
%\newcommand{\proj}{\mathrm{proj}}
%\newcommand{\id}{\blue\mathrm{id}}
%\newcommand{\im}{\blue\mathrm{Im}}
%\newcommand{\re}{\blue\mathrm{Re}}
\newcommand{\col}{\blue\mathrm{col}}
\newcommand{\nullity}{\blue\mathrm{nullity}}
\newcommand{\rank}{\blue\mathrm{rank}}
\newcommand{\spn}[1]{\blue\mathrm{span}\left\{#1\right\}}
\newcommand{\spnv}{\blue\mathrm{span}\,}
%\newcommand{\diag}[1]{\blue\mathrm{diag}\left(#1\right)\,}
%\newcommand{\satop}[2]{\stackrel{\scriptstyle{#1}}{\scriptstyle{#2}}}
%\newcommand{\llnot}{\sim\!}
\newcommand{\qed}{\hfill$\Box$}
%\newcommand{\rbullet}{\includegraphics[width=4mm]{red-bullet-on-white.ps}}
%\newcommand{\gbullet}{\includegraphics[width=3mm]{green-bullet-on-white.ps}}
%\newcommand{\ybullet}{\includegraphics[width=3mm]{yellow-bullet-on-white.ps}}
\newcommand{\vsp}{{\psset{unit=4.2mm}\begin{pspicture}(0,1)(0,0)\end{pspicture}}}
\renewcommand{\emptyset}{\varnothing}
\newcommand{\mq}{\text{\red ?}}
%\renewcommand{\emph}[1]{{\blue\textsl{#1}}}

%\newcommand{\shat}{\begin{pspicture}(0,0)(0,0)\psset{linecolor=black,linewidth=.075}\psarc*(.67,3.18){.25}{45}{225}\psline(.35,2.85)(1,3.5)\end{pspicture}}
%\newcommand{\rhat}{\begin{pspicture}(0,0)(0,0)\psset{linecolor=red,linewidth=.05}\psarc*(.675,3.175){.25}{45}{225}\psline(.35,2.85)(1,3.5)\end{pspicture}}
%\newcommand{\bhat}{\begin{pspicture}(0,0)(0,0)\psset{linecolor=blue,linewidth=.05}\psarc*(.675,3.175){.25}{45}{225}\psline(.35,2.85)(1,3.5)\end{pspicture}}
%\newcommand{\sdress}{\begin{pspicture}(0,0)(0,0)\pstriangle*[linecolor=black,linewidth=.05](1,.5)(1.2,1.75)\end{pspicture}}
%\newcommand{\rdress}{\begin{pspicture}(0,0)(0,0)\pstriangle*[linecolor=red,linewidth=.05](1,.5)(1.2,1.75)\end{pspicture}}
%\newcommand{\bdress}{\begin{pspicture}(0,0)(0,0)\pstriangle*[linecolor=brown,linewidth=.05](1,.5)(1.2,1.75)\end{pspicture}}
%\newcommand{\ydress}{\begin{pspicture}(0,0)(0,0)\pstriangle*[linecolor=amber,linewidth=.05](1,.5)(1.2,1.75)\end{pspicture}}
%\newcommand{\gdress}{\begin{pspicture}(0,0)(0,0)\pstriangle*[linecolor=green,linewidth=.05](1,.5)(1.2,1.75)\end{pspicture}}
%\newcommand{\sshoes}{\begin{pspicture}(0,0)(0,0)\psset{linecolor=black,linewidth=.05}
% \psline(.6,0)(.85,0.125)\psline(0.85,0)(0.85,0.5)\psline(1.15,0)(1.15,0.5)\psline(1.4,0)(1.15,0.125)\end{pspicture}}
%\newcommand{\rshoes}{\begin{pspicture}(0,0)(0,0)\psset{linecolor=black,linewidth=.05}
%  \psline(0.85,0)(0.85,0.5)\pscircle*[linecolor=red](.65,.05){.05}\psframe*[linecolor=red](.65,0)(.90,.1)
%  \psline(1.15,0)(1.15,0.5)\psframe*[linecolor=red](1.1,0)(1.35,0.1)\pscircle*[linecolor=red](1.35,.05){.05}\end{pspicture}}
%
%\newcommand{\woman}[1]{\begin{pspicture}(0.25,0)(1.75,3.4)
% \psset{unit=4mm,dotsize=0.09,linewidth=.03,fillcolor=black,linecolor=black,shadow=false}
% \psline(1,2)(1,2.35)\psline(0.25,1.75)(1,2)\psline(1.75,1.75)(1,2)\pscircle(1,2.85){0.5}
% \psarc(1,2.85){.25}{-140}{-40}\psdots(.85,2.95)(1.15,2.95)
% \psset{linecolor=brown,linewidth=.1}
% \psarc(1,2.85){.5}{0}{180}\psarc(1.75,2.85){.25}{180}{270}\psarc(.25,2.85){.25}{-90}{0}
% \psset{linewidth=.06}#1
% \end{pspicture}}

\newcommand{\dicei}{{\psset{unit=6mm,shadow=false,linewidth=.05}\begin{pspicture}(0,0.3)(1,1.3)\begin{psframe}(0,0)(1,1)\qdisk(0.5,0.5){0.1\psunit}\end{psframe}\end{pspicture}}}
\newcommand{\diceii}{{\psset{unit=6mm,shadow=false,linewidth=.05}\begin{pspicture}(0,0.3)(1,1.3)\begin{psframe}(0,0)(1,1)\qdisk(0.2,0.2){0.1\psunit}\qdisk(0.8,0.8){0.1\psunit}\end{psframe}\end{pspicture}}}
\newcommand{\diceiii}{{\psset{unit=6mm,shadow=false,linewidth=.05}\begin{pspicture}(0,0.3)(1,1.3)\begin{psframe}(0,0)(1,1)\qdisk(0.2,0.2){0.1\psunit}\qdisk(0.5,0.5){0.1\psunit}\qdisk(0.8,0.8){0.1\psunit}\end{psframe}\end{pspicture}}}
\newcommand{\diceiv}{{\psset{unit=6mm,shadow=false,linewidth=.05}\begin{pspicture}(0,0.3)(1,1.3)\begin{psframe}(0,0)(1,1)\qdisk(0.2,0.2){0.1\psunit}\qdisk(0.2,0.8){0.1\psunit}\qdisk(0.8,0.8){0.1\psunit}\qdisk(0.8,0.2){0.1\psunit}\end{psframe}\end{pspicture}}}
\newcommand{\dicev}{{\psset{unit=6mm,shadow=false,linewidth=.05}\begin{pspicture}(0,0.3)(1,1.3)\begin{psframe}(0,0)(1,1)\qdisk(0.2,0.2){0.1\psunit}\qdisk(0.2,0.8){0.1\psunit}\qdisk(0.5,0.5){0.1\psunit}\qdisk(0.8,0.8){0.1\psunit}\qdisk(0.8,0.2){0.1\psunit}\end{psframe}\end{pspicture}}}
\newcommand{\dicevi}{{\psset{unit=6mm,shadow=false,linewidth=.05}\begin{pspicture}(0,0.3)(1,1.3)\begin{psframe}(0,0)(1,1)\qdisk(0.2,0.2){0.1\psunit}\qdisk(0.2,0.5){0.1\psunit}\qdisk(0.2,0.8){0.1\psunit}\qdisk(0.8,0.8){0.1\psunit}\qdisk(0.8,0.5){0.1\psunit}\qdisk(0.8,0.2){0.1\psunit}\end{psframe}\end{pspicture}}}
\newcommand{\diceframe}{{\psset{unit=6mm,shadow=false,linewidth=.05}\begin{pspicture}(0,0.3)(1,1.3)\begin{psframe}(-.25,-.25)(1.25,1.25)\end{psframe}\end{pspicture}}}

\newcommand{\pigeon}{{\psset{xunit=6mm,yunit=6mm,runit=6mm,linewidth=.8pt,shadow=false,linecolor=darkgray,fillcolor=white,fillstyle=solid}
  \begin{pspicture}(0,0)(1.9, 1){\pscustom{\newpath
    \moveto(1.179, 0.088)
    \curveto(1.156, 0.239)(1.402, 0.255)(1.451, 0.464)
    \curveto(1.498, 0.668)(1.42 , 0.774)(1.75 , 0.812)
    \curveto(1.656, 0.887)(1.528, 0.997)(1.399, 0.979)
    \curveto(1.067, 0.929)(1.269, 0.646)(0.156, 0.241)
    \curveto(0    , 0.185)(0.869, 0.396)(1.037, 0.205)
    \curveto(1.201, 0.016)(0.902, 0    )(1.17 , 0.003)
    \curveto(1.37 , 0.005)(1.197, 0.003)(1.179, 0.088)
    \closepath}}
    \pscircle(1.447, 0.9){0.045}
  \end{pspicture}}}


%\newcommand{\dicegrid}[5]{\[\begin{pspicture}(-1,-1.2)(20,8)\psset{shadow=false,linecolor=darkgray}
%    \rput(0  ,0  ){\dicevi} \rput(0  ,1.2){\dicev} \rput(0  ,2.4){\diceiv}\rput(0  ,3.6){\diceiii}
%    \rput(0  ,4.8){\diceii} \rput(0  ,6  ){\dicei} \rput(1.5,7.55){\dicei} \rput(2.7,7.55){\diceii}
%    \rput(3.9,7.55){\diceiii}\rput(5.1,7.55){\diceiv}\rput(6.3,7.55){\dicev} \rput(7.5,7.55){\dicevi}
%    \psset{linecolor=gray}\psline( .75,-1)( .75,8)\psline(-.75, 6.5)(8.25,6.5)\psline(-.75,-1)(-.75,8)
%    \psline(-.75,8)(8.25,8)\psline(-.75,-1)(8.25,-1)\psline(8.25,-1)(8.25,8)\psline(-.75, 8)(.75,6.5)
%    \put(-.55,6.6){1}\put(.25,7.25){2}
%    \psset{linecolor=blue}#1\psset{linecolor=red}#2\put(9.5,4.3){{#3}}\put(9.5,3.1){{#4}}\put(9.5,1.9){{#5}}
%  \end{pspicture}\]}
%
%\newcommand{\bnaa}{\pscircle*(1.5,5.7){3pt}}\newcommand{\bnab}{\pscircle*(2.7,5.7){3pt}}\newcommand{\bnac}{\pscircle*(3.9,5.7){3pt}}\newcommand{\bnad}{\pscircle*(5.1,5.7){3pt}}\newcommand{\bnae}{\pscircle*(6.3,5.7){3pt}}\newcommand{\bnaf}{\pscircle*(7.5,5.7){3pt}}
%\newcommand{\bnba}{\pscircle*(1.5,4.5){3pt}}\newcommand{\bnbb}{\pscircle*(2.7,4.5){3pt}}\newcommand{\bnbc}{\pscircle*(3.9,4.5){3pt}}\newcommand{\bnbd}{\pscircle*(5.1,4.5){3pt}}\newcommand{\bnbe}{\pscircle*(6.3,4.5){3pt}}\newcommand{\bnbf}{\pscircle*(7.5,4.5){3pt}}
%\newcommand{\bnca}{\pscircle*(1.5,3.3){3pt}}\newcommand{\bncb}{\pscircle*(2.7,3.3){3pt}}\newcommand{\bncc}{\pscircle*(3.9,3.3){3pt}}\newcommand{\bncd}{\pscircle*(5.1,3.3){3pt}}\newcommand{\bnce}{\pscircle*(6.3,3.3){3pt}}\newcommand{\bncf}{\pscircle*(7.5,3.3){3pt}}
%\newcommand{\bnda}{\pscircle*(1.5,2.1){3pt}}\newcommand{\bndb}{\pscircle*(2.7,2.1){3pt}}\newcommand{\bndc}{\pscircle*(3.9,2.1){3pt}}\newcommand{\bndd}{\pscircle*(5.1,2.1){3pt}}\newcommand{\bnde}{\pscircle*(6.3,2.1){3pt}}\newcommand{\bndf}{\pscircle*(7.5,2.1){3pt}}
%\newcommand{\bnea}{\pscircle*(1.5, .9){3pt}}\newcommand{\bneb}{\pscircle*(2.7, .9){3pt}}\newcommand{\bnec}{\pscircle*(3.9, .9){3pt}}\newcommand{\bned}{\pscircle*(5.1, .9){3pt}}\newcommand{\bnee}{\pscircle*(6.3, .9){3pt}}\newcommand{\bnef}{\pscircle*(7.5, .9){3pt}}
%\newcommand{\bnfa}{\pscircle*(1.5,-.3){3pt}}\newcommand{\bnfb}{\pscircle*(2.7,-.3){3pt}}\newcommand{\bnfc}{\pscircle*(3.9,-.3){3pt}}\newcommand{\bnfd}{\pscircle*(5.1,-.3){3pt}}\newcommand{\bnfe}{\pscircle*(6.3,-.3){3pt}}\newcommand{\bnff}{\pscircle*(7.5,-.3){3pt}}
%
%\newcommand{\cnaa}{\pscircle(1.5,5.7){4.5pt}}\newcommand{\cnab}{\pscircle(2.7,5.7){4.5pt}}\newcommand{\cnac}{\pscircle(3.9,5.7){4.5pt}}\newcommand{\cnad}{\pscircle(5.1,5.7){4.5pt}}\newcommand{\cnae}{\pscircle(6.3,5.7){4.5pt}}\newcommand{\cnaf}{\pscircle(7.5,5.7){4.5pt}}
%\newcommand{\cnba}{\pscircle(1.5,4.5){4.5pt}}\newcommand{\cnbb}{\pscircle(2.7,4.5){4.5pt}}\newcommand{\cnbc}{\pscircle(3.9,4.5){4.5pt}}\newcommand{\cnbd}{\pscircle(5.1,4.5){4.5pt}}\newcommand{\cnbe}{\pscircle(6.3,4.5){4.5pt}}\newcommand{\cnbf}{\pscircle(7.5,4.5){4.5pt}}
%\newcommand{\cnca}{\pscircle(1.5,3.3){4.5pt}}\newcommand{\cncb}{\pscircle(2.7,3.3){4.5pt}}\newcommand{\cncc}{\pscircle(3.9,3.3){4.5pt}}\newcommand{\cncd}{\pscircle(5.1,3.3){4.5pt}}\newcommand{\cnce}{\pscircle(6.3,3.3){4.5pt}}\newcommand{\cncf}{\pscircle(7.5,3.3){4.5pt}}
%\newcommand{\cnda}{\pscircle(1.5,2.1){4.5pt}}\newcommand{\cndb}{\pscircle(2.7,2.1){4.5pt}}\newcommand{\cndc}{\pscircle(3.9,2.1){4.5pt}}\newcommand{\cndd}{\pscircle(5.1,2.1){4.5pt}}\newcommand{\cnde}{\pscircle(6.3,2.1){4.5pt}}\newcommand{\cndf}{\pscircle(7.5,2.1){4.5pt}}
%\newcommand{\cnea}{\pscircle(1.5, .9){4.5pt}}\newcommand{\cneb}{\pscircle(2.7, .9){4.5pt}}\newcommand{\cnec}{\pscircle(3.9, .9){4.5pt}}\newcommand{\cned}{\pscircle(5.1, .9){4.5pt}}\newcommand{\cnee}{\pscircle(6.3, .9){4.5pt}}\newcommand{\cnef}{\pscircle(7.5, .9){4.5pt}}
%\newcommand{\cnfa}{\pscircle(1.5,-.3){4.5pt}}\newcommand{\cnfb}{\pscircle(2.7,-.3){4.5pt}}\newcommand{\cnfc}{\pscircle(3.9,-.3){4.5pt}}\newcommand{\cnfd}{\pscircle(5.1,-.3){4.5pt}}\newcommand{\cnfe}{\pscircle(6.3,-.3){4.5pt}}\newcommand{\cnff}{\pscircle(7.5,-.3){4.5pt}}
%
\newcommand{\clearemptydoublepage}
  {\newpage{\pagestyle{empty}{\cleardoublepage}}}


% ------------------------------------------------------------------------
%  Environments
% ------------------------------------------------------------------------

\newcommand{\example}{{\sffamily\darkgreen Example }}
\newcommand{\exercise}{{\sffamily\darkgreen Exercise }}
\newcommand{\proof}{{\sffamily\darkgreen Proof }}
\newcommand{\notes}{{\sffamily\darkgreen Notes }}
\newcommand{\note}{{\sffamily\darkgreen Note }}
\newcommand{\theorem}{{\sffamily\darkgreen Theorem }}
\newcommand{\proposition}{{\sffamily\darkgreen Proposition }}
\newcommand{\corollary}{{\sffamily\darkgreen Corollary }}
\newcommand{\lemma}{{\sffamily\darkgreen Lemma }}
\newcommand{\definition}{{\sffamily\darkgreen Definition}}
\newcommand{\problem}{{\sffamily\darkgreen Problem}}
\newcommand{\remark}{{\sffamily\darkgreen Remark}}

% ------------------------------------------------------------------------

\renewcommand{\section}
  {\clearemptydoublepage\refstepcounter{section} \secdef \cmda \cmdb}
\newcommand{\cmda}[2][]
  {{\scriptsize{\textbf{MATH5505 \:\: Combinatorics}}
   \hfill{\scriptsize{\textsl{Thomas Britz}}}\vspace{0.1cm}\\
   {\bfseries\Large\S\arabic{section} \sffamily #2}}}
\newcommand{\cmdb}[1]{{\bfseries\huge\S\,\sffamily #1}}

\pagestyle{empty}

\newcommand{\frontpage}{}

\usepackage{hyperref}

\begin{document}
%% Very important: sans-serif font throughout
\sf

\newcommand{\lecturetitle}{\vphantom{ }\vspace{15mm}\begin{center}
  {\large\sc Ramsey Theory}\end{center}}

\newcommand{\lecturetitlei}{\vphantom{ }\vspace{15mm}\begin{center}
  {\large\sc Ramsey Theory}\\[3mm]
  \darkgreen Lecture 6: Applications\end{center}}

\newcommand{\pigeonthma}{{\dg The Pigeonhole Principle}
  \\[1mm]If $\bl k+1$ pigeons are put into $\bl k$ pigeonholes,\\
         then some pigeonhole contains at least two pigeons.}

\newcommand{\pigeonthmplusa}{{\dg The Pigeonhole Principle} {\gray (general)}
  \\[1mm]If $\bl km+1$ pigeons are put into $\bl k$ pigeonholes,\\
         then some pigeonhole contains at least $\bl m+1$ pigeons.}

\newcommand{\pigeonthmstrong}{{\dg The Pigeonhole Principle} {\gray (strong)}
  \\[1mm]If $\bl (n_1-1) +\cdots+(n_k-1) + 1$ pigeons are put into $\bl k$ pigeonholes,\\
         then some $\bl i$th pigeonhole contains at least $\bl n_i$ pigeons.}

\newcommand{\pigeonthmexaa}{\\[8mm]\example}
\newcommand{\pigeonthmexab}[1]{\vspace*{-1mm}\begin{center}
  \begin{pspicture}(0,0)(8.125,2.5)\psset{linecolor=darkgreen,shadow=false}
    \multiput(0,0)(3.5,0){3}{\psframe(-1.25,-1.25)( 1.25, 1.25)}\psset{linecolor=darkgray}#1
  \end{pspicture}
  \end{center}}
\newcommand{\pigeonthmexaba}{\rput(0  ,0){\pigeon}}
\newcommand{\pigeonthmexabb}{\rput(3.5,0){\pigeon}}
\newcommand{\pigeonthmexabc}{\rput(7  ,0){\pigeon}}
\newcommand{\pigeonthmexabd}{\rput(3.4,-4){$\begin{array}{rcl}\bl k &\meq&\bl 3\\\bl m &\meq&\bl 2\end{array}$}}
\newcommand{\pigeonthmexabaii}{\rput(-0.25,0){\pigeon}\rput( .25,0){\pigeon}}
\newcommand{\pigeonthmexabbii}{\rput( 3.25,0){\pigeon}\rput(3.75,0){\pigeon}}
\newcommand{\pigeonthmexabcii}{\rput( 6.75,0){\pigeon}\rput(7.25,0){\pigeon}}
\newcommand{\pigeonthmexabciii}{\rput(6.7 ,0){\pigeon}\rput(7   ,0){\pigeon}\rput(7.3,0){\pigeon}}

\newcommand{\pigeonthmexma}{\\[8mm]\example}
\newcommand{\pigeonthmexmb}[1]{\vspace*{-1mm}\begin{center}
  \begin{pspicture}(0,0)(8.125,2.5)\psset{linecolor=darkgreen,shadow=false}
    \multiput(0,0)(3.5,0){3}{\psframe(-1.25,-1.25)( 1.25, 1.25)}
    \rput[t](0  ,-2.5){$\bl n_1\meq 1$}
    \rput[t](3.5,-2.5){$\bl n_2\meq 3$}
    \rput[t](7  ,-2.5){$\bl n_3\meq 2$}
    \psset{linecolor=darkgray}#1
  \end{pspicture}
  \end{center}}

\newcommand{\dirichleta}{\\[3mm]{\dg Dirichlet's Approximation Theorem} {\gray (1889)}}
\newcommand{\dirichletb}{\\[-1mm]If $\bl\alpha\myin\mathbb{R}$ and $\bl N\myin\mathbb{N}$,
  then $\bl p,q\myin\mathbb{N}$ exist so that $\bl q\!\leq\! N$ and
  $\bl\ds \Bigl|\alpha - \frac{p}{q}\Bigr| \mlt \frac{1}{Nq}$\,.\hspace*{-6mm}}
\newcommand{\dirichletc}{\\[2mm]{\gray Each real number can thus be approximated arbitrarily well by rationals.}}

\newcommand{\dirichletpfa}{\\[2mm]\proof}
\newcommand{\dirichletpfb}{\\Replace the latter inequality equivalently by
  $\bl \bigl|q\alpha - p\bigr| \mlt \frac{1}{N}$.}
\newcommand{\dirichletpfc}{\\Partition the unit interval $\bl[0,1)$ into $\bl N$ equal subintervals.
  \begin{pspicture}(-1.5,.5)(8,.5)\psset{shadow=false}\psline(0,0)(4,0)\multirput(0,0)(.8,0){6}{\psline(0,-.2)(0,.2)}
    \rput[t](0,-.5){{\gray0}}
    \rput[t](4,-.5){{\gray1}}
    \rput[u](3.6,1.4){$\bl\frac{1}{N}$}
    \psline[linecolor=halfgray](3.2,.4)(3.2,.6)(4,.6)(4,.4)\psline[linecolor=halfgray](3.6,.6)(3.6,.8)\end{pspicture}}
\newcommand{\dirichletpfd}{\\Define $\bl N+1$ residues
  $\bl r_i\mdef i\alpha - \lfloor i\alpha\rfloor$ for $\bl i \meq 0,1,\ldots,N$.}
\newcommand{\dirichletpfe}{\\These lie in the $\bl N$ subintervals.}
\newcommand{\dirichletpff}{\\By the {\dg Pigeonhole Principle},
  residues $\bl r_i$, $\bl r_j$ ($\bl i\!\mlt\!j$) lie in some subinterval.}
%\newcommand{\dirichletpfg}{so $\bl |r_i-r_j|\mlt \frac{1}{N}$. }
\newcommand{\dirichletpfg}{\\Set $\bl p\mdef \lfloor j\alpha\rfloor - \lfloor i\alpha\rfloor$
                           and $\bl q\mdef j-i$. }
\newcommand{\dirichletpfh}{\\Then $\bl 0\mlt q\mleq N$ }
\newcommand{\dirichletpfj}[3]{and\vspace*{-2mm}
  \begin{align*}\bl
    |p-q\alpha|&\meq |\lfloor j\alpha\rfloor - \lfloor i\alpha\rfloor - (j-i)\alpha|\\[-2mm]
    &\;{#1{\meq |i\alpha - \lfloor i\alpha\rfloor - (j\alpha - \lfloor j\alpha\rfloor)|}}
    \;{#2{\meq |r_i-r_j|}}
    \;{#3{\mlt \frac{1}{N}\,\black.\qquad\Box}}\\[-12mm]\end{align*}}

\newcommand{\ramseyb}{{\dg Ramsey's Theorem} {\gray (1930)} {\halfgray (simple)}\\
   If $\bl k,m\myin\mathbb{N}$ and $\bl n$ is sufficiently large,
   then each $\bl k$-colouring of the edges of $\bl K_n$ gives a complete monochromatic subgraph $\bl K_m$.}

\newcommand{\ramseye}{{\dg Ramsey's Theorem} {\gray (1930)}\\
   If $\bl n_1,\ldots,n_k,r\myin\mathbb{N}$ and $\bl n$ is sufficiently large,\\
   then each colouring of $\bl \binom{[n]}{r}$ with colours $\bl c_1,\ldots,c_k$\\
   gives a $\bl c_i$-coloured subfamily $\bl \binom{S}{r}$
   for some $\bl i$ and $\bl n_i$-subset $\bl S\subseteq [n]$.}

\newcommand{\ramseyea}{\\[2mm]{\gray The least such $\lightblue n$ is denoted $\lightblue R_r(n_1,\ldots,n_k)$.}}

\newcommand{\convexa}{{\dg The Happy Ending Problem} {\gray (Erd\H{o}s \&\ Szekeres 1935)}
  \\If $\bl k\myin\mathbb{N}$ and $\bl n$ is sufficiently large,
  then among any $\bl n$ points in the plane, no three collinear, there are $\bl k$ that form a convex polygon.}
\newcommand{\convexb}{\\[2mm]{\gray This was first pointed out by Esther Klein
  (for $\lightblue k\gray\meq\lightblue 4$) who discussed it with Paul Erd\H{o}s and George Szekeres.
  George found a general proof and he and Erd\H{o}s published their famous 1935 paper
  containing this result and the Erd\H{o}s-Szekeres Theorem.
  The marked the beginning of Ramsey Theory for the mathematical public,
  and it marked the beginning of Esther's and George's love and life with each other,
  hence the theorem's name by Erd\H{o}s.
  The couple remained together until 70 years later,
  when in 2005 they died within an hour of each other.
  They were brilliant and humble founders of combinatorics,
  and our School here at UNSW was lucky to have George as professor, and Esther involved,
  for 41 years.}}

\newcommand{\convexexa}{\\[4mm]\example}
\newcommand{\convexexb}[1]{\begin{center}
  \begin{pspicture}(0,0)(10,6.5)\psset{shadow=false,shadow=false,fillstyle=solid,framearc=0}
  #1\psset{linecolor=darkgray}\qdisk(0,1){.1}\qdisk(1,2){.1}\qdisk(1,5){.1}\qdisk(2,4){.1}
  \qdisk(4,0){.1}\qdisk(3,1){.1}\qdisk(5,4){.1}\qdisk(4,3){.1}
  \end{pspicture}\end{center}}
\newcommand{\convexexba}{\psset{linecolor=green,fillcolor=white}
  \psline(2,4)(4,3)(3,1)(1,2)(2,4)
  \pscircle(2,4){.2}\pscircle(4,3){.2}\pscircle(3,1){.2}\pscircle(1,2){.2}}
\newcommand{\convexexbb}{\psset{linecolor=green,fillcolor=white}
  \psline(2,4)(5,4)(4,0)(1,2)(2,4)
  \pscircle(2,4){.2}\pscircle(5,4){.2}\pscircle(4,0){.2}\pscircle(1,2){.2}}
\newcommand{\convexexbc}{\psset{linecolor=red,fillcolor=white}
  \psline(2,4)(4,3)(5,4)(4,0)(1,2)(2,4)
  \pscircle(2,4){.2}\pscircle(4,3){.2}\pscircle(5,4){.2}\pscircle(4,0){.2}\pscircle(1,2){.2}}

\newcommand{\convexpfa}{\\[4mm]\proof\quad}
\newcommand{\convexpfb}[1]{Set $\bl n\mdef R_3(k,k)$.\rput(1.5,-8.5){\begin{pspicture}(0,0)(4,3)
  \pnode(0,1){a}\pnode(1,3){b}\pnode(3,2){c}\pnode(4,3){d}\pnode(3.5,0){e}
  \psset{shadow=false,shadow=false,fillstyle=solid,framearc=0}#1\end{pspicture}}}
\newcommand{\convexpfc}{\\Then, given any $\bl n$ points, label them $\bl 1,\ldots,n$ in any order.}
\newcommand{\convexpfd}{\\Colour every triple $\bl\{i,j,\ell\}$ {\dg green} if $\bl i\mlt j\mlt\ell$ is a {\dg clockwise} sequence,
  \\\ph{b}\hspace*{40mm}and {\red red} if they form an {\red anti-clockwise} sequence.}
\newcommand{\convexpfe}{\\Then there are $\bl k$ points whose triples are all {\dg green} or all {\red red}, say {\dg green}.}
\newcommand{\convexpff}{\\In other words, each triangle among these $\bl k$ points is oriented {\dg clockwise}.}
\newcommand{\convexpfg}{\\But then the $\bl k$ points form a convex polygon.\qed}
\newcommand{\convexpfbz}{\psset{linecolor=darkgray}
  \qdisk(a){.1}\qdisk(b){.1}\qdisk(c){.1}\qdisk(d){.1}\qdisk(e){.1}}
\newcommand{\convexpfby}{{\bl\small\uput[l](a){1}\uput[ul](b){2}\uput[d](2.8,2){3}\uput[ur](d){4}\uput[dr](e){5}}}
\newcommand{\convexpfbc}{\psset{linecolor=green,fillcolor=white}
  \pspolygon(c)(e)(a)\pscircle(a){.2}\pscircle(c){.2}\pscircle(e){.2}}
\newcommand{\convexpfbd}{\psset{linecolor=green,fillcolor=white}
  \pspolygon(a)(c)(b)(a)\pscircle(a){.2}\pscircle(c){.2}\pscircle(b){.2}}
\newcommand{\convexpfbe}{\psset{linecolor=red,fillcolor=white}
  \pspolygon(c)(b)(d)\pscircle(d){.2}\pscircle(c){.2}\pscircle(b){.2}}
\newcommand{\convexpfbf}{\psset{linecolor=lightgray,fillcolor=white}
  \pspolygon(b)(c)(e)(a)\pscircle(a){.2}\pscircle(b){.2}\pscircle(c){.2}\pscircle(e){.2}}
\newcommand{\convexpfbfa}{\psset{linecolor=green,fillcolor=white}
  \pspolygon(c)(e)(a)\pscircle(a){.2}\pscircle(c){.2}\pscircle(e){.2}}
\newcommand{\convexpfbfb}{\psset{linecolor=green,fillcolor=white}
  \pspolygon(b)(e)(a)\pscircle(a){.2}\pscircle(b){.2}\pscircle(e){.2}}
\newcommand{\convexpfbfc}{\psset{linecolor=green,fillcolor=white}
  \pspolygon(b)(c)(a)\pscircle(a){.2}\pscircle(b){.2}\pscircle(c){.2}}
\newcommand{\convexpfbfd}{\psset{linecolor=green,fillcolor=white}
  \pspolygon(b)(c)(e)\pscircle(b){.2}\pscircle(c){.2}\pscircle(e){.2}}
\newcommand{\convexpfbg}{\psset{linecolor=green,fillcolor=white}
  \pspolygon(b)(c)(e)(a)\pscircle(a){.2}\pscircle(b){.2}\pscircle(c){.2}\pscircle(e){.2}}

\newcommand{\erdosa}{\theorem {\gray(Erd\H{o}s 1964)}\\
Let $\bl A\subseteq\mathbb{N}$ so that for which each $\bl n\myin\mathbb{N}$
there are $\bl a,b\myin A$ with $\bl n\meq ab$.
Then for each $\bl k\myin\mathbb{N}$ there is some integer $\bl n\myin\mathbb{N}$ so that
the equation $\bl n\meq ab$ has at least $\bl k$ solutions with $\bl a,b\myin A$.}

\newcommand{\erdospfa}{\\[4mm]\proof {\gray (Ne\v{s}et\v{r}il 1984; Ne\v{s}et\v{r}il \&\ R\"odl 1985)}}
\newcommand{\erdospfb}{\\Consider square-free integers $\bl n$ only.
\\For such $\bl n$, let $\bl M(n)$ be the set of primes factors of $\bl n$.
\\We can find a partition $\bl M(n) \meq M'\cup M''$ so that\\
$\bl a \mdef \prod M'\mdef \prod_{p\in M'}p$ and
$\bl b \mdef \prod M''$ both belong to $\bl A$.
\\By {\dg Ramsey's Theorem},
there is a finite set $\bl X\subseteq\mathbb{N}$ with $\bl|X|\mgeq 2k^2$
so that for each $\bl M\myin\binom{X}{k}$,
{\red the partition $M = M'\cup M''$ is of the same~`type'.
\\Hence, there exists $Y\subseteq X$, $|Y|\geq 2k$, such that $\binom{Y}{m}\subseteq A$
for an $m\geq\frac{1}{2}$.
\\Thus, for every $M\in\binom{M}{2m}$, $n:=\prod M$ has
$\binom{2m}{m}$ solutions $n=ab$ in $A$.}\qed}

\newcommand{\erdosb}{\\[4mm]{\dg Challenge}}
\newcommand{\erdosc}{\\Figure out what this proof says and write out a clean and intelligible proof!}

\newcommand{\biglist}{Look here for many more {\dg Ramsey Theory} applications:\\[4mm]\begin{center}\url{http://www.cs.umd.edu/~gasarch/TOPICS/ramsey/ramsey.html}\end{center}}

\newcommand{\euclida}{{\dg Euclid's Lemma} aka {\dg Bezout's Identity}
\\[1mm]For any coprime $\bl a, b\myin\mathbb{Z}$,
there are $\bl x, y\myin\mathbb{Z}$ such that $\bl ax + by \meq 1$.}

\newcommand{\euclidpfa}{\\[4mm]\proof}
\newcommand{\euclidpfb}{\\Consider the $\bl b-1$ remainders of $\bl a, 2a, \ldots, (b-1)a$ modulo $\bl b$.}
\newcommand{\euclidpfc}{\\Since $\bl a$ and $\bl b$ are coprime, none of the remainders equal $\bl0$.}
\newcommand{\euclidpfd}{\\Now assume that no remainder equals $\bl1$.}
\newcommand{\euclidpfe}{\\Then the $\bl b-1$ remainders must be among the numbers $\bl 2,\ldots,b-1$.}
\newcommand{\euclidpff}{\\By the {\dg Pigeonhole Principle},
  two of the remainders must be equal}
\newcommand{\euclidpfg}{\\so $\bl ma \mequiv na \pmod{b}$ for distinct $\bl m,n\mgt 0$.}
\newcommand{\euclidpfh}{\\Hence $\bl (m - n)a\mequiv0\pmod{b}$.}
\newcommand{\euclidpfj}{\\Since $\bl a$ and $\bl b$ are coprime, $\bl m-n\mequiv 0\pmod{b}$}
\newcommand{\euclidpfk}{,\\or $\bl m\mequiv n\pmod{b}$, a contradiction.}
\newcommand{\euclidpfm}{\\Hence, $\bl xa \mequiv 1 \pmod{b}$ for some $\bl x\myin\mathbb{Z}$}
\newcommand{\euclidpfn}{\\and so $\bl ax + by \meq 1$ for some $\bl y\myin\mathbb{Z}$.\qed}

\newcommand{\euclidb}{\\[4mm]{\gray Modified just slightly,
  this proof gives the {\green Chinese Remainder Theorem}.}}

\newcommand{\proizvolova}{{\dg Proizvolov's Identity}
  \\Bipartition $\bl[2n]$ into sets
  $\bl A \meq \{a_1\mgt \cdots \mgt a_n\}$ and $\bl B \meq \{b_1\mlt\cdots\mlt b_n\}$.
  Then\vspace*{-1mm}
  \[\bl
    \sum_{i=1}^n |a_i - b_i| \meq n^2\,\black.\vspace*{-2mm}
  \]}

\newcommand{\proizvolovpfa}{\proof}
\newcommand{\proizvolovpfb}{\\Assume that $\bl a_i,b_i\mleq n$ for some $\bl i$.}
\newcommand{\proizvolovpfc}{\\Then at least $\bl n - (i - 1)$ of the $\bl a_j$s are in $\bl [n]$.}
\newcommand{\proizvolovpfd}{\\Also, at least $\bl i$ of the $\bl b_j$s are in $\bl [n]$.}
\newcommand{\proizvolovpfe}{\\Then at least $\bl n-(i-1) + i \meq n+1$ of the $\bl a_j$s and $\bl b_j$s in $\bl [n]$.}
\newcommand{\proizvolovpff}{\\By the {\dg Pigeonhole Principle}, two of these are identical, a contradiction.}
\newcommand{\proizvolovpfg}{\\Hence, $\bl a_i$ and $\bl b_i$ cannot both be less than or equal to $\bl n$.}
\newcommand{\proizvolovpfh}{\\Similarly, $\bl a_i$ and $\bl b_i$ cannot both be bigger than or equal to $\bl n+1$.}
\newcommand{\proizvolovpfj}{\\Hence, one of $\bl a_i,b_i$ is in $\bl [n]$ and the other is in $\bl[2n]-[n]\meq \{n+1,\ldots,2n\}$.}
\newcommand{\proizvolovpfk}[2]{\\Then\vspace*{-1mm}
  \begin{align*}\bl
               \sum_{i=1}^n |a_i - b_i|
    & \;{  {\meq ((n+1) + \cdots + 2n) - (1+\cdots +n)}}\\[-4mm]
    & \;{#1{\meq (n+\cdots+n) + (1+\cdots+n) - (1+\cdots+n)}}
      \;{#2{\meq n^2\,\black.\;\Box}}\\[-12mm] \end{align*}}


\newcommand{\mantela}{{\dg Mantel's Theorem} {\gray (1907)}\\[-2mm]
If $\bl G\meq (V,E)$ is a simple graph on $\bl n$ vertices with $\bl |E|\mgt\dfrac{n^2}{4}$ edges,
\\[-1mm]then $\bl G$ has at least one triangle.}
\newcommand{\mantelb}{\\[2mm]{\gray This is a special case of {\green Tur\'{a}n's Theorem} stating when $\lightblue G$ must contain~$\lightblue K_p$.}}

\newcommand{\mantelexa}{\\[8mm]\example}
\newcommand{\mantelexb}[3]{\vspace*{10mm}\begin{center}\hspace*{-35mm}\myhexagon{#1}{\uput[l](7,0){$\bl|E|\meq #3$}#2}\hspace*{35mm}\end{center}}
\newcommand{\mantelexba}{\psline(a)(d)}
\newcommand{\mantelexbb}{\psline(b)(e)}
\newcommand{\mantelexbc}{\psline(c)(f)}
\newcommand{\mantelexbd}{\psline(a)(c)}
\newcommand{\mantelexbe}{\psset{linecolor=green,linewidth=0.075,fillcolor=white}\pspolygon[fillstyle=none](a)(c)(d)\pscircle(a){.2}\pscircle(c){.2}\pscircle(d){.2}}

\newcommand{\mantelexpfa}{\\[4mm]\proof}
\newcommand{\mantelexpfb}{\\If $\bl n\meq 3$, then if $\bl |E|\mgt \frac{n^2}{4} \meq 2.25$, then $\bl |E|\meq 3$, so $\bl G$ is itself a triangle.}
\newcommand{\mantelexpfc}{\\Assume for induction that the theorem is true for all values up to $\bl n-1$.}
\newcommand{\mantelexpfd}{\\Choose an edge $\bl\{u,v\}$ of $\bl E$.}
\newcommand{\mantelexpfe}{\\Let $\bl H$ be the subgraph of $\bl G$ obtained by deleting $\bl u$ and $\bl v$.}
\newcommand{\mantelexpff}{\\If $\bl |E(H)|\!\mgt\!\bl\frac{(n-2)^2}{4}$ edges, then $\bl H$ and thus $\bl G$ has a triangle, by assumption.}
\newcommand{\mantelexpfg}{\\Suppose that $\bl |E(H)|\mleq\frac{(n-2)^2}{4} \;\lightblue(\gray=\lightblue \frac{n^2}{4} - n + 1)$.}
\newcommand{\mantelexpfh}{\\There are $\bl|E-\{u,v\}|-|E(H)|\mgt\frac{n^2}{4}-1-\frac{(n-2)^2}{4}\meq n-2$
  edges between $\bl H$ and $\bl\{u,v\}$, so at least $\bl (n-2)+1$ edges}
\newcommand{\mantelexpfj}{, but only $\bl n-2$ vertices in $\bl H$.}
\newcommand{\mantelexpfk}{\\By the {\dg Pigeonhole Principle}, some vertex in $\bl H$
  is adjacent to both $\bl u$ and~$\bl v$.}
\newcommand{\mantelexpfm}{\\Thus, $\bl G$ has a triangle.\qed}

\newcommand{\contents}{\pspicture(0,0)(0,0)\endpspicture\\[-8mm]
  {{\sc\Large Ramsey Theory}\footnotesize\dg
  \\[- .2mm]The Pigeonhole Principle
  \\[-1.6mm]Ramsey's Theory
  \\[-1.6mm]Van der Waerden's Theorem
  \\[-1.6mm]S($\bl k,m$) (The Hales-Jewett Theorem)
  \\[-1.6mm]Schur's Theorem
  \\[-1.6mm]Schur's $\bl x^m+y^m\mequiv z^m\pmod{p}$ theorem
  \\[-1.6mm]Generalisation of Schur's Theorem and Van der Waerden's Theorem
  \\[-1.6mm]Rado's Theorem, Folkman's Theorem, Hindman's Theorem
  \\[-1.6mm]Ramras' intersection set theorem
  \\[-1.6mm]The Bipartite Ramsey Theorem
  \\[-1.6mm]The Chv\'{a}tal-Harary theorem
  \\[-1.6mm]Chv\'{a}tal's theorem
  \\[-1.6mm]The Vector Space Ramsey Theorem, The Gallai-Witt Theorem
  \\[-1.6mm]Applications
  \\[-1mm]\myspace Mixon's proof of infinitely many primes
  \\[-1.6mm]\myspace Dirichlet's Approximation Theorem
  \\[-1.6mm]\myspace Fermat's Sum of Squares Theorem
  \\[-1.6mm]\myspace No lossless compression algorithm is perfect
  \\[-1.6mm]\myspace The Erd\H{o}s-Szekeres Theorem
  \\[-1.6mm]\myspace The Happy Ending Problem
  \\[-1.6mm]\myspace Erd\H{o}s' $\bl n\meq ab$ theorem
  \\[-1.6mm]\myspace Euclid's Lemma/Bezout's Identity
  \\[-1.6mm]\myspace The Chinese Remainder Theorem
  \rput[l](-9.48,-.4){\myspace Proizvolov's Identity}
  \rput[l](-9.49,-.9){\myspace Mantel's Theorem}}}

\newcommand{\references}{{\dg References}\small
  \\[2mm]M.~Aigner and G.M.~Ziegler,
  \\\myspace {\sl\dg Proofs from The Book}, 4th edition, Springer-Verlag, Berlin, 2010.
  \\[1mm]A.~Bogomolny, {\sl\dg Pigeonhole Principle},
  \\\myspace \url{http://www.cut-the-knot.org/do_you_know/pigeon.shtml}, 2016-03.
  \\[1mm]J.~Brandt, {\sl\dg Kombinatorik}, Aarhus University, lecture notes, 2001.
  \\[1mm]R.L.~Graham, M.~Gr\"otschel, and L.~Lov\'asz (eds.),
  \\\myspace {\sl\dg Handbook of Combinatorics. I--II}, North-Holland, Amsterdam, 1995.
  \\[1mm]R.L.~Graham, B.L.~Rothschild, and J.~Spencer,
  \\\myspace {\sl\dg Ramsey Theory}, 2nd edition, John Wiley \&\ Sons, Inc., New York, 1990.
  \\[1mm]B.M.~Landman and A.~Robertson,
  \\\myspace {\sl\dg Ramsey Theory on the Integers}, 2nd edition, AMS, Providence, RI, 2014.
  \\[1mm]J.H.~van Lint and R.M.~Wilson,
  \\\myspace {\sl\dg A Course in Combinatorics}, Cambridge University Press, 1992.
}

%  \\J.~Ne\v{s}et\v{r}il and V.~R\"{o}dl, Two proofs in combinatorial number theory,
%  \\\myspace{\sl Proc.\ Amer.\ Math.\ Soc.}~{\bf 93}, (1985), 185--188
%P. Erdős, On the multiplicative representation of integers, Israel J. Math. 2 (1964), 251--261.
%J. Nešetřil and V. Rödl, Simple proof of the existence of restricted Ramsey graphs by means of partite construction, Combinatorica 2 (1981), 199-202.

\coursetitle
\np\lecturetitle
\np\lecturetitlei
\np\pigeonthma
\np\pigeonthma\pigeonthmexaa
\np\pigeonthma\pigeonthmexaa\pigeonthmexab{}
\np\pigeonthma\pigeonthmexaa\pigeonthmexab{\pigeonthmexaba}
\np\pigeonthma\pigeonthmexaa\pigeonthmexab{\pigeonthmexaba\pigeonthmexabc}
\np\pigeonthma\pigeonthmexaa\pigeonthmexab{\pigeonthmexaba\pigeonthmexabb\pigeonthmexabc}
\np\pigeonthma\pigeonthmexaa\pigeonthmexab{\pigeonthmexaba\pigeonthmexabb\pigeonthmexabcii}
\np\pigeonthmplusa
\np\pigeonthmplusa\\[4mm]\ramseyb
\np\pigeonthmplusa\\[4mm]\ramseyb\\[4mm]\ramseye
\np\pigeonthmplusa\\[4mm]\ramseyb\\[4mm]\ramseye\ramseyea
\np\convexa
\np\convexa\convexb
\np\convexa\convexexa
\np\convexa\convexexa\convexexb{}
\np\convexa\convexexa\convexexb{\convexexba}
\np\convexa\convexexa\convexexb{\convexexbb}
\np\convexa\convexexa\convexexb{\convexexbc}
\np\convexa
\np\convexa\convexpfa
\np\convexa\convexpfa\convexpfb{}
\np\convexa\convexpfa\convexpfb{}\convexpfc
\np\convexa\convexpfa\convexpfb{\convexpfbz}\convexpfc
\np\convexa\convexpfa\convexpfb{\convexpfby\convexpfbz}\convexpfc
\np\convexa\convexpfa\convexpfb{\convexpfby\convexpfbz}\convexpfc\convexpfd
\np\convexa\convexpfa\convexpfb{\convexpfbc\convexpfby\convexpfbz}\convexpfc\convexpfd
\np\convexa\convexpfa\convexpfb{\convexpfbd\convexpfby\convexpfbz}\convexpfc\convexpfd
\np\convexa\convexpfa\convexpfb{\convexpfbe\convexpfby\convexpfbz}\convexpfc\convexpfd
\np\convexa\convexpfa\convexpfb{\convexpfby\convexpfbz}\convexpfc\convexpfd\convexpfe
\np\convexa\convexpfa\convexpfb{\convexpfbf\convexpfby\convexpfbz}\convexpfc\convexpfd\convexpfe
\np\convexa\convexpfa\convexpfb{\convexpfbf\convexpfby\convexpfbz}\convexpfc\convexpfd\convexpfe\convexpff
\np\convexa\convexpfa\convexpfb{\convexpfbf\convexpfbfa\convexpfby\convexpfbz}\convexpfc\convexpfd\convexpfe\convexpff
\np\convexa\convexpfa\convexpfb{\convexpfbf\convexpfbfb\convexpfby\convexpfbz}\convexpfc\convexpfd\convexpfe\convexpff
\np\convexa\convexpfa\convexpfb{\convexpfbf\convexpfbfc\convexpfby\convexpfbz}\convexpfc\convexpfd\convexpfe\convexpff
\np\convexa\convexpfa\convexpfb{\convexpfbf\convexpfbfd\convexpfby\convexpfbz}\convexpfc\convexpfd\convexpfe\convexpff
\np\convexa\convexpfa\convexpfb{\convexpfbg\convexpfby\convexpfbz}\convexpfc\convexpfd\convexpfe\convexpff
\np\convexa\convexpfa\convexpfb{\convexpfbg\convexpfby\convexpfbz}\convexpfc\convexpfd\convexpfe\convexpff\convexpfg
\np\erdosa
\np\erdosa\erdospfa
\np\erdosa\erdospfa\erdospfb
\np\erdosa\erdospfa\erdospfb\erdosb
\np\erdosa\erdospfa\erdospfb\erdosb\erdosc
\np\biglist
\np\euclida
\np\euclida\euclidpfa
\np\euclida\euclidpfa\euclidpfb
\np\euclida\euclidpfa\euclidpfb\euclidpfc
\np\euclida\euclidpfa\euclidpfb\euclidpfc\euclidpfd
\np\euclida\euclidpfa\euclidpfb\euclidpfc\euclidpfd\euclidpfe
\np\euclida\euclidpfa\euclidpfb\euclidpfc\euclidpfd\euclidpfe\euclidpff
\np\euclida\euclidpfa\euclidpfb\euclidpfc\euclidpfd\euclidpfe\euclidpff\euclidpfg
\np\euclida\euclidpfa\euclidpfb\euclidpfc\euclidpfd\euclidpfe\euclidpff\euclidpfg\euclidpfh
\np\euclida\euclidpfa\euclidpfb\euclidpfc\euclidpfd\euclidpfe\euclidpff\euclidpfg\euclidpfh\euclidpfj
\np\euclida\euclidpfa\euclidpfb\euclidpfc\euclidpfd\euclidpfe\euclidpff\euclidpfg\euclidpfh\euclidpfj\euclidpfk
\np\euclida\euclidpfa\euclidpfb\euclidpfc\euclidpfd\euclidpfe\euclidpff\euclidpfg\euclidpfh\euclidpfj\euclidpfk\euclidpfm
\np\euclida\euclidpfa\euclidpfb\euclidpfc\euclidpfd\euclidpfe\euclidpff\euclidpfg\euclidpfh\euclidpfj\euclidpfk\euclidpfm\euclidpfn
\np\euclida\euclidpfa\euclidpfb\euclidpfc\euclidpfd\euclidpfe\euclidpff\euclidpfg\euclidpfh\euclidpfj\euclidpfk\euclidpfm\euclidpfn\euclidb
\np\proizvolova
\np\proizvolova\proizvolovpfa
\np\proizvolova\proizvolovpfa\proizvolovpfb
\np\proizvolova\proizvolovpfa\proizvolovpfb\proizvolovpfc
\np\proizvolova\proizvolovpfa\proizvolovpfb\proizvolovpfc\proizvolovpfd
\np\proizvolova\proizvolovpfa\proizvolovpfb\proizvolovpfc\proizvolovpfd\proizvolovpfe
\np\proizvolova\proizvolovpfa\proizvolovpfb\proizvolovpfc\proizvolovpfd\proizvolovpfe\proizvolovpff
\np\proizvolova\proizvolovpfa\proizvolovpfb\proizvolovpfc\proizvolovpfd\proizvolovpfe\proizvolovpff\proizvolovpfg
\np\proizvolova\proizvolovpfa\proizvolovpfb\proizvolovpfc\proizvolovpfd\proizvolovpfe\proizvolovpff\proizvolovpfg\proizvolovpfh
\np\proizvolova\proizvolovpfa\proizvolovpfb\proizvolovpfc\proizvolovpfd\proizvolovpfe\proizvolovpff\proizvolovpfg\proizvolovpfh\proizvolovpfj
\np\proizvolova\proizvolovpfa\proizvolovpfb\proizvolovpfc\proizvolovpfd\proizvolovpfe\proizvolovpff\proizvolovpfg\proizvolovpfh\proizvolovpfj\proizvolovpfk{\ph}{\ph}
\np\proizvolova\proizvolovpfa\proizvolovpfb\proizvolovpfc\proizvolovpfd\proizvolovpfe\proizvolovpff\proizvolovpfg\proizvolovpfh\proizvolovpfj\proizvolovpfk{}{\ph}
\np\proizvolova\proizvolovpfa\proizvolovpfb\proizvolovpfc\proizvolovpfd\proizvolovpfe\proizvolovpff\proizvolovpfg\proizvolovpfh\proizvolovpfj\proizvolovpfk{}{}
\np\mantela
\np\mantela\mantelb
\np\mantela\mantelb\mantelexa
\np\mantela\mantelb\mantelexa\mantelexb{}{}{6}
\np\mantela\mantelb\mantelexa\mantelexb{\mantelexba}{}{7}
\np\mantela\mantelb\mantelexa\mantelexb{\mantelexba\mantelexbb}{}{8}
\np\mantela\mantelb\mantelexa\mantelexb{\mantelexba\mantelexbb\mantelexbc}{}{9}
\np\mantela\mantelb\mantelexa\mantelexb{\mantelexba\mantelexbb\mantelexbc\mantelexbd}{}{10}
\np\mantela\mantelb\mantelexa\mantelexb{\mantelexba\mantelexbb\mantelexbc\mantelexbd}{\mantelexbe}{10}
\np\mantela\mantelb
\np\mantela\mantelb\mantelexpfa
\np\mantela\mantelb\mantelexpfa\mantelexpfb
\np\mantela\mantelb\mantelexpfa\mantelexpfb\mantelexpfc
\np\mantela\mantelb\mantelexpfa\mantelexpfb\mantelexpfc\mantelexpfd
\np\mantela\mantelb\mantelexpfa\mantelexpfb\mantelexpfc\mantelexpfd\mantelexpfe
\np\mantela\mantelb\mantelexpfa\mantelexpfb\mantelexpfc\mantelexpfd\mantelexpfe\mantelexpff
\np\mantela\mantelb\mantelexpfa\mantelexpfb\mantelexpfc\mantelexpfd\mantelexpfe\mantelexpff\mantelexpfg
\np\mantela\mantelb\mantelexpfa\mantelexpfb\mantelexpfc\mantelexpfd\mantelexpfe\mantelexpff\mantelexpfg\mantelexpfh
\np\mantela\mantelb\mantelexpfa\mantelexpfb\mantelexpfc\mantelexpfd\mantelexpfe\mantelexpff\mantelexpfg\mantelexpfh\mantelexpfj
\np\mantela\mantelb\mantelexpfa\mantelexpfb\mantelexpfc\mantelexpfd\mantelexpfe\mantelexpff\mantelexpfg\mantelexpfh\mantelexpfj\mantelexpfk
\np\mantela\mantelb\mantelexpfa\mantelexpfb\mantelexpfc\mantelexpfd\mantelexpfe\mantelexpff\mantelexpfg\mantelexpfh\mantelexpfj\mantelexpfk\mantelexpfm
\np\contents
\np\references





\end{document}

%\np\pigeonthma
%\np\pigeonthma\dirichleta
%\np\pigeonthma\dirichleta\dirichletb
%\np\pigeonthma\dirichleta\dirichletb\dirichletc
%\np\pigeonthma\dirichleta\dirichletb\dirichletpfa
%\np\pigeonthma\dirichleta\dirichletb\dirichletpfa\dirichletpfb
%\np\pigeonthma\dirichleta\dirichletb\dirichletpfa\dirichletpfb\dirichletpfc
%\np\pigeonthma\dirichleta\dirichletb\dirichletpfa\dirichletpfb\dirichletpfc\dirichletpfd
%\np\pigeonthma\dirichleta\dirichletb\dirichletpfa\dirichletpfb\dirichletpfc\dirichletpfd\dirichletpfe
%\np\pigeonthma\dirichleta\dirichletb\dirichletpfa\dirichletpfb\dirichletpfc\dirichletpfd\dirichletpfe\dirichletpff
%\np\pigeonthma\dirichleta\dirichletb\dirichletpfa\dirichletpfb\dirichletpfc\dirichletpfd\dirichletpfe\dirichletpff\dirichletpfg
%\np\pigeonthma\dirichleta\dirichletb\dirichletpfa\dirichletpfb\dirichletpfc\dirichletpfd\dirichletpfe\dirichletpff\dirichletpfg\dirichletpfh
%\np\pigeonthma\dirichleta\dirichletb\dirichletpfa\dirichletpfb\dirichletpfc\dirichletpfd\dirichletpfe\dirichletpff\dirichletpfg\dirichletpfh\dirichletpfj{\ph}{\ph}{\ph}
%\np\pigeonthma\dirichleta\dirichletb\dirichletpfa\dirichletpfb\dirichletpfc\dirichletpfd\dirichletpfe\dirichletpff\dirichletpfg\dirichletpfh\dirichletpfj{}{\ph}{\ph}
%\np\pigeonthma\dirichleta\dirichletb\dirichletpfa\dirichletpfb\dirichletpfc\dirichletpfd\dirichletpfe\dirichletpff\dirichletpfg\dirichletpfh\dirichletpfj{}{}{\ph}
%\np\pigeonthma\dirichleta\dirichletb\dirichletpfa\dirichletpfb\dirichletpfc\dirichletpfd\dirichletpfe\dirichletpff\dirichletpfg\dirichletpfh\dirichletpfj{}{}{}


\newcommand{\spernerlemma}{{\dg Sperner's Lemma} {\gray(1928)}
  \\Triangulate a triangle with vertices $\bl v_1,v_2,v_3$ into smaller triangles.
  \\Colour each of the vertices in the colours $\bl\{1,2,3\}$ so that
  \\\mybullet $\bl v_i$ has colour $\bl i$ for each $\bl i\meq 1,2,3$;
  \\\mybullet each vertex along the edge from $\bl v_i$ to $\bl v_j$ has colour $\bl i$ or $\bl j$;
  \\\mybullet all other (interior) vertices may have any colour $\bl 1,2,3$.
  \\Then there is a triangle whose vertices have all three colours.}

\newcommand{\spernerlemmaexa}{\example}
\newcommand{\spernerlemmaexb}{}

\newcommand{\spernerlemmapfa}{\\[2mm]\proof}
\newcommand{\spernerlemmapfb}{\\Let $\bl G$ be the graph with a vertex in each triangle, including the outer triangle,
  and edges that cross the triangle sides whose vertices are coloured $\bl 1$ and $\bl2$.}
\newcommand{\spernerlemmapfc}{\\{\gray (Then $\lightblue G$ is a subgraph of the dual graph of the triangulation.)}}
\newcommand{\spernerlemmapfd}{\\The number of edges crossing the triangle side $\bl v1$ from $\bl v_2$ is odd.}
\newcommand{\spernerlemmapfe}{\\Hence, the degree of the other vertex of $\bl G$ is odd.}
\newcommand{\spernerlemmapff}{\\The degree of each other vertex is 1 if it lies in a tricoloured triangle and is 2 otherwise.}
\newcommand{\spernerlemmapfg}{\\By the {\dg Handshaking Lemma}, the sum of vertex degrees is even}
\newcommand{\spernerlemmapfh}{\\so there must be an odd number of $\bl 1$-degree vertices.}
\newcommand{\spernerlemmapfj}{\\In particular, there is at least one such vertex.}
\newcommand{\spernerlemmapfk}{\\Hence, there is a tri-coloured triangle.\qed}

\np\spernerlemma


\newcommand{\brouwera}{{\dg Brouwer's Fixed Point Theorem} {\gray (1912)}
  \\Every continuous function $\bl f\,:\,B\to B$ on an $\bl n$-dimensional ball $\bl B\subseteq \mathbb{R}^n$
    has a fixed point $\bl x\myin B$ with $\bl f(x)\meq (x)$.}

