%%  MATH5505 Ramsey Theory Lecture 5: Graphs and Geometry
%%
%%  by Thomas Britz 2018S1
%%

% OK Refresh Ramsey's Theorem
% OK Ramsey bipartite theorem  (prove, first proving "flower" intersection result Ramsey theorem)
% OK r(T_n,K_m) \geq ...       (prove)
% OK r(T_n,K_m) = ...          (prove)
% OK Euclidean Ramsey Theory: the Gallai-Witt Theorem (mention) - translate and scale
% OK Euclidean Ramsey Theory: bricks, regular n-gon - translate and rotate/reflect etc.
%

\documentclass[12pt,a4paper,landscape]{article}
\special{landscape}

\usepackage{latexsym,amsfonts,amsmath,amssymb} %,calc,fancybox,epsfig,amscd,tabularx}
\usepackage{pstricks,pst-plot,pst-node,pst-tree}
%\usepackage{array,graphicx,epsfig}
%\usepackage{multicol,multirow,hyperref,rotating}
%\usepackage[T1]{fontenc}
%\usepackage{yfonts}


\mag=\magstep4

\newrgbcolor{green}{0.2 0.6 0.2}
\newrgbcolor{darkgreen}{0 0.4 0}
\newrgbcolor{graydarkgreen}{0.5 0.67 0.5}
\newrgbcolor{lightgreen}{0 0.75 0}
\newrgbcolor{skyblue}{0.5 0.80 1}
\newrgbcolor{darkblue}{0 0 0.6}
\newrgbcolor{darkred}{0.6 0 0}
\newrgbcolor{halfred}{.8 0 0}
\newrgbcolor{white}{1 1 1}
\newrgbcolor{nearlywhite}{0.95 0.95 0.95}
\newrgbcolor{offwhite}{0.9 0.9 0.9}
\newrgbcolor{lightgray}{0.85 0.85 0.85}
\newrgbcolor{halfgray}{0.8 0.8 0.8}
\newrgbcolor{altgray}{0.67 0.67 0.67}
\newrgbcolor{darkyellow}{1 0.94 0.15}
\newrgbcolor{halflightyellow}{1 1 0.4}
\newrgbcolor{lightyellow}{1 1 0.7}
\newrgbcolor{gold}{0.96 0.96 0.1}
\newrgbcolor{lightblue}{0.5 0.5 1}
\newrgbcolor{amber}{1 0.75 0}
\newrgbcolor{hotpink}{1 0.41 0.71}

\setlength{\textheight}{18.0truecm}
\setlength{\textwidth}{26.0truecm}
\setlength{\hoffset}{-12.0truecm}
\setlength{\voffset}{-5.2truecm}

\parindent 0in

%\setlength{\bigskipamount}{5ex plus1.5ex minus 2ex}
%\setlength{\parindent}{0cm}
%\setlength{\parskip}{0.2cm}

\psset{unit=6mm,linewidth=.06,dotscale=1.5,fillcolor=white,fillstyle=none,
 linecolor=gray,framearc=.3,shadowcolor=offwhite,shadow=true,shadowsize=.125,
 shadowangle=-45,dash=7pt 5pt}
%\psset{unit=.4,linecolor=gray,fillcolor=offwhite,shadowsize=.2,framearc=.3}
%\psset{unit=10mm,linewidth=.03}

\def\dedge{\ncline[linestyle=dashed]}

\def\np{\newpage}
\def\bl{\blue}
\def\bk{\black}
\def\wh{\white}
\def\rd{\red}
\def\lg{\lightgray}
\def\gr{\green}
\def\dr{\darkred}
\def\dg{\darkgreen}
\def\gdg{\graydarkgreen}
\def\dgy{\darkgray}
\def\ec{\dg}

\newcommand{\ora}[1]{\overrightarrow{#1}}

\newcommand{\llb}{\\[1mm]}
\newcommand{\lb}{\\[3mm]}
\newcommand{\blb}{\\[5mm]}
\newcommand{\hlb}{\\[48mm]}

\newcommand{\mynewpage}{\newpage\vspace*{-10mm}}

\newcommand{\vc}[1]{\begin{pmatrix}#1\end{pmatrix}}

\newcommand{\mytext}[1]{\text{\black#1\blue}}

\newcommand{\qbinom}[2]{\genfrac{[}{]}{0pt}{}{#1}{#2}}

\newcommand{\dl}{\psset{linestyle=dashed,linecolor=altgray,shadow=false}}
%\newcommand{\dl}{\psset{linestyle=dashed,linecolor=blue}}

\newcommand{\myframe}[1]{\pspicture(0,0)(0,0)\psset{unit=1cm,
 shadowcolor=offwhite,shadow=true,shadowangle=-45,linewidth=.03,linecolor=gray,
 fillcolor=lightgray,shadowsize=.15,framearc=.3}\psframe#1\endpspicture}

\newcommand{\mypicture}[1]{\pspicture(0,0)(0,0)\psset{unit=1cm,
 shadowcolor=offwhite,shadow=true,shadowangle=-45,linewidth=.03,linecolor=gray,
 fillcolor=lightgray,shadowsize=.15,framearc=.3}#1\endpspicture}

\newcommand{\mymatrix}[1]{{\psset{unit=4mm,linewidth=.03,linecolor=gray,fillstyle=solid,fillcolor=offwhite,shadow=false,framearc=0}\pspicture(0,0)(6,4)
  \psframe(0,0)(7,5)#1\psframe[linecolor=gray,fillcolor=offwhite,linewidth=.03,fillstyle=none,framearc=0](0,0)(7,5)\endpspicture}}

\newcommand{\mypair}[2]{{\psset{unit=4mm,linewidth=.03,linecolor=gray,fillstyle=solid,fillcolor=offwhite,shadow=false,framearc=0}\pspicture(0,.2)(2,.8)
  \psframe(0,0)(2,1)\darkgray\rput(.5,.5){#1}\rput(1.5,.5){#2}\psframe[linecolor=gray,fillcolor=offwhite,linewidth=.03,fillstyle=none,framearc=0](0,0)(2,1)\endpspicture}}

\newcommand{\mysixtuple}[6]{{\psset{unit=4mm,linewidth=.03,linecolor=gray,fillstyle=solid,fillcolor=offwhite,shadow=false,framearc=0}\pspicture(0,.2)(6,.8)
  \psframe(0,0)(6,1)\darkgray\rput(.5,.5){#1}\rput(1.5,.5){#2}\rput(2.5,.5){#3}\rput(3.5,.5){#4}\rput(4.5,.5){#5}\rput(5.5,.5){#6}
  \psframe[linecolor=gray,fillcolor=offwhite,linewidth=.03,fillstyle=none,framearc=0](0,0)(6,1)\endpspicture}}

\newcommand{\acset}{\psset{fillstyle=none,linecolor=red}}
\newcommand{\drr}{\psset{linecolor=darkred,fillcolor=red,linewidth=.034}}
\newcommand{\dbb}{\psset{linecolor=darkblue,fillcolor=blue,linewidth=.034}}
\newcommand{\bbl}{\psset{linecolor=blue,fillcolor=lightblue}}
\newcommand{\dbhg}{\psset{linecolor=darkblue,fillcolor=halfgray}}
\newcommand{\bhg}{\psset{linecolor=blue,fillcolor=halfgray}}
\newcommand{\rhg}{\psset{linecolor=red,fillcolor=halfgray}}
\newcommand{\ghg}{\psset{linecolor=green,fillcolor=halfgray}}
\newcommand{\dgg}{\psset{linecolor=darkgreen,fillcolor=green,linewidth=.034}}

\newcommand{\mychain}{\pspicture(0,0)(0,0)\dbb
  \psset{shadowsize=.2,shadow=true,shadowcolor=offwhite,shadowangle=-45,shadowsize=.15}
  \pscircle(0,0){.09}\pscircle(0,1){.09}\pscircle(0,2){.09}\pscircle(0,3){.09}
  \psline(0,.125)(0,.875)\psline(0,1.125)(0,1.875)\psline(0,2.125)(0,2.875)\endpspicture}

\newcommand{\myantichain}{\pspicture(0,0)(0,0)\drr
  \psset{shadowsize=.2,shadow=true,shadowcolor=offwhite,shadowangle=-45,shadowsize=.15}
  \pscircle(0,0){.09}\pscircle(1,0){.09}\pscircle(2,0){.09}\pscircle(3,0){.09}\endpspicture}

\newcommand{\mybullet}{{\psset{unit=6mm,dotscale=1.5,linewidth=.05,fillcolor=black,linecolor=black,framearc=.3,shadow=true}
  \pspicture(-.3,0)(.8,.4)\qdisk(.25,.2){.1}\pscircle*[linecolor=gray](.25,.2){.07}\endpspicture}}

\newcommand{\mysmallbullet}{{\psset{unit=4mm,dotscale=1.5,linewidth=.05,fillcolor=black,linecolor=black,framearc=.3,shadow=true}
  \pspicture(-.3,0)(.8,.4)\qdisk(.25,.2){.1}\pscircle*[linecolor=gray](.25,.2){.07}\endpspicture}}

\newcommand{\mybulletv}{{\psset{unit=6mm,dotscale=1.5,linewidth=.04,linecolor=black,fillstyle=solid,
  fillcolor=gray,shadow=false}\pspicture(0,0)(.5,.4)\pscircle(.25,.2){.15}\endpspicture}}

\newcommand{\mystar}{\pspicture(0,0)(0,0)\psset{unit=1cm}\gold\large$\star$\endpspicture
  \pspicture(0,0)(0,0)\psset{unit=1cm}\rput(-.23,.193){\small\white$\star$}\endpspicture}

\newcommand{\mystep}{{\psset{unit=6mm,dotscale=1.5,linewidth=.06,linecolor=darkgreen,fillstyle=solid,
  fillcolor=lightgreen,shadow=false}\pspicture(-.3,0)(.8,.4)\pscircle(.25,.2){.15}\endpspicture}}
\newcommand{\mystepv}{{\psset{unit=6mm,dotscale=1.5,linewidth=.06,linecolor=darkgreen,fillstyle=solid,
  fillcolor=lightgreen,shadow=false}\pspicture(0,0)(.5,.4)\pscircle(.25,.2){.15}\endpspicture}}
\newcommand{\mystepvv}{{\psset{unit=6mm,dotscale=1.5,linewidth=.06,linecolor=darkred,fillstyle=solid,
  fillcolor=red,shadow=false}\pspicture(0,0)(.5,.4)\pscircle(.25,.2){.15}\endpspicture}}
\newcommand{\mystepvvv}{{\psset{unit=6mm,dotscale=1.5,linewidth=.06,linecolor=darkred,fillstyle=solid,
  fillcolor=red,shadow=false}\pspicture(0,0)(.5,.4)\endpspicture}}

\newcommand{\myi}{{\psset{unit=6mm,dotscale=1.5,linewidth=.05,linecolor=darkgreen,fillstyle=none,shadow=false}%
  \pspicture(-.3,0)(.8,.4)\pscircle(.25,.2){.25}\rput(.25,.2){\tiny\dg1}\endpspicture}}
\newcommand{\myii}{{\psset{unit=6mm,dotscale=1.5,linewidth=.05,linecolor=darkgreen,fillstyle=none,shadow=false}%
  \pspicture(-.3,0)(.8,.4)\pscircle(.25,.2){.25}\rput(.25,.2){\tiny\dg2}\endpspicture}}
\newcommand{\myiii}{{\psset{unit=6mm,dotscale=1.5,linewidth=.05,linecolor=darkgreen,fillstyle=none,shadow=false}%
  \pspicture(-.3,0)(.8,.4)\pscircle(.25,.2){.25}\rput(.25,.2){\tiny\dg3}\endpspicture}}
\newcommand{\myiv}{{\psset{unit=6mm,dotscale=1.5,linewidth=.05,linecolor=darkgreen,fillstyle=none,shadow=false}%
  \pspicture(-.3,0)(.8,.4)\pscircle(.25,.2){.25}\rput(.25,.2){\tiny\dg4}\endpspicture}}
\newcommand{\myv}{{\psset{unit=6mm,dotscale=1.5,linewidth=.05,linecolor=darkgreen,fillstyle=none,shadow=false}%
  \pspicture(-.3,0)(.8,.4)\pscircle(.25,.2){.25}\rput(.25,.2){\tiny\dg5}\endpspicture}}
\newcommand{\myvi}{{\psset{unit=6mm,dotscale=1.5,linewidth=.05,linecolor=darkgreen,fillstyle=none,shadow=false}%
  \pspicture(-.3,0)(.8,.4)\pscircle(.25,.2){.25}\rput(.25,.2){\tiny\dg6}\endpspicture}}
\newcommand{\myvii}{{\psset{unit=6mm,dotscale=1.5,linewidth=.05,linecolor=darkgreen,fillstyle=none,shadow=false}%
  \pspicture(-.3,0)(.8,.4)\pscircle(.25,.2){.25}\rput(.25,.2){\tiny\dg7}\endpspicture}}
\newcommand{\myviii}{{\psset{unit=6mm,dotscale=1.5,linewidth=.05,linecolor=darkgreen,fillstyle=none,shadow=false}%
  \pspicture(-.3,0)(.8,.4)\pscircle(.25,.2){.25}\rput(.25,.2){\tiny\dg8}\endpspicture}}
\newcommand{\myix}{{\psset{unit=6mm,dotscale=1.5,linewidth=.05,linecolor=darkgreen,fillstyle=none,shadow=false}%
  \pspicture(-.3,0)(.8,.4)\pscircle(.25,.2){.25}\rput(.25,.2){\tiny\dg9}\endpspicture}}

\newcommand{\mywo}{{\psset{unit=6mm,dotscale=1.5,linewidth=.05,linecolor=darkgreen,fillcolor=white,fillstyle=solid,shadow=false}%
  \pspicture(-.3,0)(.8,.4)\pscircle(.25,.2){.25}\rput(.25,.2){\tiny\dg0}\endpspicture}}
\newcommand{\mywi}{{\psset{unit=6mm,dotscale=1.5,linewidth=.05,linecolor=darkgreen,fillcolor=white,fillstyle=solid,shadow=false}%
  \pspicture(-.3,0)(.8,.4)\pscircle(.25,.2){.25}\rput(.25,.2){\tiny\dg1}\endpspicture}}
\newcommand{\mywii}{{\psset{unit=6mm,dotscale=1.5,linewidth=.05,linecolor=darkgreen,fillcolor=white,fillstyle=solid,shadow=false}%
  \pspicture(-.3,0)(.8,.4)\pscircle(.25,.2){.25}\rput(.25,.2){\tiny\dg2}\endpspicture}}

\newcommand{\myO}{{\psset{unit=6mm,dotscale=1.5,linewidth=.06,linecolor=darkgreen,fillstyle=none,
  shadow=false}\pspicture(0,0)(0,0)\pscircle(-.2,.225){.4}\endpspicture}}

\newcommand{\myspace}{\psset{unit=6mm,dotscale=1.5,linewidth=.06}\pspicture(-.3,0)(.8,.4)\endpspicture}

\newcommand{\mydot}{\pscircle(0,0){.1}}
\newcommand{\mydotv}{\pscircle*[linecolor=gray](0,0){.1}\pscircle*[linecolor=blue](0,0){.06}}

\newcommand{\myhexagon}[2]{\psset{fillcolor=halfgray,linecolor=gray,shadow=false,fillstyle=none,linewidth=.06}
  \pspicture(-3.5,-2.6)(3,2.6)
    \pnode(-3  , 0  ){a}
    \pnode(-1.5, 2.6){b}
    \pnode( 1.5, 2.6){c}
    \pnode( 3  , 0  ){d}
    \pnode( 1.5,-2.6){e}
    \pnode(-1.5,-2.6){f}
    \pspolygon(a)(b)(c)(d)(e)(f)#1\psset{fillstyle=solid}
    \pscircle(a){.2}
    \pscircle(b){.2}
    \pscircle(c){.2}
    \pscircle(d){.2}
    \pscircle(e){.2}
    \pscircle(f){.2}#2
  \endpspicture}

\newcommand{\myksix}[2]{\myhexagon{\pspolygon(a)(c)(e)\pspolygon(b)(d)(f)\psline(a)(d)\psline(b)(e)\psline(c)(f)#1}{#2}}

\newcommand{\lss}{$\spadesuit$}
\newcommand{\lhs}{$\darkred\heartsuit$}
\newcommand{\lds}{$\darkred\diamondsuit$}
\newcommand{\lcs}{$\clubsuit$}

\newcommand{\hatmanv}{{%
 \psset{dotscale=1.5,dotsize=0.09,linewidth=.03,fillcolor=black,linecolor=black,shadow=false}
 \psline(0,0)(0.25,0)\psline(1.75,0)(2,0)\psline(0.25,0)(1,1.2)
 \psline(1.75,0)(1,1.2)\psline(1,1.2)(1,2.5)\psline(0.25,1.75)(1,2)
 \psline(1.75,1.75)(1,2)\pscircle(1,3){0.5}\psarc(1,3){.25}{-140}{-40}
 \psdots(.85,3.15)(1.15,3.15)\psset{fillstyle=solid}\psframe(.5,3.35)(1.5,3.45)
 \psframe(.7,3.35)(1.3,4.1)}}

\newcommand{\manv}{{%
 \psset{dotscale=1.5,dotsize=0.09,linewidth=.03,fillcolor=black,linecolor=black,shadow=false}%
 \pspicture(0,0)(2,3.5)%\psline(0,0)(0.25,0)\psline(1.75,0)(2,0)
 \psline(0.25,0)(1,1.2)
 \psline(1.75,0)(1,1.2)\psline(1,1.2)(1,2.5)\psline(0.25,1.75)(1,2)
 \psline(1.75,1.75)(1,2)\pscircle(1,3){0.5}\psarc(1,3){.25}{-140}{-40}
 \psdots(.85,3.15)(1.15,3.15)%\psset{fillstyle=solid}\psframe(.5,3.35)(1.5,3.45)\psframe(.7,3.35)(1.3,4.1)
 \endpspicture}}

\newcommand{\sadmanv}{{
 \psset{dotscale=1.5,dotsize=0.09,linewidth=.03,fillcolor=black,linecolor=black,shadow=false}
 \psline(0,0)(0.25,0)\psline(1.75,0)(2,0)\psline(0.25,0)(1,1.2)
 \psline(1.75,0)(1,1.2)\psline(1,1.2)(1,2.5)\psline(0.25,1.75)(1,2)
 \psline(1.75,1.75)(1,2)\pscircle(1,3){0.5}\psarc(1,2.65){.25}{40}{140}
 \psdots(.87,3.15)(1.13,3.15)\psset{fillstyle=solid}\psframe(.5,3.35)(1.5,3.45)
 \psframe(.7,3.35)(1.3,4.1)}}

\newcommand{\womanv}{{
 \psset{dotscale=1.5,dotsize=0.09,linewidth=.03,fillcolor=black,linecolor=black,shadow=false}
 \pspicture(0,0)(2,3.5)%
 \psline(0.5,0)(0.75,0)\psline(1.5,0)(1.25,0)\psline(0.75,0)(0.75,0.5)
 \psline(1.25,0)(1.25,0.5)\psline(1,2)(1,2.35)\psline(0.25,1.75)(1,2)
 \psline(1.75,1.75)(1,2)\pscircle(1,2.85){0.5}
 \psarc(1,2.85){.25}{-140}{-40}\psdots(.85,2.95)(1.15,2.95)
 \psset{linecolor=brown,linewidth=.1}
 \psarc(1,2.85){.5}{0}{180}\psarc(1.75,2.85){.25}{180}{270}
 \psarc(0.25,2.85){.25}{-90}{0}
 \pstriangle[linecolor=red,fillstyle=solid,fillcolor=red](1,.5)(1.2,1.75)\endpspicture}}

\newcommand{\spyv}{{%
 \psset{dotscale=1.5,dotsize=0.09,linewidth=.03,fillcolor=black,linecolor=black,shadow=false}
 \pspicture(0,0)(2,3.5)%\psline(0,0)(0.25,0)\psline(1.75,0)(2,0)
 \psline(0.25,0)(1,1.2)
 \psline(1.75,0)(1,1.2)\psline(1,1.2)(1,2.5)\psline(0.25,1.75)(1,2)
 \psline(1.75,1.75)(1,2)\pscircle(1,3){0.5}\psarc(1,3){.25}{-140}{-40}
 %\psdots(.85,3.15)(1.15,3.15)
 \psset{fillstyle=solid}
 \psframe( .725,3.075)( .975,3.225)
 \psframe(1.025,3.075)(1.275,3.225)
 \psellipse*(1,3.5)(.5,.14)
 \pspolygon(.6,3.5)(1.4,3.5)( .8,3.8)
 \pspolygon(.6,3.5)(1.4,3.5)(1.2,3.8)
 \endpspicture}}

\newcommand{\puzzledspyv}{{%
 \psset{dotscale=1.5,dotsize=0.09,linewidth=.03,fillcolor=black,linecolor=black,shadow=false}
 \pspicture(0,0)(2,3.5)%\psline(0,0)(0.25,0)\psline(1.75,0)(2,0)
 \psline(0.25,0)(1,1.2)
 \psline(1.75,0)(1,1.2)\psline(1,1.2)(1,2.5)\psline(0.25,1.75)(1,2)
 \psline(1.75,1.75)(1,2)\pscircle(1,3){0.5}
 % Mouth
 \psline(.83,2.8)(1.17,2.8)
 %\psdots(.85,3.15)(1.15,3.15)
 \psset{fillstyle=solid}
 \psframe( .725,3.075)( .975,3.225)
 \psframe(1.025,3.075)(1.275,3.225)
 \psellipse*(1,3.5)(.5,.14)
 \pspolygon(.6,3.5)(1.4,3.5)( .8,3.8)
 \pspolygon(.6,3.5)(1.4,3.5)(1.2,3.8)
 \endpspicture}}

\newcommand{\man}{\psset{unit=4mm}\manv}
\newcommand{\hatman}{\psset{unit=4mm}\hatmanv}
\newcommand{\sadman}{\psset{unit=4mm}\sadmanv}
\newcommand{\woman}{\psset{unit=4mm}\womanv}
\newcommand{\spy}{\psset{unit=4mm}\spyv}
\newcommand{\puzzledspy}{\psset{unit=4mm}\puzzledspyv}


\newcommand{\coursetitle}{\vphantom{ }\vspace{7mm}\begin{center}
    {\large\darkgreen MATH5505 \:\: Combinatorics}\\[2.5mm]
    UNSW 2018S1
    \end{center}}

\newcounter{gra}

\DeclareMathAlphabet{\mathscr}{OT1}{pzc}%
                                 {m}{it}

\newcommand{\ph}{\phantom}
\newcommand{\ds}{\displaystyle}
\newcommand{\mra}{\black\rightsquigarrow\blue}
\newcommand{\tvs}{\textvisiblespace}
\newcommand{\msp}{\,}
\newcommand{\mba}{\,\mypicture{\psset{shadow=false}\psline(-.04,-.15)(-.04,.4)}}
\newcommand{\mbq}{\,\mypicture{\psset{shadow=false}\psline(-.04,-.15)(-.04,.4)\rput(-0.04,0.7){{\darkgray?}}}}
\newcommand{\meq}{\black=\blue}
\newcommand{\msim}{\black\sim\blue}
\newcommand{\mequiv}{\black\equiv\blue}
\newcommand{\mapprox}{\black\approx\blue}
\newcommand{\mneq}{\red\neq\blue}
\newcommand{\mleq}{\black\leq\blue}
\newcommand{\mgeq}{\black\geq\blue}
\newcommand{\mgt}{\black>\blue}
\newcommand{\mlt}{\black<\blue}
\newcommand{\mto}{\black\to\blue}
\newcommand{\msubseteq}{\black\subseteq\blue}
\newcommand{\mnsubseteq}{\red\not\subseteq\blue}
\newcommand{\mequ}{\black\equiv\blue}
\newcommand{\mnequ}{\red\not\equiv\blue}
\newcommand{\myin}{\black\in\blue}
\newcommand{\mypl}{\black+\blue}
\newcommand{\mdef}{\black:=\blue}
\newcommand{\mnotin}{\red\notin\blue}
\newcommand{\mynotin}{\red\notin\blue}
\newcommand{\mywhere}{\quad\text{\black where\blue}\quad}
\newcommand{\myand}{\quad\text{\black and\blue}\quad}
\newcommand{\dnd}{\black\mathchoice{\mathrel{{\kern0.1em|\kern-0.4em/}}}
  {\mathrel{{\kern0.1em|\kern-0.4em/}}}{\mathrel{{\kern0.1em|\kern-0.33em/}}}
  {\mathrel{{\kern0.1em|\kern-0.2em/}}}\blue}
\newcommand{\mdiv}{\black\,|\,\blue}
\newcommand{\mymod}[1]{\:(\textrm{mod}\,#1)}
\newcommand{\mask}[1]{}
\newcommand{\ord}{\textrm{ord}\,}
\newcommand{\ns}{\negthickspace\negthickspace}
\newcommand{\nns}{\negthickspace\negthickspace\negthickspace\negthickspace}
\newcommand{\lns}{\hspace*{-.3mm}}
\newcommand{\ri}{\,i\,}% \,\mathrm{i}}
\newcommand{\di}{\mbox{$\dg\bullet$}}
\newcommand{\dah}{\mbox{$\dg\mathbf{-}$}}
\newcommand{\tp}{\mbox{{\dg\tt p}}}
\newcommand{\Var}{\mathrm{Var}}
\newcommand{\Arg}{\mathrm{Arg}}
\renewcommand{\Re}{\mathrm{Re}}
\renewcommand{\Im}{\mathrm{Im}}
\newcommand{\bbN}{\blue\mathbb{N}}
\newcommand{\bbZ}{\blue\mathbb{Z}}
\newcommand{\bbQ}{\blue\mathbb{Q}}
\newcommand{\bbR}{\blue\mathbb{R}}
\newcommand{\bbC}{\blue\mathbb{C}}
\newcommand{\bbF}{\blue\mathbb{F}}
\newcommand{\bbP}{\blue\mathbb{P}}
%\newcommand{\calR}{\blue\mathcal{R}}
%\newcommand{\calC}{\blue\mathcal{C}}
\renewcommand{\vec}[1]{\mathbf{\blue #1}}
%\newcommand{\pv}[1]{{\blue\begin{pmatrix}#1\end{pmatrix}}}
%\newcommand{\pvn}[1]{{\begin{pmatrix}#1\end{pmatrix}}}
%\newcommand{\spv}[1]{{\blue\left(\begin{smallmatrix}#1\end{smallmatrix}\right)}}
%\newcommand{\mpv}[1]{{\blue\Biggl(\!\!\begin{array}{r}#1\end{array}\!\!\Biggr)}}
%\newcommand{\augmv}[1]{{\left(\begin{array}{rr|r}#1\end{array}\right)}}
%\newcommand{\taugmv}[1]{{\left(\begin{array}{rrr|r}#1\end{array}\right)}}
%\newcommand{\qaugmv}[1]{{\left(\begin{array}{rrrr|r}#1\end{array}\right)}}
%\newcommand{\augm}[1]{{\blue\left(\begin{array}{rr|r}#1\end{array}\right)}}
%\newcommand{\daugm}[1]{{\blue\left(\begin{array}{rr|rr}#1\end{array}\right)}}
%\newcommand{\taugm}[1]{{\blue\left(\begin{array}{rrr|r}#1\end{array}\right)}}
%\newcommand{\traugm}[1]{{\blue\left(\begin{array}{rrr|rrr}#1\end{array}\right)}}
%\newcommand{\qaugm}[1]{{\blue\left(\begin{array}{rrrr|r}#1\end{array}\right)}}
%\newcommand{\qtaugm}[1]{{\blue\left(\begin{array}{rrrr|rrr}#1\end{array}\right)}}
%\newcommand{\mydet}[1]{{\blue\left|\begin{matrix}#1\end{matrix}\right|}}
%\newcommand{\mydets}[1]{{\blue\Bigl|\begin{matrix}#1\end{matrix}\Bigr|}}
%\newcommand{\arr}[1]{\overrightarrow{#1}}
\newcommand{\lcm}{\mathrm{lcm}}
\renewcommand{\mod}{\;\mathrm{mod}\;}
%\newcommand{\proj}{\mathrm{proj}}
%\newcommand{\id}{\blue\mathrm{id}}
%\newcommand{\im}{\blue\mathrm{Im}}
%\newcommand{\re}{\blue\mathrm{Re}}
\newcommand{\col}{\blue\mathrm{col}}
\newcommand{\nullity}{\blue\mathrm{nullity}}
\newcommand{\rank}{\blue\mathrm{rank}}
\newcommand{\spn}[1]{\blue\mathrm{span}\left\{#1\right\}}
\newcommand{\spnv}{\blue\mathrm{span}\,}
%\newcommand{\diag}[1]{\blue\mathrm{diag}\left(#1\right)\,}
%\newcommand{\satop}[2]{\stackrel{\scriptstyle{#1}}{\scriptstyle{#2}}}
%\newcommand{\llnot}{\sim\!}
\newcommand{\qed}{\hfill$\Box$}
%\newcommand{\rbullet}{\includegraphics[width=4mm]{red-bullet-on-white.ps}}
%\newcommand{\gbullet}{\includegraphics[width=3mm]{green-bullet-on-white.ps}}
%\newcommand{\ybullet}{\includegraphics[width=3mm]{yellow-bullet-on-white.ps}}
\newcommand{\vsp}{{\psset{unit=4.2mm}\begin{pspicture}(0,1)(0,0)\end{pspicture}}}
\renewcommand{\emptyset}{\varnothing}
\newcommand{\mq}{\text{\red ?}}
%\renewcommand{\emph}[1]{{\blue\textsl{#1}}}

%\newcommand{\shat}{\begin{pspicture}(0,0)(0,0)\psset{linecolor=black,linewidth=.075}\psarc*(.67,3.18){.25}{45}{225}\psline(.35,2.85)(1,3.5)\end{pspicture}}
%\newcommand{\rhat}{\begin{pspicture}(0,0)(0,0)\psset{linecolor=red,linewidth=.05}\psarc*(.675,3.175){.25}{45}{225}\psline(.35,2.85)(1,3.5)\end{pspicture}}
%\newcommand{\bhat}{\begin{pspicture}(0,0)(0,0)\psset{linecolor=blue,linewidth=.05}\psarc*(.675,3.175){.25}{45}{225}\psline(.35,2.85)(1,3.5)\end{pspicture}}
%\newcommand{\sdress}{\begin{pspicture}(0,0)(0,0)\pstriangle*[linecolor=black,linewidth=.05](1,.5)(1.2,1.75)\end{pspicture}}
%\newcommand{\rdress}{\begin{pspicture}(0,0)(0,0)\pstriangle*[linecolor=red,linewidth=.05](1,.5)(1.2,1.75)\end{pspicture}}
%\newcommand{\bdress}{\begin{pspicture}(0,0)(0,0)\pstriangle*[linecolor=brown,linewidth=.05](1,.5)(1.2,1.75)\end{pspicture}}
%\newcommand{\ydress}{\begin{pspicture}(0,0)(0,0)\pstriangle*[linecolor=amber,linewidth=.05](1,.5)(1.2,1.75)\end{pspicture}}
%\newcommand{\gdress}{\begin{pspicture}(0,0)(0,0)\pstriangle*[linecolor=green,linewidth=.05](1,.5)(1.2,1.75)\end{pspicture}}
%\newcommand{\sshoes}{\begin{pspicture}(0,0)(0,0)\psset{linecolor=black,linewidth=.05}
% \psline(.6,0)(.85,0.125)\psline(0.85,0)(0.85,0.5)\psline(1.15,0)(1.15,0.5)\psline(1.4,0)(1.15,0.125)\end{pspicture}}
%\newcommand{\rshoes}{\begin{pspicture}(0,0)(0,0)\psset{linecolor=black,linewidth=.05}
%  \psline(0.85,0)(0.85,0.5)\pscircle*[linecolor=red](.65,.05){.05}\psframe*[linecolor=red](.65,0)(.90,.1)
%  \psline(1.15,0)(1.15,0.5)\psframe*[linecolor=red](1.1,0)(1.35,0.1)\pscircle*[linecolor=red](1.35,.05){.05}\end{pspicture}}
%
%\newcommand{\woman}[1]{\begin{pspicture}(0.25,0)(1.75,3.4)
% \psset{unit=4mm,dotsize=0.09,linewidth=.03,fillcolor=black,linecolor=black,shadow=false}
% \psline(1,2)(1,2.35)\psline(0.25,1.75)(1,2)\psline(1.75,1.75)(1,2)\pscircle(1,2.85){0.5}
% \psarc(1,2.85){.25}{-140}{-40}\psdots(.85,2.95)(1.15,2.95)
% \psset{linecolor=brown,linewidth=.1}
% \psarc(1,2.85){.5}{0}{180}\psarc(1.75,2.85){.25}{180}{270}\psarc(.25,2.85){.25}{-90}{0}
% \psset{linewidth=.06}#1
% \end{pspicture}}

\newcommand{\dicei}{{\psset{unit=6mm,shadow=false,linewidth=.05}\begin{pspicture}(0,0.3)(1,1.3)\begin{psframe}(0,0)(1,1)\qdisk(0.5,0.5){0.1\psunit}\end{psframe}\end{pspicture}}}
\newcommand{\diceii}{{\psset{unit=6mm,shadow=false,linewidth=.05}\begin{pspicture}(0,0.3)(1,1.3)\begin{psframe}(0,0)(1,1)\qdisk(0.2,0.2){0.1\psunit}\qdisk(0.8,0.8){0.1\psunit}\end{psframe}\end{pspicture}}}
\newcommand{\diceiii}{{\psset{unit=6mm,shadow=false,linewidth=.05}\begin{pspicture}(0,0.3)(1,1.3)\begin{psframe}(0,0)(1,1)\qdisk(0.2,0.2){0.1\psunit}\qdisk(0.5,0.5){0.1\psunit}\qdisk(0.8,0.8){0.1\psunit}\end{psframe}\end{pspicture}}}
\newcommand{\diceiv}{{\psset{unit=6mm,shadow=false,linewidth=.05}\begin{pspicture}(0,0.3)(1,1.3)\begin{psframe}(0,0)(1,1)\qdisk(0.2,0.2){0.1\psunit}\qdisk(0.2,0.8){0.1\psunit}\qdisk(0.8,0.8){0.1\psunit}\qdisk(0.8,0.2){0.1\psunit}\end{psframe}\end{pspicture}}}
\newcommand{\dicev}{{\psset{unit=6mm,shadow=false,linewidth=.05}\begin{pspicture}(0,0.3)(1,1.3)\begin{psframe}(0,0)(1,1)\qdisk(0.2,0.2){0.1\psunit}\qdisk(0.2,0.8){0.1\psunit}\qdisk(0.5,0.5){0.1\psunit}\qdisk(0.8,0.8){0.1\psunit}\qdisk(0.8,0.2){0.1\psunit}\end{psframe}\end{pspicture}}}
\newcommand{\dicevi}{{\psset{unit=6mm,shadow=false,linewidth=.05}\begin{pspicture}(0,0.3)(1,1.3)\begin{psframe}(0,0)(1,1)\qdisk(0.2,0.2){0.1\psunit}\qdisk(0.2,0.5){0.1\psunit}\qdisk(0.2,0.8){0.1\psunit}\qdisk(0.8,0.8){0.1\psunit}\qdisk(0.8,0.5){0.1\psunit}\qdisk(0.8,0.2){0.1\psunit}\end{psframe}\end{pspicture}}}
\newcommand{\diceframe}{{\psset{unit=6mm,shadow=false,linewidth=.05}\begin{pspicture}(0,0.3)(1,1.3)\begin{psframe}(-.25,-.25)(1.25,1.25)\end{psframe}\end{pspicture}}}

\newcommand{\pigeon}{{\psset{xunit=6mm,yunit=6mm,runit=6mm,linewidth=.8pt,shadow=false,linecolor=darkgray,fillcolor=white,fillstyle=solid}
  \begin{pspicture}(0,0)(1.9, 1){\pscustom{\newpath
    \moveto(1.179, 0.088)
    \curveto(1.156, 0.239)(1.402, 0.255)(1.451, 0.464)
    \curveto(1.498, 0.668)(1.42 , 0.774)(1.75 , 0.812)
    \curveto(1.656, 0.887)(1.528, 0.997)(1.399, 0.979)
    \curveto(1.067, 0.929)(1.269, 0.646)(0.156, 0.241)
    \curveto(0    , 0.185)(0.869, 0.396)(1.037, 0.205)
    \curveto(1.201, 0.016)(0.902, 0    )(1.17 , 0.003)
    \curveto(1.37 , 0.005)(1.197, 0.003)(1.179, 0.088)
    \closepath}}
    \pscircle(1.447, 0.9){0.045}
  \end{pspicture}}}


%\newcommand{\dicegrid}[5]{\[\begin{pspicture}(-1,-1.2)(20,8)\psset{shadow=false,linecolor=darkgray}
%    \rput(0  ,0  ){\dicevi} \rput(0  ,1.2){\dicev} \rput(0  ,2.4){\diceiv}\rput(0  ,3.6){\diceiii}
%    \rput(0  ,4.8){\diceii} \rput(0  ,6  ){\dicei} \rput(1.5,7.55){\dicei} \rput(2.7,7.55){\diceii}
%    \rput(3.9,7.55){\diceiii}\rput(5.1,7.55){\diceiv}\rput(6.3,7.55){\dicev} \rput(7.5,7.55){\dicevi}
%    \psset{linecolor=gray}\psline( .75,-1)( .75,8)\psline(-.75, 6.5)(8.25,6.5)\psline(-.75,-1)(-.75,8)
%    \psline(-.75,8)(8.25,8)\psline(-.75,-1)(8.25,-1)\psline(8.25,-1)(8.25,8)\psline(-.75, 8)(.75,6.5)
%    \put(-.55,6.6){1}\put(.25,7.25){2}
%    \psset{linecolor=blue}#1\psset{linecolor=red}#2\put(9.5,4.3){{#3}}\put(9.5,3.1){{#4}}\put(9.5,1.9){{#5}}
%  \end{pspicture}\]}
%
%\newcommand{\bnaa}{\pscircle*(1.5,5.7){3pt}}\newcommand{\bnab}{\pscircle*(2.7,5.7){3pt}}\newcommand{\bnac}{\pscircle*(3.9,5.7){3pt}}\newcommand{\bnad}{\pscircle*(5.1,5.7){3pt}}\newcommand{\bnae}{\pscircle*(6.3,5.7){3pt}}\newcommand{\bnaf}{\pscircle*(7.5,5.7){3pt}}
%\newcommand{\bnba}{\pscircle*(1.5,4.5){3pt}}\newcommand{\bnbb}{\pscircle*(2.7,4.5){3pt}}\newcommand{\bnbc}{\pscircle*(3.9,4.5){3pt}}\newcommand{\bnbd}{\pscircle*(5.1,4.5){3pt}}\newcommand{\bnbe}{\pscircle*(6.3,4.5){3pt}}\newcommand{\bnbf}{\pscircle*(7.5,4.5){3pt}}
%\newcommand{\bnca}{\pscircle*(1.5,3.3){3pt}}\newcommand{\bncb}{\pscircle*(2.7,3.3){3pt}}\newcommand{\bncc}{\pscircle*(3.9,3.3){3pt}}\newcommand{\bncd}{\pscircle*(5.1,3.3){3pt}}\newcommand{\bnce}{\pscircle*(6.3,3.3){3pt}}\newcommand{\bncf}{\pscircle*(7.5,3.3){3pt}}
%\newcommand{\bnda}{\pscircle*(1.5,2.1){3pt}}\newcommand{\bndb}{\pscircle*(2.7,2.1){3pt}}\newcommand{\bndc}{\pscircle*(3.9,2.1){3pt}}\newcommand{\bndd}{\pscircle*(5.1,2.1){3pt}}\newcommand{\bnde}{\pscircle*(6.3,2.1){3pt}}\newcommand{\bndf}{\pscircle*(7.5,2.1){3pt}}
%\newcommand{\bnea}{\pscircle*(1.5, .9){3pt}}\newcommand{\bneb}{\pscircle*(2.7, .9){3pt}}\newcommand{\bnec}{\pscircle*(3.9, .9){3pt}}\newcommand{\bned}{\pscircle*(5.1, .9){3pt}}\newcommand{\bnee}{\pscircle*(6.3, .9){3pt}}\newcommand{\bnef}{\pscircle*(7.5, .9){3pt}}
%\newcommand{\bnfa}{\pscircle*(1.5,-.3){3pt}}\newcommand{\bnfb}{\pscircle*(2.7,-.3){3pt}}\newcommand{\bnfc}{\pscircle*(3.9,-.3){3pt}}\newcommand{\bnfd}{\pscircle*(5.1,-.3){3pt}}\newcommand{\bnfe}{\pscircle*(6.3,-.3){3pt}}\newcommand{\bnff}{\pscircle*(7.5,-.3){3pt}}
%
%\newcommand{\cnaa}{\pscircle(1.5,5.7){4.5pt}}\newcommand{\cnab}{\pscircle(2.7,5.7){4.5pt}}\newcommand{\cnac}{\pscircle(3.9,5.7){4.5pt}}\newcommand{\cnad}{\pscircle(5.1,5.7){4.5pt}}\newcommand{\cnae}{\pscircle(6.3,5.7){4.5pt}}\newcommand{\cnaf}{\pscircle(7.5,5.7){4.5pt}}
%\newcommand{\cnba}{\pscircle(1.5,4.5){4.5pt}}\newcommand{\cnbb}{\pscircle(2.7,4.5){4.5pt}}\newcommand{\cnbc}{\pscircle(3.9,4.5){4.5pt}}\newcommand{\cnbd}{\pscircle(5.1,4.5){4.5pt}}\newcommand{\cnbe}{\pscircle(6.3,4.5){4.5pt}}\newcommand{\cnbf}{\pscircle(7.5,4.5){4.5pt}}
%\newcommand{\cnca}{\pscircle(1.5,3.3){4.5pt}}\newcommand{\cncb}{\pscircle(2.7,3.3){4.5pt}}\newcommand{\cncc}{\pscircle(3.9,3.3){4.5pt}}\newcommand{\cncd}{\pscircle(5.1,3.3){4.5pt}}\newcommand{\cnce}{\pscircle(6.3,3.3){4.5pt}}\newcommand{\cncf}{\pscircle(7.5,3.3){4.5pt}}
%\newcommand{\cnda}{\pscircle(1.5,2.1){4.5pt}}\newcommand{\cndb}{\pscircle(2.7,2.1){4.5pt}}\newcommand{\cndc}{\pscircle(3.9,2.1){4.5pt}}\newcommand{\cndd}{\pscircle(5.1,2.1){4.5pt}}\newcommand{\cnde}{\pscircle(6.3,2.1){4.5pt}}\newcommand{\cndf}{\pscircle(7.5,2.1){4.5pt}}
%\newcommand{\cnea}{\pscircle(1.5, .9){4.5pt}}\newcommand{\cneb}{\pscircle(2.7, .9){4.5pt}}\newcommand{\cnec}{\pscircle(3.9, .9){4.5pt}}\newcommand{\cned}{\pscircle(5.1, .9){4.5pt}}\newcommand{\cnee}{\pscircle(6.3, .9){4.5pt}}\newcommand{\cnef}{\pscircle(7.5, .9){4.5pt}}
%\newcommand{\cnfa}{\pscircle(1.5,-.3){4.5pt}}\newcommand{\cnfb}{\pscircle(2.7,-.3){4.5pt}}\newcommand{\cnfc}{\pscircle(3.9,-.3){4.5pt}}\newcommand{\cnfd}{\pscircle(5.1,-.3){4.5pt}}\newcommand{\cnfe}{\pscircle(6.3,-.3){4.5pt}}\newcommand{\cnff}{\pscircle(7.5,-.3){4.5pt}}
%
\newcommand{\clearemptydoublepage}
  {\newpage{\pagestyle{empty}{\cleardoublepage}}}


% ------------------------------------------------------------------------
%  Environments
% ------------------------------------------------------------------------

\newcommand{\example}{{\sffamily\darkgreen Example }}
\newcommand{\exercise}{{\sffamily\darkgreen Exercise }}
\newcommand{\proof}{{\sffamily\darkgreen Proof }}
\newcommand{\notes}{{\sffamily\darkgreen Notes }}
\newcommand{\note}{{\sffamily\darkgreen Note }}
\newcommand{\theorem}{{\sffamily\darkgreen Theorem }}
\newcommand{\proposition}{{\sffamily\darkgreen Proposition }}
\newcommand{\corollary}{{\sffamily\darkgreen Corollary }}
\newcommand{\lemma}{{\sffamily\darkgreen Lemma }}
\newcommand{\definition}{{\sffamily\darkgreen Definition}}
\newcommand{\problem}{{\sffamily\darkgreen Problem}}
\newcommand{\remark}{{\sffamily\darkgreen Remark}}

% ------------------------------------------------------------------------

\renewcommand{\section}
  {\clearemptydoublepage\refstepcounter{section} \secdef \cmda \cmdb}
\newcommand{\cmda}[2][]
  {{\scriptsize{\textbf{MATH5505 \:\: Combinatorics}}
   \hfill{\scriptsize{\textsl{Thomas Britz}}}\vspace{0.1cm}\\
   {\bfseries\Large\S\arabic{section} \sffamily #2}}}
\newcommand{\cmdb}[1]{{\bfseries\huge\S\,\sffamily #1}}

\pagestyle{empty}

\newcommand{\frontpage}{}



\begin{document}
%% Very important: sans-serif font throughout
\sf

\newcommand{\lecturetitle}{\vphantom{ }\vspace{15mm}\begin{center}
  {\large\sc Ramsey Theory}\end{center}}

\newcommand{\lecturetitlei}{\vphantom{ }\vspace{15mm}\begin{center}
  {\large\sc Ramsey Theory}\\[3mm]
  \darkgreen Lecture 5: Graphs and Geometry\end{center}}

\newcommand{\pigeonthmplusa}{{\dg The Pigeonhole Principle} {\gray (general)}
  \\[1mm]If $\bl km+1$ pigeons are put into $\bl k$ pigeonholes,\\
         then some pigeonhole contains at least $\bl m+1$ pigeons.}

\newcommand{\ramseyb}{{\dg Ramsey's Theorem} {\gray (1930)} {\halfgray (simple)}\\
   If $\bl k,m\myin\mathbb{N}$ and $\bl n$ is sufficiently large,
   then each $\bl k$-colouring of the edges of $\bl K_n$ gives a complete monochromatic subgraph $\bl K_m$.}

\newcommand{\ramseybipartitea}{\\[4mm]\theorem {\gray (Beineke \&\ Schwenk 1976)}\\
   If $\bl k,m\myin\mathbb{N}$ and $\bl n$ is sufficiently large,
   then each $\bl k$-colouring of the edges of $\bl K_{n,n}$ gives a complete monochromatic subgraph $\bl K_{m,m}$.}

\newcommand{\ramseybipartiteav}{\theorem {\gray (Beineke \&\ Schwenk 1976)}\\
   If $\bl k,m\myin\mathbb{N}$ and $\bl n$ is sufficiently large,
   then each $\bl k$-colouring of the edges of $\bl K_{n,n}$ gives a complete monochromatic subgraph $\bl K_{m,m}$.}

\newcommand{\ramseybipartiteavv}{\\[2mm]\theorem {\gray (Beineke \&\ Schwenk 1976)}\\
   If $\bl k,m\myin\mathbb{N}$ and $\bl n$ is sufficiently large,
   then each $\bl k$-colouring of the edges of $\bl K_{n,n}$ gives a complete monochromatic subgraph $\bl K_{m,m}$.}

\newcommand{\ramseysetintersecta}{\\[4mm]\theorem {\gray(Ramras 2002)}
  \\If $\bl k,m\myin\mathbb{N}$, $\bl k\mgeq 2$ and $\bl N$ is sufficiently large,
   then of any $\bl N$ $\bl N$-subsets $\bl X\msubseteq\bl[k(N-1)+1]$,
   some $\bl m$ sets have at least $\bl m$ elements in common.}

\newcommand{\ramseysetintersectb}{\\[2mm]{\gray
  Ramras proved the special case $\lightblue k\gray=\lightblue 2$; we have generalised his result here.
  \\As the proof shows,
  $\lightblue k(N-1)+1$ may be replaced by any linear function $\lightblue f(N)\gray:=\lightblue aN+b$
  where $\lightblue  a,b\gray\in\lightblue\mathbb{Z}$ and $\lightblue a\gray\geq\lightblue 1$,
  and $\lightblue b\gray\geq\lightblue1$ if $\lightblue a\gray=\lightblue 1$.
  \\The particular choice $\lightblue k(N-1)+1$ is required for the proof of the $\lightblue K_{n,n}$ theorem above.}}

\newcommand{\ramseysetintersectav}{\theorem  {\gray(Ramras 2002)}
  \\If $\bl k,m\myin\mathbb{N}$, $\bl k\mgeq 2$ and $\bl N$ is sufficiently large,
   then of any $\bl N$ $\bl N$-subsets $\bl X\msubseteq\bl[k(N-1)+1]$,
   some $\bl m$ sets have at least $\bl m$ elements in common.}

\newcommand{\ramseysetintersectavv}{\\[2mm]\theorem  {\gray(Ramras 2002)}
  \\If $\bl k,m\myin\mathbb{N}$, $\bl k\mgeq 2$ and $\bl N$ is sufficiently large,
   then of any $\bl N$ $\bl N$-subsets $\bl X\msubseteq\bl[k(N-1)+1]$,
   some $\bl m$ sets have at least $\bl m$ elements in common.}

\newcommand{\ramseysetintersectpfa}{\\[2mm]\proof}
\newcommand{\ramseysetintersectpfb}{\\Let $\bl m\myin\mathbb{N}$ and assume that the theorem is false. Set $\bl M\mdef k(N-1)+1$.}
\newcommand{\ramseysetintersectpfc}{\\Then there is
  a family $\bl\mathcal{X}$ of $\bl N$ $\bl N$-subsets $\bl X\myin\bl\binom{[M]}{N}$,
  \\each $\bl m$ sets of which have at most $\bl m-1$ elements in common.}
%\newcommand{\ramseysetintersectpfd}{\\Let $\bl\mathcal{Y}\mdef\binom{[M]}{m}$ be the family of all $\bl m$-subsets of $\bl[M]$.}
\newcommand{\ramseysetintersectpfe}[3]{\\Then
 \[\bl
        N\binom{N}{m}
   \meq \sum_{X\in\mathcal{X}}\sum_{Y\in\binom{X}{m}}\!\! 1\;
   {#1{\meq \hspace*{-1mm}\sum_{Y\in\binom{[M]}{m}}
             \sum_{\begin{smallmatrix}X\in\mathcal{X}\,:\\Y\subset X\phantom{\,:}\end{smallmatrix}}\!\!1}}\;
   {#2{\mleq\hspace*{-2mm}\sum_{Y\in\binom{[M]}{m}}\hspace*{-2mm} (m-1)}}\;
   {#3{\meq \textstyle\binom{M}{m} (m-1)\,\black.\hspace*{-8mm}}}\]}
\newcommand{\ramseysetintersectpff}{Now, $\bl N\binom{N}{m}$ is a polynomial in $\bl N$ of degree $\bl m+1$,
  \\whereas $\bl \binom{M}{m} (m\!-\!1)\meq \binom{k(N-1)+1}{m} (m\!-\!1)$ is a polynomial in $\bl N$ of degree~$\bl m$}
\newcommand{\ramseysetintersectpfg}{,
  \\so for sufficiently large $\bl N$, we have a contradiction.\qed}



\newcommand{\ramseybipartitepfa}{\\[2mm]\proof}
\newcommand{\ramseybipartitepfb}{\\Let $\bl N$ be sufficiently large as in the theorem, and set $\bl n\mdef k(N-1)+1$.}
\newcommand{\ramseybipartitepfc}{\\Colour the edges of $\bl K_{n,n}$ with $\bl k$ colours, and let $\bl A, B$ be the vertex parts.}
\newcommand{\ramseybipartitepfd}{\\Each vertex $\bl a\myin A$ is adjacent to $\bl n\meq k(N-1)+1$ edges.}
\newcommand{\ramseybipartitepfe}{\\By the {\dg Pigeonhole Principle}, $\bl a$ is incident to $\bl N$ edges with same colour~$\bl c(a)$.}
\newcommand{\ramseybipartitepff}{\\There are $\bl n$ vertices $\bl a\myin A$, each incident to $\bl N$ edges with colour $\bl c(a)$.}
\newcommand{\ramseybipartitepfg}{\\By the {\dg Pigeonhole Principle}, $\bl N$ of these $\bl n$ colours $\bl c(a)$ are the same,~say~$\bl c$.}
\newcommand{\ramseybipartitepfh}{\\So, some $\bl N$ vertices in $\bl A$ are each adjacent to $\bl N$ vertices in $\bl B$ by edges of colour~$\bl c$.}
\newcommand{\ramseybipartitepfj}{  By definition of $\bl N$, some $\bl m$ vertices in $\bl A$ are each adjacent to
  a common set of $\bl m$ vertices in~$\bl B$ via edges of colour~$\bl c$.}
\newcommand{\ramseybipartitepfk}{\\This gives us a $\bl c$-coloured subgraph $\bl K_{m,m}$.\qed\vspace*{-10mm}}

\newcommand{\ramseybipartitedensity}{\\[4mm]\theorem {\gray (Zarankiewicz 1951)}\\
   Let $\bl m\myin\mathbb{N}$ and $\bl \epsilon \mgt 0$.
   If $\bl n$ is sufficiently large and $\bl G$ is a subgraph of $\bl K_{n,n}$ with at least $\bl\epsilon n^2$ edges,
   then $\bl G$ has $\bl K_{m,m}$ as a subgraph.}

\newcommand{\graphramseyaa}[2]{For graphs $\bl G,H$,\\[3mm]
  \begin{tabular}{rl}
     {$\bl\chi(G)$}  &\hspace*{-2mm}{{is the least number of colours in a proper vertex colouring of $\bl G$}}\\ % {\gray(is the chromatic number of $\lightblue G$)}\\
  {#1{$\bl   c(H)$}} &\hspace*{-2mm}{#1{is the largest size of a connected component of $\bl H$}}\\
  {#2{$\bl r(G,H)$}} &\hspace*{-2mm}{#2{is the smallest $\bl n$ so that if $\bl E(K_n)$ is coloured {\red red} and {\bl blue},}}
                   \\&\hspace*{-2mm}{#2{then $\bl K_n$ either has a {\red red} $\red G$ or a {\bl blue} $\bl H$ as subgraph.}}
  \end{tabular}}

\newcommand{\graphramseyaexaa}{\\[4mm]\example}
\newcommand{\graphramseyaexab}[2]{\begin{center}%
   {\psset{fillcolor=halfgray,linecolor=gray,shadow=false,fillstyle=none,linewidth=.06}
    \pspicture(0,0)(10.5,4)\rput(0,0){%
      \pspicture(-1.5,-2)(1.5,2.6)
    \pnode(-1.5, 0  ){a}
    \pnode( 1.5, 0  ){b}
    \pnode( 0  , 2.6){c}
    \pspolygon(a)(b)(c)\psset{fillstyle=solid}
    \pscircle(a){.2}
    \pscircle(b){.2}
    \pscircle(c){.2}
    \rput[b](0,-1.5){$\bl G_1$}
    \rput[b](0,-3){$\bl  \chi(G_1)\meq 3$}
    \rput[b](0,-4){{#1{$\bl \,c(G_1)\meq 3$}}}
      \endpspicture}
    \rput(7,0){\pspicture(-1.5,-2)(1.5,2.6)
    \pnode(-1.5, 0  ){a}
    \pnode( 1.5, 0  ){b}
    \pnode( 0  , 2.6){c}
    \psline(a)(b)\psset{fillstyle=solid}
    \pscircle(a){.2}
    \pscircle(b){.2}
    \pscircle(c){.2}
    \rput[b](0,-1.5){$\bl G_2$}
    \rput[b](0,-3  ){$\bl \chi(G_2)\meq 2$}
    \rput[b](0,-4  ){{#1{$\bl \,c(G_2)\meq 2$}}}
     \endpspicture}
     \rput[l](12,0){{#2{$\bl r({\red\triangle},{\bl\triangle})\meq 6$}}}
    \endpspicture}\end{center}}

\newcommand{\graphramseyba}{\\[4mm]\theorem {\gray (Chv\'{a}tal \&\ Harary 1972)}\vspace*{-2mm}
  \[\bl
    r(G,H)\mgeq (c(G) - 1)(\chi(H) - 1) + 1
  \]}

\newcommand{\graphramseybav}{\theorem {\gray (Chv\'{a}tal \&\ Harary 1972)}\vspace*{-2mm}
  \[\bl
    r(G,H)\mgeq (c(G) - 1)(\chi(H) - 1) + 1
  \]}

\newcommand{\graphramseybexaa}{\\[4mm]\example}
\newcommand{\graphramseybexab}[1]{\[\bl r({\red\triangle},{\bl\triangle})
  \meq 6 \mgeq 5\meq (3 - 1)\times (3 - 1) + 1
  \,{#1{\meq (\chi({\red\triangle}) - 1)(c({\bl\triangle})-1) + 1}}\]}


\newcommand{\graphramseybpfa}{\\\proof}
\newcommand{\graphramseybpfb}{\\Set $\bl n\mdef (c(G) - 1)(\chi(H)-1)$ and consider $\bl K_n$.}
\newcommand{\graphramseybpfc}{\\We can find $\bl\chi(H)-1$ disjoint $\bl K_{c(G)-1}$ subgraphs of $\bl K_n$.}
\newcommand{\graphramseybpfd}{\\Colour the edges of these {\red red} and all other edges {\bl blue}.}
\newcommand{\graphramseybpfe}{\\Then $\bl K_n$ has no {\red red $G$} subgraph nor any {\bl blue $H$} subgraph.}
\newcommand{\graphramseybpff}{\\Hence, $\bl r(G,H)\mgeq n+1$.\qed}

\newcommand{\graphramseyca}{\theorem {\gray (Chv\'{a}tal 1977)}
  \\If $\bl T_m$ is a tree on $\bl m$ vertices, then\vspace*{-2mm}
  \[\bl
    r(T_m,K_n)\meq (m-1)(n-1) + 1\,\black.\vspace*{-2mm}
  \]}

\newcommand{\graphramseycexaa}{\example}
\newcommand{\graphramseycexab}{\\For $\bl m\meq 2$, it is clear that $\bl r(T_m,K_n)\meq n$, as claimed.}
\newcommand{\graphramseycexac}{\\Similarly for $\bl n\meq 2$, $\bl r(T_m,K_n)\meq m$, as claimed.}

\newcommand{\graphramseycpfa}{\proof}
\newcommand{\graphramseycpfb}[1]{\\By the \theorem above,\vspace*{-2mm}
  \[\bl r(T_m,K_n)\mgeq (c(T_m) - 1)(\chi(K_n) - 1) + 1\,{#1{\meq (m-1)(n-1)+1}}\,\black.\]}
\newcommand{\graphramseycpfbv}{\\By the \theorem above, $\bl r(T_m,K_n)\mgeq (m-1)(n-1)+1$.\\}
\newcommand{\graphramseycpfc}{We therefore wish to prove that $\bl r(T_m,K_n)\mleq (m-1)(n-1)+1$.}
\newcommand{\graphramseycpfd}{\\Assume that the theorem is true for all $\bl m',n'$ with $\bl m'+n'\mlt m+n$.}
\newcommand{\graphramseycpfe}{\\Now colour the edges of $\bl K_N$ {\red red} and {\blue blue} where $\bl N\mdef (m-1)(n-1)+1$.}
\newcommand{\graphramseycpff}{\\Let $\bl T_{m-1}$ be a tree obtained by removing an end-vertex from $\bl T_m$
  \\and let $\red v$ be the vertex that was adjacent to that removed vertex.}
\newcommand{\graphramseycpfg}{\\By the induction assumption, $\bl K_N$ contains either a {\blue blue} $\bl K_n$ subgraph, and we are done}
\newcommand{\graphramseycpfh}{, or it has a {\red red} $\red T_{m-1}$ subgraph; suppose the latter.}
\newcommand{\graphramseycpfj}{\\Let $\bl K_{N'}$ be the graph obtained by deleting the vertices of $\red T_{m-1}$ from $\bl K_N$.}
\newcommand{\graphramseycpfk}{\\Then $\bl K_{N'}$ has $\bl N-(m-1)\meq (m-1)(n-2)+1$ vertices.}
\newcommand{\graphramseycpfm}{\\By the induction assumption, $\bl K_{N'}$ has either a {\red red} $\red T_m$ subgraph in which case we are done,
  or it has a {\bl blue} $\bl K_{n-1}$ subgraph; suppose the latter.}
\newcommand{\graphramseycpfn}{\\{[...]}We have supposed that {\red red} $\red T_{m-1}$
  and {\bl blue} $\bl K_{n-1}$ are subgraphs of $\bl K_N$.}
\newcommand{\graphramseycpfp}{\\Now consider the edges from the vertex $\red v$ in $\red T_{m-1}$ to the vertices of $\bl K_{n-1}$.}
\newcommand{\graphramseycpfq}{\\If one of these edges is red, then add it to $\red v$ to get a {\red red} $\red T_m$ subgraph.}
\newcommand{\graphramseycpfr}{\\Otherwise, all the edges are blue, so add them to $\bl K_{n-1}$ to get a {\bl blue} $\bl K_n$.}
\newcommand{\graphramseycpfs}{\\Thus, $\bl K_N$ has either a {\red red} $\red T_m$ subgraph or a {\bl blue} $\bl K_n$ subgraph.}
\newcommand{\graphramseycpft}{\\Hence, $\bl r(T_m,K_n)\mleq (m-1)(n-1)+1$.\qed\vspace*{-2mm}}




\coursetitle
\np\lecturetitle
\np\lecturetitlei
\np\pigeonthmplusa
\np\ramseyb
\np\ramseyb\ramseybipartitea
\np\ramseyb\ramseybipartitea\ramseysetintersecta
\np\ramseyb\ramseybipartitea\ramseysetintersecta\ramseysetintersectb
\np\ramseysetintersectav
\np\ramseysetintersectav\ramseysetintersectpfa
\np\ramseysetintersectav\ramseysetintersectpfa\ramseysetintersectpfb
\np\ramseysetintersectav\ramseysetintersectpfa\ramseysetintersectpfb\ramseysetintersectpfc
\np\ramseysetintersectav\ramseysetintersectpfa\ramseysetintersectpfb\ramseysetintersectpfc\ramseysetintersectpfe{\ph}{\ph}{\ph}
\np\ramseysetintersectav\ramseysetintersectpfa\ramseysetintersectpfb\ramseysetintersectpfc\ramseysetintersectpfe{}{\ph}{\ph}
\np\ramseysetintersectav\ramseysetintersectpfa\ramseysetintersectpfb\ramseysetintersectpfc\ramseysetintersectpfe{}{}{\ph}
\np\ramseysetintersectav\ramseysetintersectpfa\ramseysetintersectpfb\ramseysetintersectpfc\ramseysetintersectpfe{}{}{}
\np\ramseysetintersectav\ramseysetintersectpfa\ramseysetintersectpfb\ramseysetintersectpfc\ramseysetintersectpfe{}{}{}\ramseysetintersectpff
\np\ramseysetintersectav\ramseysetintersectpfa\ramseysetintersectpfb\ramseysetintersectpfc\ramseysetintersectpfe{}{}{}\ramseysetintersectpff\ramseysetintersectpfg
\np\ramseysetintersectav
\np\ramseysetintersectav\ramseybipartitea
\np\ramseysetintersectav\ramseybipartitea\ramseybipartitepfa
\np\ramseysetintersectav\ramseybipartitea\ramseybipartitepfa\ramseybipartitepfb
\np\ramseysetintersectav\ramseybipartitea\ramseybipartitepfa\ramseybipartitepfb\ramseybipartitepfc
\np\ramseysetintersectav\ramseybipartitea\ramseybipartitepfa\ramseybipartitepfb\ramseybipartitepfc\ramseybipartitepfd
\np\ramseysetintersectav\ramseybipartitea\ramseybipartitepfa\ramseybipartitepfb\ramseybipartitepfc\ramseybipartitepfd\ramseybipartitepfe
\np\ramseysetintersectav\ramseybipartitea\ramseybipartitepfa\ramseybipartitepfb\ramseybipartitepfc\ramseybipartitepfd\ramseybipartitepfe\ramseybipartitepff
\np\ramseysetintersectav\ramseybipartitea\ramseybipartitepfa\ramseybipartitepfb\ramseybipartitepfc\ramseybipartitepfd\ramseybipartitepfe\ramseybipartitepff\ramseybipartitepfg
\np\ramseysetintersectav\ramseybipartitea\ramseybipartitepfa\ramseybipartitepfb\ramseybipartitepfc\ramseybipartitepfd\ramseybipartitepfe\ramseybipartitepff\ramseybipartitepfg\ramseybipartitepfh
\np\ramseysetintersectav\ramseybipartiteavv\ramseybipartitepfa\ramseybipartitepfb\ramseybipartitepfc\ramseybipartitepfd\ramseybipartitepfe\ramseybipartitepff\ramseybipartitepfg\ramseybipartitepfh\ramseybipartitepfj
\np\ramseysetintersectav\ramseybipartiteavv\ramseybipartitepfa\ramseybipartitepfb\ramseybipartitepfc\ramseybipartitepfd\ramseybipartitepfe\ramseybipartitepff\ramseybipartitepfg\ramseybipartitepfh\ramseybipartitepfj\ramseybipartitepfk
\np\ramseybipartiteav
\np\ramseybipartiteav\ramseybipartitedensity
\np\graphramseyaa{\ph}{\ph}\graphramseyaexab{\ph}{\ph}
\np\graphramseyaa{}{\ph}\graphramseyaexab{}{\ph}
\np\graphramseyaa{}{}\graphramseyaexab{}{}
\np\graphramseyaa{}{}
\np\graphramseyaa{}{}\graphramseyba
\np\graphramseyaa{}{}\graphramseyba\graphramseybexaa
\np\graphramseyaa{}{}\graphramseyba\graphramseybexaa\graphramseybexab{\ph}
\np\graphramseyaa{}{}\graphramseyba\graphramseybexaa\graphramseybexab{}
\np\graphramseyaa{}{}\graphramseyba
\np\graphramseyaa{}{}\graphramseyba\graphramseybpfa
\np\graphramseyaa{}{}\graphramseyba\graphramseybpfa\graphramseybpfb
\np\graphramseyaa{}{}\graphramseyba\graphramseybpfa\graphramseybpfb\graphramseybpfc
\np\graphramseyaa{}{}\graphramseyba\graphramseybpfa\graphramseybpfb\graphramseybpfc\graphramseybpfd
\np\graphramseyaa{}{}\graphramseyba\graphramseybpfa\graphramseybpfb\graphramseybpfc\graphramseybpfd\graphramseybpfe
\np\graphramseyaa{}{}\graphramseyba\graphramseybpfa\graphramseybpfb\graphramseybpfc\graphramseybpfd\graphramseybpfe\graphramseybpff
\np\graphramseyaa{}{}\graphramseyba
\np\graphramseyaa{}{}\graphramseyba\graphramseyca
\np\graphramseyaa{}{}\graphramseyba\graphramseyca\graphramseycexaa
\np\graphramseyaa{}{}\graphramseyba\graphramseyca\graphramseycexaa\graphramseycexab
\np\graphramseyaa{}{}\graphramseyba\graphramseyca\graphramseycexaa\graphramseycexab\graphramseycexac
\np\graphramseyaa{}{}\graphramseyba\graphramseyca
\np\graphramseyaa{}{}\graphramseyba\graphramseyca\graphramseycpfa
\np\graphramseyaa{}{}\graphramseyba\graphramseyca\graphramseycpfa\graphramseycpfb{\ph}
\np\graphramseyaa{}{}\graphramseyba\graphramseyca\graphramseycpfa\graphramseycpfb{}
\np\graphramseyca\graphramseycpfa\graphramseycpfbv
\np\graphramseyca\graphramseycpfa\graphramseycpfbv\graphramseycpfc
\np\graphramseyca\graphramseycpfa\graphramseycpfbv\graphramseycpfc\graphramseycpfd
\np\graphramseyca\graphramseycpfa\graphramseycpfbv\graphramseycpfc\graphramseycpfd\graphramseycpfe
\np\graphramseyca\graphramseycpfa\graphramseycpfbv\graphramseycpfc\graphramseycpfd\graphramseycpfe\graphramseycpff
\np\graphramseyca\graphramseycpfa\graphramseycpfbv\graphramseycpfc\graphramseycpfd\graphramseycpfe\graphramseycpff\graphramseycpfg
\np\graphramseyca\graphramseycpfa\graphramseycpfbv\graphramseycpfc\graphramseycpfd\graphramseycpfe\graphramseycpff\graphramseycpfg\graphramseycpfh
\np\graphramseyca\graphramseycpfa\graphramseycpfbv\graphramseycpfc\graphramseycpfd\graphramseycpfe\graphramseycpff\graphramseycpfg\graphramseycpfh\graphramseycpfj
\np\graphramseyca\graphramseycpfa\graphramseycpfbv\graphramseycpfc\graphramseycpfd\graphramseycpfe\graphramseycpff\graphramseycpfg\graphramseycpfh\graphramseycpfj\graphramseycpfk
\np\graphramseyca\graphramseycpfa\graphramseycpfbv\graphramseycpfc\graphramseycpfd\graphramseycpfe\graphramseycpff\graphramseycpfg\graphramseycpfh\graphramseycpfj\graphramseycpfk\graphramseycpfm
\np\graphramseyca\graphramseycpfa\graphramseycpfbv\graphramseycpfc\graphramseycpfd\graphramseycpfe\graphramseycpff\graphramseycpfn
\np\graphramseyca\graphramseycpfa\graphramseycpfbv\graphramseycpfc\graphramseycpfd\graphramseycpfe\graphramseycpff\graphramseycpfn\graphramseycpfp
\np\graphramseyca\graphramseycpfa\graphramseycpfbv\graphramseycpfc\graphramseycpfd\graphramseycpfe\graphramseycpff\graphramseycpfn\graphramseycpfp\graphramseycpfq
\np\graphramseyca\graphramseycpfa\graphramseycpfbv\graphramseycpfc\graphramseycpfd\graphramseycpfe\graphramseycpff\graphramseycpfn\graphramseycpfp\graphramseycpfq\graphramseycpfr
\np\graphramseyca\graphramseycpfa\graphramseycpfbv\graphramseycpfc\graphramseycpfd\graphramseycpfe\graphramseycpff\graphramseycpfn\graphramseycpfp\graphramseycpfq\graphramseycpfr\graphramseycpfs
\np\graphramseyca\graphramseycpfa\graphramseycpfbv\graphramseycpfc\graphramseycpfd\graphramseycpfe\graphramseycpff\graphramseycpfn\graphramseycpfp\graphramseycpfq\graphramseycpfr\graphramseycpfs\graphramseycpft

\newcommand{\ramseye}{{\dg Ramsey's Theorem} {\gray (1930)}\\
   If $\bl n_1,\ldots,n_k,r\myin\mathbb{N}$ and $\bl n$ is sufficiently large,\\
   then each colouring of $\bl \binom{[n]}{r}$ with colours $\bl c_1,\ldots,c_k$\\
   gives a $\bl c_i$-coloured subfamily $\bl \binom{S}{r}$
   for some $\bl i$ and $\bl n_i$-subset $\bl S\subseteq [n]$.}

\newcommand{\affineramsey}{\\[4mm]{\dg The Affine Ramsey Theorem} {\gray (Spencer 1975)}
  \\Let $\bl\mathbb{F}$ be a finite field.
  If $\bl k,r,t\myin\mathbb{N}$ and $\bl n$ is sufficiently large,
  then each \\$\bl r$-colouring of the $\bl t$-dimensional affine subspaces of $\bl\mathbb{F}^n$
  gives a $\bl k$-dimensional affine subspace of $\bl\mathbb{F}^n$
  whose $\bl t$-dimensional affine subspaces have the same colour.}

\newcommand{\vectorramsey}{\\[4mm]{\dg Vector Space Ramsey Theorem} {\gray (Graham, Leeb \&\ Rothschild 1972)}
  \\Let $\bl\mathbb{F}$ be a finite field.
  If $\bl k,r,t\myin\mathbb{N}$ and $\bl n$ is sufficiently large,
   then each \\$\bl r$-colouring of the $\bl t$-dimensional subspaces of $\bl\mathbb{F}^n$
   gives a $\bl k$-dimensional subspace of $\bl\mathbb{F}^n$
   whose $\bl t$-dimensional subspaces have the same colour.}

\newcommand{\euclidramseya}{{\dg The Gallai-Witt Theorem}
  \\For each finite subset $\bl V\msubseteq \mathbb{R}^n$ and each finite colouring of $\bl\mathbb{R}^n$,
  \\there is a monochromatic subset $\bl W\msubseteq \mathbb{R}^n$ that
  can be obtained from $\bl V$ by {\dg translating} and {\dg scaling}.}

\newcommand{\euclidramseyb}{\\[4mm]When {\dg scaling} is not allowed, then it is easy to find
  finite sets $\bl V\msubseteq\mathbb{R}^n$ and colourings of $\bl\mathbb{R}^n$
  so that no {\dg translation} of $\bl V$ is monochromatic.}

\newcommand{\euclidramseyc}{\\[4mm]A finite set of points $\bl V$ is {\dg\em Ramsey} if
  for each $\bl r\myin\mathbb{N}$ there is a sufficiently large $\bl n$ such that any $\bl r$-colouring
  of $\bl\mathbb{R}^n$ has a monochromatic subset $\bl W$ that is {\dg congruent} to $\bl V$,
  i.e., {\dg translating}, {\dg rotating}, and {\dg reflecting} is allowed.}

\newcommand{\euclidramseyd}{\\[4mm]\theorem
  \\\mybullet All rectangular parallelopipeds (``bricks'') are Ramsey.
  \\\mybullet In particular, all equilateral simplexes are Ramsey.
  \\\mybullet Indeed, all regular polygons and polyhedra are Ramsey.
  \\\mybullet Each Ramsey set can be placed on some sphere.}


\np\ramseye
\np\ramseye\affineramsey
\np\ramseye\affineramsey\vectorramsey
\np\euclidramseya
\np\euclidramseya\euclidramseyb
\np\euclidramseya\euclidramseyb\euclidramseyc
\np\euclidramseya\euclidramseyb\euclidramseyc\euclidramseyd



\end{document}

