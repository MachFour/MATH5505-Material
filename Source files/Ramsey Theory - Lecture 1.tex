%%  MATH5505 Ramsey Theory Lecture 1: Introduction
%%
%%  by Thomas Britz 2018S1
%%

\documentclass[12pt,a4paper,landscape]{article}
\special{landscape}

\usepackage{latexsym,amsfonts,amsmath,amssymb} %,calc,fancybox,epsfig,amscd,tabularx}
\usepackage{pstricks,pst-plot,pst-node,pst-tree}
%\usepackage{array,graphicx,epsfig}
%\usepackage{multicol,multirow,hyperref,rotating}
%\usepackage[T1]{fontenc}
%\usepackage{yfonts}


\mag=\magstep4

\newrgbcolor{green}{0.2 0.6 0.2}
\newrgbcolor{darkgreen}{0 0.4 0}
\newrgbcolor{graydarkgreen}{0.5 0.67 0.5}
\newrgbcolor{lightgreen}{0 0.75 0}
\newrgbcolor{skyblue}{0.5 0.80 1}
\newrgbcolor{darkblue}{0 0 0.6}
\newrgbcolor{darkred}{0.6 0 0}
\newrgbcolor{halfred}{.8 0 0}
\newrgbcolor{white}{1 1 1}
\newrgbcolor{nearlywhite}{0.95 0.95 0.95}
\newrgbcolor{offwhite}{0.9 0.9 0.9}
\newrgbcolor{lightgray}{0.85 0.85 0.85}
\newrgbcolor{halfgray}{0.8 0.8 0.8}
\newrgbcolor{altgray}{0.67 0.67 0.67}
\newrgbcolor{darkyellow}{1 0.94 0.15}
\newrgbcolor{halflightyellow}{1 1 0.4}
\newrgbcolor{lightyellow}{1 1 0.7}
\newrgbcolor{gold}{0.96 0.96 0.1}
\newrgbcolor{lightblue}{0.5 0.5 1}
\newrgbcolor{amber}{1 0.75 0}
\newrgbcolor{hotpink}{1 0.41 0.71}

\setlength{\textheight}{18.0truecm}
\setlength{\textwidth}{26.0truecm}
\setlength{\hoffset}{-12.0truecm}
\setlength{\voffset}{-5.2truecm}

\parindent 0in

%\setlength{\bigskipamount}{5ex plus1.5ex minus 2ex}
%\setlength{\parindent}{0cm}
%\setlength{\parskip}{0.2cm}

\psset{unit=6mm,linewidth=.06,dotscale=1.5,fillcolor=white,fillstyle=none,
 linecolor=gray,framearc=.3,shadowcolor=offwhite,shadow=true,shadowsize=.125,
 shadowangle=-45,dash=7pt 5pt}
%\psset{unit=.4,linecolor=gray,fillcolor=offwhite,shadowsize=.2,framearc=.3}
%\psset{unit=10mm,linewidth=.03}

\def\dedge{\ncline[linestyle=dashed]}

\def\np{\newpage}
\def\bl{\blue}
\def\bk{\black}
\def\wh{\white}
\def\rd{\red}
\def\lg{\lightgray}
\def\gr{\green}
\def\dr{\darkred}
\def\dg{\darkgreen}
\def\gdg{\graydarkgreen}
\def\dgy{\darkgray}
\def\ec{\dg}

\newcommand{\ora}[1]{\overrightarrow{#1}}

\newcommand{\llb}{\\[1mm]}
\newcommand{\lb}{\\[3mm]}
\newcommand{\blb}{\\[5mm]}
\newcommand{\hlb}{\\[48mm]}

\newcommand{\mynewpage}{\newpage\vspace*{-10mm}}

\newcommand{\vc}[1]{\begin{pmatrix}#1\end{pmatrix}}

\newcommand{\mytext}[1]{\text{\black#1\blue}}

\newcommand{\qbinom}[2]{\genfrac{[}{]}{0pt}{}{#1}{#2}}

\newcommand{\dl}{\psset{linestyle=dashed,linecolor=altgray,shadow=false}}
%\newcommand{\dl}{\psset{linestyle=dashed,linecolor=blue}}

\newcommand{\myframe}[1]{\pspicture(0,0)(0,0)\psset{unit=1cm,
 shadowcolor=offwhite,shadow=true,shadowangle=-45,linewidth=.03,linecolor=gray,
 fillcolor=lightgray,shadowsize=.15,framearc=.3}\psframe#1\endpspicture}

\newcommand{\mypicture}[1]{\pspicture(0,0)(0,0)\psset{unit=1cm,
 shadowcolor=offwhite,shadow=true,shadowangle=-45,linewidth=.03,linecolor=gray,
 fillcolor=lightgray,shadowsize=.15,framearc=.3}#1\endpspicture}

\newcommand{\mymatrix}[1]{{\psset{unit=4mm,linewidth=.03,linecolor=gray,fillstyle=solid,fillcolor=offwhite,shadow=false,framearc=0}\pspicture(0,0)(6,4)
  \psframe(0,0)(7,5)#1\psframe[linecolor=gray,fillcolor=offwhite,linewidth=.03,fillstyle=none,framearc=0](0,0)(7,5)\endpspicture}}

\newcommand{\mypair}[2]{{\psset{unit=4mm,linewidth=.03,linecolor=gray,fillstyle=solid,fillcolor=offwhite,shadow=false,framearc=0}\pspicture(0,.2)(2,.8)
  \psframe(0,0)(2,1)\darkgray\rput(.5,.5){#1}\rput(1.5,.5){#2}\psframe[linecolor=gray,fillcolor=offwhite,linewidth=.03,fillstyle=none,framearc=0](0,0)(2,1)\endpspicture}}

\newcommand{\mysixtuple}[6]{{\psset{unit=4mm,linewidth=.03,linecolor=gray,fillstyle=solid,fillcolor=offwhite,shadow=false,framearc=0}\pspicture(0,.2)(6,.8)
  \psframe(0,0)(6,1)\darkgray\rput(.5,.5){#1}\rput(1.5,.5){#2}\rput(2.5,.5){#3}\rput(3.5,.5){#4}\rput(4.5,.5){#5}\rput(5.5,.5){#6}
  \psframe[linecolor=gray,fillcolor=offwhite,linewidth=.03,fillstyle=none,framearc=0](0,0)(6,1)\endpspicture}}

\newcommand{\acset}{\psset{fillstyle=none,linecolor=red}}
\newcommand{\drr}{\psset{linecolor=darkred,fillcolor=red,linewidth=.034}}
\newcommand{\dbb}{\psset{linecolor=darkblue,fillcolor=blue,linewidth=.034}}
\newcommand{\bbl}{\psset{linecolor=blue,fillcolor=lightblue}}
\newcommand{\dbhg}{\psset{linecolor=darkblue,fillcolor=halfgray}}
\newcommand{\bhg}{\psset{linecolor=blue,fillcolor=halfgray}}
\newcommand{\rhg}{\psset{linecolor=red,fillcolor=halfgray}}
\newcommand{\ghg}{\psset{linecolor=green,fillcolor=halfgray}}
\newcommand{\dgg}{\psset{linecolor=darkgreen,fillcolor=green,linewidth=.034}}

\newcommand{\mychain}{\pspicture(0,0)(0,0)\dbb
  \psset{shadowsize=.2,shadow=true,shadowcolor=offwhite,shadowangle=-45,shadowsize=.15}
  \pscircle(0,0){.09}\pscircle(0,1){.09}\pscircle(0,2){.09}\pscircle(0,3){.09}
  \psline(0,.125)(0,.875)\psline(0,1.125)(0,1.875)\psline(0,2.125)(0,2.875)\endpspicture}

\newcommand{\myantichain}{\pspicture(0,0)(0,0)\drr
  \psset{shadowsize=.2,shadow=true,shadowcolor=offwhite,shadowangle=-45,shadowsize=.15}
  \pscircle(0,0){.09}\pscircle(1,0){.09}\pscircle(2,0){.09}\pscircle(3,0){.09}\endpspicture}

\newcommand{\mybullet}{{\psset{unit=6mm,dotscale=1.5,linewidth=.05,fillcolor=black,linecolor=black,framearc=.3,shadow=true}
  \pspicture(-.3,0)(.8,.4)\qdisk(.25,.2){.1}\pscircle*[linecolor=gray](.25,.2){.07}\endpspicture}}

\newcommand{\mysmallbullet}{{\psset{unit=4mm,dotscale=1.5,linewidth=.05,fillcolor=black,linecolor=black,framearc=.3,shadow=true}
  \pspicture(-.3,0)(.8,.4)\qdisk(.25,.2){.1}\pscircle*[linecolor=gray](.25,.2){.07}\endpspicture}}

\newcommand{\mybulletv}{{\psset{unit=6mm,dotscale=1.5,linewidth=.04,linecolor=black,fillstyle=solid,
  fillcolor=gray,shadow=false}\pspicture(0,0)(.5,.4)\pscircle(.25,.2){.15}\endpspicture}}

\newcommand{\mystar}{\pspicture(0,0)(0,0)\psset{unit=1cm}\gold\large$\star$\endpspicture
  \pspicture(0,0)(0,0)\psset{unit=1cm}\rput(-.23,.193){\small\white$\star$}\endpspicture}

\newcommand{\mystep}{{\psset{unit=6mm,dotscale=1.5,linewidth=.06,linecolor=darkgreen,fillstyle=solid,
  fillcolor=lightgreen,shadow=false}\pspicture(-.3,0)(.8,.4)\pscircle(.25,.2){.15}\endpspicture}}
\newcommand{\mystepv}{{\psset{unit=6mm,dotscale=1.5,linewidth=.06,linecolor=darkgreen,fillstyle=solid,
  fillcolor=lightgreen,shadow=false}\pspicture(0,0)(.5,.4)\pscircle(.25,.2){.15}\endpspicture}}
\newcommand{\mystepvv}{{\psset{unit=6mm,dotscale=1.5,linewidth=.06,linecolor=darkred,fillstyle=solid,
  fillcolor=red,shadow=false}\pspicture(0,0)(.5,.4)\pscircle(.25,.2){.15}\endpspicture}}
\newcommand{\mystepvvv}{{\psset{unit=6mm,dotscale=1.5,linewidth=.06,linecolor=darkred,fillstyle=solid,
  fillcolor=red,shadow=false}\pspicture(0,0)(.5,.4)\endpspicture}}

\newcommand{\myi}{{\psset{unit=6mm,dotscale=1.5,linewidth=.05,linecolor=darkgreen,fillstyle=none,shadow=false}%
  \pspicture(-.3,0)(.8,.4)\pscircle(.25,.2){.25}\rput(.25,.2){\tiny\dg1}\endpspicture}}
\newcommand{\myii}{{\psset{unit=6mm,dotscale=1.5,linewidth=.05,linecolor=darkgreen,fillstyle=none,shadow=false}%
  \pspicture(-.3,0)(.8,.4)\pscircle(.25,.2){.25}\rput(.25,.2){\tiny\dg2}\endpspicture}}
\newcommand{\myiii}{{\psset{unit=6mm,dotscale=1.5,linewidth=.05,linecolor=darkgreen,fillstyle=none,shadow=false}%
  \pspicture(-.3,0)(.8,.4)\pscircle(.25,.2){.25}\rput(.25,.2){\tiny\dg3}\endpspicture}}
\newcommand{\myiv}{{\psset{unit=6mm,dotscale=1.5,linewidth=.05,linecolor=darkgreen,fillstyle=none,shadow=false}%
  \pspicture(-.3,0)(.8,.4)\pscircle(.25,.2){.25}\rput(.25,.2){\tiny\dg4}\endpspicture}}
\newcommand{\myv}{{\psset{unit=6mm,dotscale=1.5,linewidth=.05,linecolor=darkgreen,fillstyle=none,shadow=false}%
  \pspicture(-.3,0)(.8,.4)\pscircle(.25,.2){.25}\rput(.25,.2){\tiny\dg5}\endpspicture}}
\newcommand{\myvi}{{\psset{unit=6mm,dotscale=1.5,linewidth=.05,linecolor=darkgreen,fillstyle=none,shadow=false}%
  \pspicture(-.3,0)(.8,.4)\pscircle(.25,.2){.25}\rput(.25,.2){\tiny\dg6}\endpspicture}}
\newcommand{\myvii}{{\psset{unit=6mm,dotscale=1.5,linewidth=.05,linecolor=darkgreen,fillstyle=none,shadow=false}%
  \pspicture(-.3,0)(.8,.4)\pscircle(.25,.2){.25}\rput(.25,.2){\tiny\dg7}\endpspicture}}
\newcommand{\myviii}{{\psset{unit=6mm,dotscale=1.5,linewidth=.05,linecolor=darkgreen,fillstyle=none,shadow=false}%
  \pspicture(-.3,0)(.8,.4)\pscircle(.25,.2){.25}\rput(.25,.2){\tiny\dg8}\endpspicture}}
\newcommand{\myix}{{\psset{unit=6mm,dotscale=1.5,linewidth=.05,linecolor=darkgreen,fillstyle=none,shadow=false}%
  \pspicture(-.3,0)(.8,.4)\pscircle(.25,.2){.25}\rput(.25,.2){\tiny\dg9}\endpspicture}}

\newcommand{\mywo}{{\psset{unit=6mm,dotscale=1.5,linewidth=.05,linecolor=darkgreen,fillcolor=white,fillstyle=solid,shadow=false}%
  \pspicture(-.3,0)(.8,.4)\pscircle(.25,.2){.25}\rput(.25,.2){\tiny\dg0}\endpspicture}}
\newcommand{\mywi}{{\psset{unit=6mm,dotscale=1.5,linewidth=.05,linecolor=darkgreen,fillcolor=white,fillstyle=solid,shadow=false}%
  \pspicture(-.3,0)(.8,.4)\pscircle(.25,.2){.25}\rput(.25,.2){\tiny\dg1}\endpspicture}}
\newcommand{\mywii}{{\psset{unit=6mm,dotscale=1.5,linewidth=.05,linecolor=darkgreen,fillcolor=white,fillstyle=solid,shadow=false}%
  \pspicture(-.3,0)(.8,.4)\pscircle(.25,.2){.25}\rput(.25,.2){\tiny\dg2}\endpspicture}}

\newcommand{\myO}{{\psset{unit=6mm,dotscale=1.5,linewidth=.06,linecolor=darkgreen,fillstyle=none,
  shadow=false}\pspicture(0,0)(0,0)\pscircle(-.2,.225){.4}\endpspicture}}

\newcommand{\myspace}{\psset{unit=6mm,dotscale=1.5,linewidth=.06}\pspicture(-.3,0)(.8,.4)\endpspicture}

\newcommand{\mydot}{\pscircle(0,0){.1}}
\newcommand{\mydotv}{\pscircle*[linecolor=gray](0,0){.1}\pscircle*[linecolor=blue](0,0){.06}}

\newcommand{\myhexagon}[2]{\psset{fillcolor=halfgray,linecolor=gray,shadow=false,fillstyle=none,linewidth=.06}
  \pspicture(-3.5,-2.6)(3,2.6)
    \pnode(-3  , 0  ){a}
    \pnode(-1.5, 2.6){b}
    \pnode( 1.5, 2.6){c}
    \pnode( 3  , 0  ){d}
    \pnode( 1.5,-2.6){e}
    \pnode(-1.5,-2.6){f}
    \pspolygon(a)(b)(c)(d)(e)(f)#1\psset{fillstyle=solid}
    \pscircle(a){.2}
    \pscircle(b){.2}
    \pscircle(c){.2}
    \pscircle(d){.2}
    \pscircle(e){.2}
    \pscircle(f){.2}#2
  \endpspicture}

\newcommand{\myksix}[2]{\myhexagon{\pspolygon(a)(c)(e)\pspolygon(b)(d)(f)\psline(a)(d)\psline(b)(e)\psline(c)(f)#1}{#2}}

\newcommand{\lss}{$\spadesuit$}
\newcommand{\lhs}{$\darkred\heartsuit$}
\newcommand{\lds}{$\darkred\diamondsuit$}
\newcommand{\lcs}{$\clubsuit$}

\newcommand{\hatmanv}{{%
 \psset{dotscale=1.5,dotsize=0.09,linewidth=.03,fillcolor=black,linecolor=black,shadow=false}
 \psline(0,0)(0.25,0)\psline(1.75,0)(2,0)\psline(0.25,0)(1,1.2)
 \psline(1.75,0)(1,1.2)\psline(1,1.2)(1,2.5)\psline(0.25,1.75)(1,2)
 \psline(1.75,1.75)(1,2)\pscircle(1,3){0.5}\psarc(1,3){.25}{-140}{-40}
 \psdots(.85,3.15)(1.15,3.15)\psset{fillstyle=solid}\psframe(.5,3.35)(1.5,3.45)
 \psframe(.7,3.35)(1.3,4.1)}}

\newcommand{\manv}{{%
 \psset{dotscale=1.5,dotsize=0.09,linewidth=.03,fillcolor=black,linecolor=black,shadow=false}%
 \pspicture(0,0)(2,3.5)%\psline(0,0)(0.25,0)\psline(1.75,0)(2,0)
 \psline(0.25,0)(1,1.2)
 \psline(1.75,0)(1,1.2)\psline(1,1.2)(1,2.5)\psline(0.25,1.75)(1,2)
 \psline(1.75,1.75)(1,2)\pscircle(1,3){0.5}\psarc(1,3){.25}{-140}{-40}
 \psdots(.85,3.15)(1.15,3.15)%\psset{fillstyle=solid}\psframe(.5,3.35)(1.5,3.45)\psframe(.7,3.35)(1.3,4.1)
 \endpspicture}}

\newcommand{\sadmanv}{{
 \psset{dotscale=1.5,dotsize=0.09,linewidth=.03,fillcolor=black,linecolor=black,shadow=false}
 \psline(0,0)(0.25,0)\psline(1.75,0)(2,0)\psline(0.25,0)(1,1.2)
 \psline(1.75,0)(1,1.2)\psline(1,1.2)(1,2.5)\psline(0.25,1.75)(1,2)
 \psline(1.75,1.75)(1,2)\pscircle(1,3){0.5}\psarc(1,2.65){.25}{40}{140}
 \psdots(.87,3.15)(1.13,3.15)\psset{fillstyle=solid}\psframe(.5,3.35)(1.5,3.45)
 \psframe(.7,3.35)(1.3,4.1)}}

\newcommand{\womanv}{{
 \psset{dotscale=1.5,dotsize=0.09,linewidth=.03,fillcolor=black,linecolor=black,shadow=false}
 \pspicture(0,0)(2,3.5)%
 \psline(0.5,0)(0.75,0)\psline(1.5,0)(1.25,0)\psline(0.75,0)(0.75,0.5)
 \psline(1.25,0)(1.25,0.5)\psline(1,2)(1,2.35)\psline(0.25,1.75)(1,2)
 \psline(1.75,1.75)(1,2)\pscircle(1,2.85){0.5}
 \psarc(1,2.85){.25}{-140}{-40}\psdots(.85,2.95)(1.15,2.95)
 \psset{linecolor=brown,linewidth=.1}
 \psarc(1,2.85){.5}{0}{180}\psarc(1.75,2.85){.25}{180}{270}
 \psarc(0.25,2.85){.25}{-90}{0}
 \pstriangle[linecolor=red,fillstyle=solid,fillcolor=red](1,.5)(1.2,1.75)\endpspicture}}

\newcommand{\spyv}{{%
 \psset{dotscale=1.5,dotsize=0.09,linewidth=.03,fillcolor=black,linecolor=black,shadow=false}
 \pspicture(0,0)(2,3.5)%\psline(0,0)(0.25,0)\psline(1.75,0)(2,0)
 \psline(0.25,0)(1,1.2)
 \psline(1.75,0)(1,1.2)\psline(1,1.2)(1,2.5)\psline(0.25,1.75)(1,2)
 \psline(1.75,1.75)(1,2)\pscircle(1,3){0.5}\psarc(1,3){.25}{-140}{-40}
 %\psdots(.85,3.15)(1.15,3.15)
 \psset{fillstyle=solid}
 \psframe( .725,3.075)( .975,3.225)
 \psframe(1.025,3.075)(1.275,3.225)
 \psellipse*(1,3.5)(.5,.14)
 \pspolygon(.6,3.5)(1.4,3.5)( .8,3.8)
 \pspolygon(.6,3.5)(1.4,3.5)(1.2,3.8)
 \endpspicture}}

\newcommand{\puzzledspyv}{{%
 \psset{dotscale=1.5,dotsize=0.09,linewidth=.03,fillcolor=black,linecolor=black,shadow=false}
 \pspicture(0,0)(2,3.5)%\psline(0,0)(0.25,0)\psline(1.75,0)(2,0)
 \psline(0.25,0)(1,1.2)
 \psline(1.75,0)(1,1.2)\psline(1,1.2)(1,2.5)\psline(0.25,1.75)(1,2)
 \psline(1.75,1.75)(1,2)\pscircle(1,3){0.5}
 % Mouth
 \psline(.83,2.8)(1.17,2.8)
 %\psdots(.85,3.15)(1.15,3.15)
 \psset{fillstyle=solid}
 \psframe( .725,3.075)( .975,3.225)
 \psframe(1.025,3.075)(1.275,3.225)
 \psellipse*(1,3.5)(.5,.14)
 \pspolygon(.6,3.5)(1.4,3.5)( .8,3.8)
 \pspolygon(.6,3.5)(1.4,3.5)(1.2,3.8)
 \endpspicture}}

\newcommand{\man}{\psset{unit=4mm}\manv}
\newcommand{\hatman}{\psset{unit=4mm}\hatmanv}
\newcommand{\sadman}{\psset{unit=4mm}\sadmanv}
\newcommand{\woman}{\psset{unit=4mm}\womanv}
\newcommand{\spy}{\psset{unit=4mm}\spyv}
\newcommand{\puzzledspy}{\psset{unit=4mm}\puzzledspyv}


\newcommand{\coursetitle}{\vphantom{ }\vspace{7mm}\begin{center}
    {\large\darkgreen MATH5505 \:\: Combinatorics}\\[2.5mm]
    UNSW 2018S1
    \end{center}}

\newcounter{gra}

\DeclareMathAlphabet{\mathscr}{OT1}{pzc}%
                                 {m}{it}

\newcommand{\ph}{\phantom}
\newcommand{\ds}{\displaystyle}
\newcommand{\mra}{\black\rightsquigarrow\blue}
\newcommand{\tvs}{\textvisiblespace}
\newcommand{\msp}{\,}
\newcommand{\mba}{\,\mypicture{\psset{shadow=false}\psline(-.04,-.15)(-.04,.4)}}
\newcommand{\mbq}{\,\mypicture{\psset{shadow=false}\psline(-.04,-.15)(-.04,.4)\rput(-0.04,0.7){{\darkgray?}}}}
\newcommand{\meq}{\black=\blue}
\newcommand{\msim}{\black\sim\blue}
\newcommand{\mequiv}{\black\equiv\blue}
\newcommand{\mapprox}{\black\approx\blue}
\newcommand{\mneq}{\red\neq\blue}
\newcommand{\mleq}{\black\leq\blue}
\newcommand{\mgeq}{\black\geq\blue}
\newcommand{\mgt}{\black>\blue}
\newcommand{\mlt}{\black<\blue}
\newcommand{\mto}{\black\to\blue}
\newcommand{\msubseteq}{\black\subseteq\blue}
\newcommand{\mnsubseteq}{\red\not\subseteq\blue}
\newcommand{\mequ}{\black\equiv\blue}
\newcommand{\mnequ}{\red\not\equiv\blue}
\newcommand{\myin}{\black\in\blue}
\newcommand{\mypl}{\black+\blue}
\newcommand{\mdef}{\black:=\blue}
\newcommand{\mnotin}{\red\notin\blue}
\newcommand{\mynotin}{\red\notin\blue}
\newcommand{\mywhere}{\quad\text{\black where\blue}\quad}
\newcommand{\myand}{\quad\text{\black and\blue}\quad}
\newcommand{\dnd}{\black\mathchoice{\mathrel{{\kern0.1em|\kern-0.4em/}}}
  {\mathrel{{\kern0.1em|\kern-0.4em/}}}{\mathrel{{\kern0.1em|\kern-0.33em/}}}
  {\mathrel{{\kern0.1em|\kern-0.2em/}}}\blue}
\newcommand{\mdiv}{\black\,|\,\blue}
\newcommand{\mymod}[1]{\:(\textrm{mod}\,#1)}
\newcommand{\mask}[1]{}
\newcommand{\ord}{\textrm{ord}\,}
\newcommand{\ns}{\negthickspace\negthickspace}
\newcommand{\nns}{\negthickspace\negthickspace\negthickspace\negthickspace}
\newcommand{\lns}{\hspace*{-.3mm}}
\newcommand{\ri}{\,i\,}% \,\mathrm{i}}
\newcommand{\di}{\mbox{$\dg\bullet$}}
\newcommand{\dah}{\mbox{$\dg\mathbf{-}$}}
\newcommand{\tp}{\mbox{{\dg\tt p}}}
\newcommand{\Var}{\mathrm{Var}}
\newcommand{\Arg}{\mathrm{Arg}}
\renewcommand{\Re}{\mathrm{Re}}
\renewcommand{\Im}{\mathrm{Im}}
\newcommand{\bbN}{\blue\mathbb{N}}
\newcommand{\bbZ}{\blue\mathbb{Z}}
\newcommand{\bbQ}{\blue\mathbb{Q}}
\newcommand{\bbR}{\blue\mathbb{R}}
\newcommand{\bbC}{\blue\mathbb{C}}
\newcommand{\bbF}{\blue\mathbb{F}}
\newcommand{\bbP}{\blue\mathbb{P}}
%\newcommand{\calR}{\blue\mathcal{R}}
%\newcommand{\calC}{\blue\mathcal{C}}
\renewcommand{\vec}[1]{\mathbf{\blue #1}}
%\newcommand{\pv}[1]{{\blue\begin{pmatrix}#1\end{pmatrix}}}
%\newcommand{\pvn}[1]{{\begin{pmatrix}#1\end{pmatrix}}}
%\newcommand{\spv}[1]{{\blue\left(\begin{smallmatrix}#1\end{smallmatrix}\right)}}
%\newcommand{\mpv}[1]{{\blue\Biggl(\!\!\begin{array}{r}#1\end{array}\!\!\Biggr)}}
%\newcommand{\augmv}[1]{{\left(\begin{array}{rr|r}#1\end{array}\right)}}
%\newcommand{\taugmv}[1]{{\left(\begin{array}{rrr|r}#1\end{array}\right)}}
%\newcommand{\qaugmv}[1]{{\left(\begin{array}{rrrr|r}#1\end{array}\right)}}
%\newcommand{\augm}[1]{{\blue\left(\begin{array}{rr|r}#1\end{array}\right)}}
%\newcommand{\daugm}[1]{{\blue\left(\begin{array}{rr|rr}#1\end{array}\right)}}
%\newcommand{\taugm}[1]{{\blue\left(\begin{array}{rrr|r}#1\end{array}\right)}}
%\newcommand{\traugm}[1]{{\blue\left(\begin{array}{rrr|rrr}#1\end{array}\right)}}
%\newcommand{\qaugm}[1]{{\blue\left(\begin{array}{rrrr|r}#1\end{array}\right)}}
%\newcommand{\qtaugm}[1]{{\blue\left(\begin{array}{rrrr|rrr}#1\end{array}\right)}}
%\newcommand{\mydet}[1]{{\blue\left|\begin{matrix}#1\end{matrix}\right|}}
%\newcommand{\mydets}[1]{{\blue\Bigl|\begin{matrix}#1\end{matrix}\Bigr|}}
%\newcommand{\arr}[1]{\overrightarrow{#1}}
\newcommand{\lcm}{\mathrm{lcm}}
\renewcommand{\mod}{\;\mathrm{mod}\;}
%\newcommand{\proj}{\mathrm{proj}}
%\newcommand{\id}{\blue\mathrm{id}}
%\newcommand{\im}{\blue\mathrm{Im}}
%\newcommand{\re}{\blue\mathrm{Re}}
\newcommand{\col}{\blue\mathrm{col}}
\newcommand{\nullity}{\blue\mathrm{nullity}}
\newcommand{\rank}{\blue\mathrm{rank}}
\newcommand{\spn}[1]{\blue\mathrm{span}\left\{#1\right\}}
\newcommand{\spnv}{\blue\mathrm{span}\,}
%\newcommand{\diag}[1]{\blue\mathrm{diag}\left(#1\right)\,}
%\newcommand{\satop}[2]{\stackrel{\scriptstyle{#1}}{\scriptstyle{#2}}}
%\newcommand{\llnot}{\sim\!}
\newcommand{\qed}{\hfill$\Box$}
%\newcommand{\rbullet}{\includegraphics[width=4mm]{red-bullet-on-white.ps}}
%\newcommand{\gbullet}{\includegraphics[width=3mm]{green-bullet-on-white.ps}}
%\newcommand{\ybullet}{\includegraphics[width=3mm]{yellow-bullet-on-white.ps}}
\newcommand{\vsp}{{\psset{unit=4.2mm}\begin{pspicture}(0,1)(0,0)\end{pspicture}}}
\renewcommand{\emptyset}{\varnothing}
\newcommand{\mq}{\text{\red ?}}
%\renewcommand{\emph}[1]{{\blue\textsl{#1}}}

%\newcommand{\shat}{\begin{pspicture}(0,0)(0,0)\psset{linecolor=black,linewidth=.075}\psarc*(.67,3.18){.25}{45}{225}\psline(.35,2.85)(1,3.5)\end{pspicture}}
%\newcommand{\rhat}{\begin{pspicture}(0,0)(0,0)\psset{linecolor=red,linewidth=.05}\psarc*(.675,3.175){.25}{45}{225}\psline(.35,2.85)(1,3.5)\end{pspicture}}
%\newcommand{\bhat}{\begin{pspicture}(0,0)(0,0)\psset{linecolor=blue,linewidth=.05}\psarc*(.675,3.175){.25}{45}{225}\psline(.35,2.85)(1,3.5)\end{pspicture}}
%\newcommand{\sdress}{\begin{pspicture}(0,0)(0,0)\pstriangle*[linecolor=black,linewidth=.05](1,.5)(1.2,1.75)\end{pspicture}}
%\newcommand{\rdress}{\begin{pspicture}(0,0)(0,0)\pstriangle*[linecolor=red,linewidth=.05](1,.5)(1.2,1.75)\end{pspicture}}
%\newcommand{\bdress}{\begin{pspicture}(0,0)(0,0)\pstriangle*[linecolor=brown,linewidth=.05](1,.5)(1.2,1.75)\end{pspicture}}
%\newcommand{\ydress}{\begin{pspicture}(0,0)(0,0)\pstriangle*[linecolor=amber,linewidth=.05](1,.5)(1.2,1.75)\end{pspicture}}
%\newcommand{\gdress}{\begin{pspicture}(0,0)(0,0)\pstriangle*[linecolor=green,linewidth=.05](1,.5)(1.2,1.75)\end{pspicture}}
%\newcommand{\sshoes}{\begin{pspicture}(0,0)(0,0)\psset{linecolor=black,linewidth=.05}
% \psline(.6,0)(.85,0.125)\psline(0.85,0)(0.85,0.5)\psline(1.15,0)(1.15,0.5)\psline(1.4,0)(1.15,0.125)\end{pspicture}}
%\newcommand{\rshoes}{\begin{pspicture}(0,0)(0,0)\psset{linecolor=black,linewidth=.05}
%  \psline(0.85,0)(0.85,0.5)\pscircle*[linecolor=red](.65,.05){.05}\psframe*[linecolor=red](.65,0)(.90,.1)
%  \psline(1.15,0)(1.15,0.5)\psframe*[linecolor=red](1.1,0)(1.35,0.1)\pscircle*[linecolor=red](1.35,.05){.05}\end{pspicture}}
%
%\newcommand{\woman}[1]{\begin{pspicture}(0.25,0)(1.75,3.4)
% \psset{unit=4mm,dotsize=0.09,linewidth=.03,fillcolor=black,linecolor=black,shadow=false}
% \psline(1,2)(1,2.35)\psline(0.25,1.75)(1,2)\psline(1.75,1.75)(1,2)\pscircle(1,2.85){0.5}
% \psarc(1,2.85){.25}{-140}{-40}\psdots(.85,2.95)(1.15,2.95)
% \psset{linecolor=brown,linewidth=.1}
% \psarc(1,2.85){.5}{0}{180}\psarc(1.75,2.85){.25}{180}{270}\psarc(.25,2.85){.25}{-90}{0}
% \psset{linewidth=.06}#1
% \end{pspicture}}

\newcommand{\dicei}{{\psset{unit=6mm,shadow=false,linewidth=.05}\begin{pspicture}(0,0.3)(1,1.3)\begin{psframe}(0,0)(1,1)\qdisk(0.5,0.5){0.1\psunit}\end{psframe}\end{pspicture}}}
\newcommand{\diceii}{{\psset{unit=6mm,shadow=false,linewidth=.05}\begin{pspicture}(0,0.3)(1,1.3)\begin{psframe}(0,0)(1,1)\qdisk(0.2,0.2){0.1\psunit}\qdisk(0.8,0.8){0.1\psunit}\end{psframe}\end{pspicture}}}
\newcommand{\diceiii}{{\psset{unit=6mm,shadow=false,linewidth=.05}\begin{pspicture}(0,0.3)(1,1.3)\begin{psframe}(0,0)(1,1)\qdisk(0.2,0.2){0.1\psunit}\qdisk(0.5,0.5){0.1\psunit}\qdisk(0.8,0.8){0.1\psunit}\end{psframe}\end{pspicture}}}
\newcommand{\diceiv}{{\psset{unit=6mm,shadow=false,linewidth=.05}\begin{pspicture}(0,0.3)(1,1.3)\begin{psframe}(0,0)(1,1)\qdisk(0.2,0.2){0.1\psunit}\qdisk(0.2,0.8){0.1\psunit}\qdisk(0.8,0.8){0.1\psunit}\qdisk(0.8,0.2){0.1\psunit}\end{psframe}\end{pspicture}}}
\newcommand{\dicev}{{\psset{unit=6mm,shadow=false,linewidth=.05}\begin{pspicture}(0,0.3)(1,1.3)\begin{psframe}(0,0)(1,1)\qdisk(0.2,0.2){0.1\psunit}\qdisk(0.2,0.8){0.1\psunit}\qdisk(0.5,0.5){0.1\psunit}\qdisk(0.8,0.8){0.1\psunit}\qdisk(0.8,0.2){0.1\psunit}\end{psframe}\end{pspicture}}}
\newcommand{\dicevi}{{\psset{unit=6mm,shadow=false,linewidth=.05}\begin{pspicture}(0,0.3)(1,1.3)\begin{psframe}(0,0)(1,1)\qdisk(0.2,0.2){0.1\psunit}\qdisk(0.2,0.5){0.1\psunit}\qdisk(0.2,0.8){0.1\psunit}\qdisk(0.8,0.8){0.1\psunit}\qdisk(0.8,0.5){0.1\psunit}\qdisk(0.8,0.2){0.1\psunit}\end{psframe}\end{pspicture}}}
\newcommand{\diceframe}{{\psset{unit=6mm,shadow=false,linewidth=.05}\begin{pspicture}(0,0.3)(1,1.3)\begin{psframe}(-.25,-.25)(1.25,1.25)\end{psframe}\end{pspicture}}}

\newcommand{\pigeon}{{\psset{xunit=6mm,yunit=6mm,runit=6mm,linewidth=.8pt,shadow=false,linecolor=darkgray,fillcolor=white,fillstyle=solid}
  \begin{pspicture}(0,0)(1.9, 1){\pscustom{\newpath
    \moveto(1.179, 0.088)
    \curveto(1.156, 0.239)(1.402, 0.255)(1.451, 0.464)
    \curveto(1.498, 0.668)(1.42 , 0.774)(1.75 , 0.812)
    \curveto(1.656, 0.887)(1.528, 0.997)(1.399, 0.979)
    \curveto(1.067, 0.929)(1.269, 0.646)(0.156, 0.241)
    \curveto(0    , 0.185)(0.869, 0.396)(1.037, 0.205)
    \curveto(1.201, 0.016)(0.902, 0    )(1.17 , 0.003)
    \curveto(1.37 , 0.005)(1.197, 0.003)(1.179, 0.088)
    \closepath}}
    \pscircle(1.447, 0.9){0.045}
  \end{pspicture}}}


%\newcommand{\dicegrid}[5]{\[\begin{pspicture}(-1,-1.2)(20,8)\psset{shadow=false,linecolor=darkgray}
%    \rput(0  ,0  ){\dicevi} \rput(0  ,1.2){\dicev} \rput(0  ,2.4){\diceiv}\rput(0  ,3.6){\diceiii}
%    \rput(0  ,4.8){\diceii} \rput(0  ,6  ){\dicei} \rput(1.5,7.55){\dicei} \rput(2.7,7.55){\diceii}
%    \rput(3.9,7.55){\diceiii}\rput(5.1,7.55){\diceiv}\rput(6.3,7.55){\dicev} \rput(7.5,7.55){\dicevi}
%    \psset{linecolor=gray}\psline( .75,-1)( .75,8)\psline(-.75, 6.5)(8.25,6.5)\psline(-.75,-1)(-.75,8)
%    \psline(-.75,8)(8.25,8)\psline(-.75,-1)(8.25,-1)\psline(8.25,-1)(8.25,8)\psline(-.75, 8)(.75,6.5)
%    \put(-.55,6.6){1}\put(.25,7.25){2}
%    \psset{linecolor=blue}#1\psset{linecolor=red}#2\put(9.5,4.3){{#3}}\put(9.5,3.1){{#4}}\put(9.5,1.9){{#5}}
%  \end{pspicture}\]}
%
%\newcommand{\bnaa}{\pscircle*(1.5,5.7){3pt}}\newcommand{\bnab}{\pscircle*(2.7,5.7){3pt}}\newcommand{\bnac}{\pscircle*(3.9,5.7){3pt}}\newcommand{\bnad}{\pscircle*(5.1,5.7){3pt}}\newcommand{\bnae}{\pscircle*(6.3,5.7){3pt}}\newcommand{\bnaf}{\pscircle*(7.5,5.7){3pt}}
%\newcommand{\bnba}{\pscircle*(1.5,4.5){3pt}}\newcommand{\bnbb}{\pscircle*(2.7,4.5){3pt}}\newcommand{\bnbc}{\pscircle*(3.9,4.5){3pt}}\newcommand{\bnbd}{\pscircle*(5.1,4.5){3pt}}\newcommand{\bnbe}{\pscircle*(6.3,4.5){3pt}}\newcommand{\bnbf}{\pscircle*(7.5,4.5){3pt}}
%\newcommand{\bnca}{\pscircle*(1.5,3.3){3pt}}\newcommand{\bncb}{\pscircle*(2.7,3.3){3pt}}\newcommand{\bncc}{\pscircle*(3.9,3.3){3pt}}\newcommand{\bncd}{\pscircle*(5.1,3.3){3pt}}\newcommand{\bnce}{\pscircle*(6.3,3.3){3pt}}\newcommand{\bncf}{\pscircle*(7.5,3.3){3pt}}
%\newcommand{\bnda}{\pscircle*(1.5,2.1){3pt}}\newcommand{\bndb}{\pscircle*(2.7,2.1){3pt}}\newcommand{\bndc}{\pscircle*(3.9,2.1){3pt}}\newcommand{\bndd}{\pscircle*(5.1,2.1){3pt}}\newcommand{\bnde}{\pscircle*(6.3,2.1){3pt}}\newcommand{\bndf}{\pscircle*(7.5,2.1){3pt}}
%\newcommand{\bnea}{\pscircle*(1.5, .9){3pt}}\newcommand{\bneb}{\pscircle*(2.7, .9){3pt}}\newcommand{\bnec}{\pscircle*(3.9, .9){3pt}}\newcommand{\bned}{\pscircle*(5.1, .9){3pt}}\newcommand{\bnee}{\pscircle*(6.3, .9){3pt}}\newcommand{\bnef}{\pscircle*(7.5, .9){3pt}}
%\newcommand{\bnfa}{\pscircle*(1.5,-.3){3pt}}\newcommand{\bnfb}{\pscircle*(2.7,-.3){3pt}}\newcommand{\bnfc}{\pscircle*(3.9,-.3){3pt}}\newcommand{\bnfd}{\pscircle*(5.1,-.3){3pt}}\newcommand{\bnfe}{\pscircle*(6.3,-.3){3pt}}\newcommand{\bnff}{\pscircle*(7.5,-.3){3pt}}
%
%\newcommand{\cnaa}{\pscircle(1.5,5.7){4.5pt}}\newcommand{\cnab}{\pscircle(2.7,5.7){4.5pt}}\newcommand{\cnac}{\pscircle(3.9,5.7){4.5pt}}\newcommand{\cnad}{\pscircle(5.1,5.7){4.5pt}}\newcommand{\cnae}{\pscircle(6.3,5.7){4.5pt}}\newcommand{\cnaf}{\pscircle(7.5,5.7){4.5pt}}
%\newcommand{\cnba}{\pscircle(1.5,4.5){4.5pt}}\newcommand{\cnbb}{\pscircle(2.7,4.5){4.5pt}}\newcommand{\cnbc}{\pscircle(3.9,4.5){4.5pt}}\newcommand{\cnbd}{\pscircle(5.1,4.5){4.5pt}}\newcommand{\cnbe}{\pscircle(6.3,4.5){4.5pt}}\newcommand{\cnbf}{\pscircle(7.5,4.5){4.5pt}}
%\newcommand{\cnca}{\pscircle(1.5,3.3){4.5pt}}\newcommand{\cncb}{\pscircle(2.7,3.3){4.5pt}}\newcommand{\cncc}{\pscircle(3.9,3.3){4.5pt}}\newcommand{\cncd}{\pscircle(5.1,3.3){4.5pt}}\newcommand{\cnce}{\pscircle(6.3,3.3){4.5pt}}\newcommand{\cncf}{\pscircle(7.5,3.3){4.5pt}}
%\newcommand{\cnda}{\pscircle(1.5,2.1){4.5pt}}\newcommand{\cndb}{\pscircle(2.7,2.1){4.5pt}}\newcommand{\cndc}{\pscircle(3.9,2.1){4.5pt}}\newcommand{\cndd}{\pscircle(5.1,2.1){4.5pt}}\newcommand{\cnde}{\pscircle(6.3,2.1){4.5pt}}\newcommand{\cndf}{\pscircle(7.5,2.1){4.5pt}}
%\newcommand{\cnea}{\pscircle(1.5, .9){4.5pt}}\newcommand{\cneb}{\pscircle(2.7, .9){4.5pt}}\newcommand{\cnec}{\pscircle(3.9, .9){4.5pt}}\newcommand{\cned}{\pscircle(5.1, .9){4.5pt}}\newcommand{\cnee}{\pscircle(6.3, .9){4.5pt}}\newcommand{\cnef}{\pscircle(7.5, .9){4.5pt}}
%\newcommand{\cnfa}{\pscircle(1.5,-.3){4.5pt}}\newcommand{\cnfb}{\pscircle(2.7,-.3){4.5pt}}\newcommand{\cnfc}{\pscircle(3.9,-.3){4.5pt}}\newcommand{\cnfd}{\pscircle(5.1,-.3){4.5pt}}\newcommand{\cnfe}{\pscircle(6.3,-.3){4.5pt}}\newcommand{\cnff}{\pscircle(7.5,-.3){4.5pt}}
%
\newcommand{\clearemptydoublepage}
  {\newpage{\pagestyle{empty}{\cleardoublepage}}}


% ------------------------------------------------------------------------
%  Environments
% ------------------------------------------------------------------------

\newcommand{\example}{{\sffamily\darkgreen Example }}
\newcommand{\exercise}{{\sffamily\darkgreen Exercise }}
\newcommand{\proof}{{\sffamily\darkgreen Proof }}
\newcommand{\notes}{{\sffamily\darkgreen Notes }}
\newcommand{\note}{{\sffamily\darkgreen Note }}
\newcommand{\theorem}{{\sffamily\darkgreen Theorem }}
\newcommand{\proposition}{{\sffamily\darkgreen Proposition }}
\newcommand{\corollary}{{\sffamily\darkgreen Corollary }}
\newcommand{\lemma}{{\sffamily\darkgreen Lemma }}
\newcommand{\definition}{{\sffamily\darkgreen Definition}}
\newcommand{\problem}{{\sffamily\darkgreen Problem}}
\newcommand{\remark}{{\sffamily\darkgreen Remark}}

% ------------------------------------------------------------------------

\renewcommand{\section}
  {\clearemptydoublepage\refstepcounter{section} \secdef \cmda \cmdb}
\newcommand{\cmda}[2][]
  {{\scriptsize{\textbf{MATH5505 \:\: Combinatorics}}
   \hfill{\scriptsize{\textsl{Thomas Britz}}}\vspace{0.1cm}\\
   {\bfseries\Large\S\arabic{section} \sffamily #2}}}
\newcommand{\cmdb}[1]{{\bfseries\huge\S\,\sffamily #1}}

\pagestyle{empty}

\newcommand{\frontpage}{}


\begin{document}
%% Very important: sans-serif font throughout
\sf

\newcommand{\coursetitlei}{\vspace{5mm}\begin{center}{\Huge\green \itshape Welcome\,!}\end{center}}

\newcommand{\formal}{\vphantom{ }\vspace{10mm}{\ec\large Formalities}}
\newcommand{\formala}{\\[4mm]\myspace  Course authority: {\ec Thomas Britz}}
\newcommand{\formalb}{\\[1mm]\myspace  Email: {\ec\tt britz@unsw.edu.au}}
\newcommand{\formalc}{\\[1mm]\myspace  Please email me if you have any questions or comments!}
\newcommand{\formald}{\\[1mm]\myspace  Office hours: Thursday 11-12 or just drop by! (Room RC-5111)}
\newcommand{\formale}{\\[1mm]\myspace  Check {\ec Moodle} for {\ec course notes}, {\ec lecture slides}, and {\ec Assignments}.}
\newcommand{\formalf}{\\[1mm]\myspace  Join the {\ec Facebook group} to easily find and share information.}


\newcommand{\courseoverview}{\vphantom{ }\vspace{10mm}
  {\dg\large Overview}\\[4mm]
  {\sc\myspace Ramsey Theory}\\[1mm]
  {\sc\myspace Matching Theory}\\[1mm]
  {\sc\myspace Enumerative Combinatorics}\\[1mm]
  {\sc\myspace Extremal Set Theory}}

\newcommand{\courseoverviewalt}{\vphantom{ }\vspace{10mm}
  {\dg\large Overview}\\[4mm]
  {\sc\myspace Ramsey Theory \gray (3 Weeks)}\\[1mm]
  {\sc\myspace Matching Theory \gray (5 Weeks)}\\[1mm]
  {\sc\myspace Enumerative Combinatorics\gray (2 Week)}\\[1mm]
  {\sc\myspace Extremal Set Theory \gray (2 Weeks)}}

\newcommand{\courseoverviewramseyi}{\vphantom{ }\vspace{10mm}
  {\dg\large Overview}\\[4mm]
  {\sc\myspace Ramsey Theory}\\[1mm]
  {\sc\myspace\lightgray Matching Theory}\\[1mm]
  {\sc\myspace\lightgray Enumerative Combinatorics}\\[1mm]
  {\sc\myspace\lightgray Extremal Set Theory}}

\newcommand{\courseoverviewramseyii}{\vphantom{ }\vspace{10mm}
  {\dg\large Overview}\\[4mm]
  {\sc\myspace Ramsey Theory}\\[1mm]
  {\sc\myspace\myspace\footnotesize The Pigeonhole Principle}\\
  {\sc\myspace\myspace\footnotesize Ramsey's Theorem}\\ % Examples, Theorem, proof, example, variations
  {\sc\myspace\myspace\footnotesize Arithmetic Progressions}\\ % Van der Waerden's Theorem, The Hales-Jewett Theorem, other extensions
  {\sc\myspace\myspace\footnotesize Equations}\\ % Schur's Theorem, Rado's Theorem, Folkman's Theorem, Hindman's Theorem
  {\sc\myspace\myspace\footnotesize Graphs and Geometry}\\ % The Gallai-Witt Theorem, Spaces?, Roth's Theorem and Szemeredi's Theorem? Euclidean Ramsey Theory; Graph Ramsey Theory
  {\sc\myspace\myspace\footnotesize Applications}\\[1mm] % Triangles or something in the plane, Turan's Theorem? Other stuff?
  {\sc\myspace\lightgray Matching Theory}\\[1mm]
  {\sc\myspace\lightgray Enumerative Combinatorics}\\[1mm]
  {\sc\myspace\lightgray Extremal Set Theory}}

\newcommand{\lecturetitle}{\vphantom{ }\vspace{15mm}\begin{center}
  {\large\sc Ramsey Theory}\end{center}}

\newcommand{\lecturetitlei}{\vphantom{ }\vspace{15mm}\begin{center}
  {\large\sc Ramsey Theory}\\[3mm]
  \darkgreen Lecture 1: The Pigeonhole Principle\end{center}}

%\psset{fillstyle=none}

\newcommand{\aaia}{\vphantom{ }\vspace{15mm}\begin{center}
  \begin{pspicture}(0,2.5)(10,2.5)
    \rput( 0,2.5){\man}
    \rput( 5,2.5){\woman}
    \rput(10,2.5){\woman}
  \end{pspicture}
  \end{center}}

\newcommand{\aaib}[1]{\vphantom{ }\vspace{15mm}\begin{center}
  \begin{pspicture}(0,2.5)(12,2.5)
    \rput( 0,0){\dicev}
    \rput( 2,0){\diceiii}
    \rput( 4,0){\dicei}
    \rput( 6,0){\diceiv}
    \rput( 8,0){\dicevi}
    \rput(10,0){\diceii}
    \rput(12,0){\diceiv}#1
  \end{pspicture}
  \end{center}}

\newcommand{\aaiba}{{\psset{linecolor=darkgreen}\rput( 6,0){\diceframe}
                    \rput(12,0){\diceframe}}}

\newcommand{\pigeonthma}{{\dg The Pigeonhole Principle}
  \\[1mm]If $\bl k+1$ pigeons are put into $\bl k$ pigeonholes,\\
         then some pigeonhole contains at least two pigeons.}

\newcommand{\pigeonthmplusa}{{\dg The Pigeonhole Principle} {\gray (general)}
  \\[1mm]If $\bl km+1$ pigeons are put into $\bl k$ pigeonholes,\\
         then some pigeonhole contains at least $\bl m+1$ pigeons.}

\newcommand{\pigeonthmstrong}{{\dg The Pigeonhole Principle} {\gray (strong)}
  \\[1mm]If $\bl (n_1-1) +\cdots+(n_k-1) + 1$ pigeons are put into $\bl k$ pigeonholes,\\
         then some $\bl i$th pigeonhole contains at least $\bl n_i$ pigeons.}

\newcommand{\pigeonthmexaa}{\\[8mm]\example}
\newcommand{\pigeonthmexab}[1]{\vspace*{-1mm}\begin{center}
  \begin{pspicture}(0,0)(8.125,2.5)\psset{linecolor=darkgreen,shadow=false}
    \multiput(0,0)(3.5,0){3}{\psframe(-1.25,-1.25)( 1.25, 1.25)}\psset{linecolor=darkgray}#1
  \end{pspicture}
  \end{center}}
\newcommand{\pigeonthmexaba}{\rput(0  ,0){\pigeon}}
\newcommand{\pigeonthmexabb}{\rput(3.5,0){\pigeon}}
\newcommand{\pigeonthmexabc}{\rput(7  ,0){\pigeon}}
\newcommand{\pigeonthmexabd}{\rput(3.4,-4){$\begin{array}{rcl}\bl k &\meq&\bl 3\\\bl m &\meq&\bl 2\end{array}$}}
\newcommand{\pigeonthmexabaii}{\rput(-0.25,0){\pigeon}\rput( .25,0){\pigeon}}
\newcommand{\pigeonthmexabbii}{\rput( 3.25,0){\pigeon}\rput(3.75,0){\pigeon}}
\newcommand{\pigeonthmexabcii}{\rput( 6.75,0){\pigeon}\rput(7.25,0){\pigeon}}
\newcommand{\pigeonthmexabciii}{\rput(6.7 ,0){\pigeon}\rput(7   ,0){\pigeon}\rput(7.3,0){\pigeon}}

\newcommand{\pigeonthmexba}{\\[8mm]\example}
\newcommand{\pigeonthmexbb}[1]{\vspace*{-25mm}\aaib{#1}}
\newcommand{\pigeonthmexbc}{\vspace*{20mm}\begin{center}$\bl k\;\meq\; 6$\\\begin{tabular}{rcl}\\
   pigeons     & $\meq$ & dice values rolled\\
   pigeonholes & $\meq$ & the six possible values\end{tabular}\end{center}}

\newcommand{\pigeonthmexca}{\\[8mm]\exercise}
\newcommand{\pigeonthmexcb}{\\[1mm]Six distinct numbers are chosen from the numbers $\bl 0,\ldots,9$.
  \\Show that at least two of the chosen numbers are consecutive.}
\newcommand{\pigeonthmexcc}[1]{\vspace*{12mm}\begin{center}{\psset{unit=5mm,fillstyle=none,shadow=false}
   \pspicture(-.5,-.5)(14,.5)\lightgray
   \rput(0  ,0){0}\rput(1.5,0){1}\rput(3   ,0){2}\rput( 4.5,0){3}\rput( 6  ,0){4}
   \rput(7.5,0){5}\rput(9  ,0){6}\rput(10.5,0){7}\rput(12  ,0){8}\rput(13.5,0){9}#1\endpspicture}\end{center}}
\newcommand{\pigeonthmexcca}{\multiput(0,0)(3,0){5}{\psframe[linecolor=darkgreen](-0.5,-0.5)(2,.5)}}
\newcommand{\pigeonthmexccb}{\darkgray
   \rput(0,0){0}\rput(6,0){4}\rput(7.5,0){5}\rput(10.5,0){7}\rput(12,0){8}\rput(13.5,0){9}}

\newcommand{\pigeonthmexga}{\\[8mm]\exercise}
\newcommand{\pigeonthmexgb}{\\[1mm]$\bl n+1$ distinct numbers are chosen from the numbers $\bl 1,\ldots,2n$.
  \\Show that at least two of the chosen numbers are coprime.}

\newcommand{\pigeonthmexha}{\\[8mm]\theorem}
\newcommand{\pigeonthmexhb}{\\[1mm]There are infinitely many primes.}
\newcommand{\pigeonthmexhc}{\\[2mm]\proof {\gray(Mixon 2012)}}
\newcommand{\pigeonthmexhd}{\\Assume that there is a finite number of primes, $\bl p_1,\ldots,p_N$.}
\newcommand{\pigeonthmexhe}{\\Choose some integer $\bl K$ such that $\bl 2^K\mgt(K+1)^N$.}
\newcommand{\pigeonthmexhf}{\\There are $\bl 2^K$ numbers (``pigeons") in the set $\bl S\meq \{1,\ldots,2^K\}$}
\newcommand{\pigeonthmexhg}[4]{\\with at most $\bl (K+1)^N$ factorisations $\bl p_1^{k_1}\cdots p_N^{k_N}$ (``pigeonholes")
  since\vspace*{-.1mm}
  \[\bl
    K\meq  \log_2 2^K \;{#1{
     \mgeq \log_2 p_1^{k_1}\cdots p_N^{k_N}}}\;{#2{
     \meq  \sum k_i \log_2 p_i}}\;{#3{
     \mgeq \sum k_i }}\;{#4{
     \mgeq \max_i k_i\,\black.}}\vspace*{-4mm}
  \]}
\newcommand{\pigeonthmexhh}{\\By the {\dg Pigeonhole Principle},
  two numbers in $\bl S$ have identical factorisation.}
\newcommand{\pigeonthmexhj}{\\They must then be equal, a contradiction.\qed\vspace*{-8mm}}

\newcommand{\pigeonthmexda}{\\[8mm]\exercise}
\newcommand{\pigeonthmexdb}{\\[1mm]$\bl 19$ points are drawn in a square with side lengths~$\bl 1$.\\
  \\[-6mm]Show that at least $\bl 3$ points lie in a circle with radius less than~$\bl\dfrac{1}{4}$.}
\newcommand{\pigeonthmexdc}[2]{\vspace*{-1mm}\begin{center}
  \begin{pspicture}(0,0)(10,6.5)\psset{shadow=false,shadow=false,fillstyle=solid,framearc=0}
    #1\psframe[linecolor=darkgreen,fillstyle=none](0,0)(6,6)#2
  \end{pspicture}\vspace*{-10mm}
  \end{center}}
\newcommand{\pigeonthmexdca}{\psset{linecolor=darkgray}
  \qdisk(1.03,0.81){.1}\qdisk(0.94,2.86){.1}\qdisk(3.14,2.39){.1}\qdisk(3.28,3.51){.1}
  \qdisk(1.62,3.73){.1}\qdisk(1.23,1.02){.1}\qdisk(3.54,2.74){.1}\qdisk(5.03,2.23){.1}
  \qdisk(2.65,5.08){.1}\qdisk(3.19,5.31){.1}\qdisk(2.9,0.51){.1}\qdisk(5.46,4.19){.1}
  \qdisk(2.54,4.14){.1}\qdisk(4.36,2.63){.1}\qdisk(5.0,5.35){.1}\qdisk(0.65,1.77){.1}
  \qdisk(3.49,0.92){.1}\qdisk(3.37,4.36){.1}\qdisk(1.03,4.2){.1}}
\newcommand{\pigeonthmexdcb}{\psset{linecolor=lightgray}
  \psline(2,0   )(2,6   )
  \psline(4,0   )(4,6   )
  \psline(0   ,2)(6   ,2)
  \psline(0   ,4)(6   ,4)}
\newcommand{\pigeonthmexdcc}{\pscircle[linecolor=darkgray,fillstyle=none](1,1){1.414}
  \rput[l](7.25,5.5){{\gray The circle has radius}}
  \rput[l](7.35,4){$\lightblue \dfrac{1}{3\sqrt{2}}\gray\approx\lightblue 0.236 \gray< \lightblue \dfrac{1}{4}$\,.}}
\newcommand{\pigeonthmexdcd}{\rput[l](7.25,1.5){{\black Can this result be improved?}}}
\newcommand{\pigeonthmexdce}{\rput[l](7.25, .5){{\black Can you generalise it?}}}

\newcommand{\halliiihe}{\\[8mm]{\dg The Erd\H{o}s-Szekeres Theorem} {\gray (1935)}
  \\Each sequence $\bl a_1,\ldots,a_{n^2\!+\!1}$ of $\blue n^2\!+\!1$ distinct integers has
  \\a subsequence with $\blue n+1$ elements that is either increasing or decreasing.}

\newcommand{\halliiihf}{\\[3mm]{\dg\example}
  \\Consider the following sequence of $\blue n^2\!+\!1$ integers where $\blue n\meq\textsf{3}$:}
\newcommand{\halliiihfa}{\\{\psset{unit=5mm}\pspicture(-7,-1.5)(10,2)
   \rput(0,0){7}\rput(1,0){8}\rput(2,0){9}\rput(3,0){4}\rput(4,0){5}
   \rput(5,0){6}\rput(6,0){1}\rput(7,0){2}\rput(8,0){3}\rput(9,0){0}\endpspicture}}
\newcommand{\halliiihfaa}{\\{\psset{unit=5mm}\pspicture(-7,-1.5)(10,2)
   \psset{linecolor=halfgray,fillcolor=lightgray,shadow=false,linewidth=.03,fillstyle=solid,framearc=0}
   \psframe(-.5,-.5)(.5,.5)\psframe(2.5,-.5)(3.5,.5)\psframe(5.5,-.5)(6.5,.5)\psframe(8.5,-.5)(9.5,.5)
   \rput(1,0){8}\rput(2,0){9}\rput(4,0){5}\rput(5,0){6}\rput(7,0){2}\rput(8,0){3}
   {\darkred\rput(0,0){7}\rput(3,0){4}\rput(6,0){1}\rput(9,0){0}}\endpspicture}}
\newcommand{\halliiihfb}{\\There is no increasing subsequence of size $\blue n+1\meq\textsf{4}$.}
\newcommand{\halliiihfc}{\\However, there is a decreasing subsequence of size $\blue n+1\meq\textsf{4}$.}

\newcommand{\halliiihepfa}{\\[2mm]\proof}
\newcommand{\halliiihepfb}{\\Suppose that no increasing subsequence has $\bl n+1$ terms.}
\newcommand{\halliiihepfc}{\\Let $\bl n_i$ be the longest length of an increasing subsequence starting in $\bl a_i$.}
\newcommand{\halliiihepfd}{\\and note that $\bl n_i$ can only be one of the values $\bl 1,\ldots,n$ (``pigeonholes'').}
\newcommand{\halliiihepfe}{\\By the {\dg Pigeonhole Principle}, there are at least $\bl n+1$ identical $\bl n_i$ values.}
\newcommand{\halliiihepff}{\\Suppose that $\bl n_i\meq n_j$ for some $\bl i\mlt j$. }
\newcommand{\halliiihepfg}{Then $\bl a_i\black\not<\bl a_j$, so $\bl a_i\mgt a_j$.}
\newcommand{\halliiihepfh}{\\We therefore have a decreasing subsequence of at least $\bl n+1$ $\bl a_i$s.\qed}

\newcommand{\halliiiheexa}{\\[3mm]{\dg\exercise}}
\newcommand{\halliiiheexb}{\\[0mm]Can you generalise this theorem?\vspace*{-8mm}}

\newcommand{\pigeonthmexea}{\\[8mm]\exercise}
\newcommand{\pigeonthmexeav}{\\[8mm]\exercise}
\newcommand{\pigeonthmexeb}{\\Let $\bl C$ be a binary code of length $\bl n\meq 6$ and minimum distance $\bl d\meq 3$.
  \\Use the {\dg Pigeonhole Principle} to show that $\bl C$ has at most $\bl 8$ codewords.}
%\newcommand{\pigeonthmexec}[2]{\\[3mm]The {\dg Sphere Packing Bound} ({\dg MATH3411}) tells us that,
%  for $\bl t\meq \frac{d-1}{2}\meq 1$,
%  \[\bl
%     |C| \mleq \Biggl\lfloor\frac{2^n}{\ds\sum_{i=0}^t\binom{n}{i}}\Biggr\rfloor
%    \;{#1{\mleq \left\lfloor\frac{2^6}{1+6}\right\rfloor}} \;{#2{\meq 9}}\]}
%\newcommand{\pigeonthmexecv}{\\[3mm]The {\dg Sphere Packing Bound} ({\dg MATH3411}) tells us that
%  $\bl |C| \mleq 9$.}
\newcommand{\pigeonthmexec}{\\[2mm]\proof}
\newcommand{\pigeonthmexed}{\quad Assume that $\bl |C|\mgeq 9$.}
\newcommand{\pigeonthmexee}{\\The codewords of $\bl C$ can end in one of the $\bl 4$ pairs:
   \mypair{0}{0}\;
   \mypair{0}{1}\;
   \mypair{1}{0}\;
   \mypair{1}{1}}
\newcommand{\pigeonthmexef}{\\By the {\dg Pigeonhole Principle}, $\bl 3$ codewords $\bl c_i$ have same ending, say \mypair{0}{0}.}
\newcommand{\pigeonthmexeg}{\\The minimum distance is $\bl 3$, so without loss of generality assume that\vspace*{-2.5mm}
  \[\bl
    \begin{array}{rcl}
      c_1 &\meq& \mysixtuple{0}{0}{0}{0}{0}{0}\\
      c_2 &\meq& \mysixtuple{1}{1}{1}{?}{0}{0}\\
      c_3 &\meq& \mysixtuple{?}{?}{?}{?}{0}{0}
    \end{array}\vspace*{-2.5mm}
  \]}
\newcommand{\pigeonthmexeh}{The first three bits of $\bl c_3$ must differ from at least two of those in~$\bl c_1$.}
\newcommand{\pigeonthmexej}{\\But then the distance between $\bl c_2$ and $\bl c_3$ is at most $\bl 2$, a contradiction.\qed\vspace*{-8mm}}

\newcommand{\pigeonthmexma}{\\[8mm]\example}
\newcommand{\pigeonthmexmb}[1]{\vspace*{-1mm}\begin{center}
  \begin{pspicture}(0,0)(8.125,2.5)\psset{linecolor=darkgreen,shadow=false}
    \multiput(0,0)(3.5,0){3}{\psframe(-1.25,-1.25)( 1.25, 1.25)}
    \rput[t](0  ,-2.5){$\bl n_1\meq 1$}
    \rput[t](3.5,-2.5){$\bl n_2\meq 3$}
    \rput[t](7  ,-2.5){$\bl n_3\meq 2$}
    \psset{linecolor=darkgray}#1
  \end{pspicture}
  \end{center}}

\newcommand{\pigeonthmexna}{\\[8mm]\exercise}
\newcommand{\pigeonthmexnb}{\\[0mm]Can we cover this slightly mutilated chess board with dominoes
   \pspicture(0,0)(1.5,1)\psset{unit=4mm,fillcolor=gray,linewidth=.03,linecolor=black,shadow=false,framearc=0,fillstyle=solid}
     \psframe(0,-.2)(2,.8)\psline(1,-.2)(1,.8)\endpspicture?\vspace*{5mm}\begin{center}
  {\psset{unit=4mm,linewidth=0,linecolor=white,fillstyle=solid,fillcolor=black,shadow=false,framearc=0}
  \begin{pspicture}(0,0)(8,8)
    \psframe[fillcolor=offwhite](0,0)(8,8)
    \psframe[fillcolor=white](0,7)(1,8)
    \psframe[fillcolor=white](7,0)(8,1)
    \multiput(1,0)(2,0){3}{\psframe(0,0)(1,1)}
    \multiput(0,1)(2,0){4}{\psframe(0,0)(1,1)}
    \multiput(1,2)(2,0){4}{\psframe(0,0)(1,1)}
    \multiput(0,3)(2,0){4}{\psframe(0,0)(1,1)}
    \multiput(1,4)(2,0){4}{\psframe(0,0)(1,1)}
    \multiput(0,5)(2,0){4}{\psframe(0,0)(1,1)}
    \multiput(1,6)(2,0){4}{\psframe(0,0)(1,1)}
    \multiput(2,7)(2,0){3}{\psframe(0,0)(1,1)}
    \psline[fillstyle=none,linecolor=black,linewidth=0.05](0,0)(0,7)(1,7)(1,8)(8,8)(8,1)(7,1)(7,0)(0,0)
  \end{pspicture}}
  \end{center}}

\newcommand{\dirichleta}{\\[3mm]{\dg Dirichlet's Approximation Theorem} {\gray (1834)}}
\newcommand{\dirichletb}{\\[-1mm]If $\bl\alpha\myin\mathbb{R}$ and $\bl N\myin\mathbb{N}$,
  then $\bl p,q\myin\mathbb{N}$ exist so that $\bl q\!\leq\! N$ and
  $\bl\ds \Bigl|\alpha - \frac{p}{q}\Bigr| \mlt \frac{1}{Nq}$\,.\hspace*{-6mm}}
\newcommand{\dirichletc}{\\[2mm]{\gray Each real number can thus be approximated arbitrarily well by rationals.}}

\newcommand{\dirichletpfa}{\\[2mm]\proof}
\newcommand{\dirichletpfb}{\\Replace the latter inequality equivalently by
  $\bl \bigl|q\alpha - p\bigr| \mlt \frac{1}{N}$.}
\newcommand{\dirichletpfc}{\\Partition the unit interval $\bl[0,1)$ into $\bl N$ equal subintervals.
  \begin{pspicture}(-1.5,.5)(8,.5)\psset{shadow=false}\psline(0,0)(4,0)\multirput(0,0)(.8,0){6}{\psline(0,-.2)(0,.2)}
    \rput[t](0,-.5){{\gray0}}
    \rput[t](4,-.5){{\gray1}}
    \rput[u](3.6,1.4){$\bl\frac{1}{N}$}
    \psline[linecolor=halfgray](3.2,.4)(3.2,.6)(4,.6)(4,.4)\psline[linecolor=halfgray](3.6,.6)(3.6,.8)\end{pspicture}}
\newcommand{\dirichletpfd}{\\Define $\bl N+1$ residues
  $\bl r_i\mdef i\alpha - \lfloor i\alpha\rfloor$ for $\bl i \meq 0,1,\ldots,N$.}
\newcommand{\dirichletpfe}{\\These lie in the $\bl N$ subintervals.}
\newcommand{\dirichletpff}{\\By the {\dg Pigeonhole Principle},
  residues $\bl r_i$, $\bl r_j$ ($\bl i\!\mlt\!j$) lie in some subinterval.}
%\newcommand{\dirichletpfg}{so $\bl |r_i-r_j|\mlt \frac{1}{N}$. }
\newcommand{\dirichletpfg}{\\Set $\bl p\mdef \lfloor j\alpha\rfloor - \lfloor i\alpha\rfloor$
                           and $\bl q\mdef j-i$. }
\newcommand{\dirichletpfh}{\\Then $\bl 0\mlt q\mleq N$ }
\newcommand{\dirichletpfj}[3]{and\vspace*{-2mm}
  \begin{align*}\bl
    |p-q\alpha|&\meq |\lfloor j\alpha\rfloor - \lfloor i\alpha\rfloor - (j-i)\alpha|\\[-2mm]
    &\;{#1{\meq |i\alpha - \lfloor i\alpha\rfloor - (j\alpha - \lfloor j\alpha\rfloor)|}}
    \;{#2{\meq |r_i-r_j|}}
    \;{#3{\mlt \frac{1}{N}\,\black.\qquad\Box}}\\[-12mm]\end{align*}}



\newcommand{\losslessa}{\\[4mm]\theorem}
\newcommand{\losslessb}{\\Let $\bl f$ be a {\dg lossless digital data compression algorithm}.
  \\If $\bl |f(B)|\mlt |B|$ for some data $\bl B$,
  then $\bl |f(A)|\mgt |A|$ for some data $\bl A$.}
\newcommand{\losslessc}{\\[2mm]{\gray In other words, no lossless compression algorithm works 100\%\ of the time.}}


\newcommand{\losslesspfa}{\\[2mm]\proof}
\newcommand{\losslesspfb}{\; Represent data as binary strings:
  $\bl |A|$ is now the string length of~$\bl A$.}
%  \\the size $\bl |A|$ of data $\bl A$ will be measured as the string length of $\bl A$.}
\newcommand{\losslesspfc}{\\Suppose that $\bl |f(B)|\mlt |B|$ for at least one string $\bl B$
  \\and assume that $\bl |f(A)|\mleq |A|$ for all strings $\bl A$.}
\newcommand{\losslesspfd}{\\Let $\bl n$ be the smallest number with
  $\bl |f(B')|\mlt |B'|\meq n$ for some string~$\bl B'$.}
\newcommand{\losslesspfe}{\\Set $\bl m\mdef |f(B')| \,{\gray(<\!\lightblue n\gray)}$
  and let $\bl\mathcal{F}$ be the set of strings $\bl B$ with $\bl |f(B)|\meq m$.}
\newcommand{\losslesspff}{\\Each of the $\bl 2^m$ binary strings $\bl A$ of length $\bl m$
  must obey $\bl |f(A)|\mgeq |A|\meq m$.}
\newcommand{\losslesspfg}{\\But $\bl |f(A)|\mleq |A|$ by assumption,
 so $\bl |f(A)|\meq |A|\meq m$ and
 so $\bl A\myin\mathcal{F}$.}
\newcommand{\losslesspfh}{\\Also, $\bl B'$ is also in $\bl \mathcal{F}$, so $\bl |\mathcal{F}|\mgeq 2^m+1$.}
\newcommand{\losslesspfj}{\\By the {\dg Pigeonhole Principle},
  some distinct $\bl B_1, B_2$ satisfy $\bl f(B_1)\meq f(B_2)$.}
\newcommand{\losslesspfk}{\\But $\bl f$ is lossless and therefore injective, a contradiction.\qed\vspace*{-8mm}}

\newcommand{\fermata}{\\[8mm]\theorem {\gray(Fermat 1640, Euler 1747)}}
\newcommand{\fermatav}{\\[2mm]\theorem {\gray(Fermat 1640, Euler 1747)}}
\newcommand{\fermatb}{\\Each prime $\bl p\mequiv 1\pmod 4$ can be written as a sum of squares $\bl p\meq x^2 + y^2$.}

\newcommand{\fermatlema}{\\[4mm]\lemma}
\newcommand{\fermatlemav}{\\[2mm]\lemma \; If $\bl p\mequiv 1\pmod 4$, then $\bl a^2\mequiv -1\pmod{p}$ for some $\bl a$.}
\newcommand{\fermatlemb}{\\If $\bl p\mequiv 1\pmod 4$, then $\bl a^2\mequiv -1\pmod{p}$ for some $\bl a$.}

\newcommand{\fermatlempfa}{\\[2mm]\proof }
\newcommand{\fermatlempfb}{\\Suppose that $\bl p\mequiv 1\pmod 4$ and set $\bl a\mdef \big(\frac{p-1}{2}\big)!$\,.}
\newcommand{\fermatlempfc}{\\Note that $\bl p-i\mequiv -i\pmod p$ and apply {\dg Wilson's Theorem}:}
\newcommand{\fermatlempfd}[5]{\[\bl
\begin{array}{rl}
   a^2 \meq \bigl(\!\big(\frac{p-1}{2}\big)!\bigr)^2\!\!\!
 &{#1{\mequiv \bigl(\!\big(\frac{p-1}{2}\big)!\bigr)^2(-1)^{\frac{p-1}{2}}\!\!\pmod{p}}}\\[1mm]
 &{#2{\mequiv 1\times 2\times\cdots\times\!\big(\frac{p-1}{2}\big)\big(\!-\!\frac{p-1}{2}\big)\times\cdots\times(-2)(-1)\!\!\pmod{p}}}\\[1mm]
 &{#3{\mequiv 1\times 2\times\cdots\times\!\big(\frac{p-1}{2}\big)\big(\frac{p+1}{2}\big)\!\times\cdots\times\!(p\!-\!2)(p\!-\!1)\!\!\pmod{p}}}\\[1mm]
 &{#4{\mequiv (p-1)!\!\!\pmod{p}}}\\[1mm]
 &{#5{\mequiv{\blue-1}\!\!\pmod{p}\,\black.\qquad\Box}}
\end{array}
\]}

\newcommand{\fermatpfa}{\\[1mm]\proof }
\newcommand{\fermatpfb}{\\By the lemma, we can find some $\bl a$ with $\bl a^2\mequiv -1\pmod{p}$.}
\newcommand{\fermatpfc}{\\Now, consider the pairs of integers $\bl (x,y)$ with $\bl 0 \mleq x,y \mlt \sqrt{p}$.}
\newcommand{\fermatpfd}{\\There are $\bl (\lfloor\sqrt{p}\rfloor+1)^2\mgt p$ such pairs
                          and at most $\bl p$ values $\bl ax-y$ mod~$\bl p$.}
\newcommand{\fermatpfe}{\\By the {\dg Pigeonhole Principle},
  two distinct pairs $\bl(x_1,y_1)$ and $\bl(x_2,y_2)$ must satisfy $\bl ax_1-y_1 \mequiv ax_2-y_2 \pmod p$.}
\newcommand{\fermatpff}{ Set $\bl x\mdef x_1-x_2$ and $\bl y\mdef y_1-y_2$.}
\newcommand{\fermatpfg}{\\Then $\bl x^2+y^2\mequiv -a^2x^2+y^2\mequiv
                         -(a(x_1 - x_2))^2 + (y_1-y_2)^2\mequiv0\pmod p$.}
\newcommand{\fermatpfh}[1]{\\Since $\bl(x_1,y_1)$ and $\bl(x_2,y_2)$ are distinct and $\bl0 \mleq x_i,y_i \mlt\sqrt{p}$,
                        \\we see that $\bl 0\mlt x^2+y^2 \mlt (\sqrt{p})^2 + (\sqrt{p})^2 \meq 2p$.
                        \\{#1{Hence, $\bl x^2 + y^2 \meq p$.\qed\vspace*{-15mm}}}}



\coursetitle
\np\coursetitle\coursetitlei
\np\formal
\np\formal\formala
\np\formal\formala\formalb\formalc
\np\formal\formala\formalb\formalc\formald
\np\formal\formala\formalb\formalc\formald\formale
\np\formal\formala\formalb\formalc\formald\formale\formalf
\np\courseoverview
\np\courseoverviewalt
\np\courseoverviewramseyi
\np\courseoverviewramseyii
\np\lecturetitle
\np\lecturetitlei
%\np\aaia
%\np\aaib{}
\np\pigeonthma
\np\pigeonthma\pigeonthmexaa
\np\pigeonthma\pigeonthmexaa\pigeonthmexab{}
\np\pigeonthma\pigeonthmexaa\pigeonthmexab{\pigeonthmexaba}
\np\pigeonthma\pigeonthmexaa\pigeonthmexab{\pigeonthmexaba\pigeonthmexabc}
\np\pigeonthma\pigeonthmexaa\pigeonthmexab{\pigeonthmexaba\pigeonthmexabb\pigeonthmexabc}
\np\pigeonthma\pigeonthmexaa\pigeonthmexab{\pigeonthmexaba\pigeonthmexabb\pigeonthmexabcii}
\np\pigeonthma\pigeonthmexba
\np\pigeonthma\pigeonthmexba\pigeonthmexbb{}
\np\pigeonthma\pigeonthmexba\pigeonthmexbb{}\pigeonthmexbc
\np\pigeonthma\pigeonthmexba\pigeonthmexbb{\aaiba}\pigeonthmexbc
\np\pigeonthma\pigeonthmexca
\np\pigeonthma\pigeonthmexca\pigeonthmexcb
\np\pigeonthma\pigeonthmexca\pigeonthmexcb\pigeonthmexcc{}
\np\pigeonthma\pigeonthmexca\pigeonthmexcb\pigeonthmexcc{\pigeonthmexcca}
\np\pigeonthma\pigeonthmexca\pigeonthmexcb\pigeonthmexcc{\pigeonthmexcca\pigeonthmexccb}
\np\pigeonthma\pigeonthmexca\pigeonthmexcb
\np\pigeonthma\pigeonthmexca\pigeonthmexcb\pigeonthmexga
\np\pigeonthma\pigeonthmexca\pigeonthmexcb\pigeonthmexga\pigeonthmexgb
\np\pigeonthma
\np\pigeonthma\pigeonthmexna
\np\pigeonthma\pigeonthmexna\pigeonthmexnb
\np\pigeonthma
\np\pigeonthma\pigeonthmexha
\np\pigeonthma\pigeonthmexha\pigeonthmexhb
\np\pigeonthma\pigeonthmexha\pigeonthmexhb\pigeonthmexhc
\np\pigeonthma\pigeonthmexha\pigeonthmexhb\pigeonthmexhc\pigeonthmexhd
\np\pigeonthma\pigeonthmexha\pigeonthmexhb\pigeonthmexhc\pigeonthmexhd\pigeonthmexhe
\np\pigeonthma\pigeonthmexha\pigeonthmexhb\pigeonthmexhc\pigeonthmexhd\pigeonthmexhe\pigeonthmexhf
\np\pigeonthma\pigeonthmexha\pigeonthmexhb\pigeonthmexhc\pigeonthmexhd\pigeonthmexhe\pigeonthmexhf\pigeonthmexhg{\ph}{\ph}{\ph}{\ph}
\np\pigeonthma\pigeonthmexha\pigeonthmexhb\pigeonthmexhc\pigeonthmexhd\pigeonthmexhe\pigeonthmexhf\pigeonthmexhg{}{\ph}{\ph}{\ph}
\np\pigeonthma\pigeonthmexha\pigeonthmexhb\pigeonthmexhc\pigeonthmexhd\pigeonthmexhe\pigeonthmexhf\pigeonthmexhg{}{}{\ph}{\ph}
\np\pigeonthma\pigeonthmexha\pigeonthmexhb\pigeonthmexhc\pigeonthmexhd\pigeonthmexhe\pigeonthmexhf\pigeonthmexhg{}{}{}{\ph}
\np\pigeonthma\pigeonthmexha\pigeonthmexhb\pigeonthmexhc\pigeonthmexhd\pigeonthmexhe\pigeonthmexhf\pigeonthmexhg{}{}{}{}
\np\pigeonthma\pigeonthmexha\pigeonthmexhb\pigeonthmexhc\pigeonthmexhd\pigeonthmexhe\pigeonthmexhf\pigeonthmexhg{}{}{}{}\pigeonthmexhh
\np\pigeonthma\pigeonthmexha\pigeonthmexhb\pigeonthmexhc\pigeonthmexhd\pigeonthmexhe\pigeonthmexhf\pigeonthmexhg{}{}{}{}\pigeonthmexhh\pigeonthmexhj
\np\pigeonthma
\np\pigeonthma\dirichleta
\np\pigeonthma\dirichleta\dirichletb
\np\pigeonthma\dirichleta\dirichletb\dirichletc
\np\pigeonthma\dirichleta\dirichletb\dirichletpfa
\np\pigeonthma\dirichleta\dirichletb\dirichletpfa\dirichletpfb
\np\pigeonthma\dirichleta\dirichletb\dirichletpfa\dirichletpfb\dirichletpfc
\np\pigeonthma\dirichleta\dirichletb\dirichletpfa\dirichletpfb\dirichletpfc\dirichletpfd
\np\pigeonthma\dirichleta\dirichletb\dirichletpfa\dirichletpfb\dirichletpfc\dirichletpfd\dirichletpfe
\np\pigeonthma\dirichleta\dirichletb\dirichletpfa\dirichletpfb\dirichletpfc\dirichletpfd\dirichletpfe\dirichletpff
\np\pigeonthma\dirichleta\dirichletb\dirichletpfa\dirichletpfb\dirichletpfc\dirichletpfd\dirichletpfe\dirichletpff\dirichletpfg
\np\pigeonthma\dirichleta\dirichletb\dirichletpfa\dirichletpfb\dirichletpfc\dirichletpfd\dirichletpfe\dirichletpff\dirichletpfg\dirichletpfh
\np\pigeonthma\dirichleta\dirichletb\dirichletpfa\dirichletpfb\dirichletpfc\dirichletpfd\dirichletpfe\dirichletpff\dirichletpfg\dirichletpfh\dirichletpfj{\ph}{\ph}{\ph}
\np\pigeonthma\dirichleta\dirichletb\dirichletpfa\dirichletpfb\dirichletpfc\dirichletpfd\dirichletpfe\dirichletpff\dirichletpfg\dirichletpfh\dirichletpfj{}{\ph}{\ph}
\np\pigeonthma\dirichleta\dirichletb\dirichletpfa\dirichletpfb\dirichletpfc\dirichletpfd\dirichletpfe\dirichletpff\dirichletpfg\dirichletpfh\dirichletpfj{}{}{\ph}
\np\pigeonthma\dirichleta\dirichletb\dirichletpfa\dirichletpfb\dirichletpfc\dirichletpfd\dirichletpfe\dirichletpff\dirichletpfg\dirichletpfh\dirichletpfj{}{}{}
\np\pigeonthma
\np\pigeonthma\fermata
\np\pigeonthma\fermata\fermatb
\np\pigeonthma\fermata\fermatb\fermatlema
\np\pigeonthma\fermata\fermatb\fermatlema\fermatlemb
\np\pigeonthma\fermatlema\fermatlemb\fermatlempfa
\np\pigeonthma\fermatlema\fermatlemb\fermatlempfa\fermatlempfb
\np\pigeonthma\fermatlema\fermatlemb\fermatlempfa\fermatlempfb\fermatlempfc
\np\pigeonthma\fermatlema\fermatlemb\fermatlempfa\fermatlempfb\fermatlempfc\fermatlempfd{\ph}{\ph}{\ph}{\ph}{\ph}
\np\pigeonthma\fermatlema\fermatlemb\fermatlempfa\fermatlempfb\fermatlempfc\fermatlempfd{}{\ph}{\ph}{\ph}{\ph}
\np\pigeonthma\fermatlema\fermatlemb\fermatlempfa\fermatlempfb\fermatlempfc\fermatlempfd{}{}{\ph}{\ph}{\ph}
\np\pigeonthma\fermatlema\fermatlemb\fermatlempfa\fermatlempfb\fermatlempfc\fermatlempfd{}{}{}{\ph}{\ph}
\np\pigeonthma\fermatlema\fermatlemb\fermatlempfa\fermatlempfb\fermatlempfc\fermatlempfd{}{}{}{}{\ph}
\np\pigeonthma\fermatlema\fermatlemb\fermatlempfa\fermatlempfb\fermatlempfc\fermatlempfd{}{}{}{}{}
\np\pigeonthma\fermatlemav\fermatav\fermatb
\np\pigeonthma\fermatlemav\fermatav\fermatb\fermatpfa
\np\pigeonthma\fermatlemav\fermatav\fermatb\fermatpfa\fermatpfb
\np\pigeonthma\fermatlemav\fermatav\fermatb\fermatpfa\fermatpfb\fermatpfc
\np\pigeonthma\fermatlemav\fermatav\fermatb\fermatpfa\fermatpfb\fermatpfc\fermatpfd
\np\pigeonthma\fermatlemav\fermatav\fermatb\fermatpfa\fermatpfb\fermatpfc\fermatpfd\fermatpfe
\np\pigeonthma\fermatlemav\fermatav\fermatb\fermatpfa\fermatpfb\fermatpfc\fermatpfd\fermatpfe\fermatpff
\np\pigeonthma\fermatlemav\fermatav\fermatb\fermatpfa\fermatpfb\fermatpfc\fermatpfd\fermatpfe\fermatpff\fermatpfg
\np\pigeonthma\fermatlemav\fermatav\fermatb\fermatpfa\fermatpfb\fermatpfc\fermatpfd\fermatpfe\fermatpff\fermatpfg\fermatpfh{\ph}
\np\pigeonthma\fermatlemav\fermatav\fermatb\fermatpfa\fermatpfb\fermatpfc\fermatpfd\fermatpfe\fermatpff\fermatpfg\fermatpfh{}
\np\pigeonthma
\np\pigeonthma\losslessa
\np\pigeonthma\losslessa\losslessb
\np\pigeonthma\losslessa\losslessb\losslessc
\np\pigeonthma\losslessa\losslessb\losslesspfa
\np\pigeonthma\losslessa\losslessb\losslesspfa\losslesspfb
\np\pigeonthma\losslessa\losslessb\losslesspfa\losslesspfb\losslesspfc
\np\pigeonthma\losslessa\losslessb\losslesspfa\losslesspfb\losslesspfc\losslesspfd
\np\pigeonthma\losslessa\losslessb\losslesspfa\losslesspfb\losslesspfc\losslesspfd\losslesspfe
\np\pigeonthma\losslessa\losslessb\losslesspfa\losslesspfb\losslesspfc\losslesspfd\losslesspfe\losslesspff
\np\pigeonthma\losslessa\losslessb\losslesspfa\losslesspfb\losslesspfc\losslesspfd\losslesspfe\losslesspff\losslesspfg
\np\pigeonthma\losslessa\losslessb\losslesspfa\losslesspfb\losslesspfc\losslesspfd\losslesspfe\losslesspff\losslesspfg\losslesspfh
\np\pigeonthma\losslessa\losslessb\losslesspfa\losslesspfb\losslesspfc\losslesspfd\losslesspfe\losslesspff\losslesspfg\losslesspfh\losslesspfj
\np\pigeonthma\losslessa\losslessb\losslesspfa\losslesspfb\losslesspfc\losslesspfd\losslesspfe\losslesspff\losslesspfg\losslesspfh\losslesspfj\losslesspfk
\np\pigeonthmplusa
\np\pigeonthmplusa\pigeonthmexaa
\np\pigeonthmplusa\pigeonthmexaa\pigeonthmexab{}
\np\pigeonthmplusa\pigeonthmexaa\pigeonthmexab{\pigeonthmexabd}
\np\pigeonthmplusa\pigeonthmexaa\pigeonthmexab{\pigeonthmexaba\pigeonthmexabd}
\np\pigeonthmplusa\pigeonthmexaa\pigeonthmexab{\pigeonthmexaba\pigeonthmexabc\pigeonthmexabd}
\np\pigeonthmplusa\pigeonthmexaa\pigeonthmexab{\pigeonthmexabaii\pigeonthmexabc\pigeonthmexabd}
\np\pigeonthmplusa\pigeonthmexaa\pigeonthmexab{\pigeonthmexabaii\pigeonthmexabb\pigeonthmexabc\pigeonthmexabd}
\np\pigeonthmplusa\pigeonthmexaa\pigeonthmexab{\pigeonthmexabaii\pigeonthmexabb\pigeonthmexabcii\pigeonthmexabd}
\np\pigeonthmplusa\pigeonthmexaa\pigeonthmexab{\pigeonthmexabaii\pigeonthmexabbii\pigeonthmexabcii\pigeonthmexabd}
\np\pigeonthmplusa\pigeonthmexaa\pigeonthmexab{\pigeonthmexabaii\pigeonthmexabbii\pigeonthmexabciii\pigeonthmexabd}
\np\pigeonthmplusa
\np\pigeonthmplusa\pigeonthmexda
\np\pigeonthmplusa\pigeonthmexda\pigeonthmexdb
\np\pigeonthmplusa\pigeonthmexda\pigeonthmexdb\pigeonthmexdc{}{}
\np\pigeonthmplusa\pigeonthmexda\pigeonthmexdb\pigeonthmexdc{}{\pigeonthmexdca}
\np\pigeonthmplusa\pigeonthmexda\pigeonthmexdb\pigeonthmexdc{\pigeonthmexdcb}{\pigeonthmexdca}
\np\pigeonthmplusa\pigeonthmexda\pigeonthmexdb\pigeonthmexdc{\pigeonthmexdcb}{\pigeonthmexdca\pigeonthmexdcc}
\np\pigeonthmplusa\pigeonthmexda\pigeonthmexdb\pigeonthmexdc{}{\pigeonthmexdca\pigeonthmexdcc}
\np\pigeonthmplusa\pigeonthmexda\pigeonthmexdb\pigeonthmexdc{}{\pigeonthmexdca\pigeonthmexdcc\pigeonthmexdcd}
\np\pigeonthmplusa\pigeonthmexda\pigeonthmexdb\pigeonthmexdc{}{\pigeonthmexdca\pigeonthmexdcc\pigeonthmexdcd\pigeonthmexdce}
\np\pigeonthmplusa
\np\pigeonthmplusa\halliiihe
\np\pigeonthmplusa\halliiihe\halliiihf
\np\pigeonthmplusa\halliiihe\halliiihf\halliiihfa
\np\pigeonthmplusa\halliiihe\halliiihf\halliiihfa\halliiihfb
\np\pigeonthmplusa\halliiihe\halliiihf\halliiihfa\halliiihfb\halliiihfc
\np\pigeonthmplusa\halliiihe\halliiihf\halliiihfaa\halliiihfb\halliiihfc
\np\pigeonthmplusa\halliiihe
\np\pigeonthmplusa\halliiihe\halliiihepfa
\np\pigeonthmplusa\halliiihe\halliiihepfa\halliiihepfb
\np\pigeonthmplusa\halliiihe\halliiihepfa\halliiihepfb\halliiihepfc
\np\pigeonthmplusa\halliiihe\halliiihepfa\halliiihepfb\halliiihepfc\halliiihepfd
\np\pigeonthmplusa\halliiihe\halliiihepfa\halliiihepfb\halliiihepfc\halliiihepfd\halliiihepfe
\np\pigeonthmplusa\halliiihe\halliiihepfa\halliiihepfb\halliiihepfc\halliiihepfd\halliiihepfe\halliiihepff
\np\pigeonthmplusa\halliiihe\halliiihepfa\halliiihepfb\halliiihepfc\halliiihepfd\halliiihepfe\halliiihepff\halliiihepfg
\np\pigeonthmplusa\halliiihe\halliiihepfa\halliiihepfb\halliiihepfc\halliiihepfd\halliiihepfe\halliiihepff\halliiihepfg\halliiihepfh
\np\pigeonthmplusa\halliiihe\halliiihepfa\halliiihepfb\halliiihepfc\halliiihepfd\halliiihepfe\halliiihepff\halliiihepfg\halliiihepfh\halliiiheexa
\np\pigeonthmplusa\halliiihe\halliiihepfa\halliiihepfb\halliiihepfc\halliiihepfd\halliiihepfe\halliiihepff\halliiihepfg\halliiihepfh\halliiiheexa\halliiiheexb
\np\pigeonthmplusa
\np\pigeonthmplusa\pigeonthmexea
\np\pigeonthmplusa\pigeonthmexea\pigeonthmexeb
\np\pigeonthmplusa\pigeonthmexea\pigeonthmexeb\pigeonthmexec
\np\pigeonthmplusa\pigeonthmexea\pigeonthmexeb\pigeonthmexec\pigeonthmexed
\np\pigeonthmplusa\pigeonthmexea\pigeonthmexeb\pigeonthmexec\pigeonthmexed\pigeonthmexee
\np\pigeonthmplusa\pigeonthmexea\pigeonthmexeb\pigeonthmexec\pigeonthmexed\pigeonthmexee\pigeonthmexef
\np\pigeonthmplusa\pigeonthmexeav\pigeonthmexeb\pigeonthmexec\pigeonthmexed\pigeonthmexee\pigeonthmexef\pigeonthmexeg
\np\pigeonthmplusa\pigeonthmexeav\pigeonthmexeb\pigeonthmexec\pigeonthmexed\pigeonthmexee\pigeonthmexef\pigeonthmexeg\pigeonthmexeh
\np\pigeonthmplusa\pigeonthmexeav\pigeonthmexeb\pigeonthmexec\pigeonthmexed\pigeonthmexee\pigeonthmexef\pigeonthmexeg\pigeonthmexeh\pigeonthmexej
\np\pigeonthmstrong
\np\pigeonthmstrong\pigeonthmexma
\np\pigeonthmstrong\pigeonthmexma\pigeonthmexmb{}
\np\pigeonthmstrong\pigeonthmexma\pigeonthmexmb{\pigeonthmexabb}
\np\pigeonthmstrong\pigeonthmexma\pigeonthmexmb{\pigeonthmexabb\pigeonthmexabc}
\np\pigeonthmstrong\pigeonthmexma\pigeonthmexmb{\pigeonthmexabbii\pigeonthmexabc}
\np\pigeonthmstrong\pigeonthmexma\pigeonthmexmb{\pigeonthmexaba\pigeonthmexabbii\pigeonthmexabc}
%\np\pigeonthmstrong\pigeonthmexma\pigeonthmexmb{\pigeonthmexaba\pigeonthmexabbii\pigeonthmexabc}

\end{document}


