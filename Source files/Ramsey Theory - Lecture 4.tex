%%  MATH5505 Ramsey Theory Lecture 4: Equations
%%
%%  by Thomas Britz 2018S1
%%

% Equations: Schur's Theorem, Rado's Theorem, Folkman's Theorem, Hindman's Theorem

%   Refresh
%   Schur's Theorem   (prove)
%   Corollary         (prove)
%   Rado's Theorem    (just mention)
%   Folkman's Theorem (prove if there's room)
%   Hindman's Theorem etc. (mention)
%


\documentclass[12pt,a4paper,landscape]{article}
\special{landscape}

\usepackage{latexsym,amsfonts,amsmath,amssymb} %,calc,fancybox,epsfig,amscd,tabularx}
\usepackage{pstricks,pst-plot,pst-node,pst-tree}
%\usepackage{array,graphicx,epsfig}
%\usepackage{multicol,multirow,hyperref,rotating}
%\usepackage[T1]{fontenc}
%\usepackage{yfonts}


\mag=\magstep4

\newrgbcolor{green}{0.2 0.6 0.2}
\newrgbcolor{darkgreen}{0 0.4 0}
\newrgbcolor{graydarkgreen}{0.5 0.67 0.5}
\newrgbcolor{lightgreen}{0 0.75 0}
\newrgbcolor{skyblue}{0.5 0.80 1}
\newrgbcolor{darkblue}{0 0 0.6}
\newrgbcolor{darkred}{0.6 0 0}
\newrgbcolor{halfred}{.8 0 0}
\newrgbcolor{white}{1 1 1}
\newrgbcolor{nearlywhite}{0.95 0.95 0.95}
\newrgbcolor{offwhite}{0.9 0.9 0.9}
\newrgbcolor{lightgray}{0.85 0.85 0.85}
\newrgbcolor{halfgray}{0.8 0.8 0.8}
\newrgbcolor{altgray}{0.67 0.67 0.67}
\newrgbcolor{darkyellow}{1 0.94 0.15}
\newrgbcolor{halflightyellow}{1 1 0.4}
\newrgbcolor{lightyellow}{1 1 0.7}
\newrgbcolor{gold}{0.96 0.96 0.1}
\newrgbcolor{lightblue}{0.5 0.5 1}
\newrgbcolor{amber}{1 0.75 0}
\newrgbcolor{hotpink}{1 0.41 0.71}

\setlength{\textheight}{18.0truecm}
\setlength{\textwidth}{26.0truecm}
\setlength{\hoffset}{-12.0truecm}
\setlength{\voffset}{-5.2truecm}

\parindent 0in

%\setlength{\bigskipamount}{5ex plus1.5ex minus 2ex}
%\setlength{\parindent}{0cm}
%\setlength{\parskip}{0.2cm}

\psset{unit=6mm,linewidth=.06,dotscale=1.5,fillcolor=white,fillstyle=none,
 linecolor=gray,framearc=.3,shadowcolor=offwhite,shadow=true,shadowsize=.125,
 shadowangle=-45,dash=7pt 5pt}
%\psset{unit=.4,linecolor=gray,fillcolor=offwhite,shadowsize=.2,framearc=.3}
%\psset{unit=10mm,linewidth=.03}

\def\dedge{\ncline[linestyle=dashed]}

\def\np{\newpage}
\def\bl{\blue}
\def\bk{\black}
\def\wh{\white}
\def\rd{\red}
\def\lg{\lightgray}
\def\gr{\green}
\def\dr{\darkred}
\def\dg{\darkgreen}
\def\gdg{\graydarkgreen}
\def\dgy{\darkgray}
\def\ec{\dg}

\newcommand{\ora}[1]{\overrightarrow{#1}}

\newcommand{\llb}{\\[1mm]}
\newcommand{\lb}{\\[3mm]}
\newcommand{\blb}{\\[5mm]}
\newcommand{\hlb}{\\[48mm]}

\newcommand{\mynewpage}{\newpage\vspace*{-10mm}}

\newcommand{\vc}[1]{\begin{pmatrix}#1\end{pmatrix}}

\newcommand{\mytext}[1]{\text{\black#1\blue}}

\newcommand{\qbinom}[2]{\genfrac{[}{]}{0pt}{}{#1}{#2}}

\newcommand{\dl}{\psset{linestyle=dashed,linecolor=altgray,shadow=false}}
%\newcommand{\dl}{\psset{linestyle=dashed,linecolor=blue}}

\newcommand{\myframe}[1]{\pspicture(0,0)(0,0)\psset{unit=1cm,
 shadowcolor=offwhite,shadow=true,shadowangle=-45,linewidth=.03,linecolor=gray,
 fillcolor=lightgray,shadowsize=.15,framearc=.3}\psframe#1\endpspicture}

\newcommand{\mypicture}[1]{\pspicture(0,0)(0,0)\psset{unit=1cm,
 shadowcolor=offwhite,shadow=true,shadowangle=-45,linewidth=.03,linecolor=gray,
 fillcolor=lightgray,shadowsize=.15,framearc=.3}#1\endpspicture}

\newcommand{\mymatrix}[1]{{\psset{unit=4mm,linewidth=.03,linecolor=gray,fillstyle=solid,fillcolor=offwhite,shadow=false,framearc=0}\pspicture(0,0)(6,4)
  \psframe(0,0)(7,5)#1\psframe[linecolor=gray,fillcolor=offwhite,linewidth=.03,fillstyle=none,framearc=0](0,0)(7,5)\endpspicture}}

\newcommand{\mypair}[2]{{\psset{unit=4mm,linewidth=.03,linecolor=gray,fillstyle=solid,fillcolor=offwhite,shadow=false,framearc=0}\pspicture(0,.2)(2,.8)
  \psframe(0,0)(2,1)\darkgray\rput(.5,.5){#1}\rput(1.5,.5){#2}\psframe[linecolor=gray,fillcolor=offwhite,linewidth=.03,fillstyle=none,framearc=0](0,0)(2,1)\endpspicture}}

\newcommand{\mysixtuple}[6]{{\psset{unit=4mm,linewidth=.03,linecolor=gray,fillstyle=solid,fillcolor=offwhite,shadow=false,framearc=0}\pspicture(0,.2)(6,.8)
  \psframe(0,0)(6,1)\darkgray\rput(.5,.5){#1}\rput(1.5,.5){#2}\rput(2.5,.5){#3}\rput(3.5,.5){#4}\rput(4.5,.5){#5}\rput(5.5,.5){#6}
  \psframe[linecolor=gray,fillcolor=offwhite,linewidth=.03,fillstyle=none,framearc=0](0,0)(6,1)\endpspicture}}

\newcommand{\acset}{\psset{fillstyle=none,linecolor=red}}
\newcommand{\drr}{\psset{linecolor=darkred,fillcolor=red,linewidth=.034}}
\newcommand{\dbb}{\psset{linecolor=darkblue,fillcolor=blue,linewidth=.034}}
\newcommand{\bbl}{\psset{linecolor=blue,fillcolor=lightblue}}
\newcommand{\dbhg}{\psset{linecolor=darkblue,fillcolor=halfgray}}
\newcommand{\bhg}{\psset{linecolor=blue,fillcolor=halfgray}}
\newcommand{\rhg}{\psset{linecolor=red,fillcolor=halfgray}}
\newcommand{\ghg}{\psset{linecolor=green,fillcolor=halfgray}}
\newcommand{\dgg}{\psset{linecolor=darkgreen,fillcolor=green,linewidth=.034}}

\newcommand{\mychain}{\pspicture(0,0)(0,0)\dbb
  \psset{shadowsize=.2,shadow=true,shadowcolor=offwhite,shadowangle=-45,shadowsize=.15}
  \pscircle(0,0){.09}\pscircle(0,1){.09}\pscircle(0,2){.09}\pscircle(0,3){.09}
  \psline(0,.125)(0,.875)\psline(0,1.125)(0,1.875)\psline(0,2.125)(0,2.875)\endpspicture}

\newcommand{\myantichain}{\pspicture(0,0)(0,0)\drr
  \psset{shadowsize=.2,shadow=true,shadowcolor=offwhite,shadowangle=-45,shadowsize=.15}
  \pscircle(0,0){.09}\pscircle(1,0){.09}\pscircle(2,0){.09}\pscircle(3,0){.09}\endpspicture}

\newcommand{\mybullet}{{\psset{unit=6mm,dotscale=1.5,linewidth=.05,fillcolor=black,linecolor=black,framearc=.3,shadow=true}
  \pspicture(-.3,0)(.8,.4)\qdisk(.25,.2){.1}\pscircle*[linecolor=gray](.25,.2){.07}\endpspicture}}

\newcommand{\mysmallbullet}{{\psset{unit=4mm,dotscale=1.5,linewidth=.05,fillcolor=black,linecolor=black,framearc=.3,shadow=true}
  \pspicture(-.3,0)(.8,.4)\qdisk(.25,.2){.1}\pscircle*[linecolor=gray](.25,.2){.07}\endpspicture}}

\newcommand{\mybulletv}{{\psset{unit=6mm,dotscale=1.5,linewidth=.04,linecolor=black,fillstyle=solid,
  fillcolor=gray,shadow=false}\pspicture(0,0)(.5,.4)\pscircle(.25,.2){.15}\endpspicture}}

\newcommand{\mystar}{\pspicture(0,0)(0,0)\psset{unit=1cm}\gold\large$\star$\endpspicture
  \pspicture(0,0)(0,0)\psset{unit=1cm}\rput(-.23,.193){\small\white$\star$}\endpspicture}

\newcommand{\mystep}{{\psset{unit=6mm,dotscale=1.5,linewidth=.06,linecolor=darkgreen,fillstyle=solid,
  fillcolor=lightgreen,shadow=false}\pspicture(-.3,0)(.8,.4)\pscircle(.25,.2){.15}\endpspicture}}
\newcommand{\mystepv}{{\psset{unit=6mm,dotscale=1.5,linewidth=.06,linecolor=darkgreen,fillstyle=solid,
  fillcolor=lightgreen,shadow=false}\pspicture(0,0)(.5,.4)\pscircle(.25,.2){.15}\endpspicture}}
\newcommand{\mystepvv}{{\psset{unit=6mm,dotscale=1.5,linewidth=.06,linecolor=darkred,fillstyle=solid,
  fillcolor=red,shadow=false}\pspicture(0,0)(.5,.4)\pscircle(.25,.2){.15}\endpspicture}}
\newcommand{\mystepvvv}{{\psset{unit=6mm,dotscale=1.5,linewidth=.06,linecolor=darkred,fillstyle=solid,
  fillcolor=red,shadow=false}\pspicture(0,0)(.5,.4)\endpspicture}}

\newcommand{\myi}{{\psset{unit=6mm,dotscale=1.5,linewidth=.05,linecolor=darkgreen,fillstyle=none,shadow=false}%
  \pspicture(-.3,0)(.8,.4)\pscircle(.25,.2){.25}\rput(.25,.2){\tiny\dg1}\endpspicture}}
\newcommand{\myii}{{\psset{unit=6mm,dotscale=1.5,linewidth=.05,linecolor=darkgreen,fillstyle=none,shadow=false}%
  \pspicture(-.3,0)(.8,.4)\pscircle(.25,.2){.25}\rput(.25,.2){\tiny\dg2}\endpspicture}}
\newcommand{\myiii}{{\psset{unit=6mm,dotscale=1.5,linewidth=.05,linecolor=darkgreen,fillstyle=none,shadow=false}%
  \pspicture(-.3,0)(.8,.4)\pscircle(.25,.2){.25}\rput(.25,.2){\tiny\dg3}\endpspicture}}
\newcommand{\myiv}{{\psset{unit=6mm,dotscale=1.5,linewidth=.05,linecolor=darkgreen,fillstyle=none,shadow=false}%
  \pspicture(-.3,0)(.8,.4)\pscircle(.25,.2){.25}\rput(.25,.2){\tiny\dg4}\endpspicture}}
\newcommand{\myv}{{\psset{unit=6mm,dotscale=1.5,linewidth=.05,linecolor=darkgreen,fillstyle=none,shadow=false}%
  \pspicture(-.3,0)(.8,.4)\pscircle(.25,.2){.25}\rput(.25,.2){\tiny\dg5}\endpspicture}}
\newcommand{\myvi}{{\psset{unit=6mm,dotscale=1.5,linewidth=.05,linecolor=darkgreen,fillstyle=none,shadow=false}%
  \pspicture(-.3,0)(.8,.4)\pscircle(.25,.2){.25}\rput(.25,.2){\tiny\dg6}\endpspicture}}
\newcommand{\myvii}{{\psset{unit=6mm,dotscale=1.5,linewidth=.05,linecolor=darkgreen,fillstyle=none,shadow=false}%
  \pspicture(-.3,0)(.8,.4)\pscircle(.25,.2){.25}\rput(.25,.2){\tiny\dg7}\endpspicture}}
\newcommand{\myviii}{{\psset{unit=6mm,dotscale=1.5,linewidth=.05,linecolor=darkgreen,fillstyle=none,shadow=false}%
  \pspicture(-.3,0)(.8,.4)\pscircle(.25,.2){.25}\rput(.25,.2){\tiny\dg8}\endpspicture}}
\newcommand{\myix}{{\psset{unit=6mm,dotscale=1.5,linewidth=.05,linecolor=darkgreen,fillstyle=none,shadow=false}%
  \pspicture(-.3,0)(.8,.4)\pscircle(.25,.2){.25}\rput(.25,.2){\tiny\dg9}\endpspicture}}

\newcommand{\mywo}{{\psset{unit=6mm,dotscale=1.5,linewidth=.05,linecolor=darkgreen,fillcolor=white,fillstyle=solid,shadow=false}%
  \pspicture(-.3,0)(.8,.4)\pscircle(.25,.2){.25}\rput(.25,.2){\tiny\dg0}\endpspicture}}
\newcommand{\mywi}{{\psset{unit=6mm,dotscale=1.5,linewidth=.05,linecolor=darkgreen,fillcolor=white,fillstyle=solid,shadow=false}%
  \pspicture(-.3,0)(.8,.4)\pscircle(.25,.2){.25}\rput(.25,.2){\tiny\dg1}\endpspicture}}
\newcommand{\mywii}{{\psset{unit=6mm,dotscale=1.5,linewidth=.05,linecolor=darkgreen,fillcolor=white,fillstyle=solid,shadow=false}%
  \pspicture(-.3,0)(.8,.4)\pscircle(.25,.2){.25}\rput(.25,.2){\tiny\dg2}\endpspicture}}

\newcommand{\myO}{{\psset{unit=6mm,dotscale=1.5,linewidth=.06,linecolor=darkgreen,fillstyle=none,
  shadow=false}\pspicture(0,0)(0,0)\pscircle(-.2,.225){.4}\endpspicture}}

\newcommand{\myspace}{\psset{unit=6mm,dotscale=1.5,linewidth=.06}\pspicture(-.3,0)(.8,.4)\endpspicture}

\newcommand{\mydot}{\pscircle(0,0){.1}}
\newcommand{\mydotv}{\pscircle*[linecolor=gray](0,0){.1}\pscircle*[linecolor=blue](0,0){.06}}

\newcommand{\myhexagon}[2]{\psset{fillcolor=halfgray,linecolor=gray,shadow=false,fillstyle=none,linewidth=.06}
  \pspicture(-3.5,-2.6)(3,2.6)
    \pnode(-3  , 0  ){a}
    \pnode(-1.5, 2.6){b}
    \pnode( 1.5, 2.6){c}
    \pnode( 3  , 0  ){d}
    \pnode( 1.5,-2.6){e}
    \pnode(-1.5,-2.6){f}
    \pspolygon(a)(b)(c)(d)(e)(f)#1\psset{fillstyle=solid}
    \pscircle(a){.2}
    \pscircle(b){.2}
    \pscircle(c){.2}
    \pscircle(d){.2}
    \pscircle(e){.2}
    \pscircle(f){.2}#2
  \endpspicture}

\newcommand{\myksix}[2]{\myhexagon{\pspolygon(a)(c)(e)\pspolygon(b)(d)(f)\psline(a)(d)\psline(b)(e)\psline(c)(f)#1}{#2}}

\newcommand{\lss}{$\spadesuit$}
\newcommand{\lhs}{$\darkred\heartsuit$}
\newcommand{\lds}{$\darkred\diamondsuit$}
\newcommand{\lcs}{$\clubsuit$}

\newcommand{\hatmanv}{{%
 \psset{dotscale=1.5,dotsize=0.09,linewidth=.03,fillcolor=black,linecolor=black,shadow=false}
 \psline(0,0)(0.25,0)\psline(1.75,0)(2,0)\psline(0.25,0)(1,1.2)
 \psline(1.75,0)(1,1.2)\psline(1,1.2)(1,2.5)\psline(0.25,1.75)(1,2)
 \psline(1.75,1.75)(1,2)\pscircle(1,3){0.5}\psarc(1,3){.25}{-140}{-40}
 \psdots(.85,3.15)(1.15,3.15)\psset{fillstyle=solid}\psframe(.5,3.35)(1.5,3.45)
 \psframe(.7,3.35)(1.3,4.1)}}

\newcommand{\manv}{{%
 \psset{dotscale=1.5,dotsize=0.09,linewidth=.03,fillcolor=black,linecolor=black,shadow=false}%
 \pspicture(0,0)(2,3.5)%\psline(0,0)(0.25,0)\psline(1.75,0)(2,0)
 \psline(0.25,0)(1,1.2)
 \psline(1.75,0)(1,1.2)\psline(1,1.2)(1,2.5)\psline(0.25,1.75)(1,2)
 \psline(1.75,1.75)(1,2)\pscircle(1,3){0.5}\psarc(1,3){.25}{-140}{-40}
 \psdots(.85,3.15)(1.15,3.15)%\psset{fillstyle=solid}\psframe(.5,3.35)(1.5,3.45)\psframe(.7,3.35)(1.3,4.1)
 \endpspicture}}

\newcommand{\sadmanv}{{
 \psset{dotscale=1.5,dotsize=0.09,linewidth=.03,fillcolor=black,linecolor=black,shadow=false}
 \psline(0,0)(0.25,0)\psline(1.75,0)(2,0)\psline(0.25,0)(1,1.2)
 \psline(1.75,0)(1,1.2)\psline(1,1.2)(1,2.5)\psline(0.25,1.75)(1,2)
 \psline(1.75,1.75)(1,2)\pscircle(1,3){0.5}\psarc(1,2.65){.25}{40}{140}
 \psdots(.87,3.15)(1.13,3.15)\psset{fillstyle=solid}\psframe(.5,3.35)(1.5,3.45)
 \psframe(.7,3.35)(1.3,4.1)}}

\newcommand{\womanv}{{
 \psset{dotscale=1.5,dotsize=0.09,linewidth=.03,fillcolor=black,linecolor=black,shadow=false}
 \pspicture(0,0)(2,3.5)%
 \psline(0.5,0)(0.75,0)\psline(1.5,0)(1.25,0)\psline(0.75,0)(0.75,0.5)
 \psline(1.25,0)(1.25,0.5)\psline(1,2)(1,2.35)\psline(0.25,1.75)(1,2)
 \psline(1.75,1.75)(1,2)\pscircle(1,2.85){0.5}
 \psarc(1,2.85){.25}{-140}{-40}\psdots(.85,2.95)(1.15,2.95)
 \psset{linecolor=brown,linewidth=.1}
 \psarc(1,2.85){.5}{0}{180}\psarc(1.75,2.85){.25}{180}{270}
 \psarc(0.25,2.85){.25}{-90}{0}
 \pstriangle[linecolor=red,fillstyle=solid,fillcolor=red](1,.5)(1.2,1.75)\endpspicture}}

\newcommand{\spyv}{{%
 \psset{dotscale=1.5,dotsize=0.09,linewidth=.03,fillcolor=black,linecolor=black,shadow=false}
 \pspicture(0,0)(2,3.5)%\psline(0,0)(0.25,0)\psline(1.75,0)(2,0)
 \psline(0.25,0)(1,1.2)
 \psline(1.75,0)(1,1.2)\psline(1,1.2)(1,2.5)\psline(0.25,1.75)(1,2)
 \psline(1.75,1.75)(1,2)\pscircle(1,3){0.5}\psarc(1,3){.25}{-140}{-40}
 %\psdots(.85,3.15)(1.15,3.15)
 \psset{fillstyle=solid}
 \psframe( .725,3.075)( .975,3.225)
 \psframe(1.025,3.075)(1.275,3.225)
 \psellipse*(1,3.5)(.5,.14)
 \pspolygon(.6,3.5)(1.4,3.5)( .8,3.8)
 \pspolygon(.6,3.5)(1.4,3.5)(1.2,3.8)
 \endpspicture}}

\newcommand{\puzzledspyv}{{%
 \psset{dotscale=1.5,dotsize=0.09,linewidth=.03,fillcolor=black,linecolor=black,shadow=false}
 \pspicture(0,0)(2,3.5)%\psline(0,0)(0.25,0)\psline(1.75,0)(2,0)
 \psline(0.25,0)(1,1.2)
 \psline(1.75,0)(1,1.2)\psline(1,1.2)(1,2.5)\psline(0.25,1.75)(1,2)
 \psline(1.75,1.75)(1,2)\pscircle(1,3){0.5}
 % Mouth
 \psline(.83,2.8)(1.17,2.8)
 %\psdots(.85,3.15)(1.15,3.15)
 \psset{fillstyle=solid}
 \psframe( .725,3.075)( .975,3.225)
 \psframe(1.025,3.075)(1.275,3.225)
 \psellipse*(1,3.5)(.5,.14)
 \pspolygon(.6,3.5)(1.4,3.5)( .8,3.8)
 \pspolygon(.6,3.5)(1.4,3.5)(1.2,3.8)
 \endpspicture}}

\newcommand{\man}{\psset{unit=4mm}\manv}
\newcommand{\hatman}{\psset{unit=4mm}\hatmanv}
\newcommand{\sadman}{\psset{unit=4mm}\sadmanv}
\newcommand{\woman}{\psset{unit=4mm}\womanv}
\newcommand{\spy}{\psset{unit=4mm}\spyv}
\newcommand{\puzzledspy}{\psset{unit=4mm}\puzzledspyv}


\newcommand{\coursetitle}{\vphantom{ }\vspace{7mm}\begin{center}
    {\large\darkgreen MATH5505 \:\: Combinatorics}\\[2.5mm]
    UNSW 2018S1
    \end{center}}

\newcounter{gra}

\DeclareMathAlphabet{\mathscr}{OT1}{pzc}%
                                 {m}{it}

\newcommand{\ph}{\phantom}
\newcommand{\ds}{\displaystyle}
\newcommand{\mra}{\black\rightsquigarrow\blue}
\newcommand{\tvs}{\textvisiblespace}
\newcommand{\msp}{\,}
\newcommand{\mba}{\,\mypicture{\psset{shadow=false}\psline(-.04,-.15)(-.04,.4)}}
\newcommand{\mbq}{\,\mypicture{\psset{shadow=false}\psline(-.04,-.15)(-.04,.4)\rput(-0.04,0.7){{\darkgray?}}}}
\newcommand{\meq}{\black=\blue}
\newcommand{\msim}{\black\sim\blue}
\newcommand{\mequiv}{\black\equiv\blue}
\newcommand{\mapprox}{\black\approx\blue}
\newcommand{\mneq}{\red\neq\blue}
\newcommand{\mleq}{\black\leq\blue}
\newcommand{\mgeq}{\black\geq\blue}
\newcommand{\mgt}{\black>\blue}
\newcommand{\mlt}{\black<\blue}
\newcommand{\mto}{\black\to\blue}
\newcommand{\msubseteq}{\black\subseteq\blue}
\newcommand{\mnsubseteq}{\red\not\subseteq\blue}
\newcommand{\mequ}{\black\equiv\blue}
\newcommand{\mnequ}{\red\not\equiv\blue}
\newcommand{\myin}{\black\in\blue}
\newcommand{\mypl}{\black+\blue}
\newcommand{\mdef}{\black:=\blue}
\newcommand{\mnotin}{\red\notin\blue}
\newcommand{\mynotin}{\red\notin\blue}
\newcommand{\mywhere}{\quad\text{\black where\blue}\quad}
\newcommand{\myand}{\quad\text{\black and\blue}\quad}
\newcommand{\dnd}{\black\mathchoice{\mathrel{{\kern0.1em|\kern-0.4em/}}}
  {\mathrel{{\kern0.1em|\kern-0.4em/}}}{\mathrel{{\kern0.1em|\kern-0.33em/}}}
  {\mathrel{{\kern0.1em|\kern-0.2em/}}}\blue}
\newcommand{\mdiv}{\black\,|\,\blue}
\newcommand{\mymod}[1]{\:(\textrm{mod}\,#1)}
\newcommand{\mask}[1]{}
\newcommand{\ord}{\textrm{ord}\,}
\newcommand{\ns}{\negthickspace\negthickspace}
\newcommand{\nns}{\negthickspace\negthickspace\negthickspace\negthickspace}
\newcommand{\lns}{\hspace*{-.3mm}}
\newcommand{\ri}{\,i\,}% \,\mathrm{i}}
\newcommand{\di}{\mbox{$\dg\bullet$}}
\newcommand{\dah}{\mbox{$\dg\mathbf{-}$}}
\newcommand{\tp}{\mbox{{\dg\tt p}}}
\newcommand{\Var}{\mathrm{Var}}
\newcommand{\Arg}{\mathrm{Arg}}
\renewcommand{\Re}{\mathrm{Re}}
\renewcommand{\Im}{\mathrm{Im}}
\newcommand{\bbN}{\blue\mathbb{N}}
\newcommand{\bbZ}{\blue\mathbb{Z}}
\newcommand{\bbQ}{\blue\mathbb{Q}}
\newcommand{\bbR}{\blue\mathbb{R}}
\newcommand{\bbC}{\blue\mathbb{C}}
\newcommand{\bbF}{\blue\mathbb{F}}
\newcommand{\bbP}{\blue\mathbb{P}}
%\newcommand{\calR}{\blue\mathcal{R}}
%\newcommand{\calC}{\blue\mathcal{C}}
\renewcommand{\vec}[1]{\mathbf{\blue #1}}
%\newcommand{\pv}[1]{{\blue\begin{pmatrix}#1\end{pmatrix}}}
%\newcommand{\pvn}[1]{{\begin{pmatrix}#1\end{pmatrix}}}
%\newcommand{\spv}[1]{{\blue\left(\begin{smallmatrix}#1\end{smallmatrix}\right)}}
%\newcommand{\mpv}[1]{{\blue\Biggl(\!\!\begin{array}{r}#1\end{array}\!\!\Biggr)}}
%\newcommand{\augmv}[1]{{\left(\begin{array}{rr|r}#1\end{array}\right)}}
%\newcommand{\taugmv}[1]{{\left(\begin{array}{rrr|r}#1\end{array}\right)}}
%\newcommand{\qaugmv}[1]{{\left(\begin{array}{rrrr|r}#1\end{array}\right)}}
%\newcommand{\augm}[1]{{\blue\left(\begin{array}{rr|r}#1\end{array}\right)}}
%\newcommand{\daugm}[1]{{\blue\left(\begin{array}{rr|rr}#1\end{array}\right)}}
%\newcommand{\taugm}[1]{{\blue\left(\begin{array}{rrr|r}#1\end{array}\right)}}
%\newcommand{\traugm}[1]{{\blue\left(\begin{array}{rrr|rrr}#1\end{array}\right)}}
%\newcommand{\qaugm}[1]{{\blue\left(\begin{array}{rrrr|r}#1\end{array}\right)}}
%\newcommand{\qtaugm}[1]{{\blue\left(\begin{array}{rrrr|rrr}#1\end{array}\right)}}
%\newcommand{\mydet}[1]{{\blue\left|\begin{matrix}#1\end{matrix}\right|}}
%\newcommand{\mydets}[1]{{\blue\Bigl|\begin{matrix}#1\end{matrix}\Bigr|}}
%\newcommand{\arr}[1]{\overrightarrow{#1}}
\newcommand{\lcm}{\mathrm{lcm}}
\renewcommand{\mod}{\;\mathrm{mod}\;}
%\newcommand{\proj}{\mathrm{proj}}
%\newcommand{\id}{\blue\mathrm{id}}
%\newcommand{\im}{\blue\mathrm{Im}}
%\newcommand{\re}{\blue\mathrm{Re}}
\newcommand{\col}{\blue\mathrm{col}}
\newcommand{\nullity}{\blue\mathrm{nullity}}
\newcommand{\rank}{\blue\mathrm{rank}}
\newcommand{\spn}[1]{\blue\mathrm{span}\left\{#1\right\}}
\newcommand{\spnv}{\blue\mathrm{span}\,}
%\newcommand{\diag}[1]{\blue\mathrm{diag}\left(#1\right)\,}
%\newcommand{\satop}[2]{\stackrel{\scriptstyle{#1}}{\scriptstyle{#2}}}
%\newcommand{\llnot}{\sim\!}
\newcommand{\qed}{\hfill$\Box$}
%\newcommand{\rbullet}{\includegraphics[width=4mm]{red-bullet-on-white.ps}}
%\newcommand{\gbullet}{\includegraphics[width=3mm]{green-bullet-on-white.ps}}
%\newcommand{\ybullet}{\includegraphics[width=3mm]{yellow-bullet-on-white.ps}}
\newcommand{\vsp}{{\psset{unit=4.2mm}\begin{pspicture}(0,1)(0,0)\end{pspicture}}}
\renewcommand{\emptyset}{\varnothing}
\newcommand{\mq}{\text{\red ?}}
%\renewcommand{\emph}[1]{{\blue\textsl{#1}}}

%\newcommand{\shat}{\begin{pspicture}(0,0)(0,0)\psset{linecolor=black,linewidth=.075}\psarc*(.67,3.18){.25}{45}{225}\psline(.35,2.85)(1,3.5)\end{pspicture}}
%\newcommand{\rhat}{\begin{pspicture}(0,0)(0,0)\psset{linecolor=red,linewidth=.05}\psarc*(.675,3.175){.25}{45}{225}\psline(.35,2.85)(1,3.5)\end{pspicture}}
%\newcommand{\bhat}{\begin{pspicture}(0,0)(0,0)\psset{linecolor=blue,linewidth=.05}\psarc*(.675,3.175){.25}{45}{225}\psline(.35,2.85)(1,3.5)\end{pspicture}}
%\newcommand{\sdress}{\begin{pspicture}(0,0)(0,0)\pstriangle*[linecolor=black,linewidth=.05](1,.5)(1.2,1.75)\end{pspicture}}
%\newcommand{\rdress}{\begin{pspicture}(0,0)(0,0)\pstriangle*[linecolor=red,linewidth=.05](1,.5)(1.2,1.75)\end{pspicture}}
%\newcommand{\bdress}{\begin{pspicture}(0,0)(0,0)\pstriangle*[linecolor=brown,linewidth=.05](1,.5)(1.2,1.75)\end{pspicture}}
%\newcommand{\ydress}{\begin{pspicture}(0,0)(0,0)\pstriangle*[linecolor=amber,linewidth=.05](1,.5)(1.2,1.75)\end{pspicture}}
%\newcommand{\gdress}{\begin{pspicture}(0,0)(0,0)\pstriangle*[linecolor=green,linewidth=.05](1,.5)(1.2,1.75)\end{pspicture}}
%\newcommand{\sshoes}{\begin{pspicture}(0,0)(0,0)\psset{linecolor=black,linewidth=.05}
% \psline(.6,0)(.85,0.125)\psline(0.85,0)(0.85,0.5)\psline(1.15,0)(1.15,0.5)\psline(1.4,0)(1.15,0.125)\end{pspicture}}
%\newcommand{\rshoes}{\begin{pspicture}(0,0)(0,0)\psset{linecolor=black,linewidth=.05}
%  \psline(0.85,0)(0.85,0.5)\pscircle*[linecolor=red](.65,.05){.05}\psframe*[linecolor=red](.65,0)(.90,.1)
%  \psline(1.15,0)(1.15,0.5)\psframe*[linecolor=red](1.1,0)(1.35,0.1)\pscircle*[linecolor=red](1.35,.05){.05}\end{pspicture}}
%
%\newcommand{\woman}[1]{\begin{pspicture}(0.25,0)(1.75,3.4)
% \psset{unit=4mm,dotsize=0.09,linewidth=.03,fillcolor=black,linecolor=black,shadow=false}
% \psline(1,2)(1,2.35)\psline(0.25,1.75)(1,2)\psline(1.75,1.75)(1,2)\pscircle(1,2.85){0.5}
% \psarc(1,2.85){.25}{-140}{-40}\psdots(.85,2.95)(1.15,2.95)
% \psset{linecolor=brown,linewidth=.1}
% \psarc(1,2.85){.5}{0}{180}\psarc(1.75,2.85){.25}{180}{270}\psarc(.25,2.85){.25}{-90}{0}
% \psset{linewidth=.06}#1
% \end{pspicture}}

\newcommand{\dicei}{{\psset{unit=6mm,shadow=false,linewidth=.05}\begin{pspicture}(0,0.3)(1,1.3)\begin{psframe}(0,0)(1,1)\qdisk(0.5,0.5){0.1\psunit}\end{psframe}\end{pspicture}}}
\newcommand{\diceii}{{\psset{unit=6mm,shadow=false,linewidth=.05}\begin{pspicture}(0,0.3)(1,1.3)\begin{psframe}(0,0)(1,1)\qdisk(0.2,0.2){0.1\psunit}\qdisk(0.8,0.8){0.1\psunit}\end{psframe}\end{pspicture}}}
\newcommand{\diceiii}{{\psset{unit=6mm,shadow=false,linewidth=.05}\begin{pspicture}(0,0.3)(1,1.3)\begin{psframe}(0,0)(1,1)\qdisk(0.2,0.2){0.1\psunit}\qdisk(0.5,0.5){0.1\psunit}\qdisk(0.8,0.8){0.1\psunit}\end{psframe}\end{pspicture}}}
\newcommand{\diceiv}{{\psset{unit=6mm,shadow=false,linewidth=.05}\begin{pspicture}(0,0.3)(1,1.3)\begin{psframe}(0,0)(1,1)\qdisk(0.2,0.2){0.1\psunit}\qdisk(0.2,0.8){0.1\psunit}\qdisk(0.8,0.8){0.1\psunit}\qdisk(0.8,0.2){0.1\psunit}\end{psframe}\end{pspicture}}}
\newcommand{\dicev}{{\psset{unit=6mm,shadow=false,linewidth=.05}\begin{pspicture}(0,0.3)(1,1.3)\begin{psframe}(0,0)(1,1)\qdisk(0.2,0.2){0.1\psunit}\qdisk(0.2,0.8){0.1\psunit}\qdisk(0.5,0.5){0.1\psunit}\qdisk(0.8,0.8){0.1\psunit}\qdisk(0.8,0.2){0.1\psunit}\end{psframe}\end{pspicture}}}
\newcommand{\dicevi}{{\psset{unit=6mm,shadow=false,linewidth=.05}\begin{pspicture}(0,0.3)(1,1.3)\begin{psframe}(0,0)(1,1)\qdisk(0.2,0.2){0.1\psunit}\qdisk(0.2,0.5){0.1\psunit}\qdisk(0.2,0.8){0.1\psunit}\qdisk(0.8,0.8){0.1\psunit}\qdisk(0.8,0.5){0.1\psunit}\qdisk(0.8,0.2){0.1\psunit}\end{psframe}\end{pspicture}}}
\newcommand{\diceframe}{{\psset{unit=6mm,shadow=false,linewidth=.05}\begin{pspicture}(0,0.3)(1,1.3)\begin{psframe}(-.25,-.25)(1.25,1.25)\end{psframe}\end{pspicture}}}

\newcommand{\pigeon}{{\psset{xunit=6mm,yunit=6mm,runit=6mm,linewidth=.8pt,shadow=false,linecolor=darkgray,fillcolor=white,fillstyle=solid}
  \begin{pspicture}(0,0)(1.9, 1){\pscustom{\newpath
    \moveto(1.179, 0.088)
    \curveto(1.156, 0.239)(1.402, 0.255)(1.451, 0.464)
    \curveto(1.498, 0.668)(1.42 , 0.774)(1.75 , 0.812)
    \curveto(1.656, 0.887)(1.528, 0.997)(1.399, 0.979)
    \curveto(1.067, 0.929)(1.269, 0.646)(0.156, 0.241)
    \curveto(0    , 0.185)(0.869, 0.396)(1.037, 0.205)
    \curveto(1.201, 0.016)(0.902, 0    )(1.17 , 0.003)
    \curveto(1.37 , 0.005)(1.197, 0.003)(1.179, 0.088)
    \closepath}}
    \pscircle(1.447, 0.9){0.045}
  \end{pspicture}}}


%\newcommand{\dicegrid}[5]{\[\begin{pspicture}(-1,-1.2)(20,8)\psset{shadow=false,linecolor=darkgray}
%    \rput(0  ,0  ){\dicevi} \rput(0  ,1.2){\dicev} \rput(0  ,2.4){\diceiv}\rput(0  ,3.6){\diceiii}
%    \rput(0  ,4.8){\diceii} \rput(0  ,6  ){\dicei} \rput(1.5,7.55){\dicei} \rput(2.7,7.55){\diceii}
%    \rput(3.9,7.55){\diceiii}\rput(5.1,7.55){\diceiv}\rput(6.3,7.55){\dicev} \rput(7.5,7.55){\dicevi}
%    \psset{linecolor=gray}\psline( .75,-1)( .75,8)\psline(-.75, 6.5)(8.25,6.5)\psline(-.75,-1)(-.75,8)
%    \psline(-.75,8)(8.25,8)\psline(-.75,-1)(8.25,-1)\psline(8.25,-1)(8.25,8)\psline(-.75, 8)(.75,6.5)
%    \put(-.55,6.6){1}\put(.25,7.25){2}
%    \psset{linecolor=blue}#1\psset{linecolor=red}#2\put(9.5,4.3){{#3}}\put(9.5,3.1){{#4}}\put(9.5,1.9){{#5}}
%  \end{pspicture}\]}
%
%\newcommand{\bnaa}{\pscircle*(1.5,5.7){3pt}}\newcommand{\bnab}{\pscircle*(2.7,5.7){3pt}}\newcommand{\bnac}{\pscircle*(3.9,5.7){3pt}}\newcommand{\bnad}{\pscircle*(5.1,5.7){3pt}}\newcommand{\bnae}{\pscircle*(6.3,5.7){3pt}}\newcommand{\bnaf}{\pscircle*(7.5,5.7){3pt}}
%\newcommand{\bnba}{\pscircle*(1.5,4.5){3pt}}\newcommand{\bnbb}{\pscircle*(2.7,4.5){3pt}}\newcommand{\bnbc}{\pscircle*(3.9,4.5){3pt}}\newcommand{\bnbd}{\pscircle*(5.1,4.5){3pt}}\newcommand{\bnbe}{\pscircle*(6.3,4.5){3pt}}\newcommand{\bnbf}{\pscircle*(7.5,4.5){3pt}}
%\newcommand{\bnca}{\pscircle*(1.5,3.3){3pt}}\newcommand{\bncb}{\pscircle*(2.7,3.3){3pt}}\newcommand{\bncc}{\pscircle*(3.9,3.3){3pt}}\newcommand{\bncd}{\pscircle*(5.1,3.3){3pt}}\newcommand{\bnce}{\pscircle*(6.3,3.3){3pt}}\newcommand{\bncf}{\pscircle*(7.5,3.3){3pt}}
%\newcommand{\bnda}{\pscircle*(1.5,2.1){3pt}}\newcommand{\bndb}{\pscircle*(2.7,2.1){3pt}}\newcommand{\bndc}{\pscircle*(3.9,2.1){3pt}}\newcommand{\bndd}{\pscircle*(5.1,2.1){3pt}}\newcommand{\bnde}{\pscircle*(6.3,2.1){3pt}}\newcommand{\bndf}{\pscircle*(7.5,2.1){3pt}}
%\newcommand{\bnea}{\pscircle*(1.5, .9){3pt}}\newcommand{\bneb}{\pscircle*(2.7, .9){3pt}}\newcommand{\bnec}{\pscircle*(3.9, .9){3pt}}\newcommand{\bned}{\pscircle*(5.1, .9){3pt}}\newcommand{\bnee}{\pscircle*(6.3, .9){3pt}}\newcommand{\bnef}{\pscircle*(7.5, .9){3pt}}
%\newcommand{\bnfa}{\pscircle*(1.5,-.3){3pt}}\newcommand{\bnfb}{\pscircle*(2.7,-.3){3pt}}\newcommand{\bnfc}{\pscircle*(3.9,-.3){3pt}}\newcommand{\bnfd}{\pscircle*(5.1,-.3){3pt}}\newcommand{\bnfe}{\pscircle*(6.3,-.3){3pt}}\newcommand{\bnff}{\pscircle*(7.5,-.3){3pt}}
%
%\newcommand{\cnaa}{\pscircle(1.5,5.7){4.5pt}}\newcommand{\cnab}{\pscircle(2.7,5.7){4.5pt}}\newcommand{\cnac}{\pscircle(3.9,5.7){4.5pt}}\newcommand{\cnad}{\pscircle(5.1,5.7){4.5pt}}\newcommand{\cnae}{\pscircle(6.3,5.7){4.5pt}}\newcommand{\cnaf}{\pscircle(7.5,5.7){4.5pt}}
%\newcommand{\cnba}{\pscircle(1.5,4.5){4.5pt}}\newcommand{\cnbb}{\pscircle(2.7,4.5){4.5pt}}\newcommand{\cnbc}{\pscircle(3.9,4.5){4.5pt}}\newcommand{\cnbd}{\pscircle(5.1,4.5){4.5pt}}\newcommand{\cnbe}{\pscircle(6.3,4.5){4.5pt}}\newcommand{\cnbf}{\pscircle(7.5,4.5){4.5pt}}
%\newcommand{\cnca}{\pscircle(1.5,3.3){4.5pt}}\newcommand{\cncb}{\pscircle(2.7,3.3){4.5pt}}\newcommand{\cncc}{\pscircle(3.9,3.3){4.5pt}}\newcommand{\cncd}{\pscircle(5.1,3.3){4.5pt}}\newcommand{\cnce}{\pscircle(6.3,3.3){4.5pt}}\newcommand{\cncf}{\pscircle(7.5,3.3){4.5pt}}
%\newcommand{\cnda}{\pscircle(1.5,2.1){4.5pt}}\newcommand{\cndb}{\pscircle(2.7,2.1){4.5pt}}\newcommand{\cndc}{\pscircle(3.9,2.1){4.5pt}}\newcommand{\cndd}{\pscircle(5.1,2.1){4.5pt}}\newcommand{\cnde}{\pscircle(6.3,2.1){4.5pt}}\newcommand{\cndf}{\pscircle(7.5,2.1){4.5pt}}
%\newcommand{\cnea}{\pscircle(1.5, .9){4.5pt}}\newcommand{\cneb}{\pscircle(2.7, .9){4.5pt}}\newcommand{\cnec}{\pscircle(3.9, .9){4.5pt}}\newcommand{\cned}{\pscircle(5.1, .9){4.5pt}}\newcommand{\cnee}{\pscircle(6.3, .9){4.5pt}}\newcommand{\cnef}{\pscircle(7.5, .9){4.5pt}}
%\newcommand{\cnfa}{\pscircle(1.5,-.3){4.5pt}}\newcommand{\cnfb}{\pscircle(2.7,-.3){4.5pt}}\newcommand{\cnfc}{\pscircle(3.9,-.3){4.5pt}}\newcommand{\cnfd}{\pscircle(5.1,-.3){4.5pt}}\newcommand{\cnfe}{\pscircle(6.3,-.3){4.5pt}}\newcommand{\cnff}{\pscircle(7.5,-.3){4.5pt}}
%
\newcommand{\clearemptydoublepage}
  {\newpage{\pagestyle{empty}{\cleardoublepage}}}


% ------------------------------------------------------------------------
%  Environments
% ------------------------------------------------------------------------

\newcommand{\example}{{\sffamily\darkgreen Example }}
\newcommand{\exercise}{{\sffamily\darkgreen Exercise }}
\newcommand{\proof}{{\sffamily\darkgreen Proof }}
\newcommand{\notes}{{\sffamily\darkgreen Notes }}
\newcommand{\note}{{\sffamily\darkgreen Note }}
\newcommand{\theorem}{{\sffamily\darkgreen Theorem }}
\newcommand{\proposition}{{\sffamily\darkgreen Proposition }}
\newcommand{\corollary}{{\sffamily\darkgreen Corollary }}
\newcommand{\lemma}{{\sffamily\darkgreen Lemma }}
\newcommand{\definition}{{\sffamily\darkgreen Definition}}
\newcommand{\problem}{{\sffamily\darkgreen Problem}}
\newcommand{\remark}{{\sffamily\darkgreen Remark}}

% ------------------------------------------------------------------------

\renewcommand{\section}
  {\clearemptydoublepage\refstepcounter{section} \secdef \cmda \cmdb}
\newcommand{\cmda}[2][]
  {{\scriptsize{\textbf{MATH5505 \:\: Combinatorics}}
   \hfill{\scriptsize{\textsl{Thomas Britz}}}\vspace{0.1cm}\\
   {\bfseries\Large\S\arabic{section} \sffamily #2}}}
\newcommand{\cmdb}[1]{{\bfseries\huge\S\,\sffamily #1}}

\pagestyle{empty}

\newcommand{\frontpage}{}


\begin{document}
%% Very important: sans-serif font throughout
\sf

\newcommand{\lecturetitle}{\vphantom{ }\vspace{15mm}\begin{center}
  {\large\sc Ramsey Theory}\end{center}}

\newcommand{\lecturetitlei}{\vphantom{ }\vspace{15mm}\begin{center}
  {\large\sc Ramsey Theory}\\[3mm]
  \darkgreen Lecture 4: Equations\end{center}}

\newcommand{\pigeonthma}{{\dg The Pigeonhole Principle}
  \\[1mm]If $\bl k+1$ pigeons are put into $\bl k$ pigeonholes,\\
         then some pigeonhole contains at least two pigeons.}

\newcommand{\pigeonthmplusa}{{\dg The Pigeonhole Principle} {\gray (general)}
  \\[1mm]If $\bl km+1$ pigeons are put into $\bl k$ pigeonholes,\\
         then some pigeonhole contains at least $\bl m+1$ pigeons.}

\newcommand{\pigeonthmstrong}{{\dg The Pigeonhole Principle} {\gray (strong)}
  \\[1mm]If $\bl (n_1-1) +\cdots+(n_k-1) + 1$ pigeons are put into $\bl k$ pigeonholes,\\
         then some $\bl i$th pigeonhole contains at least $\bl n_i$ pigeons.}

\newcommand{\ramseyaa}{{\dg Ramsey Theory} makes repeated use of the {\dg Pigeonhole Principle}
to show that order exists whenever (certain) random and seemingly unordered structures are sufficiently large.}

\newcommand{\ramseyb}{{\dg Ramsey's Theorem} {\gray (1930)} {\halfgray (simple)}\\
   If $\bl k,m\myin\mathbb{N}$ and $\bl n$ is sufficiently large,
   then each $\bl k$-colouring of the edges of $\bl K_n$ gives a complete subgraph $\bl K_m$
   with monochromatic edges.}
\newcommand{\ramseyba}{\\[2mm]{\gray The least such $\lightblue n$ is denoted $\lightblue R(m;k)$.}}

\newcommand{\ramseynotaa}{\\[6mm]Define $\bl [n] \black :\meq \{1,\ldots,n\}$
                                 \; and \; $\bl\binom{S}{k} \black :\meq \{X\subseteq S\;:\; |X| \meq k\}$.}
\newcommand{\ramseynotaav}{Define $\bl [n] \black :\meq \{1,\ldots,n\}$
                                 \; and \; $\bl\binom{S}{k} \black :\meq \{X\subseteq S\;:\; |X| \meq k\}$.}

\newcommand{\ramseye}{{\dg Ramsey's Theorem} {\gray (1930)}\\
   If $\bl n_1,\ldots,n_k,r\myin\mathbb{N}$ and $\bl n$ is sufficiently large,\\
   then each colouring of $\bl \binom{[n]}{r}$ with colours $\bl c_1,\ldots,c_k$\\
   gives a $\bl c_i$-coloured subfamily $\bl \binom{S}{r}$
   for some $\bl i$ and $\bl n_i$-subset $\bl S\subseteq [n]$.}

\newcommand{\ramseyea}{\\[2mm]{\gray The least such $\lightblue n$ is denoted $\lightblue R_r(n_1,\ldots,n_k)$.}}

\newcommand{\arithma}{An {\dg arithmetic progression} is a sequence of numbers of the form
  \[\bl\hspace*{-27mm}{\white a+[0,k)d \white\,\;=}\;\;\,
    a\,,\; a+d\,,\; a+2d\,, \;\ldots\,, \;a+kd\,.
    \begin{pspicture}(-4,0)(-4,0)\psset{linecolor=darkgreen,framearc=0,fillstyle=solid,fillcolor=darkgreen}
    \psframe(0,0)(.5,1)\psframe(1,0)(1.5,1.3)\psframe(2,0)(2.5,1.6)\end{pspicture}\]}
\newcommand{\arithmav}{An {\dg arithmetic progression} is a sequence of numbers of the form
  \[\bl\hspace*{-27mm}{\white a+[0,k)d \white\;\,=}\;\;\,
    a\,,\; a+d\,,\; a+2d\,, \;\ldots\,, \;a+kd\,.
    \begin{pspicture}(-4,0)(-4,0)\psset{linecolor=darkgreen,framearc=0,fillstyle=solid,fillcolor=darkgreen}
    \psframe(0,0)(.5,1)\psframe(1,0)(1.5,1.3)\psframe(2,0)(2.5,1.6)\end{pspicture}\vspace*{6mm}\]}

\newcommand{\vanderwa}{{\dg Van der Waerden's Theorem} {\gray (1927)}\\
   If $\bl k,r\myin\mathbb{N}$ and $\bl n$ is sufficiently large,
   then each $\bl r$-colouring of $\bl [n]$ gives a monochromatic arithmetic progression of length~$\bl k$.}

\newcommand{\vanderwaa}{\\[2mm]{\gray The least such $\lightblue n$ is denoted $\lightblue W(k,r)$.}}

\newcommand{\vanderwad}{\\[12mm]Let \,{\dg S($\bl k,m$)}\, be the statement that\vspace*{-2mm}
\begin{center}{\small
  \begin{tabular}{l}for each $\bl r\myin\mathbb{N}$, there is some $\bl N\black:=\bl N(k,m,r)$ so that
    \\for each $\bl r$-colouring $\bl\chi:[N]\mto[r]$ there are $\bl a,d_1,\ldots,d_m\myin\mathbb{N}$ so that
    \\$\bl \chi(a+\sum x_id_i)$ is well-defined and constant on each $\bl k$-equivalence class of $\bl[0,k ]^m$.
  \end{tabular}}
\end{center}}

\newcommand{\vanderwadv}{\begin{pspicture}(.5,-.6)(10,1.7)\psset{linecolor=offwhite,fillcolor=nearlywhite,shadow=false,framearc=0}
    \psframe(-.25,-0.6)(22.25,2.35)\psline(-.25,1.15)(2.4,1.15)(2.4,2.35)
    \rput[l](0    ,1.7 ){{\dg S($\bl k,m$)}}
    \rput[l](7.333,1.7 ){\small for each $\bl r\myin\mathbb{N}$, there is some $\bl N\black:=\bl N(k,m,r)$ so that}
    \rput[l](3.322,0.85){\small for each $\bl r$-colouring $\bl\chi:[N]\mto[r]$ there are $\bl a,d_1,\ldots,d_m\myin\mathbb{N}$ so that}
    \rput[l](0  ,0  ){\small$\bl \chi(a+\sum x_id_i)$ is well-defined and constant on each $\bl k$-equivalence class of $\bl[0,k ]^m$.}
  \end{pspicture}}

\newcommand{\schura}{{\dg Schur's Theorem} {\gray (1916)}
  \\If $\bl\mathbb{N}$ is finitely coloured,
    then $\bl a + b \meq c$ for some same-coloured $\bl a,b,c\myin\mathbb{N}$.\hspace*{-5mm}}

\newcommand{\schurava}{\\[4mm]From the proof, we also get the following finite version.}
\newcommand{\schuravb}{\\[4mm]{\dg Schur's Theorem (finite)} {\gray (1916)}
  \\If $\bl[n]$ is finitely coloured for sufficiently large $\bl n$,\\
    then $\bl a + b \meq c$ for some same-coloured $\bl a,b,c\myin[n]$.}
\newcommand{\schuravbv}{{\dg Schur's Theorem (finite)} {\gray (1916)}
  \\If $\bl[n]$ is finitely coloured for sufficiently large $\bl n$,\\
    then $\bl a + b \meq c$ for some same-coloured $\bl a,b,c\myin[n]$.}
\newcommand{\schuravc}{\\[4mm]{\gray All infinite Ramsey theory theorems have finite versions,
  by the \\{\em\gdg Compactness Principle},
  a diagonal argument using the {\gdg Axiom of Choice}.}}

\newcommand{\schuraexaa}{\\[8mm]\example}
\newcommand{\schuraexab}[1]{\\{\psset{unit=5mm}\pspicture(-7,-1.5)(10,2)#1\endpspicture}}
\newcommand{\schuraexaba}{\rput(0,0){1}\rput(1,0){2}\rput(2,0){3}\rput(3,0){4}\rput(4,0){5}
   \rput(5,0){6}\rput(6,0){7}\rput(7,0){8}\rput(8,0){9}}
\newcommand{\schuraexabb}{{\darkred\footnotesize\rput(1,0){2}\rput(2,0){3}\rput(4,0){5}\rput(7,0){8}
                             \darkgreen\normalsize\rput(0,0){1}\rput(3,0){4}\rput(5,0){6}\rput(6,0){7}\rput(8,0){9}}}
\newcommand{\schuraexabc}{
   \psset{linecolor=halfgray,fillcolor=lightgray,shadow=false,linewidth=.03,fillstyle=solid,framearc=0}
   \psframe(-.5,-.5)(.5,.5)\psframe(4.5,-.5)(5.5,.5)\psframe(5.5,-.5)(6.5,.5)}
\newcommand{\schuraexabd}{
   \psset{linecolor=halfgray,fillcolor=lightgray,shadow=false,linewidth=.03,fillstyle=solid,framearc=0}
   \psframe(.5,-.5)(1.5,.5)\psframe(1.5,-.5)(2.5,.5)\psframe(3.5,-.5)(4.5,.5)}
\newcommand{\schuraexabe}{
   \psset{linecolor=halfgray,fillcolor=lightgray,shadow=false,linewidth=.03,fillstyle=solid,framearc=0}
   \psframe(1.5,-.5)(2.5,.5)\psframe(3.5,-.5)(4.5,.5)\psframe(6.5,-.5)(7.5,.5)}

\newcommand{\schurapfa}{\\[4mm]\proof}
\newcommand{\schurapfb}{\\Suppose that $\bl\chi:\mathbb{N}\mto [r]$ is an $\bl r$-colouring of $\bl\mathbb{N}$.}
\newcommand{\schurapfc}{\\By {\dg Ramsey's Theorem}, we may define $\bl N\mdef R(3;r)$.}
\newcommand{\schurapfd}{\\Define an edge $\bl r$-colouring $\bl\chi'$ of $\bl K_N$ by $\bl\chi'(\{i,j\})\mdef \chi(|j-i|)$.}
\newcommand{\schurapfe}{\\By definition of $\bl N$,
  there is a $\bl\chi'$-monochromatic triangle in $\bl K_N$, \\with vertices $\bl i\mlt j\mlt k$:
  \, $\bl\chi'(\{j,k\}) \meq \chi'(\{i,j\})\meq \chi'(\{i,k\})$.}
\newcommand{\schurapff}{\\Define $\bl a\mdef k-j$, $\bl b\mdef j-i$ and $\bl c\mdef k-i$.}
\newcommand{\schurapfg}{\\Then $\bl a+b\meq c$ }
\newcommand{\schurapfh}{and $\bl\chi(a)\meq \chi(b)\meq \chi(c)$.\qed}

\newcommand{\schurb}{\\[4mm]\theorem {\gray (Schur 1916)}
  \\For each $\bl m\myin\mathbb{N}$ and sufficiently big prime $\bl p$,
    there are solutions $\bl x,y,z\myin\mathbb{N}^+$\hspace*{-3mm}\\to
    \[\bl x^m + y^m \mequiv z^m\pmod{p}\black\,.\]}
\newcommand{\schurbv}{\theorem {\gray (Schur 1916)}
  \\For each $\bl m\myin\mathbb{N}$ and sufficiently big prime $\bl p$,
    there are solutions $\bl x,y,z\myin\mathbb{N}^+$\hspace*{-3mm}\\to
    \[\bl x^m + y^m \mequiv z^m\pmod{p}\black\,.\]}

\newcommand{\schurbpfa}{\proof}
\newcommand{\schurbpfb}{\\By {\dg Schur's Theorem (finite)}, there is a sufficiently large prime $\bl p$
  so that if $\bl[p-1]$ is $\bl m$-coloured, then same-coloured $\bl a,b,c\myin[p]$ exist with $\bl a+b\meq c$.}
\newcommand{\schurbpfc}{\\Consider $\bl\mathbb{Z}^*_p$, the multiplicative group of units in $\bl\mathbb{Z}_p$.}
\newcommand{\schurbpfd}{\\Define $\bl H\mdef\{x^m : x\myin\mathbb{Z}^*_p\}$. }
\newcommand{\schurbpfe}{\\This is a subgroup of $\bl\mathbb{Z}^*_p$ of index
   $\bl r\mdef |\mathbb{Z}^*_p:H|\meq \gcd(p-1,m)\mleq m$.}
\newcommand{\schurbpff}{\\The $\bl r$ cosets $\bl gH$ partition $\bl\mathbb{Z}^*_p$,
  defining an $\bl r$-colouring $\bl\chi$ of $\bl\mathbb{Z}^*_p$:\\
  $\bl\chi(a)\meq \chi(b)$ if and only if $\bl a,b\myin gH$ for some $\bl g$;
  i.e., when $\bl ab^{-1}\myin H$.}
\newcommand{\schurbpfg}{\\Since $\bl r\mleq m$,
  $\bl\chi$ as an $\bl m$-colouring of $\bl\mathbb{Z}^*_p$ (or, equivalently, $\bl[p-1]$).}
\newcommand{\schurbpfh}{\\There are same-coloured $\bl a,b,c\myin\mathbb{Z}^*_p$ with $\bl a+b\meq c$, or $\bl ac^{-1} + bc^{-1}\meq 1$.}
\newcommand{\schurbpfj}{\\Thus, $\bl ac^{-1}, bc^{-1}\myin H$,
  so $\bl ac^{-1}\meq x^m$, $\bl bc^{-1}\meq y^m$, $\bl 1\meq z^m$ for $\bl x,y,z\myin\mathbb{Z}^*_p$.}
\newcommand{\schurbpfk}{\\Then $\bl x^m+y^m\meq z^m$ in $\bl\mathbb{Z}_p$.\qed}

\newcommand{\schurca}{\theorem}
\newcommand{\schurcb}{\\For each $\bl k,r,s\mgeq 1$, there is $\bl n\mdef n(k,r,s)$ so that
   if $\bl [n]$ is $\bl r$-coloured, then for some $\bl a,d\mgt 0$,
   the following set is in $\bl [n]$ and is monochromatic:\vspace*{-2mm}
   \[\bl
     \{a, a+d, \ldots, a+kd\}\cup\{sd\}\,\black.%\vspace*{-2mm}
   \]}
\newcommand{\schurcc}{{\gray This result generalises {\gdg Van der Waerden's Theorem} and {\gdg Schur's Theorem}.}}
\newcommand{\schurcpfa}{\proof}
\newcommand{\schurcpfb}{\\Use induction on $\bl r$.}
\newcommand{\schurcpfc}{ Certainly, $\bl n(k,1,s)$ exists {\gray and equals $\lightblue\max\{k+1,s\}$}.}
\newcommand{\schurcpfd}{\\Assume that the theorem is true for $\bl r-1$ and set $\bl N\mdef n(k,r-1,s)$.}
\newcommand{\schurcpfe}{\\By {\dg Van der Waerden's Theorem},
  we may set $\bl W\mdef W(kN,r)$ \&\ $\bl n\mdef sW$.}
\newcommand{\schurcpff}{\\Let $\bl\chi$ be an $\bl r$-colouring of $\bl[n]$.}
\newcommand{\schurcpfg}{ Then $\bl[W]$ contains an arithmetic progression
  $\bl\{a,a+d',\ldots,a+(kN)d'\}$ of constant colour $\bl c$.}
\newcommand{\schurcpfh}{\\If $\bl\chi(sjd')\meq c$ for some $\bl j \myin[N]$,
  then set $\bl d\mdef jd'$, and we are done:
  \\$\bl\{a,a+d,\ldots,a+kd\}\cup\{sd\}$ has constant colour $\bl c$.}
\newcommand{\schurcpfj}{\\Suppose that $\bl\chi(sjd')\mneq c$ for all $\bl j \myin[N]$.}
\newcommand{\schurcpfk}{\\Define a colouring $\bl\chi'$ on $\bl[N]$ by $\bl\chi'(j)\mdef \chi(sjd')$.}
\newcommand{\schurcpfm}{ Since $\bl N\meq n(k,r-1,s)$,
there is a $\bl\chi'$-monochromatic set $\bl\{a',a'+d'',\ldots,a'+kd''\}\cup\{sd''\}$ in $\bl[N]$.}
\newcommand{\schurcpfn}{\\Set $\bl a\mdef sa'd'$ and $\bl d\mdef sd''d'$.}
\newcommand{\schurcpfp}{\\Then $\bl\{a,a+d,\ldots,a+kd\}\cup\{sd\}$ in $\bl[sNd']\msubseteq[n]$ is monochromatic.\qed\vspace*{-10mm}}

\newcommand{\schurda}{\corollary}
\newcommand{\schurdb}{\\For each $\bl k,r,s\mgeq 1$, there is $\bl n\mdef n(k,r,s)$ so that
   if $\bl [n]$ is $\bl r$-coloured, then for some $\bl a,d\mgt 0$,
   the following set is in $\bl [n]$ and is monochromatic:\vspace*{-2mm}
   \[\bl
     \{a + \lambda d: |\lambda|\mleq k\}\cup\{sd\}\,\black.
   \]}
\newcommand{\schurdc}{{\gray This illustrates how to shift the arithmetic progression with respect to $\lightblue sd$.}}
\newcommand{\schurdpfa}{\\[4mm]\proof}
\newcommand{\schurdpfb}{\\Applying $\bl k'\meq 2k$ to the theorem above gives $\bl a',d\mgt 0$ \\
  for which $\bl\{a'+\lambda d : \lambda \meq 0,1,\ldots,2k\}\cup\{sd\}$ in $\bl[n]$ is monochromatic.}
\newcommand{\schurdpfc}{\\Now re-label: $\bl a\mdef a'+kd$.\qed}

\newcommand{\schurdd}{These results are special cases of {\dg Rado's Theorem}.}

\newcommand{\radoa}{\\[2mm]{\dg Rado's Theorem {\gray(simple) (1933)}}}
\newcommand{\radoav}{{\dg Rado's Theorem {\gray(simple) (1933)}}}
\newcommand{\radob}{\\Let $\bl c_1,\ldots,c_n\myin\mathbb{Z}$.
    Then for any finite colouring of $\bl\mathbb{N}$,\vspace*{-2mm}
  \[\bl
    c_1 x_1 + \cdots + c_n x_n \meq 0\vspace*{-2mm}
  \]
  has a monochromatic solution $\bl x_1,\ldots,x_n\myin\mathbb{N}$ if and only if\\
  $\ds\bl\sum_{i\in I} c_i\meq 0$ for some nonempty subset $\bl I\msubseteq [n]$.\vspace*{-2mm}}

\newcommand{\radoc}{\\[4mm]{\dg Rado's Theorem} extends this to homogenous systems of linear equations.}
\newcommand{\radocv}{{\dg Rado's Theorem} extends this to homogenous systems of linear equations.}

\newcommand{\radoexaa}{\\[4mm]\example}
\newcommand{\radoexab}{\\The numbers $\bl a,a+d,\ldots,a+kd,sd$ form a solution $\bl x_0,\ldots,x_{k+1}$
  \\to the homogenous system of linear equations
  \[\bl
    \begin{array}{rl}
     x_1 - x_0         & \!\meq x_2 - x_1    \\
      \vdots           &                     \\\bl
     x_{k-1} - x_{k-2} & \!\meq x_k - x_{k-1}\\\bl
     x_{k+1}           & \!\meq s(x_1-x_0)
    \end{array}
    \quad\text{\black; that is,}\quad
    \begin{array}{rl}
     - x_0 + 2x_1 - x_2         & \!\meq 0\\
      \vdots      &                       \\\bl
     - x_{k-2} + 2x_{k-1} - x_k & \!\meq 0\\\bl
      sx_0 - sx_1 + x_{k+1}     & \!\meq 0\,\black.
    \end{array}
  \]}

\newcommand{\sumseta}[1]{For any set $\bl S\subseteq \mathbb{N}$,
      \\{\dg     sum set} $\bl\Sigma(S)$ is the set of all finite     sums of elements of~$\bl S${#1{;}}
  \\{#1{{\dg product set} $\bl\Pi(S)$    is the set of all finite products of elements of~$\bl S$.}}}
\newcommand{\sumsetexaa}{\\[2mm]\example\vspace*{-5mm}}
\newcommand{\sumsetexab}[1]{\begin{align*}\bl
   \Sigma(\{1,2,4\})   &    \meq \{1,2,3,4,5,6,7\}\\\bl
  {#1{\Pi(\{1,2,4\})}} &{#1{\meq \{1,2,4,8\}}}
  \end{align*}}

\newcommand{\folkmana}{{\dg Folkman's Theorem} {\gray(Rado 1970, Sanders 1968)}}
\newcommand{\folkmanb}{\\For $\bl c,k\myin\mathbb{N}$,
there is some sufficiently large $\bl M\myin\mathbb{N}$
  so that for any $\bl c$-colouring of $\bl\mathbb{N}$,
  there is a $\bl k$-subset $\bl S\subseteq[M]$ with monochromatic $\bl\Sigma(S)$.}
\newcommand{\folkmanc}{\\[2mm]{\gray This theorem follows as a special case from {\gdg Rado's Theorem}.}}

\newcommand{\productthma}{\\[2mm]\theorem}
\newcommand{\productthmb}{\\For $\bl c,k\myin\mathbb{N}$,
  there is some sufficiently large $\bl M\myin\mathbb{N}$
  so that for any $\bl c$-colouring of $\bl\mathbb{N}$,
  there is a $\bl k$-subset $\bl S\subseteq\mathbb[M]$ with monochromatic $\bl\Pi(S)$.}

\newcommand{\productthmpfa}{\\[2mm]\proof}
\newcommand{\productthmpfb}{\\This result follows from {\dg Folkman's Theorem}
  since any sum set in $\bl [N]$ will induce a product set in $\bl\{2^n \,:\, n\myin[N]\}$.\qed\vspace*{-4mm}}

\newcommand{\finiteunionthema}{\\[3mm]{\dg The Finite Union Theorem}}
\newcommand{\finiteunionthemb}{\\If the finite subsets of $\bl\mathbb{N}$ are finitely coloured,
  \\then there are arbitrarily large families of disjoint subsets $\bl\mathcal{D}\meq\{D_i\}_I$
  \\whose union sets $\ds\bl\{ \bigcup_{i\in J} D_i\,:\,J\msubseteq I\,, 0\mlt|J|\mlt\infty\}$
  are monochromatic.}
\newcommand{\finiteunionthemc}{\\[2mm]{\gray It is not hard to show that the finite version of this theorem
  is equivalent to {\gdg Folkman's Theorem}, via the correspondence between sets $\lightblue S$
  and binary (characteristic) vectors $\lightblue\mathbf{1}_S$ and thus numbers $\lightblue\sum_{i\in S} 2^{i-1}$.}}


\newcommand{\hindmana}{\\[4mm]{\dg Hindman's Theorem} {\gray (1974)}}
\newcommand{\hindmanb}{\\For any finite colouring of $\bl\mathbb{N}$,
  there is an infinite subset $\bl S\subseteq\mathbb{N}$ whose sum set $\bl\Sigma(S)$ is monochromatic.}
\newcommand{\hindmanc}{\\[2mm]{\gray This is much harder to prove than {\gdg Folkman's Theorem}.}}

\newcommand{\sumproductsetsa}{\\[4mm]It is an open question whether arbitrarily large finite sets $\bl S\msubseteq\mathbb{N}$
  can be exist with monochromatic $\bl\Sigma(S)\cup\Pi(S)$ when $\bl\mathbb{N}$ is finitely coloured.}
\newcommand{\sumproductsetsb}{\\Infinitely large such sets do not always exist: Hindman has shown this.}
\newcommand{\sumproductsetsc}{\\This question is open for even the smallest sets $\bl S\meq \{a,b\}$
  and their sets $\bl\Sigma(S)\cup\Pi(S)\meq\{a,b,a+b,ab\}$.}



\coursetitle
\np\lecturetitle
\np\lecturetitlei
\np\schura
\np\schura\schuraexaa
\np\schura\schuraexaa\schuraexab{\schuraexaba}
\np\schura\schuraexaa\schuraexab{\schuraexabb}
\np\schura\schuraexaa\schuraexab{\schuraexabc\schuraexabb}
\np\schura\schuraexaa\schuraexab{\schuraexabd\schuraexabb}
\np\schura\schuraexaa\schuraexab{\schuraexabe\schuraexabb}
\np\schura\schuraexaa\schuraexab{\schuraexabe\schuraexabb}\schurapfa
\np\schura\schuraexaa\schuraexab{\schuraexabe\schuraexabb}\schurapfa\schurapfb
\np\schura\schuraexaa\schuraexab{\schuraexabe\schuraexabb}\schurapfa\schurapfb\schurapfc
\np\schura\schuraexaa\schuraexab{\schuraexabe\schuraexabb}\schurapfa\schurapfb\schurapfc\schurapfd
\np\schura\schuraexaa\schuraexab{\schuraexabe\schuraexabb}\schurapfa\schurapfb\schurapfc\schurapfd\schurapfe
\np\schura\schuraexaa\schuraexab{\schuraexabe\schuraexabb}\schurapfa\schurapfb\schurapfc\schurapfd\schurapfe\schurapff
\np\schura\schuraexaa\schuraexab{\schuraexabe\schuraexabb}\schurapfa\schurapfb\schurapfc\schurapfd\schurapfe\schurapff\schurapfg
\np\schura\schuraexaa\schuraexab{\schuraexabe\schuraexabb}\schurapfa\schurapfb\schurapfc\schurapfd\schurapfe\schurapff\schurapfg\schurapfh
\np\schura
\np\schura\schurava
\np\schura\schurava\schuravb
\np\schura\schurava\schuravb\schuravc
\np\schura\schurava\schuravb\schuravc\schurb
\np\schuravbv\schurb\schurbpfa
\np\schuravbv\schurb\schurbpfa\schurbpfb
\np\schuravbv\schurb\schurbpfa\schurbpfb\schurbpfc
\np\schuravbv\schurb\schurbpfa\schurbpfb\schurbpfc\schurbpfd
\np\schuravbv\schurb\schurbpfa\schurbpfb\schurbpfc\schurbpfd\schurbpfe
\np\schuravbv\schurb\schurbpfa\schurbpfb\schurbpfc\schurbpfd\schurbpfe\schurbpff
\np\schuravbv\schurb\schurbpfa\schurbpfb\schurbpfc\schurbpfd\schurbpfe\schurbpff\schurbpfg
\np\schurbv\schurbpfa\schurbpfb\schurbpfc\schurbpfd\schurbpfe\schurbpff\schurbpfg
\np\schurbv\schurbpfa\schurbpfb\schurbpfc\schurbpfd\schurbpfe\schurbpff\schurbpfg\schurbpfh
\np\schurbv\schurbpfa\schurbpfb\schurbpfc\schurbpfd\schurbpfe\schurbpff\schurbpfg\schurbpfh\schurbpfj
\np\schurbv\schurbpfa\schurbpfb\schurbpfc\schurbpfd\schurbpfe\schurbpff\schurbpfg\schurbpfh\schurbpfj\schurbpfk
\np\schurca
\np\schurca\schurcb
\np\schurca\schurcb\schurcc
\np\schurca\schurcb\schurcpfa
\np\schurca\schurcb\schurcpfa\schurcpfb
\np\schurca\schurcb\schurcpfa\schurcpfb\schurcpfc
\np\schurca\schurcb\schurcpfa\schurcpfb\schurcpfc\schurcpfd
\np\schurca\schurcb\schurcpfa\schurcpfb\schurcpfc\schurcpfd\schurcpfe
\np\schurca\schurcb\schurcpfa\schurcpfb\schurcpfc\schurcpfd\schurcpfe\schurcpff
\np\schurca\schurcb\schurcpfa\schurcpfb\schurcpfc\schurcpfd\schurcpfe\schurcpff\schurcpfg
\np\schurca\schurcb\schurcpfa\schurcpfb\schurcpfc\schurcpfd\schurcpfe\schurcpff\schurcpfg\schurcpfh
\np\schurca\schurcb\schurcpfa\schurcpfb\schurcpfc\schurcpfd\schurcpfe\schurcpff\schurcpfg\schurcpfh\schurcpfj
\np\schurca\schurcb\schurcpfa\schurcpfb\schurcpfc\schurcpfd\schurcpfe\schurcpff\schurcpfg\schurcpfh\schurcpfj\schurcpfk
\np\schurca\schurcb\schurcpfa\schurcpfb\schurcpfc\schurcpfd\schurcpfe\schurcpff\schurcpfg\schurcpfh\schurcpfj\schurcpfk\schurcpfm
\np\schurca\schurcb\schurcpfa\schurcpfb\schurcpfc\schurcpfd\schurcpfe\schurcpff\schurcpfg\schurcpfh\schurcpfj\schurcpfk\schurcpfm\schurcpfn
\np\schurca\schurcb\schurcpfa\schurcpfb\schurcpfc\schurcpfd\schurcpfe\schurcpff\schurcpfg\schurcpfh\schurcpfj\schurcpfk\schurcpfm\schurcpfn\schurcpfp
\np\schurca\schurcb
\np\schurca\schurcb\schurda
\np\schurca\schurcb\schurda\schurdb
\np\schurca\schurcb\schurda\schurdb\schurdc
\np\schurca\schurcb\schurda\schurdb\schurdc\schurdpfa
\np\schurca\schurcb\schurda\schurdb\schurdc\schurdpfa\schurdpfb
\np\schurca\schurcb\schurda\schurdb\schurdc\schurdpfa\schurdpfb\schurdpfc
\np\schurca\schurcb\schurda\schurdb
\np\schurca\schurcb\schurda\schurdb\schurdd
\np\schurca\schurcb\schurda\schurdb\schurdd\radoa\radob
\np\schurca\schurcb\schurda\schurdb\schurdd\radoa\radob\radoc
\np\schurca\schurcb\radocv
\np\schurca\schurcb\radocv\radoexaa
\np\schurca\schurcb\radocv\radoexaa\radoexab
\np\sumseta{\ph}
\np\sumseta{\ph}\sumsetexaa
\np\sumseta{\ph}\sumsetexaa\sumsetexab{\ph}
\np\sumseta{\ph}\sumsetexaa\sumsetexab{\ph}\folkmana
\np\sumseta{\ph}\sumsetexaa\sumsetexab{\ph}\folkmana\folkmanb
\np\sumseta{\ph}\sumsetexaa\sumsetexab{\ph}\folkmana\folkmanb\folkmanc
\np\sumseta{}\sumsetexaa\sumsetexab{\ph}\folkmana\folkmanb\folkmanc
\np\sumseta{}\sumsetexaa\sumsetexab{}\folkmana\folkmanb\folkmanc
\np\sumseta{}\sumsetexaa\sumsetexab{}\folkmana\folkmanb\folkmanc\productthma
\np\sumseta{}\sumsetexaa\sumsetexab{}\folkmana\folkmanb\folkmanc\productthma\productthmb
\np\sumseta{}\sumsetexaa\sumsetexab{}\folkmana\folkmanb\productthma\productthmb\productthmpfa
\np\sumseta{}\sumsetexaa\sumsetexab{}\folkmana\folkmanb\productthma\productthmb\productthmpfa\productthmpfb
\np\folkmana\folkmanb\productthma\productthmb\finiteunionthema
\np\folkmana\folkmanb\productthma\productthmb\finiteunionthema\finiteunionthemb
\np\folkmana\folkmanb\productthma\productthmb\finiteunionthema\finiteunionthemb\finiteunionthemc
\np\folkmana\folkmanb\productthma\productthmb\finiteunionthema\finiteunionthemb\hindmana
\np\folkmana\folkmanb\productthma\productthmb\finiteunionthema\finiteunionthemb\hindmana\hindmanb
\np\folkmana\folkmanb\productthma\productthmb\finiteunionthema\finiteunionthemb\hindmana\hindmanb\hindmanc
\np\folkmana\folkmanb\hindmana\hindmanb\hindmanc
\np\folkmana\folkmanb\hindmana\hindmanb\hindmanc\sumproductsetsa
\np\folkmana\folkmanb\hindmana\hindmanb\hindmanc\sumproductsetsa\sumproductsetsb
\np\folkmana\folkmanb\hindmana\hindmanb\hindmanc\sumproductsetsa\sumproductsetsb\sumproductsetsc

%1. Prove this?\\
%2. Mention Hindman's Theorem.\\
%3. Mention Finite Union Theorem. (Prove equivalance with Folkman, or just the one direction, or not at all?)\\
%4. Mention product version.


\end{document}


