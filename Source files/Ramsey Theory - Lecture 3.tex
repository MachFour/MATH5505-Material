%%  MATH5505 Ramsey Theory Lecture 3: Arithmetic Progressions
%%
%%  by Thomas Britz 2018S1
%%


% OK Pigeonhole + Ramsey refresh
% OK Van der Waerden's Theorem
% OK Example
% OK Proof
%    Mention polynomial versions (Purely Combinatorial Proofs of Van Der Waerden-Type Theorems William Gasarch and Andy Parrish March 5, 2009)
%    The Hales-Jewett Theorem (just mention)
%    Mention Szemeredi's Theorem, primes/Tao

\documentclass[12pt,a4paper,landscape]{article}
\special{landscape}

\usepackage{latexsym,amsfonts,amsmath,amssymb} %,calc,fancybox,epsfig,amscd,tabularx}
\usepackage{pstricks,pst-plot,pst-node,pst-tree}
%\usepackage{array,graphicx,epsfig}
%\usepackage{multicol,multirow,hyperref,rotating}
%\usepackage[T1]{fontenc}
%\usepackage{yfonts}


\mag=\magstep4

\newrgbcolor{green}{0.2 0.6 0.2}
\newrgbcolor{darkgreen}{0 0.4 0}
\newrgbcolor{graydarkgreen}{0.5 0.67 0.5}
\newrgbcolor{lightgreen}{0 0.75 0}
\newrgbcolor{skyblue}{0.5 0.80 1}
\newrgbcolor{darkblue}{0 0 0.6}
\newrgbcolor{darkred}{0.6 0 0}
\newrgbcolor{halfred}{.8 0 0}
\newrgbcolor{white}{1 1 1}
\newrgbcolor{nearlywhite}{0.95 0.95 0.95}
\newrgbcolor{offwhite}{0.9 0.9 0.9}
\newrgbcolor{lightgray}{0.85 0.85 0.85}
\newrgbcolor{halfgray}{0.8 0.8 0.8}
\newrgbcolor{altgray}{0.67 0.67 0.67}
\newrgbcolor{darkyellow}{1 0.94 0.15}
\newrgbcolor{halflightyellow}{1 1 0.4}
\newrgbcolor{lightyellow}{1 1 0.7}
\newrgbcolor{gold}{0.96 0.96 0.1}
\newrgbcolor{lightblue}{0.5 0.5 1}
\newrgbcolor{amber}{1 0.75 0}
\newrgbcolor{hotpink}{1 0.41 0.71}

\setlength{\textheight}{18.0truecm}
\setlength{\textwidth}{26.0truecm}
\setlength{\hoffset}{-12.0truecm}
\setlength{\voffset}{-5.2truecm}

\parindent 0in

%\setlength{\bigskipamount}{5ex plus1.5ex minus 2ex}
%\setlength{\parindent}{0cm}
%\setlength{\parskip}{0.2cm}

\psset{unit=6mm,linewidth=.06,dotscale=1.5,fillcolor=white,fillstyle=none,
 linecolor=gray,framearc=.3,shadowcolor=offwhite,shadow=true,shadowsize=.125,
 shadowangle=-45,dash=7pt 5pt}
%\psset{unit=.4,linecolor=gray,fillcolor=offwhite,shadowsize=.2,framearc=.3}
%\psset{unit=10mm,linewidth=.03}

\def\dedge{\ncline[linestyle=dashed]}

\def\np{\newpage}
\def\bl{\blue}
\def\bk{\black}
\def\wh{\white}
\def\rd{\red}
\def\lg{\lightgray}
\def\gr{\green}
\def\dr{\darkred}
\def\dg{\darkgreen}
\def\gdg{\graydarkgreen}
\def\dgy{\darkgray}
\def\ec{\dg}

\newcommand{\ora}[1]{\overrightarrow{#1}}

\newcommand{\llb}{\\[1mm]}
\newcommand{\lb}{\\[3mm]}
\newcommand{\blb}{\\[5mm]}
\newcommand{\hlb}{\\[48mm]}

\newcommand{\mynewpage}{\newpage\vspace*{-10mm}}

\newcommand{\vc}[1]{\begin{pmatrix}#1\end{pmatrix}}

\newcommand{\mytext}[1]{\text{\black#1\blue}}

\newcommand{\qbinom}[2]{\genfrac{[}{]}{0pt}{}{#1}{#2}}

\newcommand{\dl}{\psset{linestyle=dashed,linecolor=altgray,shadow=false}}
%\newcommand{\dl}{\psset{linestyle=dashed,linecolor=blue}}

\newcommand{\myframe}[1]{\pspicture(0,0)(0,0)\psset{unit=1cm,
 shadowcolor=offwhite,shadow=true,shadowangle=-45,linewidth=.03,linecolor=gray,
 fillcolor=lightgray,shadowsize=.15,framearc=.3}\psframe#1\endpspicture}

\newcommand{\mypicture}[1]{\pspicture(0,0)(0,0)\psset{unit=1cm,
 shadowcolor=offwhite,shadow=true,shadowangle=-45,linewidth=.03,linecolor=gray,
 fillcolor=lightgray,shadowsize=.15,framearc=.3}#1\endpspicture}

\newcommand{\mymatrix}[1]{{\psset{unit=4mm,linewidth=.03,linecolor=gray,fillstyle=solid,fillcolor=offwhite,shadow=false,framearc=0}\pspicture(0,0)(6,4)
  \psframe(0,0)(7,5)#1\psframe[linecolor=gray,fillcolor=offwhite,linewidth=.03,fillstyle=none,framearc=0](0,0)(7,5)\endpspicture}}

\newcommand{\mypair}[2]{{\psset{unit=4mm,linewidth=.03,linecolor=gray,fillstyle=solid,fillcolor=offwhite,shadow=false,framearc=0}\pspicture(0,.2)(2,.8)
  \psframe(0,0)(2,1)\darkgray\rput(.5,.5){#1}\rput(1.5,.5){#2}\psframe[linecolor=gray,fillcolor=offwhite,linewidth=.03,fillstyle=none,framearc=0](0,0)(2,1)\endpspicture}}

\newcommand{\mysixtuple}[6]{{\psset{unit=4mm,linewidth=.03,linecolor=gray,fillstyle=solid,fillcolor=offwhite,shadow=false,framearc=0}\pspicture(0,.2)(6,.8)
  \psframe(0,0)(6,1)\darkgray\rput(.5,.5){#1}\rput(1.5,.5){#2}\rput(2.5,.5){#3}\rput(3.5,.5){#4}\rput(4.5,.5){#5}\rput(5.5,.5){#6}
  \psframe[linecolor=gray,fillcolor=offwhite,linewidth=.03,fillstyle=none,framearc=0](0,0)(6,1)\endpspicture}}

\newcommand{\acset}{\psset{fillstyle=none,linecolor=red}}
\newcommand{\drr}{\psset{linecolor=darkred,fillcolor=red,linewidth=.034}}
\newcommand{\dbb}{\psset{linecolor=darkblue,fillcolor=blue,linewidth=.034}}
\newcommand{\bbl}{\psset{linecolor=blue,fillcolor=lightblue}}
\newcommand{\dbhg}{\psset{linecolor=darkblue,fillcolor=halfgray}}
\newcommand{\bhg}{\psset{linecolor=blue,fillcolor=halfgray}}
\newcommand{\rhg}{\psset{linecolor=red,fillcolor=halfgray}}
\newcommand{\ghg}{\psset{linecolor=green,fillcolor=halfgray}}
\newcommand{\dgg}{\psset{linecolor=darkgreen,fillcolor=green,linewidth=.034}}

\newcommand{\mychain}{\pspicture(0,0)(0,0)\dbb
  \psset{shadowsize=.2,shadow=true,shadowcolor=offwhite,shadowangle=-45,shadowsize=.15}
  \pscircle(0,0){.09}\pscircle(0,1){.09}\pscircle(0,2){.09}\pscircle(0,3){.09}
  \psline(0,.125)(0,.875)\psline(0,1.125)(0,1.875)\psline(0,2.125)(0,2.875)\endpspicture}

\newcommand{\myantichain}{\pspicture(0,0)(0,0)\drr
  \psset{shadowsize=.2,shadow=true,shadowcolor=offwhite,shadowangle=-45,shadowsize=.15}
  \pscircle(0,0){.09}\pscircle(1,0){.09}\pscircle(2,0){.09}\pscircle(3,0){.09}\endpspicture}

\newcommand{\mybullet}{{\psset{unit=6mm,dotscale=1.5,linewidth=.05,fillcolor=black,linecolor=black,framearc=.3,shadow=true}
  \pspicture(-.3,0)(.8,.4)\qdisk(.25,.2){.1}\pscircle*[linecolor=gray](.25,.2){.07}\endpspicture}}

\newcommand{\mysmallbullet}{{\psset{unit=4mm,dotscale=1.5,linewidth=.05,fillcolor=black,linecolor=black,framearc=.3,shadow=true}
  \pspicture(-.3,0)(.8,.4)\qdisk(.25,.2){.1}\pscircle*[linecolor=gray](.25,.2){.07}\endpspicture}}

\newcommand{\mybulletv}{{\psset{unit=6mm,dotscale=1.5,linewidth=.04,linecolor=black,fillstyle=solid,
  fillcolor=gray,shadow=false}\pspicture(0,0)(.5,.4)\pscircle(.25,.2){.15}\endpspicture}}

\newcommand{\mystar}{\pspicture(0,0)(0,0)\psset{unit=1cm}\gold\large$\star$\endpspicture
  \pspicture(0,0)(0,0)\psset{unit=1cm}\rput(-.23,.193){\small\white$\star$}\endpspicture}

\newcommand{\mystep}{{\psset{unit=6mm,dotscale=1.5,linewidth=.06,linecolor=darkgreen,fillstyle=solid,
  fillcolor=lightgreen,shadow=false}\pspicture(-.3,0)(.8,.4)\pscircle(.25,.2){.15}\endpspicture}}
\newcommand{\mystepv}{{\psset{unit=6mm,dotscale=1.5,linewidth=.06,linecolor=darkgreen,fillstyle=solid,
  fillcolor=lightgreen,shadow=false}\pspicture(0,0)(.5,.4)\pscircle(.25,.2){.15}\endpspicture}}
\newcommand{\mystepvv}{{\psset{unit=6mm,dotscale=1.5,linewidth=.06,linecolor=darkred,fillstyle=solid,
  fillcolor=red,shadow=false}\pspicture(0,0)(.5,.4)\pscircle(.25,.2){.15}\endpspicture}}
\newcommand{\mystepvvv}{{\psset{unit=6mm,dotscale=1.5,linewidth=.06,linecolor=darkred,fillstyle=solid,
  fillcolor=red,shadow=false}\pspicture(0,0)(.5,.4)\endpspicture}}

\newcommand{\myi}{{\psset{unit=6mm,dotscale=1.5,linewidth=.05,linecolor=darkgreen,fillstyle=none,shadow=false}%
  \pspicture(-.3,0)(.8,.4)\pscircle(.25,.2){.25}\rput(.25,.2){\tiny\dg1}\endpspicture}}
\newcommand{\myii}{{\psset{unit=6mm,dotscale=1.5,linewidth=.05,linecolor=darkgreen,fillstyle=none,shadow=false}%
  \pspicture(-.3,0)(.8,.4)\pscircle(.25,.2){.25}\rput(.25,.2){\tiny\dg2}\endpspicture}}
\newcommand{\myiii}{{\psset{unit=6mm,dotscale=1.5,linewidth=.05,linecolor=darkgreen,fillstyle=none,shadow=false}%
  \pspicture(-.3,0)(.8,.4)\pscircle(.25,.2){.25}\rput(.25,.2){\tiny\dg3}\endpspicture}}
\newcommand{\myiv}{{\psset{unit=6mm,dotscale=1.5,linewidth=.05,linecolor=darkgreen,fillstyle=none,shadow=false}%
  \pspicture(-.3,0)(.8,.4)\pscircle(.25,.2){.25}\rput(.25,.2){\tiny\dg4}\endpspicture}}
\newcommand{\myv}{{\psset{unit=6mm,dotscale=1.5,linewidth=.05,linecolor=darkgreen,fillstyle=none,shadow=false}%
  \pspicture(-.3,0)(.8,.4)\pscircle(.25,.2){.25}\rput(.25,.2){\tiny\dg5}\endpspicture}}
\newcommand{\myvi}{{\psset{unit=6mm,dotscale=1.5,linewidth=.05,linecolor=darkgreen,fillstyle=none,shadow=false}%
  \pspicture(-.3,0)(.8,.4)\pscircle(.25,.2){.25}\rput(.25,.2){\tiny\dg6}\endpspicture}}
\newcommand{\myvii}{{\psset{unit=6mm,dotscale=1.5,linewidth=.05,linecolor=darkgreen,fillstyle=none,shadow=false}%
  \pspicture(-.3,0)(.8,.4)\pscircle(.25,.2){.25}\rput(.25,.2){\tiny\dg7}\endpspicture}}
\newcommand{\myviii}{{\psset{unit=6mm,dotscale=1.5,linewidth=.05,linecolor=darkgreen,fillstyle=none,shadow=false}%
  \pspicture(-.3,0)(.8,.4)\pscircle(.25,.2){.25}\rput(.25,.2){\tiny\dg8}\endpspicture}}
\newcommand{\myix}{{\psset{unit=6mm,dotscale=1.5,linewidth=.05,linecolor=darkgreen,fillstyle=none,shadow=false}%
  \pspicture(-.3,0)(.8,.4)\pscircle(.25,.2){.25}\rput(.25,.2){\tiny\dg9}\endpspicture}}

\newcommand{\mywo}{{\psset{unit=6mm,dotscale=1.5,linewidth=.05,linecolor=darkgreen,fillcolor=white,fillstyle=solid,shadow=false}%
  \pspicture(-.3,0)(.8,.4)\pscircle(.25,.2){.25}\rput(.25,.2){\tiny\dg0}\endpspicture}}
\newcommand{\mywi}{{\psset{unit=6mm,dotscale=1.5,linewidth=.05,linecolor=darkgreen,fillcolor=white,fillstyle=solid,shadow=false}%
  \pspicture(-.3,0)(.8,.4)\pscircle(.25,.2){.25}\rput(.25,.2){\tiny\dg1}\endpspicture}}
\newcommand{\mywii}{{\psset{unit=6mm,dotscale=1.5,linewidth=.05,linecolor=darkgreen,fillcolor=white,fillstyle=solid,shadow=false}%
  \pspicture(-.3,0)(.8,.4)\pscircle(.25,.2){.25}\rput(.25,.2){\tiny\dg2}\endpspicture}}

\newcommand{\myO}{{\psset{unit=6mm,dotscale=1.5,linewidth=.06,linecolor=darkgreen,fillstyle=none,
  shadow=false}\pspicture(0,0)(0,0)\pscircle(-.2,.225){.4}\endpspicture}}

\newcommand{\myspace}{\psset{unit=6mm,dotscale=1.5,linewidth=.06}\pspicture(-.3,0)(.8,.4)\endpspicture}

\newcommand{\mydot}{\pscircle(0,0){.1}}
\newcommand{\mydotv}{\pscircle*[linecolor=gray](0,0){.1}\pscircle*[linecolor=blue](0,0){.06}}

\newcommand{\myhexagon}[2]{\psset{fillcolor=halfgray,linecolor=gray,shadow=false,fillstyle=none,linewidth=.06}
  \pspicture(-3.5,-2.6)(3,2.6)
    \pnode(-3  , 0  ){a}
    \pnode(-1.5, 2.6){b}
    \pnode( 1.5, 2.6){c}
    \pnode( 3  , 0  ){d}
    \pnode( 1.5,-2.6){e}
    \pnode(-1.5,-2.6){f}
    \pspolygon(a)(b)(c)(d)(e)(f)#1\psset{fillstyle=solid}
    \pscircle(a){.2}
    \pscircle(b){.2}
    \pscircle(c){.2}
    \pscircle(d){.2}
    \pscircle(e){.2}
    \pscircle(f){.2}#2
  \endpspicture}

\newcommand{\myksix}[2]{\myhexagon{\pspolygon(a)(c)(e)\pspolygon(b)(d)(f)\psline(a)(d)\psline(b)(e)\psline(c)(f)#1}{#2}}

\newcommand{\lss}{$\spadesuit$}
\newcommand{\lhs}{$\darkred\heartsuit$}
\newcommand{\lds}{$\darkred\diamondsuit$}
\newcommand{\lcs}{$\clubsuit$}

\newcommand{\hatmanv}{{%
 \psset{dotscale=1.5,dotsize=0.09,linewidth=.03,fillcolor=black,linecolor=black,shadow=false}
 \psline(0,0)(0.25,0)\psline(1.75,0)(2,0)\psline(0.25,0)(1,1.2)
 \psline(1.75,0)(1,1.2)\psline(1,1.2)(1,2.5)\psline(0.25,1.75)(1,2)
 \psline(1.75,1.75)(1,2)\pscircle(1,3){0.5}\psarc(1,3){.25}{-140}{-40}
 \psdots(.85,3.15)(1.15,3.15)\psset{fillstyle=solid}\psframe(.5,3.35)(1.5,3.45)
 \psframe(.7,3.35)(1.3,4.1)}}

\newcommand{\manv}{{%
 \psset{dotscale=1.5,dotsize=0.09,linewidth=.03,fillcolor=black,linecolor=black,shadow=false}%
 \pspicture(0,0)(2,3.5)%\psline(0,0)(0.25,0)\psline(1.75,0)(2,0)
 \psline(0.25,0)(1,1.2)
 \psline(1.75,0)(1,1.2)\psline(1,1.2)(1,2.5)\psline(0.25,1.75)(1,2)
 \psline(1.75,1.75)(1,2)\pscircle(1,3){0.5}\psarc(1,3){.25}{-140}{-40}
 \psdots(.85,3.15)(1.15,3.15)%\psset{fillstyle=solid}\psframe(.5,3.35)(1.5,3.45)\psframe(.7,3.35)(1.3,4.1)
 \endpspicture}}

\newcommand{\sadmanv}{{
 \psset{dotscale=1.5,dotsize=0.09,linewidth=.03,fillcolor=black,linecolor=black,shadow=false}
 \psline(0,0)(0.25,0)\psline(1.75,0)(2,0)\psline(0.25,0)(1,1.2)
 \psline(1.75,0)(1,1.2)\psline(1,1.2)(1,2.5)\psline(0.25,1.75)(1,2)
 \psline(1.75,1.75)(1,2)\pscircle(1,3){0.5}\psarc(1,2.65){.25}{40}{140}
 \psdots(.87,3.15)(1.13,3.15)\psset{fillstyle=solid}\psframe(.5,3.35)(1.5,3.45)
 \psframe(.7,3.35)(1.3,4.1)}}

\newcommand{\womanv}{{
 \psset{dotscale=1.5,dotsize=0.09,linewidth=.03,fillcolor=black,linecolor=black,shadow=false}
 \pspicture(0,0)(2,3.5)%
 \psline(0.5,0)(0.75,0)\psline(1.5,0)(1.25,0)\psline(0.75,0)(0.75,0.5)
 \psline(1.25,0)(1.25,0.5)\psline(1,2)(1,2.35)\psline(0.25,1.75)(1,2)
 \psline(1.75,1.75)(1,2)\pscircle(1,2.85){0.5}
 \psarc(1,2.85){.25}{-140}{-40}\psdots(.85,2.95)(1.15,2.95)
 \psset{linecolor=brown,linewidth=.1}
 \psarc(1,2.85){.5}{0}{180}\psarc(1.75,2.85){.25}{180}{270}
 \psarc(0.25,2.85){.25}{-90}{0}
 \pstriangle[linecolor=red,fillstyle=solid,fillcolor=red](1,.5)(1.2,1.75)\endpspicture}}

\newcommand{\spyv}{{%
 \psset{dotscale=1.5,dotsize=0.09,linewidth=.03,fillcolor=black,linecolor=black,shadow=false}
 \pspicture(0,0)(2,3.5)%\psline(0,0)(0.25,0)\psline(1.75,0)(2,0)
 \psline(0.25,0)(1,1.2)
 \psline(1.75,0)(1,1.2)\psline(1,1.2)(1,2.5)\psline(0.25,1.75)(1,2)
 \psline(1.75,1.75)(1,2)\pscircle(1,3){0.5}\psarc(1,3){.25}{-140}{-40}
 %\psdots(.85,3.15)(1.15,3.15)
 \psset{fillstyle=solid}
 \psframe( .725,3.075)( .975,3.225)
 \psframe(1.025,3.075)(1.275,3.225)
 \psellipse*(1,3.5)(.5,.14)
 \pspolygon(.6,3.5)(1.4,3.5)( .8,3.8)
 \pspolygon(.6,3.5)(1.4,3.5)(1.2,3.8)
 \endpspicture}}

\newcommand{\puzzledspyv}{{%
 \psset{dotscale=1.5,dotsize=0.09,linewidth=.03,fillcolor=black,linecolor=black,shadow=false}
 \pspicture(0,0)(2,3.5)%\psline(0,0)(0.25,0)\psline(1.75,0)(2,0)
 \psline(0.25,0)(1,1.2)
 \psline(1.75,0)(1,1.2)\psline(1,1.2)(1,2.5)\psline(0.25,1.75)(1,2)
 \psline(1.75,1.75)(1,2)\pscircle(1,3){0.5}
 % Mouth
 \psline(.83,2.8)(1.17,2.8)
 %\psdots(.85,3.15)(1.15,3.15)
 \psset{fillstyle=solid}
 \psframe( .725,3.075)( .975,3.225)
 \psframe(1.025,3.075)(1.275,3.225)
 \psellipse*(1,3.5)(.5,.14)
 \pspolygon(.6,3.5)(1.4,3.5)( .8,3.8)
 \pspolygon(.6,3.5)(1.4,3.5)(1.2,3.8)
 \endpspicture}}

\newcommand{\man}{\psset{unit=4mm}\manv}
\newcommand{\hatman}{\psset{unit=4mm}\hatmanv}
\newcommand{\sadman}{\psset{unit=4mm}\sadmanv}
\newcommand{\woman}{\psset{unit=4mm}\womanv}
\newcommand{\spy}{\psset{unit=4mm}\spyv}
\newcommand{\puzzledspy}{\psset{unit=4mm}\puzzledspyv}


\newcommand{\coursetitle}{\vphantom{ }\vspace{7mm}\begin{center}
    {\large\darkgreen MATH5505 \:\: Combinatorics}\\[2.5mm]
    UNSW 2018S1
    \end{center}}

\newcounter{gra}

\DeclareMathAlphabet{\mathscr}{OT1}{pzc}%
                                 {m}{it}

\newcommand{\ph}{\phantom}
\newcommand{\ds}{\displaystyle}
\newcommand{\mra}{\black\rightsquigarrow\blue}
\newcommand{\tvs}{\textvisiblespace}
\newcommand{\msp}{\,}
\newcommand{\mba}{\,\mypicture{\psset{shadow=false}\psline(-.04,-.15)(-.04,.4)}}
\newcommand{\mbq}{\,\mypicture{\psset{shadow=false}\psline(-.04,-.15)(-.04,.4)\rput(-0.04,0.7){{\darkgray?}}}}
\newcommand{\meq}{\black=\blue}
\newcommand{\msim}{\black\sim\blue}
\newcommand{\mequiv}{\black\equiv\blue}
\newcommand{\mapprox}{\black\approx\blue}
\newcommand{\mneq}{\red\neq\blue}
\newcommand{\mleq}{\black\leq\blue}
\newcommand{\mgeq}{\black\geq\blue}
\newcommand{\mgt}{\black>\blue}
\newcommand{\mlt}{\black<\blue}
\newcommand{\mto}{\black\to\blue}
\newcommand{\msubseteq}{\black\subseteq\blue}
\newcommand{\mnsubseteq}{\red\not\subseteq\blue}
\newcommand{\mequ}{\black\equiv\blue}
\newcommand{\mnequ}{\red\not\equiv\blue}
\newcommand{\myin}{\black\in\blue}
\newcommand{\mypl}{\black+\blue}
\newcommand{\mdef}{\black:=\blue}
\newcommand{\mnotin}{\red\notin\blue}
\newcommand{\mynotin}{\red\notin\blue}
\newcommand{\mywhere}{\quad\text{\black where\blue}\quad}
\newcommand{\myand}{\quad\text{\black and\blue}\quad}
\newcommand{\dnd}{\black\mathchoice{\mathrel{{\kern0.1em|\kern-0.4em/}}}
  {\mathrel{{\kern0.1em|\kern-0.4em/}}}{\mathrel{{\kern0.1em|\kern-0.33em/}}}
  {\mathrel{{\kern0.1em|\kern-0.2em/}}}\blue}
\newcommand{\mdiv}{\black\,|\,\blue}
\newcommand{\mymod}[1]{\:(\textrm{mod}\,#1)}
\newcommand{\mask}[1]{}
\newcommand{\ord}{\textrm{ord}\,}
\newcommand{\ns}{\negthickspace\negthickspace}
\newcommand{\nns}{\negthickspace\negthickspace\negthickspace\negthickspace}
\newcommand{\lns}{\hspace*{-.3mm}}
\newcommand{\ri}{\,i\,}% \,\mathrm{i}}
\newcommand{\di}{\mbox{$\dg\bullet$}}
\newcommand{\dah}{\mbox{$\dg\mathbf{-}$}}
\newcommand{\tp}{\mbox{{\dg\tt p}}}
\newcommand{\Var}{\mathrm{Var}}
\newcommand{\Arg}{\mathrm{Arg}}
\renewcommand{\Re}{\mathrm{Re}}
\renewcommand{\Im}{\mathrm{Im}}
\newcommand{\bbN}{\blue\mathbb{N}}
\newcommand{\bbZ}{\blue\mathbb{Z}}
\newcommand{\bbQ}{\blue\mathbb{Q}}
\newcommand{\bbR}{\blue\mathbb{R}}
\newcommand{\bbC}{\blue\mathbb{C}}
\newcommand{\bbF}{\blue\mathbb{F}}
\newcommand{\bbP}{\blue\mathbb{P}}
%\newcommand{\calR}{\blue\mathcal{R}}
%\newcommand{\calC}{\blue\mathcal{C}}
\renewcommand{\vec}[1]{\mathbf{\blue #1}}
%\newcommand{\pv}[1]{{\blue\begin{pmatrix}#1\end{pmatrix}}}
%\newcommand{\pvn}[1]{{\begin{pmatrix}#1\end{pmatrix}}}
%\newcommand{\spv}[1]{{\blue\left(\begin{smallmatrix}#1\end{smallmatrix}\right)}}
%\newcommand{\mpv}[1]{{\blue\Biggl(\!\!\begin{array}{r}#1\end{array}\!\!\Biggr)}}
%\newcommand{\augmv}[1]{{\left(\begin{array}{rr|r}#1\end{array}\right)}}
%\newcommand{\taugmv}[1]{{\left(\begin{array}{rrr|r}#1\end{array}\right)}}
%\newcommand{\qaugmv}[1]{{\left(\begin{array}{rrrr|r}#1\end{array}\right)}}
%\newcommand{\augm}[1]{{\blue\left(\begin{array}{rr|r}#1\end{array}\right)}}
%\newcommand{\daugm}[1]{{\blue\left(\begin{array}{rr|rr}#1\end{array}\right)}}
%\newcommand{\taugm}[1]{{\blue\left(\begin{array}{rrr|r}#1\end{array}\right)}}
%\newcommand{\traugm}[1]{{\blue\left(\begin{array}{rrr|rrr}#1\end{array}\right)}}
%\newcommand{\qaugm}[1]{{\blue\left(\begin{array}{rrrr|r}#1\end{array}\right)}}
%\newcommand{\qtaugm}[1]{{\blue\left(\begin{array}{rrrr|rrr}#1\end{array}\right)}}
%\newcommand{\mydet}[1]{{\blue\left|\begin{matrix}#1\end{matrix}\right|}}
%\newcommand{\mydets}[1]{{\blue\Bigl|\begin{matrix}#1\end{matrix}\Bigr|}}
%\newcommand{\arr}[1]{\overrightarrow{#1}}
\newcommand{\lcm}{\mathrm{lcm}}
\renewcommand{\mod}{\;\mathrm{mod}\;}
%\newcommand{\proj}{\mathrm{proj}}
%\newcommand{\id}{\blue\mathrm{id}}
%\newcommand{\im}{\blue\mathrm{Im}}
%\newcommand{\re}{\blue\mathrm{Re}}
\newcommand{\col}{\blue\mathrm{col}}
\newcommand{\nullity}{\blue\mathrm{nullity}}
\newcommand{\rank}{\blue\mathrm{rank}}
\newcommand{\spn}[1]{\blue\mathrm{span}\left\{#1\right\}}
\newcommand{\spnv}{\blue\mathrm{span}\,}
%\newcommand{\diag}[1]{\blue\mathrm{diag}\left(#1\right)\,}
%\newcommand{\satop}[2]{\stackrel{\scriptstyle{#1}}{\scriptstyle{#2}}}
%\newcommand{\llnot}{\sim\!}
\newcommand{\qed}{\hfill$\Box$}
%\newcommand{\rbullet}{\includegraphics[width=4mm]{red-bullet-on-white.ps}}
%\newcommand{\gbullet}{\includegraphics[width=3mm]{green-bullet-on-white.ps}}
%\newcommand{\ybullet}{\includegraphics[width=3mm]{yellow-bullet-on-white.ps}}
\newcommand{\vsp}{{\psset{unit=4.2mm}\begin{pspicture}(0,1)(0,0)\end{pspicture}}}
\renewcommand{\emptyset}{\varnothing}
\newcommand{\mq}{\text{\red ?}}
%\renewcommand{\emph}[1]{{\blue\textsl{#1}}}

%\newcommand{\shat}{\begin{pspicture}(0,0)(0,0)\psset{linecolor=black,linewidth=.075}\psarc*(.67,3.18){.25}{45}{225}\psline(.35,2.85)(1,3.5)\end{pspicture}}
%\newcommand{\rhat}{\begin{pspicture}(0,0)(0,0)\psset{linecolor=red,linewidth=.05}\psarc*(.675,3.175){.25}{45}{225}\psline(.35,2.85)(1,3.5)\end{pspicture}}
%\newcommand{\bhat}{\begin{pspicture}(0,0)(0,0)\psset{linecolor=blue,linewidth=.05}\psarc*(.675,3.175){.25}{45}{225}\psline(.35,2.85)(1,3.5)\end{pspicture}}
%\newcommand{\sdress}{\begin{pspicture}(0,0)(0,0)\pstriangle*[linecolor=black,linewidth=.05](1,.5)(1.2,1.75)\end{pspicture}}
%\newcommand{\rdress}{\begin{pspicture}(0,0)(0,0)\pstriangle*[linecolor=red,linewidth=.05](1,.5)(1.2,1.75)\end{pspicture}}
%\newcommand{\bdress}{\begin{pspicture}(0,0)(0,0)\pstriangle*[linecolor=brown,linewidth=.05](1,.5)(1.2,1.75)\end{pspicture}}
%\newcommand{\ydress}{\begin{pspicture}(0,0)(0,0)\pstriangle*[linecolor=amber,linewidth=.05](1,.5)(1.2,1.75)\end{pspicture}}
%\newcommand{\gdress}{\begin{pspicture}(0,0)(0,0)\pstriangle*[linecolor=green,linewidth=.05](1,.5)(1.2,1.75)\end{pspicture}}
%\newcommand{\sshoes}{\begin{pspicture}(0,0)(0,0)\psset{linecolor=black,linewidth=.05}
% \psline(.6,0)(.85,0.125)\psline(0.85,0)(0.85,0.5)\psline(1.15,0)(1.15,0.5)\psline(1.4,0)(1.15,0.125)\end{pspicture}}
%\newcommand{\rshoes}{\begin{pspicture}(0,0)(0,0)\psset{linecolor=black,linewidth=.05}
%  \psline(0.85,0)(0.85,0.5)\pscircle*[linecolor=red](.65,.05){.05}\psframe*[linecolor=red](.65,0)(.90,.1)
%  \psline(1.15,0)(1.15,0.5)\psframe*[linecolor=red](1.1,0)(1.35,0.1)\pscircle*[linecolor=red](1.35,.05){.05}\end{pspicture}}
%
%\newcommand{\woman}[1]{\begin{pspicture}(0.25,0)(1.75,3.4)
% \psset{unit=4mm,dotsize=0.09,linewidth=.03,fillcolor=black,linecolor=black,shadow=false}
% \psline(1,2)(1,2.35)\psline(0.25,1.75)(1,2)\psline(1.75,1.75)(1,2)\pscircle(1,2.85){0.5}
% \psarc(1,2.85){.25}{-140}{-40}\psdots(.85,2.95)(1.15,2.95)
% \psset{linecolor=brown,linewidth=.1}
% \psarc(1,2.85){.5}{0}{180}\psarc(1.75,2.85){.25}{180}{270}\psarc(.25,2.85){.25}{-90}{0}
% \psset{linewidth=.06}#1
% \end{pspicture}}

\newcommand{\dicei}{{\psset{unit=6mm,shadow=false,linewidth=.05}\begin{pspicture}(0,0.3)(1,1.3)\begin{psframe}(0,0)(1,1)\qdisk(0.5,0.5){0.1\psunit}\end{psframe}\end{pspicture}}}
\newcommand{\diceii}{{\psset{unit=6mm,shadow=false,linewidth=.05}\begin{pspicture}(0,0.3)(1,1.3)\begin{psframe}(0,0)(1,1)\qdisk(0.2,0.2){0.1\psunit}\qdisk(0.8,0.8){0.1\psunit}\end{psframe}\end{pspicture}}}
\newcommand{\diceiii}{{\psset{unit=6mm,shadow=false,linewidth=.05}\begin{pspicture}(0,0.3)(1,1.3)\begin{psframe}(0,0)(1,1)\qdisk(0.2,0.2){0.1\psunit}\qdisk(0.5,0.5){0.1\psunit}\qdisk(0.8,0.8){0.1\psunit}\end{psframe}\end{pspicture}}}
\newcommand{\diceiv}{{\psset{unit=6mm,shadow=false,linewidth=.05}\begin{pspicture}(0,0.3)(1,1.3)\begin{psframe}(0,0)(1,1)\qdisk(0.2,0.2){0.1\psunit}\qdisk(0.2,0.8){0.1\psunit}\qdisk(0.8,0.8){0.1\psunit}\qdisk(0.8,0.2){0.1\psunit}\end{psframe}\end{pspicture}}}
\newcommand{\dicev}{{\psset{unit=6mm,shadow=false,linewidth=.05}\begin{pspicture}(0,0.3)(1,1.3)\begin{psframe}(0,0)(1,1)\qdisk(0.2,0.2){0.1\psunit}\qdisk(0.2,0.8){0.1\psunit}\qdisk(0.5,0.5){0.1\psunit}\qdisk(0.8,0.8){0.1\psunit}\qdisk(0.8,0.2){0.1\psunit}\end{psframe}\end{pspicture}}}
\newcommand{\dicevi}{{\psset{unit=6mm,shadow=false,linewidth=.05}\begin{pspicture}(0,0.3)(1,1.3)\begin{psframe}(0,0)(1,1)\qdisk(0.2,0.2){0.1\psunit}\qdisk(0.2,0.5){0.1\psunit}\qdisk(0.2,0.8){0.1\psunit}\qdisk(0.8,0.8){0.1\psunit}\qdisk(0.8,0.5){0.1\psunit}\qdisk(0.8,0.2){0.1\psunit}\end{psframe}\end{pspicture}}}
\newcommand{\diceframe}{{\psset{unit=6mm,shadow=false,linewidth=.05}\begin{pspicture}(0,0.3)(1,1.3)\begin{psframe}(-.25,-.25)(1.25,1.25)\end{psframe}\end{pspicture}}}

\newcommand{\pigeon}{{\psset{xunit=6mm,yunit=6mm,runit=6mm,linewidth=.8pt,shadow=false,linecolor=darkgray,fillcolor=white,fillstyle=solid}
  \begin{pspicture}(0,0)(1.9, 1){\pscustom{\newpath
    \moveto(1.179, 0.088)
    \curveto(1.156, 0.239)(1.402, 0.255)(1.451, 0.464)
    \curveto(1.498, 0.668)(1.42 , 0.774)(1.75 , 0.812)
    \curveto(1.656, 0.887)(1.528, 0.997)(1.399, 0.979)
    \curveto(1.067, 0.929)(1.269, 0.646)(0.156, 0.241)
    \curveto(0    , 0.185)(0.869, 0.396)(1.037, 0.205)
    \curveto(1.201, 0.016)(0.902, 0    )(1.17 , 0.003)
    \curveto(1.37 , 0.005)(1.197, 0.003)(1.179, 0.088)
    \closepath}}
    \pscircle(1.447, 0.9){0.045}
  \end{pspicture}}}


%\newcommand{\dicegrid}[5]{\[\begin{pspicture}(-1,-1.2)(20,8)\psset{shadow=false,linecolor=darkgray}
%    \rput(0  ,0  ){\dicevi} \rput(0  ,1.2){\dicev} \rput(0  ,2.4){\diceiv}\rput(0  ,3.6){\diceiii}
%    \rput(0  ,4.8){\diceii} \rput(0  ,6  ){\dicei} \rput(1.5,7.55){\dicei} \rput(2.7,7.55){\diceii}
%    \rput(3.9,7.55){\diceiii}\rput(5.1,7.55){\diceiv}\rput(6.3,7.55){\dicev} \rput(7.5,7.55){\dicevi}
%    \psset{linecolor=gray}\psline( .75,-1)( .75,8)\psline(-.75, 6.5)(8.25,6.5)\psline(-.75,-1)(-.75,8)
%    \psline(-.75,8)(8.25,8)\psline(-.75,-1)(8.25,-1)\psline(8.25,-1)(8.25,8)\psline(-.75, 8)(.75,6.5)
%    \put(-.55,6.6){1}\put(.25,7.25){2}
%    \psset{linecolor=blue}#1\psset{linecolor=red}#2\put(9.5,4.3){{#3}}\put(9.5,3.1){{#4}}\put(9.5,1.9){{#5}}
%  \end{pspicture}\]}
%
%\newcommand{\bnaa}{\pscircle*(1.5,5.7){3pt}}\newcommand{\bnab}{\pscircle*(2.7,5.7){3pt}}\newcommand{\bnac}{\pscircle*(3.9,5.7){3pt}}\newcommand{\bnad}{\pscircle*(5.1,5.7){3pt}}\newcommand{\bnae}{\pscircle*(6.3,5.7){3pt}}\newcommand{\bnaf}{\pscircle*(7.5,5.7){3pt}}
%\newcommand{\bnba}{\pscircle*(1.5,4.5){3pt}}\newcommand{\bnbb}{\pscircle*(2.7,4.5){3pt}}\newcommand{\bnbc}{\pscircle*(3.9,4.5){3pt}}\newcommand{\bnbd}{\pscircle*(5.1,4.5){3pt}}\newcommand{\bnbe}{\pscircle*(6.3,4.5){3pt}}\newcommand{\bnbf}{\pscircle*(7.5,4.5){3pt}}
%\newcommand{\bnca}{\pscircle*(1.5,3.3){3pt}}\newcommand{\bncb}{\pscircle*(2.7,3.3){3pt}}\newcommand{\bncc}{\pscircle*(3.9,3.3){3pt}}\newcommand{\bncd}{\pscircle*(5.1,3.3){3pt}}\newcommand{\bnce}{\pscircle*(6.3,3.3){3pt}}\newcommand{\bncf}{\pscircle*(7.5,3.3){3pt}}
%\newcommand{\bnda}{\pscircle*(1.5,2.1){3pt}}\newcommand{\bndb}{\pscircle*(2.7,2.1){3pt}}\newcommand{\bndc}{\pscircle*(3.9,2.1){3pt}}\newcommand{\bndd}{\pscircle*(5.1,2.1){3pt}}\newcommand{\bnde}{\pscircle*(6.3,2.1){3pt}}\newcommand{\bndf}{\pscircle*(7.5,2.1){3pt}}
%\newcommand{\bnea}{\pscircle*(1.5, .9){3pt}}\newcommand{\bneb}{\pscircle*(2.7, .9){3pt}}\newcommand{\bnec}{\pscircle*(3.9, .9){3pt}}\newcommand{\bned}{\pscircle*(5.1, .9){3pt}}\newcommand{\bnee}{\pscircle*(6.3, .9){3pt}}\newcommand{\bnef}{\pscircle*(7.5, .9){3pt}}
%\newcommand{\bnfa}{\pscircle*(1.5,-.3){3pt}}\newcommand{\bnfb}{\pscircle*(2.7,-.3){3pt}}\newcommand{\bnfc}{\pscircle*(3.9,-.3){3pt}}\newcommand{\bnfd}{\pscircle*(5.1,-.3){3pt}}\newcommand{\bnfe}{\pscircle*(6.3,-.3){3pt}}\newcommand{\bnff}{\pscircle*(7.5,-.3){3pt}}
%
%\newcommand{\cnaa}{\pscircle(1.5,5.7){4.5pt}}\newcommand{\cnab}{\pscircle(2.7,5.7){4.5pt}}\newcommand{\cnac}{\pscircle(3.9,5.7){4.5pt}}\newcommand{\cnad}{\pscircle(5.1,5.7){4.5pt}}\newcommand{\cnae}{\pscircle(6.3,5.7){4.5pt}}\newcommand{\cnaf}{\pscircle(7.5,5.7){4.5pt}}
%\newcommand{\cnba}{\pscircle(1.5,4.5){4.5pt}}\newcommand{\cnbb}{\pscircle(2.7,4.5){4.5pt}}\newcommand{\cnbc}{\pscircle(3.9,4.5){4.5pt}}\newcommand{\cnbd}{\pscircle(5.1,4.5){4.5pt}}\newcommand{\cnbe}{\pscircle(6.3,4.5){4.5pt}}\newcommand{\cnbf}{\pscircle(7.5,4.5){4.5pt}}
%\newcommand{\cnca}{\pscircle(1.5,3.3){4.5pt}}\newcommand{\cncb}{\pscircle(2.7,3.3){4.5pt}}\newcommand{\cncc}{\pscircle(3.9,3.3){4.5pt}}\newcommand{\cncd}{\pscircle(5.1,3.3){4.5pt}}\newcommand{\cnce}{\pscircle(6.3,3.3){4.5pt}}\newcommand{\cncf}{\pscircle(7.5,3.3){4.5pt}}
%\newcommand{\cnda}{\pscircle(1.5,2.1){4.5pt}}\newcommand{\cndb}{\pscircle(2.7,2.1){4.5pt}}\newcommand{\cndc}{\pscircle(3.9,2.1){4.5pt}}\newcommand{\cndd}{\pscircle(5.1,2.1){4.5pt}}\newcommand{\cnde}{\pscircle(6.3,2.1){4.5pt}}\newcommand{\cndf}{\pscircle(7.5,2.1){4.5pt}}
%\newcommand{\cnea}{\pscircle(1.5, .9){4.5pt}}\newcommand{\cneb}{\pscircle(2.7, .9){4.5pt}}\newcommand{\cnec}{\pscircle(3.9, .9){4.5pt}}\newcommand{\cned}{\pscircle(5.1, .9){4.5pt}}\newcommand{\cnee}{\pscircle(6.3, .9){4.5pt}}\newcommand{\cnef}{\pscircle(7.5, .9){4.5pt}}
%\newcommand{\cnfa}{\pscircle(1.5,-.3){4.5pt}}\newcommand{\cnfb}{\pscircle(2.7,-.3){4.5pt}}\newcommand{\cnfc}{\pscircle(3.9,-.3){4.5pt}}\newcommand{\cnfd}{\pscircle(5.1,-.3){4.5pt}}\newcommand{\cnfe}{\pscircle(6.3,-.3){4.5pt}}\newcommand{\cnff}{\pscircle(7.5,-.3){4.5pt}}
%
\newcommand{\clearemptydoublepage}
  {\newpage{\pagestyle{empty}{\cleardoublepage}}}


% ------------------------------------------------------------------------
%  Environments
% ------------------------------------------------------------------------

\newcommand{\example}{{\sffamily\darkgreen Example }}
\newcommand{\exercise}{{\sffamily\darkgreen Exercise }}
\newcommand{\proof}{{\sffamily\darkgreen Proof }}
\newcommand{\notes}{{\sffamily\darkgreen Notes }}
\newcommand{\note}{{\sffamily\darkgreen Note }}
\newcommand{\theorem}{{\sffamily\darkgreen Theorem }}
\newcommand{\proposition}{{\sffamily\darkgreen Proposition }}
\newcommand{\corollary}{{\sffamily\darkgreen Corollary }}
\newcommand{\lemma}{{\sffamily\darkgreen Lemma }}
\newcommand{\definition}{{\sffamily\darkgreen Definition}}
\newcommand{\problem}{{\sffamily\darkgreen Problem}}
\newcommand{\remark}{{\sffamily\darkgreen Remark}}

% ------------------------------------------------------------------------

\renewcommand{\section}
  {\clearemptydoublepage\refstepcounter{section} \secdef \cmda \cmdb}
\newcommand{\cmda}[2][]
  {{\scriptsize{\textbf{MATH5505 \:\: Combinatorics}}
   \hfill{\scriptsize{\textsl{Thomas Britz}}}\vspace{0.1cm}\\
   {\bfseries\Large\S\arabic{section} \sffamily #2}}}
\newcommand{\cmdb}[1]{{\bfseries\huge\S\,\sffamily #1}}

\pagestyle{empty}

\newcommand{\frontpage}{}


\begin{document}
%% Very important: sans-serif font throughout
\sf

\newcommand{\lecturetitle}{\vphantom{ }\vspace{15mm}\begin{center}
  {\large\sc Ramsey Theory}\end{center}}

\newcommand{\lecturetitlei}{\vphantom{ }\vspace{15mm}\begin{center}
  {\large\sc Ramsey Theory}\\[3mm]
  \darkgreen Lecture 3: Arithmetic Progressions\end{center}}

\newcommand{\aaia}{\vphantom{ }\vspace{15mm}\begin{center}
  \begin{pspicture}(0,2.5)(10,2.5)
    \rput( 0,2.5){\man}
    \rput( 5,2.5){\woman}
    \rput(10,2.5){\woman}
  \end{pspicture}
  \end{center}}

\newcommand{\pigeonthma}{{\dg The Pigeonhole Principle}
  \\[1mm]If $\bl k+1$ pigeons are put into $\bl k$ pigeonholes,\\
         then some pigeonhole contains at least two pigeons.}

\newcommand{\pigeonthmplusa}{{\dg The Pigeonhole Principle} {\gray (general)}
  \\[1mm]If $\bl km+1$ pigeons are put into $\bl k$ pigeonholes,\\
         then some pigeonhole contains at least $\bl m+1$ pigeons.}

\newcommand{\pigeonthmstrong}{{\dg The Pigeonhole Principle} {\gray (strong)}
  \\[1mm]If $\bl (n_1-1) +\cdots+(n_k-1) + 1$ pigeons are put into $\bl k$ pigeonholes,\\
         then some $\bl i$th pigeonhole contains at least $\bl n_i$ pigeons.}

\newcommand{\pigeonthmexaa}{\\[8mm]\example}
\newcommand{\pigeonthmexab}[1]{\vspace*{-1mm}\begin{center}
  \begin{pspicture}(0,0)(8.125,2.5)\psset{linecolor=darkgreen,shadow=false}
    \multiput(0,0)(3.5,0){3}{\psframe(-1.25,-1.25)( 1.25, 1.25)}\psset{linecolor=darkgray}#1
  \end{pspicture}
  \end{center}}
\newcommand{\pigeonthmexaba}{\rput(0  ,0){\pigeon}}
\newcommand{\pigeonthmexabb}{\rput(3.5,0){\pigeon}}
\newcommand{\pigeonthmexabc}{\rput(7  ,0){\pigeon}}
\newcommand{\pigeonthmexabd}{\rput(3.4,-4){$\begin{array}{rcl}\bl k &\meq&\bl 3\\\bl m &\meq&\bl 2\end{array}$}}
\newcommand{\pigeonthmexabaii}{\rput(-0.25,0){\pigeon}\rput( .25,0){\pigeon}}
\newcommand{\pigeonthmexabbii}{\rput( 3.25,0){\pigeon}\rput(3.75,0){\pigeon}}
\newcommand{\pigeonthmexabcii}{\rput( 6.75,0){\pigeon}\rput(7.25,0){\pigeon}}
\newcommand{\pigeonthmexabciii}{\rput(6.7 ,0){\pigeon}\rput(7   ,0){\pigeon}\rput(7.3,0){\pigeon}}

\newcommand{\pigeonthmexga}{\\[8mm]\exercise}
\newcommand{\pigeonthmexgb}{\\[1mm]$\bl n+1$ distinct numbers are chosen from the numbers $\bl 1,\ldots,2n$.
  \\Show that at least two of the chosen numbers are coprime.}

\newcommand{\pigeonthmexma}{\\[8mm]\example}
\newcommand{\pigeonthmexmb}[1]{\vspace*{-1mm}\begin{center}
  \begin{pspicture}(0,0)(8.125,2.5)\psset{linecolor=darkgreen,shadow=false}
    \multiput(0,0)(3.5,0){3}{\psframe(-1.25,-1.25)( 1.25, 1.25)}
    \rput[t](0  ,-2.5){$\bl n_1\meq 1$}
    \rput[t](3.5,-2.5){$\bl n_2\meq 3$}
    \rput[t](7  ,-2.5){$\bl n_3\meq 2$}
    \psset{linecolor=darkgray}#1
  \end{pspicture}
  \end{center}}

%\newcommand{\ramseyaa}{{\dg Ramsey Theory} makes repeated use of the {\dg Pigeonhole Principle}
%to show that order exists whenever (certain) random structures are sufficiently big.}

\newcommand{\ramseyaa}{{\dg Ramsey Theory} makes repeated use of the {\dg Pigeonhole Principle}
to show that order exists whenever (certain) random and seemingly unordered structures are sufficiently large.}

\newcommand{\ramseyaaexaa}{\\[8mm]\exercise}
\newcommand{\ramseyaaexab}{\\Colour each edge of the complete graph $\bl K_6$ either {\red red} or {\bl blue}.
                           \\Show that there must be a red triangle $\red\triangle$ or a blue triangle $\bl\triangle$.}

\newcommand{\ramseyaaexac}[1]{\vspace*{10mm}\begin{center}\myksix{#1}{}\hspace*{15mm}\ph{b}\end{center}}
\newcommand{\ramseyaaexacalt}[2]{\vspace*{10mm}\begin{center}\myksix{#1}{#2}\hspace*{15mm}\ph{b}\end{center}}
\newcommand{\ramseyaaexaca}{{\psset{linecolor=red}
  \psline(a)(c)\psline(e)(f)\psline(a)(e)\psline(c)(f)
  \psline(a)(d)\psline(b)(e)\psline(b)(d)}{\psset{linecolor=blue}
  \psline(a)(b)\psline(b)(f)\psline(c)(e)\psline(d)(f)
  \psline(d)(e)\psline(a)(f)\psline(b)(c)\psline(c)(d)}}
\newcommand{\ramseyaaexacb}{{\psset{linecolor=blue,fillcolor=lightblue,fillstyle=solid,linearc=0,opacity=.3}\pspolygon(a)(b)(f)}}
\newcommand{\ramseyaaexacc}{{\psset{linecolor=blue,fillcolor=lightblue,fillstyle=solid,linearc=0,opacity=.3}\pspolygon(c)(d)(e)}}
\newcommand{\ramseyaaexacalta}{{\pscircle[linecolor=darkgreen,fillcolor=green](a){.2}}}
\newcommand{\ramseyaaexacd}{
  {\psset{linecolor=lightgray}\psline(b)(c)\psline(b)(d)\psline(b)(e)\psline(b)(f)
                              \psline(c)(d)\psline(c)(e)\psline(c)(f)
                              \psline(d)(e)\psline(d)(f)\psline(e)(f)}
  {\psset{linecolor=red }\psline(a)(c)\psline(a)(e)\psline(a)(d)}
  {\psset{linecolor=blue}\psline(a)(b)\psline(a)(f)}}
\newcommand{\ramseyaaexace}{{\psset{linecolor=blue}\psline(c)(d)(e)(c)}}
\newcommand{\ramseyaaexacf}{\ramseyaaexacc}
\newcommand{\ramseyaaexacg}{{\psset{linecolor=blue}\psline(c)(e)(d)}{\psset{linecolor=red}\psline(a)(c)(d)}}
\newcommand{\ramseyaaexach}{{\psset{linecolor=blue}\psline(c)(e)(d)}{\psset{linecolor=red,fillcolor=red,fillstyle=solid,linearc=0,opacity=.3}\pspolygon(a)(c)(d)}}

\newcommand{\ramseyb}{{\dg Ramsey's Theorem} {\gray (1930)} {\halfgray (simple)}\\
   If $\bl k,m\myin\mathbb{N}$ and $\bl n$ is sufficiently large,
   then each $\bl k$-colouring of the edges of $\bl K_n$ gives a complete subgraph $\bl K_m$
   with monochromatic edges.}
\newcommand{\ramseyba}{\\[2mm]{\gray The least such $\lightblue n$ is denoted $\lightblue R(m;k)$.}}

\newcommand{\ramseybexaa}{\\[6mm]\example}
\newcommand{\ramseybexab}[1]{\\Colour each edge of $\bl K_n\meq K_6$ either {\red red} or {\bl blue} ($\bl k\meq 2$).
                          \\There must be a red triangle $\red K_3 \black = \red \triangle$ or a blue triangle $\bl K_3\meq \triangle$.
                          \vspace*{4mm}\begin{center}\myksix{\ramseyaaexaca#1}{}\hspace*{15mm}\ph{b}\end{center}}
\newcommand{\ramseybexaba}{\rput[l](4,-1.4){\begin{tabular}{l}\exercise\\Show that $\bl R(3;2) \meq 6$.\end{tabular}}}

\newcommand{\ramseybexaav}{\\[6mm]\example}
\newcommand{\ramseybexabv}[1]{\\Colour each edge ($\bl r\meq 2$) of $\bl K_n\meq K_6$ either {\red red} or {\bl blue}.
                          \\There must be a red triangle $\red K_3 \black = \red \triangle$ or a blue triangle $\bl K_3\meq \triangle$.
                          \vspace*{4mm}\begin{center}\myksix{\ramseyaaexaca#1}{}\hspace*{15mm}\ph{b}\end{center}}
\newcommand{\ramseybexabav}{\rput[l](5,0){$\bl R(3,3) \meq 6$}}


\newcommand{\ramseyc}{{\dg Ramsey's Theorem} {\gray (1930)} {\lightgray (reduced)}\\
   If $\bl n_1,\ldots,n_k\myin\mathbb{N}$ and $\bl n$ is sufficiently large,\\
   then each colouring of the edges of $\bl K_n$ with colours $\bl c_1,\ldots,c_k$\\
   gives a subgraph $\bl K_{n_i}$ with all edges coloured $\bl c_i$ for some $\bl i$.}
\newcommand{\ramseyca}{\\[2mm]{\gray The least such $\lightblue n$ is denoted $\lightblue R(n_1,\ldots,n_k)$.}}
\newcommand{\ramseycb}{\\{\gray Note that $\lightblue R(k;m) = R(\underbrace{m,\ldots,m}_k)$.}}

\newcommand{\ramseybexcaa}{\\[6mm]\example}
\newcommand{\ramseybexcab}{\\$\bl R(3,3) \meq R(2;3) \meq 6$}

\newcommand{\ramseycpfia}{\\[6mm]{\dg Proof 1 }}
\newcommand{\ramseycpfib}{\\[.5mm]This is the same proof as before, except now use
  the strong version of the {\dg Pigeonhole Principle}.}

\newcommand{\ramseybexcba}{\\[6mm]\exercise}
\newcommand{\ramseybexcbb}{\\Fill in the details of this proof!}

\newcommand{\ramseybexda}{\\[6mm]\lemma {\gray (Erd\H{o}s \&\ Szekeres 1935)}}
\newcommand{\ramseybexdav}{\lemma}
\newcommand{\ramseybexdb}{\\[.5mm]$\bl R(k,\ell)\mleq R(k-1,\ell) + R(k,\ell-1)$}

\newcommand{\ramseybexdaexaa}{\\[6mm]\example}
\newcommand{\ramseybexdaexab}{\\[.5mm]For any $\bl k\geq 2$, we note that $\bl R(k,2) \meq R(2,k)\meq k$. {\gray (Why?)}}
\newcommand{\ramseybexdaexac}{\\The above bound is therefore tight for $\bl R(3,3)$:
   \[\bl R(3,3) \meq 6 \mleq 3 + 3 \meq R(2,3) + R(3,2)\,\black.\vspace*{-6mm}\]}


\newcommand{\ramseybexea}{\\[6mm]\lemma}
\newcommand{\ramseybexeav}{\lemma}
\newcommand{\ramseybexeb}{\\$\bl R(n_1,\ldots,n_k)\mleq R\bigl(n_1,\ldots,n_{k-2},R(n_{k-1},n_k)\bigr)$}


\newcommand{\ramseynotaa}{\\[6mm]Define $\bl [n] \black :\meq \{1,\ldots,n\}$
                                 \; and \; $\bl\binom{S}{k} \black :\meq \{X\subseteq S\;:\; |X| \meq k\}$.}
\newcommand{\ramseynotaav}{Define $\bl [n] \black :\meq \{1,\ldots,n\}$
                                 \; and \; $\bl\binom{S}{k} \black :\meq \{X\subseteq S\;:\; |X| \meq k\}$.}

\newcommand{\ramseynotaaexa}{\\[6mm]\example}
\newcommand{\ramseynotaaexb}{\vspace*{-3mm}\[\bl [3] \meq \{1,2,3\} \qquad \binom{[3]}{2} \meq \bigl\{\{1,2\},\{1,3\},\{2,3\}\bigr\}\]}

\newcommand{\ramseyd}{{\dg Ramsey's Theorem} {\gray (1930)} {\lightgray (reduced)}\\
   If $\bl n_1,\ldots,n_k\myin\mathbb{N}$ and $\bl n$ is sufficiently large,\\
   then each colouring of $\bl \binom{[n]}{2}$ with colours $\bl c_1,\ldots,c_k$\\
   gives a $\bl c_i$-coloured subfamily $\bl \binom{S}{2}$
   for some $\bl i$ and $\bl n_i$-subset $\bl S\subseteq [n]$.}

\newcommand{\ramseydv}{{\dg Ramsey's Theorem} {\gray (1930)} {\lightgray (reduced)}\\
   If $\bl n_1,\ldots,n_k\myin\mathbb{N}$ and $\bl n$ is sufficiently large,\\
   then each colouring of $\bl \binom{[n]}{\red2\bl}$ with colours $\bl c_1,\ldots,c_k$\\
   gives a $\bl c_i$-coloured subfamily $\bl \binom{S}{\red2\bl}$
   for some $\bl i$ and $\bl n_i$-subset $\bl S\subseteq [n]$.}

\newcommand{\ramseye}{{\dg Ramsey's Theorem} {\gray (1930)}\\
   If $\bl n_1,\ldots,n_k,r\myin\mathbb{N}$ and $\bl n$ is sufficiently large,\\
   then each colouring of $\bl \binom{[n]}{r}$ with colours $\bl c_1,\ldots,c_k$\\
   gives a $\bl c_i$-coloured subfamily $\bl \binom{S}{r}$
   for some $\bl i$ and $\bl n_i$-subset $\bl S\subseteq [n]$.}

\newcommand{\ramseyea}{\\[2mm]{\gray The least such $\lightblue n$ is denoted $\lightblue R_r(n_1,\ldots,n_k)$.}}

%\newcommand{\arithma}{An {\dg arithmetic progression} is a sequence of numbers of the form
%  \[\bl\hspace*{-27mm}{\lightblue a+[0,k)d \gray\,\;=}\;\;\,
%    a\,,\; a+d\,,\; a+2d\,, \;\ldots\,, \;a+kd\,.
%    \begin{pspicture}(-4,0)(-4,0)\psset{linecolor=darkgreen,framearc=0,fillstyle=solid,fillcolor=darkgreen}
%    \psframe(0,0)(.5,1)\psframe(1,0)(1.5,1.3)\psframe(2,0)(2.5,1.6)\end{pspicture}\]}
%\newcommand{\arithmav}{An {\dg arithmetic progression} is a sequence of numbers of the form
%  \[\bl\hspace*{-27mm}{\lightblue a+[0,k)d \gray\;\,=}\;\;\,
%    a\,,\; a+d\,,\; a+2d\,, \;\ldots\,, \;a+kd\,.
%    \begin{pspicture}(-4,0)(-4,0)\psset{linecolor=darkgreen,framearc=0,fillstyle=solid,fillcolor=darkgreen}
%    \psframe(0,0)(.5,1)\psframe(1,0)(1.5,1.3)\psframe(2,0)(2.5,1.6)\end{pspicture}\vspace*{6mm}\]}
\newcommand{\arithma}{An {\dg arithmetic progression} is a sequence of numbers of the form
  \[\bl\hspace*{-27mm}{\white a+[0,k)d \white\,\;=}\;\;\,
    a\,,\; a+d\,,\; a+2d\,, \;\ldots\,, \;a+kd\,.
    \begin{pspicture}(-4,0)(-4,0)\psset{linecolor=darkgreen,framearc=0,fillstyle=solid,fillcolor=darkgreen}
    \psframe(0,0)(.5,1)\psframe(1,0)(1.5,1.3)\psframe(2,0)(2.5,1.6)\end{pspicture}\]}
\newcommand{\arithmav}{An {\dg arithmetic progression} is a sequence of numbers of the form
  \[\bl\hspace*{-27mm}{\white a+[0,k)d \white\;\,=}\;\;\,
    a\,,\; a+d\,,\; a+2d\,, \;\ldots\,, \;a+kd\,.
    \begin{pspicture}(-4,0)(-4,0)\psset{linecolor=darkgreen,framearc=0,fillstyle=solid,fillcolor=darkgreen}
    \psframe(0,0)(.5,1)\psframe(1,0)(1.5,1.3)\psframe(2,0)(2.5,1.6)\end{pspicture}\vspace*{6mm}\]}

\newcommand{\arithmaexa}{\example}
\newcommand{\arithmaexb}[2]{\[\bl
  1,2,3,4,5
  \qquad\qquad {#1{10,13,16,19}}
  \qquad\qquad {#2{\red 3,5,7,11\psset{linecolor=red}\psline*(0.3,0)(-3,.44)\psline*(0.3,0.44)(-3,0)}}
  \vspace*{6mm}\]}

\newcommand{\vanderwa}{{\dg Van der Waerden's Theorem} {\gray (1927)}\\
   If $\bl k,r\myin\mathbb{N}$ and $\bl n$ is sufficiently large,
   then each $\bl r$-colouring of $\bl [n]$ gives a monochromatic arithmetic progression of length~$\bl k$.}

\newcommand{\vanderwaa}{\\[2mm]{\gray The least such $\lightblue n$ is denoted $\lightblue W(k,r)$.}}

\newcommand{\vanderwaexaa}{\\[8mm]\example}
\newcommand{\vanderwaexab}[1]{\\{\psset{unit=5mm}\pspicture(-7,-1.5)(10,2)#1\endpspicture}}
\newcommand{\vanderwaexaba}{\rput(0,0){1}\rput(1,0){2}\rput(2,0){3}\rput(3,0){4}\rput(4,0){5}
   \rput(5,0){6}\rput(6,0){7}\rput(7,0){8}\rput(8,0){9}}
\newcommand{\vanderwaexabb}{{\darkred\footnotesize\rput(1,0){2}\rput(2,0){3}\rput(4,0){5}\rput(7,0){8}
                             \darkgreen\normalsize\rput(0,0){1}\rput(3,0){4}\rput(5,0){6}\rput(6,0){7}\rput(8,0){9}}}
\newcommand{\vanderwaexabc}{
   \psset{linecolor=halfgray,fillcolor=lightgray,shadow=false,linewidth=.03,fillstyle=solid,framearc=0}
   \psframe(-.5,-.5)(.5,.5)\psframe(2.5,-.5)(3.5,.5)\psframe(5.5,-.5)(6.5,.5)}
\newcommand{\vanderwaexabd}{
   \psset{linecolor=halfgray,fillcolor=lightgray,shadow=false,linewidth=.03,fillstyle=solid,framearc=0}
   \psframe(.5,-.5)(1.5,.5)\psframe(3.5,-.5)(4.5,.5)\psframe(6.5,-.5)(7.5,.5)}

\newcommand{\vanderwaexba}{\\[8mm]\exercise}
\newcommand{\vanderwaexbb}{\\Explain why $\bl W(2,r)\meq r+1$.}

\newcommand{\vanderwaexca}{\\[4mm]\example}
\newcommand{\vanderwaexcb}{\\Let us prove the theorem for $\bl k\meq 3$, $\bl r\meq 2$ by showing that $\bl W(3,2)\mleq 325$.}
\newcommand{\vanderwaexcc}[1]{\begin{center}\begin{pspicture}(8.6,-.6)(8.6,6.6)
    \psframe(-.5, 5.6 )(4.5, 6.4 )\rput(0,6){1}\rput(2,6){$\cdots$}\rput(4,6){ 5}\rput(-1.5,6){$\bl B_1$}
    \psframe(-.5, 4.6 )(4.5, 5.4 )\rput(0,5){6}\rput(2,5){$\cdots$}\rput(4,5){10}\rput(-1.5,5){$\bl B_2$}
    \multips(2  , 3.25)(0  ,  .25){3}{\pscircle[linecolor=gray,fillcolor=lightgray,fillstyle=solid,shadow=false](0,0){.1}}
    \psframe(-.5, 1.6 )(4.5, 2.4 )\rput(0,2){\footnotesize161}\rput(2,2){$\cdots$}\rput(4,2){\footnotesize165}\rput(-1.5,2){$\bl B_{33}$}
    \multips(2  ,  .25)(0  ,  .25){3}{\pscircle[linecolor=gray,fillcolor=lightgray,fillstyle=solid,shadow=false](0,0){.1}}
    \psframe(-.5,-1.4 )(4.5, -.6 )\rput(0,-1){\footnotesize321}\rput(2,-1){$\cdots$}\rput(4,-1){\footnotesize325}\rput(-1.5,-1){$\bl B_{65}$}#1
  \end{pspicture}\end{center}}
\newcommand{\vanderwaexcca}{
   \rput[l](6,6){By the {\dg Pigeonhole Principle}, the first $\bl 33\mgt 2^5$ $\bl B_\ell$s}
   \rput[l](6,5){contain $\bl B_i$ and $\bl B_{i+d}$ with same $\bl 2$-colouring, say:}
   \psframe(9.5,2.6)(14.5,3.4)\rput(8.5,3){$\bl B_i$}{\psset{linecolor=darkred,fillcolor=red,fillstyle=solid,shadow=false}
     \pscircle(10,3){.2}\pscircle(12,3){.2}\pscircle(14,3){.2}
     \psset{linecolor=darkgreen,fillstyle=none}
     \pscircle(11,3){.2}\pscircle(13,3){.2}}
   \psframe(9.5,0.6)(14.5,1.4)\rput(8.5,1){$\bl B_{i+d}$}{\psset{linecolor=darkred,fillcolor=red,fillstyle=solid,shadow=false}
     \pscircle(10,1){.2}\pscircle(12,1){.2}\pscircle(14,1){.2}
     \psset{linecolor=darkgreen,fillstyle=none}
     \pscircle(11,1){.2}\pscircle(13,1){.2}}}
\newcommand{\vanderwaexccav}{
   \rput[l](6,6){By the {\dg Pigeonhole Principle}, the first $\bl 33\mgt 2^5$ $\bl B_\ell$s}
   \rput[l](6,5){contain $\bl B_i$ and $\bl B_{i+d}$ with same $\bl 2$-colouring, say:}
   \psframe(9.5,2.6)(14.5,3.4)\rput(8.5,3){$\bl B_i$}{\psset{linecolor=darkred,fillcolor=red,fillstyle=solid,shadow=false}
     \pscircle(10,3){.2}\pscircle(12,3){.2}
     \psset{linecolor=darkgreen,fillstyle=none}
     \pscircle(11,3){.2}\pscircle(13,3){.2}\pscircle(14,3){.2}}
   \psframe(9.5,0.6)(14.5,1.4)\rput(8.5,1){$\bl B_{i+d}$}{\psset{linecolor=darkred,fillcolor=red,fillstyle=solid,shadow=false}
     \pscircle(10,1){.2}\pscircle(12,1){.2}
     \psset{linecolor=darkgreen,fillstyle=none}
     \pscircle(11,1){.2}\pscircle(13,1){.2}\pscircle(14,1){.2}}}
\newcommand{\vanderwaexccb}{\psframe[shadow=false,linecolor=darkgreen](9.7,2.7)(12.3,3.3)
     \psframe[shadow=false,linecolor=darkgreen](9.7,0.7)(12.3,1.3)
   \rput[l](6,-1){By the {\dg Pigeonhole Principle}, $\!\bl 2$ colors are the same.}}
\newcommand{\vanderwaexcd}{\psframe(9.5,-1.4)(14.5,-.6)\rput(8.5,-1){$\bl B_{i+2d}$}}
\newcommand{\vanderwaexcda}{\rput(14,-1){?}}
\newcommand{\vanderwaexcdb}{{\psset{linecolor=darkgreen,fillstyle=none,shadow=false}
   \pscircle(14,-1){.2}\psline(14,2.8)(14,1.2)\psline(14,0.8)(14,-.8)}}
\newcommand{\vanderwaexcdc}{{\psset{linecolor=darkred,fillcolor=red,fillstyle=solid,shadow=false}
   \psline(10,3)(14,-1)\pscircle(10,3){.2}\pscircle(12,1){.2}\pscircle(14,-1){.2}}}

\newcommand{\vanderwab}{\\[6mm]Proofs of the theorem use this type of ``Cantor diagonalisation'',
  \\applied to blocks of blocks of blocks.}

%\newcommand{\vanderwac}{\\[6mm]Let $\bl [a,b] \meq \{a,a+1,\ldots,b\}$.}
\newcommand{\vanderwac}{\\[6mm]The {\dg$\bl i$th $\bl k$-equivalence class} of $\bl[0,k ]^m\meq \{0,1,\ldots,k \}^m$ is the set of
   vectors
  \[\bl (x_1,\ldots,x_{m-i},\underbrace{k,\ldots,k}_i)\qquad\mytext{where}\qquad \bl x_1,\ldots,x_{m-i}\mneq k \,.\]}
\newcommand{\vanderwacv}{The {\dg$\bl i$th $\bl k $-equivalence class} of $\bl[0,k ]^m\meq \{0,1,\ldots,k \}^m$ is the set of
   vectors
  \[\bl (x_1,\ldots,x_{m-i},\underbrace{k,\ldots,k}_i)\qquad\mytext{where}\qquad \bl x_1,\ldots,x_{m-i}\mneq k \,.\]}

\newcommand{\vanderwacexaa}{\example}
\newcommand{\vanderwacexab}{\\The three $\bl3$-equivalence classes of $\bl[0,3]^2$:
  \begin{pspicture}(-2.5,1.9)(-2.5,1.9)\psset{fillstyle=solid,fillcolor=offwhite,linecolor=lightgray,shadow=false,framearc=0}
    \psframe(-.25, -.25)(2.25,2.25)
    \psframe(-.25, 2.75)(2.25,3.25)
    \psframe(2.75, 2.75)(3.25,3.25)\scriptsize
    \rput(-.75,0){0}
    \rput(-.75,1){1}
    \rput(-.75,2){2}
    \rput(-.75,3){3}
    \rput(0,-.75){0}
    \rput(1,-.75){1}
    \rput(2,-.75){2}
    \rput(3,-.75){3}
    \multips(0,0)(1,0){4}{\multips(0,0)(0,1){4}{\pscircle[linecolor=gray,fillcolor=halfgray,fillstyle=solid,shadow=false](0,0){.1}}}
  \end{pspicture}}

\newcommand{\vanderwad}{\\[12mm]Let \,{\dg S($\bl k,m$)}\, be the statement that\vspace*{-2mm}
\begin{center}{\small
  \begin{tabular}{l}for each $\bl r\myin\mathbb{N}$, there is some $\bl N\black:=\bl N(k,m,r)$ so that
    \\for each $\bl r$-colouring $\bl\chi:[N]\mto[r]$ there are $\bl a,d_1,\ldots,d_m\myin\mathbb{N}$ so that
    \\$\bl \chi(a+\sum x_id_i)$ is well-defined and constant on each $\bl k$-equivalence class of $\bl[0,k ]^m$.
  \end{tabular}}
\end{center}}

\newcommand{\vanderwadv}{\begin{pspicture}(.5,-.6)(10,1.7)\psset{linecolor=offwhite,fillcolor=nearlywhite,shadow=false,framearc=0}
    \psframe(-.25,-0.6)(22.25,2.35)\psline(-.25,1.15)(2.4,1.15)(2.4,2.35)
    \rput[l](0    ,1.7 ){{\dg S($\bl k,m$)}}
    \rput[l](7.333,1.7 ){\small for each $\bl r\myin\mathbb{N}$, there is some $\bl N\black:=\bl N(k,m,r)$ so that}
    \rput[l](3.322,0.85){\small for each $\bl r$-colouring $\bl\chi:[N]\mto[r]$ there are $\bl a,d_1,\ldots,d_m\myin\mathbb{N}$ so that}
    \rput[l](0  ,0  ){\small$\bl \chi(a+\sum x_id_i)$ is well-defined and constant on each $\bl k$-equivalence class of $\bl[0,k ]^m$.}
  \end{pspicture}}

\newcommand{\vanderwae}{\\[2mm]\lemma}
\newcommand{\vanderwaev}{\lemma}
\newcommand{\vanderwaea}{\quad If \,{\dg S($\bl k,m$)}\, for some $\bl m\mgeq 1$, then \,{\dg S($\bl k,m+1$)}\,.}
\newcommand{\vanderwaeav}{\\If \,{\dg S($\bl k,m$)}\, for some $\bl m\mgeq 1$, then \,{\dg S($\bl k,m+1$)}\,.}



\newcommand{\vanderwaepfa}{\\[2mm]\proof}
\newcommand{\vanderwaepfav}{\\[2mm]\proof {\gray(continued)}}
\newcommand{\vanderwaepfb}{\\Suppose that \,{\dg S($\bl k,m$)}\, for some $\bl m\mgeq 1$
  and note that this implies \,{\dg S($\bl k,1$)}\,.}
\newcommand{\vanderwaepfc}{\\We wish to show that \,{\dg S($\bl k,m+1$)}\,.}
\newcommand{\vanderwaepfd}{\\Letting $\bl r\myin\mathbb{N}$,
  we can find
      $\bl M \mdef N(k,m,r)$
  and $\bl M'\mdef N(k,1,r^M)$.}
\newcommand{\vanderwaepfe}{\\Set $\bl N\meq MM'$ and let $\bl \chi:[N]\mto[r]$ be given.}
\newcommand{\vanderwaepff}{\\There are at most distinct $\bl r^M$ vectors $\bl\bigl(\chi((\ell-1)M+1),\ldots,\chi(\ell M)\bigr)$
  where $\bl \ell\myin[M']$,
  so number these vectors, inducing a colouring $\bl\chi':[M']\mto[r^M]$.}
\newcommand{\vanderwaepfg}{\\Since $\bl M'\meq N(k,1,r^M)$, we can find $\bl a',d'$
  so that $\bl\chi'(a'+x_1d')$
  is well-defined and constant for all $\bl x_1\myin[0,k-1]$.}
\newcommand{\vanderwaepfh}{\\Set $\bl d_1\mdef d'M$;
  then $\bl\bigl(\chi((a'-1)M+x_1d_1+1),\ldots,\chi(a'M+x_1d_1)\bigr)$
  is well-defined and constant for all $\bl x_1\myin[0,k-1]$.}
\newcommand{\vanderwaepfj}{\\Fix $\bl x_1$ and define $\bl\chi'':[M]\mto[r]$ by
  $\bl\chi''(s)\mdef\chi((a'-1)M+x_1d_1+s)$.}
\newcommand{\vanderwaepfk}{\\Since $\bl M\meq N(k,m,r)$,
  there are $\bl a'',d_2,\ldots,d_{m+1}$
  so that $\bl\chi''(a''+\sum_{i=2}^{m+1}x_id_i)$ is well-defined and constant on each $\bl k$-equivalence class of $\bl[0,k]^m$.}
\newcommand{\vanderwaepfm}{\\Setting $\bl a\mdef a''+(a'-1)M$,
  we see that $\bl\chi(a+\sum_{i=1}^{m+1}x_id_i)$ is well-defined and constant on each $\bl k$-equivalence class of $\bl[0,k]^{m+1}$.}
\newcommand{\vanderwaepfn}{\\Hence, $\bl N\meq MM'\geq N(k,m,r)$, and so \,{\dg S($\bl k,m+1$)}\,.\qed}

\newcommand{\vanderwaf}{\\[2mm]\lemma}
\newcommand{\vanderwafa}{\\If \,{\dg S($\bl k,m$)}\, for all $\bl m\mgeq 1$, then \,{\dg S($\bl k+1,1$)}\,.}
\newcommand{\vanderwafav}{\quad If \,{\dg S($\bl k,m$)}\, for all $\bl m\mgeq 1$, then \,{\dg S($\bl k+1,1$)}\,.}

\newcommand{\vanderwafpfa}{\\[2mm]\proof}
\newcommand{\vanderwafpfb}{\\[-1mm]Suppose that \,{\dg S($\bl k,m$)}\, for all $\bl m\mgeq 1$. }
\newcommand{\vanderwafpfc}{We wish to show that \,{\dg S($\bl k+1,1$)}\,.}
\newcommand{\vanderwafpfd}{\\Therefore, let $\bl r\myin\mathbb{N}$ and $\bl\chi:[N(k,r,r)]\mto [r]$.}
\newcommand{\vanderwafpfe}{\\Then there are $\bl a',d_1,\ldots,d_r\myin\mathbb{N}$ so that $\bl\chi(a'+\sum x_id_i)$
is well-defined and constant on each $\bl k$-equivalence class of $\bl[0,k]^r$.}
\newcommand{\vanderwafpfev}{\\Then there are $\bl a',d_1,\ldots,d_r\myin\mathbb{N}$ so that
  \rput(0,0){\psframe[fillstyle=solid,fillcolor=offwhite,linecolor=lightgray,shadow=false,framearc=0](-.12,-.3)(4.45,.65)}%
  $\bl\chi(a'+\sum x_id_i)$ is well-defined and constant on each $\bl k$-equivalence class of $\bl[0,k]^r$.}
\newcommand{\vanderwafpff}{\\There are at most $\bl r$ function values of $\bl\chi$,
  so by the {\dg Pigeonhole Principle},
  at least two of the $\bl r+1$ values $\bl\chi(a'),\chi(a'+k d_r),\ldots,\chi(a'+\sum_{i=1}^rk d_i)$
  must be the same,}
\newcommand{\vanderwafpfg}{ say
  $\bl\chi(a'+\sum_{i=u}^rk d_i) \meq \chi(a'+\sum_{i=v}^rk d_i)$ for $\bl u\mlt v$.}
\newcommand{\vanderwafpfh}{\\Set $\bl a\mdef a'+\sum_{i=v}^r k d_i$ and $\bl d\mdef \sum_{i=u}^{v-1} d_i$.}
\newcommand{\vanderwafpfj}{\\We have seen that $\bl\chi(a+xd)
  \meq \chi(a'+ \sum_{i=1}^{u-1} 0d_i+ \sum_{i=u}^{v-1} xd_i + \sum_{i=v}^r k d_i)$
  is constant for $\bl x\myin[0,k-1]$. }
\newcommand{\vanderwafpfjv}{\\We have seen that
  $\bl\chi(a+xd)\meq
  \rput(0,0){\psframe[fillstyle=solid,fillcolor=offwhite,linecolor=lightgray,shadow=false,framearc=0](-.12,-.3)(11.87,.65)}%
  \chi(a'+ \sum_{i=1}^{u-1} 0d_i+ \sum_{i=u}^{v-1} xd_i + \sum_{i=v}^r k d_i)$
  is constant for $\bl x\myin[0,k-1]$. }
\newcommand{\vanderwafpfk}[3]{It also assumes this constant value in $\bl x\meq k$:\vspace*{-2.5mm}
  \begin{center}$\bl{#1{\bl\chi(a+k d) \meq \chi(a'+\sum_{i=u}^rk d_i)}}\;
   {#2{\meq \chi(a'+\sum_{i=v}^rk d_i)}}\;
   {#3{\meq \chi(a+0d)}}\,.$\vspace*{-2.5mm}\end{center}}
\newcommand{\vanderwafpfm}[1]{Hence, $\bl\chi(a+xd)$ is constant for $\bl x\myin[0,k]$.
   {#1{Therefore, \,{\dg S($\bl k+1,1$)}\,.\qed}}\vspace*{-8mm}}

\newcommand{\vanderwag}{\\[2mm]\theorem}
\newcommand{\vanderwaga}{\\\,{\dg S($\bl k,m$)}\, for all $\bl k,m\mgeq 1$.}

\newcommand{\vanderwagpfa}{\\[2mm]\proof}
\newcommand{\vanderwagpfb}{\\The $\bl 1$-equivalence classes of $\bl[0,1]^1$ are $\bl\{0\}$, $\bl\{1\}$,
  so \;{\dg S($\bl 1,\!1$)}\; is trivially~true.}
\newcommand{\vanderwagpfc}{\\The theorem now follows by induction from the lemmas.\qed}

\newcommand{\vanderwapfa}{\\[6mm]{\dg Proof} of {\dg Van der Waerden's Theorem} }
\newcommand{\vanderwapfb}{\\{\dg S($\bl k,1$)}.\qed}


\newcommand{\vanderwgena}{\\[4mm]{\dg Polynomial Van der Waerden's Theorem} {\gray (Bergelson \&\ Leibman 1996)}\\
   If $\bl k,r\myin\mathbb{N}$, $\bl p_1,\ldots,p_k\myin\mathbb{Z}[x]$ with $\bl p_i(0)\meq 0$
   and $\bl n$ sufficiently~large,
   then any $\bl r$-colouring of $\bl [n]$ gives monochromatic $\bl a,a\!+\!p_1(d),\ldots,a\!+\!p_k(d)$.}

\newcommand{\halesjewett}{\\[4mm]{\dg The Hales-Jewett Theorem} {\gray(1963)}
 \\If $\bl k,m,r\myin\mathbb{N}$ and $\bl n$ is sufficiently large,
   then each $\bl r$-colouring of any {\dg cube} $\bl C\meq \{a + \sum_{i=1}^n x_i d_i \::\: 0\mleq x_i\mleq k\}$ of dimension $\bl n$ and length $\bl k$
   gives a monochromatic subcube $\bl C'\subseteq C$ of dimension $\bl m$ and length $\bl k$.}

\newcommand{\szemeredi}{\\[4mm]{\dg Szemer\'{e}di's Theorem} {\gray(1975)}
  \\[-2mm]If $\bl k\in\mathbb{N}$ and $\bl S\subseteq\mathbb{N}$ has positive upper density
  ($\bl\ds \limsup_{n\to\infty} \frac{|S\cap [n]|}{n} \mgt 0$),
  \\[-1mm]then $\bl A$ contains infinitely many arithmetic progressions of length $\bl k$.}

\newcommand{\greentao}{\\[4mm]{\dg The Green-Tao Theorem} {\gray(2004)}
  \\The set of prime numbers contains arbitrarily long arithmetic progressions.\vspace*{-12mm}}



\coursetitle
\np\lecturetitle
\np\lecturetitlei
\np\pigeonthmplusa
\np\pigeonthmplusa\\[6mm]\ramseyaa
\np\pigeonthmplusa\\[6mm]\ramseyaa\\[6mm]\ramseye\ramseyea
\np\ramseye\ramseyea\ramseybexaav\ramseybexabv{\ramseybexabav}
\np\arithma
\np\arithma\arithmaexa
\np\arithma\arithmaexa\arithmaexb{\ph}{\ph}
\np\arithma\arithmaexa\arithmaexb{}{\ph}
\np\arithma\arithmaexa\arithmaexb{}{}
\np\arithma\arithmaexa\arithmaexb{}{}\vanderwa
\np\arithma\arithmaexa\arithmaexb{}{}\vanderwa\vanderwaa
\np\arithmav\vanderwa\vanderwaa\vanderwaexaa
\np\arithmav\vanderwa\vanderwaa\vanderwaexaa\vanderwaexab{\vanderwaexaba}
\np\arithmav\vanderwa\vanderwaa\vanderwaexaa\vanderwaexab{\vanderwaexabb}
\np\arithmav\vanderwa\vanderwaa\vanderwaexaa\vanderwaexab{\vanderwaexabc\vanderwaexabb}
\np\arithmav\vanderwa\vanderwaa\vanderwaexaa\vanderwaexab{\vanderwaexabd\vanderwaexabb}
\np\vanderwa\vanderwaa\vanderwaexaa\vanderwaexab{\vanderwaexabd\vanderwaexabb}
\np\vanderwa\vanderwaa\vanderwaexaa\vanderwaexab{\vanderwaexabd\vanderwaexabb}\vanderwaexba
\np\vanderwa\vanderwaa\vanderwaexaa\vanderwaexab{\vanderwaexabd\vanderwaexabb}\vanderwaexba\vanderwaexbb
\np\vanderwa\vanderwaa\vanderwaexca
\np\vanderwa\vanderwaa\vanderwaexca\vanderwaexcb
\np\vanderwa\vanderwaa\vanderwaexca\vanderwaexcb\vanderwaexcc{}
\np\vanderwa\vanderwaa\vanderwaexca\vanderwaexcb\vanderwaexcc{\vanderwaexcca}
\np\vanderwa\vanderwaa\vanderwaexca\vanderwaexcb\vanderwaexcc{\vanderwaexcca\vanderwaexccb}
\np\vanderwa\vanderwaa\vanderwaexca\vanderwaexcb\vanderwaexcc{\vanderwaexccav\vanderwaexccb}
\np\vanderwa\vanderwaa\vanderwaexca\vanderwaexcb\vanderwaexcc{\vanderwaexccav\vanderwaexcd\vanderwaexcda}
\np\vanderwa\vanderwaa\vanderwaexca\vanderwaexcb\vanderwaexcc{\vanderwaexccav\vanderwaexcd\vanderwaexcdb}
\np\vanderwa\vanderwaa\vanderwaexca\vanderwaexcb\vanderwaexcc{\vanderwaexccav\vanderwaexcd\vanderwaexcdc}
\np\vanderwa\vanderwaa\vanderwab
\np\vanderwa\vanderwab\vanderwac
\np\vanderwa\vanderwab\vanderwac\vanderwacexaa
\np\vanderwa\vanderwab\vanderwac\vanderwacexaa\vanderwacexab
\np\vanderwacv\vanderwacexaa\vanderwacexab
\np\vanderwacv\vanderwacexaa\vanderwacexab\vanderwad
\np\vanderwadv
\np\vanderwadv\vanderwae
\np\vanderwadv\vanderwae\vanderwaea
\np\vanderwadv\vanderwae\vanderwaea\vanderwaepfa
\np\vanderwadv\vanderwae\vanderwaea\vanderwaepfa\vanderwaepfb
\np\vanderwadv\vanderwae\vanderwaea\vanderwaepfa\vanderwaepfb\vanderwaepfc
\np\vanderwadv\vanderwae\vanderwaea\vanderwaepfa\vanderwaepfb\vanderwaepfc\vanderwaepfd
\np\vanderwadv\vanderwae\vanderwaea\vanderwaepfa\vanderwaepfb\vanderwaepfc\vanderwaepfd\vanderwaepfe
\np\vanderwadv\vanderwae\vanderwaea\vanderwaepfa\vanderwaepfb\vanderwaepfc\vanderwaepfd\vanderwaepfe\vanderwaepff
\np\vanderwadv\vanderwae\vanderwaea\vanderwaepfa\vanderwaepfb\vanderwaepfc\vanderwaepfd\vanderwaepfe\vanderwaepff\vanderwaepfg
\np\vanderwadv\vanderwae\vanderwaea\vanderwaepfa\vanderwaepfb\vanderwaepfc\vanderwaepfd\vanderwaepfe\vanderwaepff\vanderwaepfg\vanderwaepfh
\np\vanderwadv\vanderwae\vanderwaea\vanderwaepfa\vanderwaepfb\vanderwaepfc\vanderwaepfd\vanderwaepfe\vanderwaepff\vanderwaepfg\vanderwaepfh\vanderwaepfj
\np\vanderwadv\vanderwae\vanderwaea\vanderwaepfav\vanderwaepfd\vanderwaepfe\\{[...]}\vanderwaepfh\vanderwaepfj
\np\vanderwadv\vanderwae\vanderwaea\vanderwaepfav\vanderwaepfd\vanderwaepfe\\{[...]}\vanderwaepfh\vanderwaepfj\vanderwaepfk
\np\vanderwadv\vanderwae\vanderwaea\vanderwaepfav\vanderwaepfd\vanderwaepfe\\{[...]}\vanderwaepfh\vanderwaepfj\vanderwaepfk\vanderwaepfm
\np\vanderwadv\vanderwae\vanderwaea\vanderwaepfav\vanderwaepfd\vanderwaepfe\\{[...]}\vanderwaepfh\vanderwaepfj\vanderwaepfk\vanderwaepfm\vanderwaepfn
\np\vanderwadv\vanderwaf
\np\vanderwadv\vanderwaf\vanderwafav\vanderwafpfa
\np\vanderwadv\vanderwaf\vanderwafav\vanderwafpfa\vanderwafpfb
\np\vanderwadv\vanderwaf\vanderwafav\vanderwafpfa\vanderwafpfb\vanderwafpfc
\np\vanderwadv\vanderwaf\vanderwafav\vanderwafpfa\vanderwafpfb\vanderwafpfc\vanderwafpfd
\np\vanderwadv\vanderwaf\vanderwafav\vanderwafpfa\vanderwafpfb\vanderwafpfc\vanderwafpfd\vanderwafpfe
\np\vanderwadv\vanderwaf\vanderwafav\vanderwafpfa\vanderwafpfb\vanderwafpfc\vanderwafpfd\vanderwafpfe\vanderwafpff
\np\vanderwadv\vanderwaf\vanderwafav\vanderwafpfa\vanderwafpfb\vanderwafpfc\vanderwafpfd\vanderwafpfe\vanderwafpff\vanderwafpfg
\np\vanderwadv\vanderwaf\vanderwafav\vanderwafpfa\vanderwafpfb\vanderwafpfc\vanderwafpfd\vanderwafpfe\vanderwafpff\vanderwafpfg\vanderwafpfh
\np\vanderwadv\vanderwaf\vanderwafav\vanderwafpfa\vanderwafpfb\vanderwafpfc\vanderwafpfd\vanderwafpfev\vanderwafpff\vanderwafpfg\vanderwafpfh\vanderwafpfjv
\np\vanderwadv\vanderwaf\vanderwafav\vanderwafpfa\vanderwafpfb\vanderwafpfc\vanderwafpfd\vanderwafpfev\vanderwafpff\vanderwafpfg\vanderwafpfh\vanderwafpfjv\vanderwafpfk{\ph}{\ph}{\ph}
\np\vanderwadv\vanderwaf\vanderwafav\vanderwafpfa\vanderwafpfb\vanderwafpfc\vanderwafpfd\vanderwafpfev\vanderwafpff\vanderwafpfg\vanderwafpfh\vanderwafpfjv\vanderwafpfk{}{\ph}{\ph}
\np\vanderwadv\vanderwaf\vanderwafav\vanderwafpfa\vanderwafpfb\vanderwafpfc\vanderwafpfd\vanderwafpfev\vanderwafpff\vanderwafpfg\vanderwafpfh\vanderwafpfjv\vanderwafpfk{}{}{\ph}
\np\vanderwadv\vanderwaf\vanderwafav\vanderwafpfa\vanderwafpfb\vanderwafpfc\vanderwafpfd\vanderwafpfev\vanderwafpff\vanderwafpfg\vanderwafpfh\vanderwafpfjv\vanderwafpfk{}{}{}
\np\vanderwadv\vanderwaf\vanderwafav\vanderwafpfa\vanderwafpfb\vanderwafpfc\vanderwafpfd\vanderwafpfev\vanderwafpff\vanderwafpfg\vanderwafpfh\vanderwafpfjv\vanderwafpfk{}{}{}\vanderwafpfm{\ph}
\np\vanderwadv\vanderwaf\vanderwafav\vanderwafpfa\vanderwafpfb\vanderwafpfc\vanderwafpfd\vanderwafpfev\vanderwafpff\vanderwafpfg\vanderwafpfh\vanderwafpfjv\vanderwafpfk{}{}{}\vanderwafpfm{}
\np\vanderwadv\vanderwae\vanderwaeav\vanderwaf\vanderwafa
\np\vanderwadv\vanderwae\vanderwaeav\vanderwaf\vanderwafa\vanderwag
\np\vanderwadv\vanderwae\vanderwaeav\vanderwaf\vanderwafa\vanderwag\vanderwaga
\np\vanderwadv\vanderwae\vanderwaeav\vanderwaf\vanderwafa\vanderwag\vanderwaga\vanderwagpfa
\np\vanderwadv\vanderwae\vanderwaeav\vanderwaf\vanderwafa\vanderwag\vanderwaga\vanderwagpfa\vanderwagpfb
\np\vanderwadv\vanderwae\vanderwaeav\vanderwaf\vanderwafa\vanderwag\vanderwaga\vanderwagpfa\vanderwagpfb\vanderwagpfc
\np\vanderwadv\vanderwae\vanderwaeav\vanderwaf\vanderwafa\vanderwag\vanderwaga\vanderwagpfa\vanderwagpfb\vanderwagpfc\vanderwapfa
\np\vanderwadv\vanderwae\vanderwaeav\vanderwaf\vanderwafa\vanderwag\vanderwaga\vanderwagpfa\vanderwagpfb\vanderwagpfc\vanderwapfa\vanderwapfb
\np\begin{picture}(0,0)(0,0)\end{picture}\\[-8mm]\vanderwa
\np\begin{picture}(0,0)(0,0)\end{picture}\\[-8mm]\vanderwa\vanderwgena
\np\begin{picture}(0,0)(0,0)\end{picture}\\[-8mm]\vanderwa\vanderwgena\halesjewett
\np\begin{picture}(0,0)(0,0)\end{picture}\\[-8mm]\vanderwa\vanderwgena\halesjewett\szemeredi
\np\begin{picture}(0,0)(0,0)\end{picture}\\[-8mm]\vanderwa\vanderwgena\halesjewett\szemeredi\greentao



\end{document}


