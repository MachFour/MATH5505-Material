\documentclass[a4paper]{article}

\usepackage{amsmath}
\usepackage{amssymb}
\usepackage{amsthm}
\usepackage{fancyhdr}
\usepackage{framed}
\usepackage{hyperref}
%\usepackage{mathrsfs}
%\usepackage[english]{babel}
%\usepackage{graphicx}
%\usepackage{enumerate}

%\theoremstyle{definition}
%\newtheorem{theorem}{Theorem}[section]
%\newtheorem{proposition}[theorem]{Proposition}
%\newtheorem{lemma}[theorem]{Lemma}
%\newtheorem{example}[theorem]{Example}
%\newtheorem{exercise}[theorem]{Exercise}

\expandafter\let\expandafter\oldproof\csname\string\proof\endcsname
\let\oldendproof\endproof
\renewenvironment{proof}[1][\proofname]{%
  \oldproof[\scshape \noindent {\bfseries \text{Proof}}]%
}{\oldendproof}


\setlength{\topmargin}{0mm}
\addtolength{\oddsidemargin}{-.8\oddsidemargin}
\addtolength{\evensidemargin}{-.8\evensidemargin}
\addtolength{\textwidth}{+.3\textwidth}
\addtolength{\textheight}{+.05\textheight}

\newcommand{\example} {\bigskip\noindent{\bf Example.}\;\;}
\newcommand{\exercise}{\bigskip\noindent{\bf Exercise.}\;\;}
\newcommand{\lemma}{\bigskip\noindent{\bf Lemma.}\;\;}
\newcommand{\proposition}{\bigskip\noindent{\bf Proposition.}\;\;}
\newcommand{\remark}{\bigskip\noindent{\bf Remark.}\;\;}
\newcommand{\definition}{\bigskip\noindent{\bf Definition.}\;\;}
\newcommand{\challenge}{\bigskip\noindent{\bf Challenge.}\;\;}
\newcommand{\corollary}{\bigskip\noindent{\bf Corollary.}\;\;}

\newenvironment{thm}[1]{
	\begin{framed}
	\noindent
	{\bfseries #1}\\}{
	\end{framed}
}

\newcommand{\fix} {\begin{picture}(0,0)(0,0)\put(-50,0){\huge Fix!}\end{picture}}

\newcommand{\CC}{\mathbb{C}}
\newcommand{\FF}{\mathbb{F}}
\newcommand{\Ff}{\mathcal{F}}
%\newcommand{\mathbb{N}}{\mathbb{N}}
\newcommand{\RR}{\mathbb{R}}
\newcommand{\Xf}{\mathcal{X}}
\newcommand{\Yf}{\mathcal{Y}}
\newcommand{\ZZ}{\mathbb{Z}}

\pagestyle{fancyplain}

\title{\Large\sc Chapter 1: Ramsey Theory}
\author{Thomas Britz}
\date{}

\begin{document}

\maketitle
\lhead{MATH5505}
\rhead{Combinatorics}

\section{The Pigeonhole Principle}

\begin{thm}{The Pigeonhole Principle (simple)}
If $k+1$ pigeons are put into $k$ pigeonholes,
then some pigeonhole contains at least two pigeons.
\end{thm}

\example
Roll seven six-sided dice.
Let the dice values rolled be ``pigeons"
and the six possible values be ``pigeonholes".
The Pigeonhole Principle tells us that at least one value must appear more than~once.

\exercise
Six distinct numbers are chosen from the set $\{0,\ldots,9\}$.\\
Show that at least two of the chosen numbers must be consecutive.

\exercise
$n+1$ distinct numbers are chosen from the numbers $1,\ldots,2n$.\\
Show that at least two of the chosen numbers are coprime.

\exercise
Can we cover a chessboard with two opposite corners removed using dominoes?

\begin{thm}{Theorem (new proof by Mixon 2012)}
There are infinitely many primes.
\end{thm}

\begin{proof}
Assume that there is a finite number of primes, $p_1,\ldots,p_N$.
Choose some integer $K$ such that $2^K > (K+1)^N$.
There are $2^K$ numbers (``pigeons") in the set $S = \{1,\ldots, 2^K\}$ with
at most $(K+1)^N$ factorisations $p_1^{k_1}, \ldots, p_N^{k_N}$ (``pigeonholes"),
since
\[
  K  = \log_2 2^K
	 \geq \log_2 p_1^{k_1} \ldots p_N^{k_N}
	 = \sum_{i=1}^N k_i \log_2 p_i
	 \geq \sum_{i=1}^N k_i
	 \geq \max_{1 \leq i \leq N} k_i\,.
\]
By the Pigeonhole Principle, two numbers in $S$ have identical factorisation.
They must then be equal, giving a contradiction.
\end{proof}

The next theorem shows that every real number can be approximated by a rational,
and that this approximation converges quickly.

\begin{thm}{Dirichlet's Approximation Theorem (1889)}
If $\alpha \in \mathbb{R}$ and $N \in \mathbb{N}$,
then $p,q \in \mathbb{N}$ exist so that $q \leq N$ and $|\alpha - \frac{p}{q}| < \frac{1}{Nq}$.
\end{thm}

\begin{proof}
Replace the latter inequality equivalently by $| q\alpha - p| < \frac{1}{N}$.
Partition the unit interval $[0,1)$ into $N$ equal subintervals.
Define $N+1$ residues $r_i := i \alpha - \lfloor i\alpha \rfloor$ for $i = 0,1,\ldots, N$.
These lie in the $N$ subintervals.
Let $i<j$.
By the Pigeonhole Principle, residues $r_i, r_j$ ($i<j$) lie in some common subinterval.
Set $p: = \lfloor j\alpha \rfloor - \lfloor i \alpha \rfloor$ and $q: = j-i$.
Then $0 < q \leq N$, and
\[
  |q\alpha - p|
    = |\lfloor j\alpha \rfloor - \lfloor i\alpha \rfloor - (j-i)\alpha|
	= |(i\alpha - \lfloor i \alpha \rfloor) - (j\alpha - \lfloor j \alpha \rfloor)|
	= | r_i - r_j|
	< \frac{1}{N}\,.\qedhere
\]
\end{proof}

\lemma
If $p \equiv 1 \pmod 4$ where $p$ is prime, then $a^2 \equiv -1\pmod p$ for some $a$.

\begin{proof}
Suppose that $p \equiv 1 \pmod 4$ and set $a := \left(\frac{p-1}{2}\right)!$. Note that $p-i \equiv -i \pmod p$ and apply Wilson's Theorem, which states that $(p-1)! \equiv -1 \pmod p$. Then
\begin{align*}
	a^2 = \left(\left(\frac{p-1}{2}\right)!\right)^2
	& \equiv \left(\left(\frac{p-1}{2}\right)!\right)^2 (-1)^{(p-1)/2} \pmod p\\
	& \equiv 1 \times 2 \times \ldots \times \left(\frac{p-1}{2}\right)\left(-\frac{p-1}{2}\right) \times \ldots \times (-2)(-1) \pmod p\\
	& \equiv 1 \times 2 \times \ldots \times \left(\frac{p-1}{2}\right)\left(\frac{p+1}{2}\right) \times \ldots \times (-2)(-1) \pmod p\\
	& \equiv 1 \times 2 \times \ldots \times (p-1) \pmod p\\
	& \equiv -1 \pmod p. \qedhere
\end{align*}
\end{proof}

\begin{thm}{Theorem (Fermat 1640, Euler 1747)}
Each prime $p \equiv 1 \pmod 4$ can be written as a sum of squares $p = x^2 + y^2$.
\end{thm}

\begin{proof}
By the above lemma, we can find some $a$ with $a^2 \equiv -1 \pmod p$.

Now, consider the pairs of integers $(x,y)$ with $0 \leq x,y < \sqrt{p}$.
There are $(\lfloor\sqrt{p}\rfloor +1)^2 > p$ such pairs and at most $p$ values $ax - y \pmod p$.
By the Pigeonhole Principle,
two distinct pairs $(x_1, y_1)$ and $(x_2, y_2)$ must satisfy
$ax_1 -y_1 \equiv ax_2 - y_2 \pmod p$.
Set $x:= x_1 - x_2$ and $y: y_1 - y_2$.
Then
\[
  x^2 + y^2
  \equiv -a^2x^2 + y^2
  \equiv -(a(x_1 - x_2))^2 + (y_1 - y_2)^2
  \equiv 0 \pmod p\,.
\]

Since $(x_1, y_1)$ and $(x_2, y_2)$ are distinct and $0 \leq x_i, y_i < \sqrt{p}$,
we see that
\[
  0 < x^2 + y^2
	< (\sqrt{p})^2 + (\sqrt{p})^2
	= 2p\,.
\]
Hence, $x^2 + y^2 = p$.
\end{proof}

The following theorem states that no lossless compression algorithm works 100\% of the time!

\begin{thm}{Theorem}
Let $f$ be a \emph{lossless digital data compression algorithm}.\\
If $|f(B)| < |B|$ for some data $B$, then $|f(A)| > |A|$ for some data $A$.
\end{thm}

\begin{proof}
Represent data as binary strings and let $|A|$ now denote the string length of $A$.\\
Suppose that $|f(B)| < |B|$ for at least one string $B$
and assume that $|f(A)| \leq |A|$ for all strings $A$.

Let $n$ be the smallest numbers with $|f(B')| < |B'| = n$ for some string $B'$.
Set $m:= |f(B')| (< n)$ and let $\Ff$ be the set of strings $B$ with $|f(B)| = m$.
Each of the $2^m$ binary strings $A$ of length $m$ must obey $|f(A)| \geq |A| = m$.
By the assumption, we have that $|f(A)| = |A| = m$ and so $A \in \Ff$.

Also, $B'$ is also in $\Ff$ so $|\Ff| \geq 2^m + 1$.
So, by the Pigeonhole Principle,
some distinct $B_1$ and $B_2$ satisfy $f(B_1) = f(B_2)$.
But $f$ is lossless and therefore injective, which raises a contradiction.
\end{proof}

\begin{thm}{The Pigeonhole Principle (general)}
If $km +1$ pigeons are put into $k$ pigeonholes,
then some pigeonhole contains at least $m+1$ pigeons.
\end{thm}

\example
If there are 3 pigeonholes and 7 pigeons,
then some pigeonhole must contain at least 3 pigeons. (Here, $m = 2$ and $k=3$.)

\exercise
19 points are drawn in a square with side lengths 1.\\
Show that at least 3 points lie in a circle with radius less than $\frac{1}{4}$.

\begin{thm}{The Erd\H{o}s-Szekeres Theorem (1935)}
Each sequence $a_1, \ldots, a_{n^2+1}$ of $n^2+1$ distinct integers has
a subsequence with $n+1$ elements that is either increasing or decreasing.
\end{thm}

\begin{proof}
Suppose that there is no increasing subsequence that has $n+1$ terms.
Let $n_i$ be the longest length of an increasing subsequence starting in $a_i$,
and note that the ``pigeons" $n_i$ can only be one of the ``pigeonhole" values $1,\ldots, n$.

By the Pigeonhole Principle, there are at least $n+1$ identical $n_i$ values.
Suppose that $n_i = n_j$ for some $i < j$. Then $a_i \not< a_j$, so $a_i > a_j$.
We thus have a decreasing subsequence of at least $n+1$ $a_i$'s.
\end{proof}

\example
Consider the following sequence of $n^2+1$ integers where $n=3$:
\[
  7\ 8\ 9\ 4\ 5\ 6\ 1\ 2\ 3\ 0
\]
There is no increasing subsequence of size $n+1 = 4$.
However, there are 27 decreasing subsequences of size 4:
any of the first three, followed by any of the second three,
followed by any of the third three, followed by 0.

\exercise
Let $C$ be a binary code of length $n=6$ and minimum distance $d=3$.\\
Use the Pigeonhole Principle to show that $C$ has at most 8 codewords.

%Solution:
%Suppose that $|C| = 9$. Then 9 codewords can end in one of the four pairs 00, 01, 10, 11. By the Pigeonhole Principle, 3 codeowrds $c_i$ have the same ending, say 00. The minimum distance is 3, so without loss of generality, assume that
%\begin{align*}
%	c_1 & =  000000,\\
%	c_2 & = 111?00,\\
%	c_3 & = ????00.
%\end{align*}
%The first three bits of $c_3$ must differ from at least two of those of $c_1$. Then the distance between $c_2$ and $c_3$ is at most 2, which presents a contradiction.

\begin{thm}{The Pigeonhole Principle (strong)}
If $(n_1 - 1) + \cdots + (n_k - 1) + 1$ pigeons are put into $k$ pigeonholes,\\
then some $i^{th}$ pigeonhole contains at least $n_i$ pigeons.
\end{thm}



\newpage
\section{Ramsey's Theorem}

Ramsey Theory makes repeated use of the Pigeonhole Principle to show
that order exists whenever (certain) random
and seemingly unordered structures are sufficiently large.

\exercise
Colour each edge of the complete graph $K_6$ either red or blue.\\
Show that there must be a red triangle or a blue triangle.


\begin{thm}{Ramsey's Theorem (1930, simple)}
If $k,m \in \mathbb{N}$ and $n$ is sufficiently large,
then each $k$-colouring of the edges of $K_n$ gives
a complete subgraph $K_m$ with monochromatic edges.
\end{thm}

\begin{proof}
Choose a vertex $v_0$ in $V_0 := V(K_n)$.
Let $c_1$ be a most-occurring edge colour from $v_0$,
and let $V_1$ be the vertices with $c_1$-coloured edge to $v_0$.
Then choose $v_1 \in V_1$ and let $c_2$ be the most-occurring edge colour from $v_1$ to vertices in $V_1$.
Let $V_2$ be the $V_1$ vertices with $c_2$-coloured edges to $v_1$.

For big enough $n$, we can choose $N:= k(m-2) + 1$ colours $c_1, c_2,\ldots, c_N$
and vertices $v_0, v_1, \ldots, v_N$.
By the Pigeonhole Principle, some $m-1$ colours $c_i$ are identical, say colour $c$,
and the $m$ corresponding vertices $v_i$ induces a $K_m$ subgraph with $c$-coloured edges.
\end{proof}

\remark
The least such $n$ is denoted $R(m; k)$.

\exercise
Show that $R(3; 2) = 6$.

\exercise
Use the method of the proof of Ramsey's Theorem to show that $R(3;2) \leq 6$.

\begin{thm}{Ramsey's Theorem (1930, reduced)}
If $n_1, \ldots, n_k\in \mathbb{N}$ and $n$ is sufficiently large,
then each colouring of the edges of $K_n$ with colours $c_1, \ldots, c_k$ gives
a subgraph $K_{n_i}$ with all edges coloured $c_i$ for some $i$.
\end{thm}

\remark
The least such $n$ is denoted $R(n_1, \ldots, n_k)$.
Note that $R(m; k) = R(\underbrace{m, \ldots, m}_{k \text{ times}})$.\vspace*{-4mm}

\example
$R(3,3) = R(3; 2) = 6$.

\bigskip\noindent
The proof of this version of Ramsey's Theorem
is identical to the first proof
except that uses the strong version of the Pigeonhole Principle.

\exercise
Complete the proof of the theorem above.

\lemma (Erd\H{o}s \& Szekeres 1935)
	\[R(k,\ell) \leq R(k-1, \ell) + R(k, \ell-1)\]
	
\begin{proof}
We use induction on $k+ \ell$, noting that $R(k,1) = R(1,k) = 1$ for all $k$.
Assume that $R(k-1, \ell)$ and $R(k, \ell-1)$ are well-defined
and set $N: = R(k-1, \ell) + R(k, \ell-1)$.
Colour each edge of $K_N$ either red or blue and choose a vertex $v$.
Let $R$ and $B$ be the vertices with red and blue edges to $v$, respectively.
Then
\[
    |R| + |B|
  = N-1
  = R(k-1, \ell) + R(k, \ell-1) - 1\,.
\]
Therefore, $|R| \geq R(k-1,\ell)$ or $|B| \geq R(k,\ell-1)$.
Assume without loss of generality that $|R| \geq R(k-1,\ell)$.
If the vertices $R$ induce a blue-edged $K_\ell$, then we are done.
Otherwise, by definition of $R(k-1,\ell)$,
$R$~induces a red-edged $K_{k-1}$.
By adding the $v$ and its red edges, we have a red-edged $K_k$.
\end{proof}
	
\example
For any $k \geq 2$, we note that $R(k,2) = R(2,k) = k$.
The bound in the above lemma is therefore tight for $R(3,3)$:
	\[ R(3,3) = 6 \leq 3 + 3 = R(2,3) + R(3,2)\,.\]
	
\lemma
\[ R(n_1, \ldots, n_k) \leq R(n_1, \ldots, n_{k-2}, R(n_{k-1}, n_k))\]
	
\begin{proof}
We will use induction on $k \geq 2$, noting that the previous lemma gives the case $k=2$.\\
Assume that $R(m_1, \ldots, m_{k-1})$ is well-defined for all $m_i$
and set $N: = R(n_1, \ldots, n_{k-2}, R(n_{k-1}, n_k))$.

Colour each edge of $K_N$ in colours $c_1, \ldots, c_k$.
Now, recolour each $c_{k-1}$- and $c_k$-coloured edges by a new colours $c_{k-1}'$.
By definition, $K_N$ contains either a subgraph $K_{n_i}$ with $c_i$-coloured edges
for some $i \leq k-2$ or a $c_{k-1}'$-coloured $K_{R(n_{k-1}, n_k)}$ subgraph.

Suppose that $K_N$ contains the latter subgraph. This subgraph originally had colours $c_{k-1}, c_k$, so it (and thus $K_N$) contained either a $c_{k-1}$-coloured $K_{n_{k-1}}$ or a $c_k$-coloured $K_{n_k}$. Hence, $K_N$ had a $c_i$-coloured $K_{n_i}$ subgraph for some $i \leq n$.
\end{proof}

\noindent{\bfseries Alternate proof of Ramsey's Theorem (reduced).}
Apply the lemma above.\hfill$\Box$

\definition
\[\hspace*{-5mm}
  [n]: = \{1,\ldots, n\}\hspace*{10mm}\text{and}\hspace*{10mm} \binom{S}{k} := \{ X \subseteq S : |X| = k\}\vspace*{-4mm}
\]

\example\vspace*{-4mm}
\begin{align*}
	[3] = \{1,2,3\} && \binom{[3]}{2} = \bigl\{\{1,2\},\{1,3\},\{2,3\}\bigr\}
\end{align*}

We will now use this new notation to rewrite Ramsey's Theorem.

\begin{thm}{Ramsey's Theorem (1930)}
If $n_1, \ldots, n_k, r\in \mathbb{N}$ and $n$ is sufficiently large,
then each colouring of $\binom{[n]}{r}$ with colours $c_1, \ldots, c_k$ gives
a $c_i$-coloured subfamily $\binom{S}{r}$ for some $i$ and $n_i$-subset $S \subseteq [n]$.
\end{thm}

\remark
The least such $n$ is denoted $R_r (n_1, \ldots, n_k)$.


\begin{proof}
We generalise the original proof, using induction on $r$,
having already proved the case $r = 2$.
Assume that the theorem is true for $r-1$ and set $t: = R_{r-1} (n_1, \ldots, n_k)$.

Choose $r-2$ elements $v_1, \ldots, v_{r-2}$
and set $V_{r-1} : = [n] - \{v_1, \ldots, v_{r-2}\}$.
Continue to choose elements $v_{r-1}, v_r, \ldots$ and sets $V_r, V_{r+1}, \ldots$ as follows.
Suppose that we have chosen $v_1, \ldots, v_{i-1}$ and $V_i$.
Choose $v_i \in V_i$.
Let $V_{i+1}$ be a maximal subsets of $V_i - \{v_i\}$
so that $T \cup \{v\}$ and $T \cup \{w\}$ have the same colour
for all $v, w \in V_{i+1}$ and $(r-1)$-subsets $T \subseteq \{v_1, \ldots, v_i\}$.
If $n$ is sufficiently large, then it is possible to choose $t$ elements $v_1, \ldots, v_t$.

Now colour each $(r-1)$-subset $T$ of these,
giving $T$ the same colour as $T \cup \{v_t\}$ if $v_t \not\in V_t$
and colouring it randomly otherwise.
By definition of $t$,
there is a $n_i$-subset $S$ of the $v_i$'s
whose $(r-1)$-subsets each have colour $c_i$ for some $i$.

Let $U \subseteq S$ be an $r$-subset and let $v$ be the greatest element in $U$.
Then $T :=U - \{v\}$ is a $c_i$-coloured $(r-1)$-sized subset of $S$
and $T \cup \{v_t\}$ must therefore be $c_i$-coloured.
By construction, $T \cup \{v_t\}$ and $U = T \cup \{v\}$ have the same colour, $c_i$.
We see that each $r$-subset $U \subseteq S$ is $c_i$-coloured.
\end{proof}

\remark
The reduced version of Ramsey's Theorem occurs when $r = 2$.


\newpage
\section{Arithmetic Progressions}

\definition
An {\em arithmetic progression} is a sequence of integers of the form $a, a + d, a + 2d, \ldots, a +kd$.

\begin{thm}{Van der Waerden's Theorem (1927)}
If $k,r\in \mathbb{N}$ and $n$ is sufficiently large,
then each $k$-colouring of $[n]$ gives a monochromatic arithmetic progression of length $r$.
\end{thm}

\remark
The least such $n$ is denoted $W(r,k)$.

\exercise
Explain why $W(2,k) = k+1$.

\example
Let us prove the theorem for $r=3$, $k = 2$ by showing that $W(3,2) \leq 325$.

We begin by partitioning the set $[325]$ into blocks $B_1, \ldots, B_{65}$ each of five consecutive numbers.
By the Pigeonhole Principle, the first $33 > 2^5$ $B_\ell$'s contain $B_i$ and $B_{i+d''}$
with the same 2-colouring pattern.
If these contain a monochromatic arithmetic progression of length~3, we are done.

Otherwise, note that the Pigeonhole Principle implies that
at least two of the first 3 numbers of $B_{i}$,
say $i+i'$ and $i+i'+d'$ must have the same colour, say red.
Now consider the number $N:=i+2d''+i'+2d'$ in the block $B_{i+2d}$.
If $N$ is red, then
the numbers $a,a+d,a+2d(=N)$ where $a:= i+i'$ and $d:=d'+d''$ is a red arithmetic progression.
Otherwise, $N$ is green.
Since we assumed that $B_i$ and $B_{i+d''}$ contained
no monochromatic arithmetic progression of length~3,
the numbers $i+2d'$ and $i+2d'+d''$ must be green,
so the numbers $a,a+d,a+2d(=N)$ where $a:= i+i'+2d'$ and $d:=d''$ is a green arithmetic progression.

\bigskip
Ramsey theorem often features arguments such as Cantor diagonalisation type argument in the example above.
This is also at the core of the proof of Van der Waerden's Theorem, below,
applying it to recursively to blocks of blocks of blocks.
To express this without unduly cumbersome notation, let us prove a more general theorem
whose proof relies on the following notions.

The $i^{th}$ $k$-equivalence class of $[0,k]^m = \{0, \ldots, k\}^m$ is the set of vectors
	\[ (x_1, \ldots, x_{m-i}, \underbrace{k, \ldots, k}_{i \text{ times}}),\]
where $x_1, \ldots, x_{m-i} \neq k$.\\

\example
The three $3$-equivalence classes of $[0,3]^2$ are
\begin{align*}
	&\{(0,0), (0,1), (0,2), (1,0), (1,1), (1,2), (2,0), (2,1), (2,2)\}\,;\\
	&\{(0,3), (1,3), (2,3)\}\,;\\
	&\{(3,3)\}\,.
\end{align*}

%
%\fix - what to fix?
%
\noindent
Let $S(k,m)$ be the statement that
\begin{center}
\begin{minipage}{.9\textwidth}
for each $r \in \mathbb{N}$,
there is some $N:= N(k,m,r)$ so that for each $r$-colouring $\chi: [N] \to [r]$
there are $a, d_1, \ldots, d_m \in \mathbb{N}$ so that,
for $x_i \in [k-1]$,
$\chi(a + \sum_i x_i d_i)$ is well-defined and constant on each $k$-equivalence class of $[0,k]^m$.
\end{minipage}
\end{center}

\lemma
If $S(k,m)$ for some $m \geq 1$, then $S(k, m+1)$.

\begin{proof}
Suppose that $S(k,m)$ for some $m \geq 1$ and note that this implies $S(k,1)$.
We wish to show that $S(k, m+1)$.
Letting $r \in \mathbb{N}$, we can find $M:= N(k,m,r)$ and $M':= N(k,1,r^M)$.
Set $N = MM'$ and let $\chi : [N] \to [r]$ be given.
There are at most distinct $r^M$ vectors $(\chi((\ell-1)M+1), \ldots, \chi(\ell M))$
where $\ell \in [M']$, so number these vectors,
inducing a colouring $\chi' : [M'] \to [r^M]$.
Since $M' = N(k,1,r^M)$, we can find $a', d'$ so that
$\chi'(a' + x_1 d')$ is well-defined and constant for all $x_1 \in [0,k-1]$.

Set $d_1 := d' M$.
Then $(\chi((a'-1)M + x_1 d_1 + 1), \ldots, \chi(a' M + x_1 d_1))$ is
well-defined and constant for all $x_1 \in [0,k-1]$.
Fix $x_1$ and define $\chi'' : [M] \to [r]$ by $\chi''(s):=\chi((a'-1)M + x_1 d_1 + s)$.
Since $M = N(k,m,r)$, there are $a'', d_2, \ldots, d_{m+1}$ so that
$\chi'' (a'' + \sum_{i=2}^{m+1} x_i d_i)$ is well-defined and
constant on each $k$-equivalence class of $[0,k]^m$.
Setting $a:= a'' + (a'-1)M$,
we see that $\chi(a + \sum_{i=1}^{m+1} x_i d_i)$ is well-defined
and constant on each $k$-equivalence class of $[0,k]^{m+1}$.

Hence, $N = MM' \geq N(k,m,r)$, and so $S(k,m+1)$.
\end{proof}

\lemma
If $S(k,m)$ for all $m \geq 1$, then $S(k+1,1)$.

\begin{proof}
Suppose that $S(k,m)$ for all $m \leq 1$.
We wish to show that $S(k+1,1)$.
Therefore, let $r \in \mathbb{N}$ and $\chi : [N(k,r,r)] \to [r]$.
Then there are $a', d_1, \ldots, d_r \in \mathbb{N}$ so that
$\chi(a' + \sum_i x_i d_i)$ is well-defined and constant on
each $k$-equivalence class of $[0,k]^r$.

There are at most $r$ function values of $\chi$,
so by the Pigeonhole Principle,
at least two of the $r+1$ values $\chi(a'), \chi(a' + kd_r), \ldots, \chi(a' + \sum_{i=1}^r kd_i)$
must be the same, so $\chi(a' + \sum_{i=u}^r kd_i) = \chi(a' + \sum_{i=v}^r kd_i)$ for $u < v$.
Set $a: = a' + \sum_{i=v}^r kd_i$ and $d:= \sum_{i=u}^{v-1} d_i$.
We have seen that
$\chi(a + xd) = \chi(a' + \sum_{i=1}^{u-1} 0 d_i + \sum_{i=u}^{v-1} xd_i + \sum_{i=v}^r kd_i)$
is constant for $x \in [0,k-1]$.
It also assumes this constant value in $x=k$:\vspace*{-1mm}
\[
	\chi(a+kd)
  = \chi\Bigl(a' + \sum_{i=u}^r kd_i\Bigr)
  = \chi\Bigl(a' + \sum_{i=v}^r kd_i\Bigr)
  = \chi(a + 0d)
\]
Hence, $\chi(a + xd)$ is constant for $x \in [0,k]$.
Therefore, $S(k+1, 1)$.
\end{proof}

\begin{thm}{Theorem}
$S(k,m)$ for all $k,m \geq 1$.
\end{thm}

\begin{proof}
The equivalence classes of $[0,1]^1$ are $\{0\}$ and $\{1\}$, so $S(1,1)$ is trivially true.
The theorem therefore follows from the two above lemmas.
\end{proof}

Note that $S(k,1)$ is exactly Van der Waerden's Theorem, which we have now proved.

\begin{thm}{Polynomial Van der Waerden's Theorem (Bergelson \&\ Leibman 1996)}
If $k,r,n \in \mathbb{N}, p_1, \ldots, p_k \in \ZZ[x]$ with $p_i (0) = 0$
and $n$ sufficiently large, then any $r$-colouring of $[n]$ gives
monochromatic $a, a+ p_1 (d), \ldots, a + p_k (d)$.
\end{thm}

The following theorem is one of the jewels of Ramsey Theorem,
from which many other Ramsey results follow.
We have essentially proved it by proving $S(k,m)$.

\begin{thm}{The Hales-Jewett Theorem (1963)}
If $k,m,r \in \mathbb{N}$ and $n$ is big enough,
then any $r$-colouring of any cube
$C = \{ a+ \sum_{i=1}^n x_i d_i : x_i \in [0,k]\}$ of dimension $n$
and length $k$ has a monochromatic subcube $C' \subseteq C$ of dimension $m$ and length $k$.
\end{thm}

In each of the Ramsey Theory results that we have seen so far,
certain monochromatic substructures are guaranteed - but their particular colour is not given.
Szemer\'edis Theorem (1975) goes beyond these theorems by
guaranteeing monochromatic substructures of some {\em particular colour} if it has sufficient density.
It is one of the most celebrated - and most difficult to prove - results
in Ramsey Theory, and it and its proof form the basis of other great theorems,
including the Green-Tao Theorem (2004).

%\begin{thm}{Szemer\'edis Theorem (1975)}
%If $k \in \mathbb{N}$ and $S \subseteq \mathbb{N}$ has
%positive upper density $\left( \limsup_{n\to\infty} \frac{|S \cap [n]|}{n} \geq 0\right)$,
%then $A$ contains infinitely many arithmetic progressions of length $k$.
%\end{thm}


\newpage
\section{Equations}

\begin{thm}{Schur's Theorem (1916)}
If $\mathbb{N}$ is finitely coloured,
then $a+b = c$ for some same-coloured $a,b,c \in \mathbb{N}$.
\end{thm}

\begin{proof}
Suppose that $\chi: \mathbb{N} \to [r]$ is an $r$-colouring of $\mathbb{N}$.
By Ramsey's Theorem, we may define $N:= R(3;r)$.
Define an edge $r$-colouring $\chi'$ of $K_N$ by $\chi' (\{i,j\}) := \chi(|j-i|)$.
By definition of $N$,
there is a $\chi'$-monochromatic triangle in $K_N$ with vertices $i < j < k$.
Namely, $\chi'(\{j,k\}) = \chi' (\{i,j\}) = \chi'(\{i,k\})$.
Define $a := k-j$, $b:= j-i$ and $c:= k-i$.
Then $a + b = c$ and $\chi(a) = \chi(b) = \chi(c)$.
\end{proof}

From the proof, we also get the following finite version.

\begin{thm}{Schur's Theorem (finite version)}
If $[n]$ is finitely coloured for sufficiently large $n$,
then $a + b = c$ for some same-coloured $a,b,c \in [n]$.
\end{thm}

In fact, all infinite Ramsey theory theorems have finite versions,
by the \emph{Compactness Principle},
a diagonal argument using the Axiom of Choice.

\begin{thm}{Theorem (Schur 1916)}
For each $m \in \mathbb{N}$ and sufficiently large prime $p$,
there are solutions $x,y,z \in \mathbb{N}^+$ to\vspace*{-1mm}
\[ x^m + y^m \equiv z^m \pmod{p}\,.\vspace*{-2mm}\]
\end{thm}

\begin{proof}
By the finite version of Schur's theorem,
there is a large prime $p$ so that if $[p-1]$ is $m$-coloured,
then same-coloured $a,b,c\in[p]$ exist with $a+b=c$.
Consider $\ZZ_p^*$, the multiplicative group of units in $\ZZ_p$.
Define $H := \{x^m : x \in \ZZ_p^*\}$.
This is a subgroup of $\ZZ_p^*$ of index $r:= |\ZZ_p^* : H | = \gcd(p-1,m) \leq m$.
The $r$ cosets $gH$ partition $\ZZ_p^*$,
defining an $r$-colouring $\chi$ of $\ZZ_p^*$: $\chi(a) = \chi(b)$ by
if and only if $a,b \in gH$ for some $g$, i.e. when $ab^{-1} \in H$.
Since $r \leq m$, $\chi$ as an $m$-colouring of $\ZZ_p^*$ (or, equivalently, $[p-1]$).
There are same-coloured $a,b,c \in \ZZ_p^*$ with $a+b = c$ or $ac^{-1} + bc^{-1} = 1$.
Thus, $ac^{-1}, bc^{-1} \in H$,
so $ac^{-1} \in x^m$, $bc^{-1} = y^m$, $1 = z^m$ for $x,y,z \in \ZZ_p^*$.
Then $x^m + y^m = z^m$ in $\ZZ_p$.
\end{proof}

The following theorem generalises for Van der Waerden's Theorem and Schur's Theorem.

\begin{thm}{Theorem}
For each $k,r,s \geq 1$, there is $n:= n(k,r,s)$ so that if $[n]$ is $r$-coloured,
then for some $a,d > 0$, the following set is in $[n]$ and is monochromatic:\vspace*{-1mm}
\[\{a,a+d, \ldots, a +kd\} \cup \{ sd\}\,.\vspace*{-2mm}\]
\end{thm}

\begin{proof}
Use induction on $r$.
Certainly, $n(k,1,s)$ exists and equals $\max\{k+1,s\}$.
Assume that the theorem is true for $r-1$ and set $N:= n(k,r-1,s)$.
By Van der Waerden's Theorem, we may set $W:=W(kN,r)$ and $n:=sW$.
Let $\chi$ be an $r$-colouring of $[n]$.
Then $[W]$ contains an arithmetic progression $\{a, a + d', \ldots, a + (kN)d'\}$
of constant colour $c$.
If $\chi(sjd') = c$ for some $j \in [N]$, then set $d:=jd'$ and we are done:
$\{a,a+d,\ldots, a+kd\} \cup \{ sd\}$ has constant colour $c$.

Suppose that $\chi(sjd') \neq c$ for all $j \in [N]$.
Define a colouring $\chi'$ on $[N]$ by $\chi'(j) : = \chi(sjd')$. \\
Since $N = n(k,r-1,s)$,
there is a $\chi'$-monochromatic set $\{a', a' + d'' , \ldots, a' + kd''\} \cup\{sd''\}$ in $[N]$. \\
Set $a:= sa'd'$ and $d:= sd''d'$.
Then $\{a, a + d, \ldots, a +kd\} \cup \{sd\}$ in $[sNd'] \subseteq [n]$ is monochromatic.
\end{proof}

The following corollary illustrates how to shift the arithmetic progression with respect to $sd$.

\corollary
For each $k,r,s \geq $, there is $n:= n(k,r,s)$ so that if $[n]$ if $r$-coloured, then for some $a,d > 0$, the following set is in $[n]$ and is monochromatic:
	\[ \{a + \lambda d: |\lambda| \leq k \} \cup \{sd\}\,.\]
	
\begin{proof}
Applying $k' = 2k$ to the preceding theorem gives $a', d > 0$
for which $\{a' + \lambda d: \lambda \in [0,2k]\} \cup \{ sd\}$ in $[n]$ is monochromatic.
Now relabel $a:= a' + kd$.
\end{proof}

\begin{thm}{Rado's Theorem (1933, simple)}
Let $c_1, \ldots, c_n \in \ZZ$.
Then for any finite colouring of $\mathbb{N}$,
\[ c_1 x_1 + \cdots + c_n x_n = 0\]
has a monochromatic solution $x_1, \ldots, x_n \in \mathbb{N}$
if and only if $\sum_{i \in I} c_i = 0$ for some subset $I \subseteq [n]$.
\end{thm}

The full version of Rado's Theorem extends this to homogenous systems of linear equations.\\

\example
The numbers $a, a+d, \ldots, a+kd, sd$ form a solution $x_1, \ldots, x_{k+1}$ to the homogenous system of linear equations
  \[
    \begin{array}{rl}
     x_1 - x_0         & = x_2 - x_1    \\
      \vdots           &                     \\
     x_{k-1} - x_{k-2} & = x_k - x_{k-1}\\
     x_{k+1}           & = s(x_1-x_0)
    \end{array}
    \quad\text{; that is,}\quad
    \begin{array}{rl}
     - x_0 + 2x_1 - x_2         & = 0\\
      \vdots      &                       \\
     - x_{k-2} + 2x_{k-1} - x_k & = 0\\
      sx_0 - sx_1 + x_{k+1}     & = 0\,.
    \end{array}
  \]


\definition
For any $S \subseteq \mathbb{N}$,
the \emph{sum set} $\Sigma(S)$ is the set of all finite sums of elements of $S$.

\example
\[ \Sigma(\{1,2,4\}) = \{1,2,3,4,5,6,7\}\]
	
The following theorem follows as a special case of Rado's Theorem.
	
\begin{thm}{Folkman's Theorem (Rado 1970, Sanders 1968)}
For $c,k \in \mathbb{N}$,
there is some sufficiently large $M \in \mathbb{N}$ so that for
any $c$-colouring of $\mathbb{N}$,\\
there is a $k$-subset $S\subseteq [M]$ with monochromatic $\Sigma(S)$.
\end{thm}

\definition
For any set $S \subseteq \mathbb{N}$,
the \emph{product set} $\Pi(S)$ is the set of all finite products of elements of $S$.

\example
\[ \Pi (\{1,2,4\}) = \{1,2,4,8\}\]

\begin{thm}{Theorem}
For $c,k \in \mathbb{N}$, there is some sufficiently large $M \in \mathbb{N}$
so that for any $c$-colouring of $\mathbb{N}$,\\
there is a $k$-subsets $S \subseteq [M]$ with monochromatic $\Pi(S)$.
\end{thm}

\begin{proof}
This result follows from Folkman's Theorem
since any sum set in $[N]$ will induce a product set in $\{2^n : n \in [N]\}$.
\end{proof}

\newpage

\begin{thm}{The Finite Union Theorem}
If the finite subsets of $\mathbb{N}$ are finitely coloured,
then there are arbitrarily large families of disjoint subsets $\mathcal{D} = \{D_i\}_{i \in I}$
whose union sets $\{ \bigcup_{i \in J} D_i : J \subseteq I, 0 < |J| < \infty\}$ are monochromatic.
\end{thm}

It is not hard to show that
the finite version of this theorem is equivalent to Folkman's Theorem,
via the correspondence between sets $S$ and binary (characteristic) vectors $1_S$
and thus numbers $\sum_{i \in S} 2^{i-1}$.

\begin{thm}{Hindman's Theorem (1974)}
For any finite colouring of $\mathbb{N}$,
there is an infinite subsets $S \subseteq \mathbb{N}$
whose sum set $\Sigma(S)$ is monochromatic.
\end{thm}

It is an open question whether arbitrarily large finite sets $S \subseteq \mathbb{N}$
can exist with monochromatic $\Sigma(S) \cup \Pi(S)$ when $\mathbb{N}$ is finitely coloured.
Infinitely large such sets do not always exist: Hindman has shown this.
This question is open for even the smallest sets $S = \{a,b\}$
and their sets $\Sigma(S) \cup \Pi(S) = \{a,b,a+b,ab\}$.


\newpage
\section{Graphs and Geometry}

\begin{thm}{Theorem (Ramras 2002)}
If $k,m \in \mathbb{N}$ such that $k \geq 2$ and $N$ is sufficiently large,
then among any $N$ $N$-subsets $X \subseteq [k(N-1)+1]$,
some $m$ sets have at least $m$ elements in common.
\end{thm}

\begin{proof}
Let $m \in \mathbb{N}$ and assume that the theorem is false.
Set $M:=k(N-1)+1$.
Then there is a family $\Xf$ of $N$ $N$-subsets $X \in \binom{[M]}{N}$,
each $m$ sets of which have at most $m-1$ elements in common.
Then
\[
	N \binom{N}{m}
  = \sum_{X \in \Xf} \sum_{Y \in \binom{X}{m}} 1
  = \sum_{Y \in \binom{[M]}{m}} \sum_{X \in \Xf : Y \subset X} 1
 \leq \sum_{Y \in \binom{[M]}{m}} (m-1)
  = \binom{M}{m} (m-1)\,.
\]
Now, $N\binom{N}{m}$ is a polynomial in $N$ of degree $m+1$,
whereas $\binom{M}{m} (m-1) = \binom{k(N-1)+1}{m} (m-1)$ is a polynomial in $N$ of degree $m$,
so for sufficiently large $N$, we have a contradiction.
\end{proof}

\begin{thm}{The Bipartite Ramsey Theorem (Beineke \&\ Schwenk 1976)}
If $k,m \in \mathbb{N}$  and $n$ is sufficiently large,
then each $k$-colouring of the edges of $K_{n,n}$ gives
a complete monochromatic subgraph $K_{m,m}$.
\end{thm}

\begin{proof}
Let $N$ be sufficiently large as the theorem above,
and set $n:= k(N-1)+1$.
Colour the edges of $K_{n,n}$ with $k$ colours and let $A,B$ be the vertex parts.
Each vertex $a \in A$ is adjacent to $n= k(N-1)+1$ edges.
By the Pigeonhole Principle, $a$ is incident to $N$ edges with the same colour $c(a)$.

There are $n$ vertices $a \in A$, each incident to $N$ edges with colour $c(a)$.
Again by the Pigeonhole Principle,
$N$ of these $n$ colours $c(a)$ are the same, say $c$.
So, some $N$ vertices in $A$ are each adjacent to $N$ vertices in $B$ by edges of colour $c$.
By definition of $N$,
some $m$ vertices in $A$ are each adjacent to
a common set of $m$ vertices in $B$ via edges of colour $c$.
This gives us a $c$-coloured subgraph $K_{m,m}$.
\end{proof}

\begin{thm}{Theorem (Zarankiewicz 1952)}
Let $m \in \mathbb{N}$ and $\epsilon > 0$.
If $n$ is sufficiently large and $G$ is a subgraph of $K_{n,n}$ with at least $\epsilon n^2$ edges, then $G$ has $K_{m,m}$ as a subgraph.
\end{thm}

\definition
For graphs $G,H$,
\begin{center}
\begin{minipage}{.9\textwidth}
\begin{enumerate}
  \item[$\chi(G)$] is the least number of colours in a proper vertex colouring of $G$.\\
    This is the {\em chromatic number} of $G$.
  \item[$c(H)$] is the largest size of a connected component of $H$.
  \item[$r(G,H)$] is the smallest $n$ so that if $E(K_n)$ is coloured red and blue,\\
    then $K_n$ either has a red $G$ or a blue $H$ as a subgraph.
\end{enumerate}
\end{minipage}
\end{center}

\example
$\chi(K_n) = c(K_n) = n$ whereas $\chi(P_n) = 2$ for any path $P_n$ on $n$ vertices,
and $c(mP_n) = n$ where $mP_n$ denotes a disjoint set of $m$ such paths.
We have previously seen that $r(K_3,K_3) = 6$.


\begin{thm}{Theorem (Chv\'atal \&\ Harary 1972)}
	\[r(G,H) \geq (c(G) - 1)(\chi(H) - 1) + 1\]
\end{thm}

\begin{proof}
Set $n:= (c(G) - 1)(\chi(H) - 1)$ and consider $K_n$.
We can find $\chi(H) - 1$ disjoint $K_{c(G) - 1}$ subgraphs of $K_n$.
Colour the edges of these red and all other edges blue.
Then $K_n$ has no red $G$ subgraph nor any blue $H$ subgraph. Hence, $r(G,H) \geq n+1$.
\end{proof}

\begin{thm}{Theorem (Chv\'atal 1977)}
If $T_m$ is a tree on $m$ vertices, then
\[ r(T_m, K_m) = (m-1)(n-1)+1\,.\]
\end{thm}

\example
For $m = 2$, it is clear that $r(T_m, K_n) = n$, as claimed.
Similarly for $n = 2$, $r(T_m, K_n) = m$.

\begin{proof} By the above theorem,
\[
        r(T_m, K_n)
  \geq (c(T_m) - 1)(\chi(K_n) - 1) + 1
    =    (m-1)(n-1)+1\,.
\]
We therefore wish to prove that $r(T_m, K_n) \leq (m-1)(n-1)+1$.
Assume that the theorem is true for all $m', n'$ with $m' + n' < m + n$.

Now, arbitrarily colour the edges of $K_N$ red and blue, where $N:= (m-1)(n-1)+1$.
Let $T_{m-1}$ be a tree obtained by removing an end-vertex from $T_m$
and let $v$ be the vertex that was adjacent to that removed vertex.
By the induction assumption, $K_N$ contains either a blue $K_n$ subgraph, and we are done,
or it has a red $T_{m-1}$ subgraph. Suppose the latter.

Let $K_{N'}$ be the graph obtained by deleting the vertices of $T_{m-1}$ from $K_N$.
Then $K_{N'}$ has $N-(m-1) = (m-1)(n-2) +1$ vertices.
By the induction assumption, $K_{N'}$ has either a red $T_m$ subgraph,
in which case we are done, or it has a blue $K_{n-1}$ subgraph.
Suppose the latter.

We have supposed that red $T_{m-1}$ and blue $K_{n-1}$ are subgraphs of $K_N$.
Now consider the edges from the vertex $v$ in $T_{m-1}$ to the vertices of $K_{n-1}$.
If one of these edges is red, then add it to $v$ to get a red $T_m$ subgraph.
Otherwise, all edges are blue, so add them to $K_{n-1}$ to get a blue $K_n$.
Thus, $K_N$ has either a red $T_m$ subgraph or a blue $K_n$ subgraph.

Hence, $r(T_m, K_n) \leq (m-1)(n-1)+1$, and we are done.
\end{proof}

Surprisingly often, we find that combinatorial results on subsets have analogues on vector subspaces.
This is also true of Ramsey's Theorem which has the following vector space analogues.

\begin{thm}{The Affine Ramsey Theorem (Spencer 1975)}
  Let $\FF$ be a finite field.
  If $k,r,t \in \mathbb{N}$ and $n$ is sufficiently large,
  then each $r$-colouring of the $t$-dimensional subspaces of $\FF^n$ gives
  a $k$-dimensional affine subspace of $\FF^n$
  whose $t$-dimensional subspaces have the same colour.
\end{thm}

\begin{thm}{Vector Space Ramsey Theorem (Graham, Leeb \&\ Rothschild 1972)}
Let $\FF$ be a finite field.
If $k,r,t \in \mathbb{N}$ and $n$ is sufficiently large,
then each $r$-colouring of the $t$-dimensional subspaces of $\FF^n$ gives
a $k$-dimensional vector subspace of $\FF^n$
whose $t$-dimensional subspaces have the same colour.
\end{thm}

\begin{thm}{The Gallai-Witt Theorem}
For each finite subset $V \subseteq \RR^n$ and each finite colouring of $\RR^n$,
there is a monochromatic subset $W \subseteq \RR^n$ that
can be obtained from $V$ by translating and scaling.
\end{thm}

When scaling is not allowed,
then it is easy to find finite sets $V \subseteq \RR^n$ and colourings of $\RR^n$ so that
no translation of $V$ is monochromatic.

\definition
A finite set of points $V$ is \emph{Ramsey}
if for each $r \in \mathbb{N}$ there is a sufficiently large $n$ such that
any $r$-colouring $\RR^n$ has a monochromatic subsets $W$ that is congruent to $V$,
i.e., translating, rotating, and reflecting is allowed.

\begin{thm}{Theorem}
	The following hold:
	\begin{enumerate}
		\item All rectangular parallelepipeds (``bricks") are Ramsey.
		\item In particular, all equilateral simplexes are Ramsey.
		\item Indeed, all regular polygons and polyhedra are Ramsey.
		\item Each Ramsey set can be placed on some sphere.
	\end{enumerate}
\end{thm}




\newpage
\section{Applications}

\begin{thm}{The Happy Ending Problem (Erd\H{o}s \&\ Szekeres 1935)}
If $k \in \mathbb{N}$ and $n$ is sufficiently large,
then among any $n$ points in the plane, no three collinear,
there are $k$ that form a convex polygon.
\end{thm}

\begin{proof}
Set $n:= R_3 (k,k)$.
Then, given any $n$ points, label them $1, \ldots, n$ in any order.
Colour every triple $\{i,j,\ell\}$ green if $i < j < \ell$ is a clockwise sequence,
and red if they form an anti-clockwise sequence.
Then there are $k$ points whose triples are all green or all red.
Without loss of generality, we may assume that this colour is green.
In other words, each triangle among these $k$ points is oriented clockwise.
Then these $k$ points form a convex polygon, and we are done.
\end{proof}


\begin{thm}{Theorem (Erd\H{o}s 1964)}
  Let $A \subseteq \mathbb{N}$ so that for each $n \in \mathbb{N}$,
  there are $a,b \in A$ with $n=ab$.
  Then for each $k \in \mathbb{N}$,
  there is some integer $n \in \mathbb{N}$ so that
  the equation $n=ab$ has at least $k$ solutions with $a,b \in A$.
\end{thm}

\begin{proof}
Consider square-free integers $n$ only.
For such $n$, let $M(n)$ be the set of prime factors of $N$.
We can find a partition $M(n) = M' \cup M''$ so that
$a:= \prod M' := \prod_{p \in M'} p$ and $b:= \prod M''$ both belong to $A$.\\

{\em By Ramsey's theorem, there is a finite set $X \subseteq \mathbb{N}$ with $|X| \geq 2k^2$ so that
for each $M \in \binom{X}{k}$, the partition $M = M' \cup M''$ is of the same `type'.
Hence, there exists $Y \subseteq X$, $|Y| \geq 2k$,
such that $\binom{Y}{m} \subseteq A$ for an $m \geq \frac{1}{2}$.
Thus, for every $M \in \binom{M}{2m}$,
$n:= \prod M$ has $\binom{2m}{m}$ solutions $n=ab$ in $A$.}
\end{proof}

\challenge
Rewrite this proof so that the cursive part is correct and comprehensible!

\begin{thm}{Euclid's Lemma {\normalfont or} Bezout's Identity}
For any coprime $a,b \in \ZZ$, there are $x,y \in \ZZ$ such that $ax + by = 1$.
\end{thm}

\begin{proof}
Consider the $b-1$ remainders of $a,2a, \ldots, (b-1)n$ modulo $b$.
Since $a$ and $b$ are coprime, none of the remainders equal 0.
Now assume that no remainder equals 1.
Then the $b-1$ remainders must be among the numbers $2, \ldots, b-1$.
By the Pigeonhole Principle, two of the remainders must be equal,
so $ma \equiv na \pmod b$ for distinct $m,n > 0$. Hence, $(m-n)a \equiv 0 \pmod b$.
Since $a$ and $b$ are coprime, $m-n \equiv 0 \pmod b$, or $m \equiv n \pmod b$, a contradiction.
Hence, $xa \equiv 1 \pmod b$ for some $x \in \ZZ$ and so $ax + by = 1$ for some $y \in \ZZ$.
\end{proof}

Modified slightly, this proof gives the Chinese Remainder Theorem.

%\newpage
\begin{thm}{Proizvolov's Identity}
Bipartition $[2n]$ into sets $A = \{a_1 > \cdots > a_n\}$ and $B = \{b_1 < \cdots < b_n\}$.
Then
  \[ \sum_{i=1}^n |a_i - b_i| = n^2\,.\]
\end{thm}

\newpage
\begin{proof}
Assume that $a_i, b_i \leq n$ for some $i$.
Then at least $n-(i-1)$ of the $a_j$'s are in $[n]$.
Also, at least $i$ of the $b_j$'s are in $[n]$.
Then at least $n-(i-1)+i = n+1$ of the $a_j$'s and $b_j$'s are in $[n]$.

By the Pigeonhole Principle, two of these are identical, which gives a contradiction.
Hence, $a_i$ and $b_i$ cannot both be less than or equal to $[n]$.
Similarly, $a_i$ and $b_i$ cannot both be bigger than equal to $[n]$.
Hence, one of $a_i$ and $b_i$ is in $[n]$
and the other is in $[2n]-[n] = \{n+1, \ldots, 2n\}$. Then,
\begin{align*}
	\sum_{i=1}^n |a_i - b_i| & = ((n+1) + \cdots + 2n) - (1 + \cdots + n)\\
	& = (\underbrace{n + \cdots + n}_{n \text{ times}}) + (1 + \cdots + n) - (1 + \cdots + n)\\
	& = n^2\qedhere
\end{align*}
\end{proof}

The following theorem is a special case of Tur\'an's Theorem stating when $G$ must contain $K_p$.

\begin{thm}{Mantel's Theorem (1907)}
If $G = (V,E)$ is a simple graph on $n$ vertices with $|E| > \frac{n^2}{4}$ edges,
then $G$ has at least one triangle.
\end{thm}

\begin{proof}
If $n=3$, then if $|E| > \frac{n^2}{4} = 2.25$, so $G$ itself is a triangle.
Assume for induction that the theorem is true for all values up to $n-1$.
Choose an edge $\{u,v\}$ of $G$, and let $H$ be the subgraph of $G$ obtained by deleting $u$ and $v$.

If $|E(H)| > \frac{(n-2)^2}{4}$ edges,
then $H$ and thus $G$ has a triangle, by assumption.\\
Suppose instead that $|E(H)| \leq \frac{(n-2)^2}{4} = \frac{n^2}{4} - n + 1$.
There are
\[
  |E-\{u,v\}| - |E(H)| > \frac{n^2}{4} - \frac{(n-2)^2}{4} - 1 = n-2
\]
edges between $H$ and $\{u,v\}$;
that is, there are at least $n-1$ such edges, and only $n-2$ vertices in $H$.
By the Pigeonhole Principle, some vertex in $H$
is adjacent to both $u$ and~$v$.
Thus, $G$ has a triangle.
\end{proof}


\begin{thebibliography}{9}

\bibitem{AiZi10}
M.~Aigner and G.M.~Ziegler,
{\sl Proofs from The Book}, 4th edition, Springer-Verlag, Berlin, 2010.

\bibitem{bogomolny}
A.~Bogomolny, {\sl Pigeonhole Principle},\\ \url{http://www.cut-the-knot.org/do_you_know/pigeon.shtml}, 2016--03.

\bibitem{brandt01}
J.~Brandt, {\sl Kombinatorik}, Aarhus University, lecture notes, 2001.

\bibitem{GrGrLo95}
R.L.~Graham, M.~Gr\"otschel, and L.~Lov\'asz (eds.),
{\sl Handbook of Combinatorics. I--II}, North-Holland, Amsterdam, 1995.

\bibitem{GrRoSp90}
R.L.~Graham, B.L.~Rothschild, and J.~Spencer,
{\sl Ramsey Theory}, 2nd edition, John Wiley \&\ Sons, Inc., New York, 1990.

\bibitem{LaRob14}
B.M.~Landman and A.~Robertson,
{\sl Ramsey Theory on the Integers}, 2nd edition, AMS, Providence, RI, 2014.

\bibitem{vLiWi92}
J.H.~van Lint and R.M.~Wilson,
{\sl A Course in Combinatorics}, Cambridge University Press, 1992.

\end{thebibliography}

\end{document} 